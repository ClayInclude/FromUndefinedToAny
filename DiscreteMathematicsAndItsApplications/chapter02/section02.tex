%%
%% Author: Clay
%% 2019/12/17
%%

\section{集合运算}
{
    \subsection{引言}
    {
        两个或多个集合可以以许多不同的方式结合在一起。

        \begin{defines}
            令 $A$ 和 $B$ 为集合。
            集合 $A$ 和 $B$ 的并集,用 $A \cup B$ 表示,是一个集合,它包含 $A$ 或 $B$ 或同时在 $A$ 和 $B$ 中的元素。
        \end{defines}

        \begin{defines}
            令 $A$ 和 $B$ 为集合。
            集合 $A$ 和 $B$ 的交集,用 $A \cap B$ 表示,是一个集合,它包含同时在 $A$ 和 $B$ 中的元素。
        \end{defines}

        \begin{defines}
            两个集合称为是不相交的,如果它们的交集为空集。
        \end{defines}

        $$|A \cup B| = |A| + |B| - |A \cap B|$$
        把这一结果推广到任意多个集合的并集就是所谓的包含排斥原理或简称\emreg{容斥原理(principle of inclusion-exclusion)}。

        \begin{defines}
            令 $A$ 和 $B$ 为集合。
            $A$ 和 $B$ 的差集,用 $A - B$ 表示,是一个集合,它包含属于 $A$ 而不属于 $B$ 的元素。
            $A$ 和 $B$ 的差集也成为 $B$ 相对于 $A$ 的补集。
        \end{defines}

        \begin{defines}
            集合 $A$ 和 $B$ 的差集有时候也记为 $A \setminus B$ 。
        \end{defines}

        \begin{defines}
            令 $U$ 为全集。
            集合 $A$ 的补集,用 $\bar{A}$ 表示,是 $A$ 相对于 $U$ 的补集。
            所以,集合 $A$ 的补集是 $U - A$ 。
        \end{defines}
    }

    \subsection{集合恒等式}
    {
        \begin{table}[htb]
            \centering

            \begin{tabular}{l|l}
                \hline
                \multicolumn{1}{c|}{等价式} & \multicolumn{1}{c}{名称}\\
                \hline
                $A \cap U = A$ & \multirow{2}*{恒等律} \\
                $A \cup \varnothing = A$ & \\
                \hline
                $A \cup U = U$ & \multirow{2}*{支配律} \\
                $A \cap \varnothing = \varnothing$ & \\
                \hline
                $A \cup A = A$ & \multirow{2}*{幂等律} \\
                $A \cap A = A$ & \\
                \hline
                $\bar{(\bar{A})} = A$ & 补律 \\
                \hline
                $A \cup B = B \cup A$ & \multirow{2}*{交换律} \\
                $A \cap B = B \cap A$ & \\
                \hline
                $A \cup (B \cup C) = (A \cup B) \cup C$ & \multirow{2}*{结合律} \\
                $A \cap (B \cap C) = (A \cap B) \cap c$ & \\
                \hline
                $A \cup (B \cap C) = (A \cup B) \cap (A \cup C)$ & \multirow{2}*{分配律} \\
                $A \cap (B \cup C) = (A \cap B) \cup (A \cap C)$ & \\
                \hline
                $\bar{A \cap B} = \bar{A} \cup \bar{B}$ & \multirow{2}*{德·摩根律} \\
                $\bar{A \cup B} = \bar{A} \cap \bar{B}$ & \\
                \hline
                $A \cup (A \cap B) = A$ & \multirow{2}*{吸收律} \\
                $A \cap (A \cup B) = A$ & \\
                \hline
                $A \cup \bar{A} = U$ & \multirow{2}*{互补律} \\
                $A \cap \bar{A} = \varnothing$ & \\
                \hline
            \end{tabular}

            \caption{集合恒等式}
        \end{table}

        证明集合相等的另一种方法是证明每一个是另一个的子集。

        集合恒等式还可以通过\emreg{成员表}来证明。
    }

    \subsection{扩展的并集和交集}
    {
        \begin{defines}
            一组集合的并集是包含那些至少是这组集合中一个集合成员的元素的集合。
        \end{defines}

        用下列符号
        $$A_1 \cup A_2 \cup \cdots \cup A_n \cup \cdots = \bigcup_{i = 1}^{\infty} A_i$$
        表示集合 $A_1, A_2, \cdots , A_n, \cdots$ 的并集。

        \begin{defines}
            一组集合的交集是包含那些属于这组集合中所有成员集合的元素的集合。
        \end{defines}

        用下列符号
        $$A_1 \cap A_2 \cap \cdots \cap A_n \cap \cdots = \bigcap_{i = 1}^{\infty} A_i$$
        表示集合 $A_1, A_2, \cdots , A_n, \cdots$ 的交集。

        更一般地,当 $I$ 是一个集合时,我们有 $\cap_{i \in I} A_i = \{x | \forall i \in I (x \in A_i)\}$ 和 $\cup_{i \in I} A_i = \{x | \exists i \in I (x \in A_i)\}$ 。
    }

    \subsection{集合的计算机表示}
    {
        假定全集 $U$ 是有限的。
        首先为 $U$ 的元素任意规定一个顺序,例如 $a_1, a_2, \cdots , a_n$ 。
        于是可以用长度为 $n$ 的位串来表示 $U$ 的子集 $A$ :
        其中位串中第 $i$ 位是$1$ ,如果 $a_i$ 属于 $A$ ;
        是 $0$ ,如果 $a_i$ 不属于 $A$ 。

        用位串表示集合便于计算集合的补集、并集、交集和差集。
    }

    \subsection{练习}
    {
        %1.
        \begin{practices}
            \begin{enumerate}[A.]
                \item 住在离学校一英里以内且走路上学的学生。
                \item 住在离学校一英里以内或走路上学的学生。
                \item 住在离学校一英里以内,并且不是走路上学的学生。
                \item 走路上学,并且不是住在离学校一英里以内的学生。
            \end{enumerate}
        \end{practices}

        %2.
        \begin{practices}
            \begin{enumerate}[A.]
                \item $A \cap B$
                \item $A - B$
                \item $A \cup B$
                \item $\bar{A \cup B}$
            \end{enumerate}
        \end{practices}

        %3.
        \begin{practices}
            \begin{enumerate}[A.]
                \item $\{0, 1, 2, 3, 4, 5, 6\}$
                \item $\{3\}$
                \item $\{1, 2, 4, 5\}$
                \item $\{0, 6\}$
            \end{enumerate}
        \end{practices}

        %4.
        \begin{practices}
            \begin{enumerate}[A.]
                \item $B$
                \item $A$
                \item $\varnothing$
                \item $\{f, g, h\}$
            \end{enumerate}
        \end{practices}

        %5.
        \begin{practices}
            $\bar{\bar{A}} = \{x | x \notin \bar{A}\} = \{x | \neg (\neg x \in A)\} = \{x | x \in A\} = A$
        \end{practices}

        %6.
        \begin{practices}
            \begin{enumerate}[A.]
                \item $A \cup \varnothing = \{x | x \in A \vee x \in \varnothing\} = \{x | x \in A\} = A$
                \item $A \cap \U = \{x | x \in A \wedge x \in \U\} = \{x | x \in A\} = A$
            \end{enumerate}
        \end{practices}

        %7.
        \begin{practices}
            \begin{enumerate}[A.]
                \item $A \cup \U = \{x | x \in A \vee x \in U\} = \{x | x \in U\} = U$
                \item $A \cap \varnothing = \{x | x \in A \wedge x \in \varnothing\} = \{x | x \in \varnothing\} = \varnothing$
            \end{enumerate}
        \end{practices}

        %8.
        \begin{practices}
            \begin{enumerate}[A.]
                \item $A \cup A = \{x | x \in A \vee x \in A\} = \{x | x \in A\} = A$
                \item $A \cap A = \{x | x \in A \wedge x \in A\} = \{x | x \in A\} = A$
            \end{enumerate}
        \end{practices}

        %9.
        \begin{practices}
            \begin{enumerate}[A.]
                \item $A \cup \bar{A} = \{x | x \in A \vee x \in \bar{A}\} = \{x | x \in U\} = U$
                \item $A \cap \bar{A} = \{x | x \in A \wedge x \in \bar{A}\} = \varnothing$
            \end{enumerate}
        \end{practices}

        %10.
        \begin{practices}
            \begin{enumerate}[A.]
                \item $A - \varnothing = \{x | x \in A \wedge x \notin \varnothing\} = \{x | x \in A\} = A$
                \item $\varnothing - A = \{x | x \in \varnothing \wedge x \notin A\} = \varnothing$
            \end{enumerate}
        \end{practices}

        %11.
        \begin{practices}
            \begin{enumerate}[A.]
                \item $A \cup B = \{x | x \in A \vee x \in B\} = B \cup A$
                \item $A \cap B = \{x | x \in A \wedge x \in B\} = B \cap A$
            \end{enumerate}
        \end{practices}

        %12.
        \begin{practices}
            假设 $x \in (A \cap B)$ ,则 $x \in A \wedge x \in B$ ,并且 $x \in (A \cup (A \wedge B))$ 。
            如果 $x \notin (A \cap B)$ ,则 $x \notin A, x \notin (A \cap B)$ 。
        \end{practices}

        %13.
        \begin{practices}
            假设 $x \in (A \cap (A \cup B))$ ,则 $x \in A \wedge x \in (A \cup B)$ 。
            如果 $x \notin (A \cap (A \cup B))$ ,则 $x \notin A \vee x \notin (A \cup B)$ 。
        \end{practices}

        %14.
        \begin{practices}
            $A = (A - B) \cup (A \cap B) = \{1, 3, 5, 6, 7, 8, 9\}$

            $B = (B - A) \cup (A \cap B) = \{2, 3, 6, 9, 10\}$
        \end{practices}

        %15.
        \begin{practices}
            \begin{align*}
                \bar{A \cup B}
                &= \{x | x \notin (A \cup B)\} \\
                &= \{x | x \notin A \wedge x \notin B\} \\
                &= \bar{A} \cap \bar{B}
            \end{align*}
        \end{practices}

        %16.
        \begin{practices}
            \begin{enumerate}[A.]
                \item $(x \in (A \cap B)) = (x \in A \wedge x \in B) \rightarrow (x \in A)$
                \item $(x \in A) \rightarrow (x \in A \vee x \in B) = (x \in A \cup B)$
                \item $(x \in A \wedge x \notin B) \rightarrow (x \in A)$
                \item $(x \in A \wedge (x \in B \wedge x \notin A)) = (\varnothing)$
            \end{enumerate}
        \end{practices}

        %17.
        \begin{practices}
            \begin{align*}
                \bar{A \cap B \cap C}
                &= \{x | x \notin (A \cap B \cap C)\} \\
                &= \{x | x \notin A \vee x \notin B \vee x \notin C\} \\
                &= \bar{A} \cup \bar{B} \cup \bar{C}
            \end{align*}
        \end{practices}

        %18.
        \begin{practices}
            \begin{enumerate}[A.]
                \item $(x \in A \vee x \in B) \rightarrow (x \in A \vee x \in B \vee x \in C)$
                \item $(x \in A \wedge x \in B \wedge x \in C) \rightarrow (x \in A \wedge x \in B)$
                \item $x \in A \wedge x \notin B \wedge x \notin C \rightarrow x \in A \wedge x \notin C$
                \item $x \in A \wedge x \notin C \wedge x \in C \wedge x \notin B \rightarrow \varnothing$
                \item $(x \in B \wedge x \notin A) \vee (x \in C \wedge x \notin A) \rightarrow (x \in B \vee x \in C) \wedge x \notin A$
            \end{enumerate}
        \end{practices}

        %19.
        \begin{practices}
            \begin{enumerate}[A.]
                \item $x \in A \wedge x \notin B \rightarrow x \in A \wedge x \in \bar{B}$
                \item $(x \in A \wedge x \in B) \vee (x \in A \wedge x \notin B) \rightarrow (x \in A) \vee (x \in \varnothing) \rightarrow x \in A$
            \end{enumerate}
        \end{practices}

        %20.
        \begin{practices}
            \begin{enumerate}[A.]
                \item
                {
                    如果 $x \in A$ ,则 $x \in B$ 。
                    所以 $x \in A \rightarrow x \in B \rightarrow (x \in A \vee x \in B)$ 。
                }
                \item
                {
                    如果 $x \in A$ ,则 $x \in (A \cap B)$ 。
                }
            \end{enumerate}
        \end{practices}

        %21.
        \begin{practices}
            \begin{align*}
                x \in A \vee (x \in B \vee x \in C) \\
                &= (x \in A \vee x \in B) \vee x \in C
            \end{align*}
        \end{practices}

        %22.
        \begin{practices}
            \begin{align*}
                x \in A \wedge (x \in B \wedge x \in C) \\
                &= (x \in \wedge x \in B) \wedge x \in C
            \end{align*}
        \end{practices}

        %23.
        \begin{practices}
            \begin{align*}
                x \in A \vee (x \in B \wedge x \in C) \\
                &= (x \in A \vee x \in B) \wedge (x \in A \vee x \in C)
            \end{align*}
        \end{practices}

        %24.
        \begin{practices}
            \begin{align*}
                (x \in A \wedge x \notin B) \wedge x \notin C \\
                &= (x \in A \wedge x \notin C) \wedge x \notin B \\
                &= (x \in A \wedge x \notin C) \wedge x \notin (B - C)
            \end{align*}
        \end{practices}

        %25.
        \begin{practices}
            \begin{enumerate}[A.]
                \item $\{4, 6\}$
                \item $\{0, 1, 2, 3, 4, 5, 6, 7, 8, 9, 10\}$
                \item $\{4, 6, 8, 10\}$
                \item $\{2, 4, 5, 6, 7, 8, 9, 10\}$
            \end{enumerate}
        \end{practices}

        %26.
        \begin{practices}

        \end{practices}

        %27.
        \begin{practices}

        \end{practices}

        %28.
        \begin{practices}

        \end{practices}

        %29.
        \begin{practices}
            \begin{enumerate}[A.]
                \item $B \subseteq A$
                \item $A \subseteq B$
                \item $A \cap B = \varnothing$
                \item 交换律
                \item $A = B \wedge A - B = \varnothing \wedge B - A = \varnothing$
            \end{enumerate}
        \end{practices}

        %30.
        \begin{practices}
            \begin{enumerate}[A.]
                \item 不能
                \item 不能
                \item 能
            \end{enumerate}
        \end{practices}

        %31.
        \begin{practices}
            \begin{align*}
                x \in A \rightarrow x \in B
                &\equiv \neg (x \in B) \rightarrow \neg (x \in A) \\
                &\equiv x \notin B \rightarrow x \notin A
            \end{align*}
        \end{practices}

        %32.
        \begin{practices}
            $\{2, 5\}$
        \end{practices}

        %33.
        \begin{practices}
            主修计算机科学或主修数学的学生并且不同时主修另一门学科。
        \end{practices}

        %34.
        \begin{practices}
            
        \end{practices}

        %35.
        \begin{practices}
            定义。
        \end{practices}

        %36.
        \begin{practices}
            定义。
        \end{practices}

        %37.
        \begin{practices}
            \begin{enumerate}[A.]
                \item $(A \cup A) - (A \cap A) = \varnothing$
                \item $(A \cup \varnothing) - (A \cap \varnothing) = A$
                \item $(A \cup U) - (A \cap U) = \bar{A}$
                \item $(A \cup \bar{A}) - (A \cap \bar{A}) = U$
            \end{enumerate}
        \end{practices}

        %38.
        \begin{practices}
            \begin{enumerate}[A.]
                \item $(A \cup B) - (A \cap B) = (B \cup A) - (B \cap A)$
                \item
                {
                    \begin{align*}
                        (((A \cup B) - (A \cap B)) \cup B) - (((A \cup B) - (A \cap B)) \cap B) \\
                        &= (A \cup B) - B \\
                        &= A
                    \end{align*}
                }
            \end{enumerate}
        \end{practices}

        %39.
        \begin{practices}
            $B = \varnothing$
        \end{practices}

        %40.
        \begin{practices}
            成立。
        \end{practices}

        %41.
        \begin{practices}
            是。
        \end{practices}

        %42.
        \begin{practices}
            成立。
        \end{practices}

        %43.
        \begin{practices}
            成立。
        \end{practices}

        %44.
        \begin{practices}
            如果 $|A| = n, |B| = m$ ,则 $|A + B| <= m + n$ 。
        \end{practices}

        %45.
        \begin{practices}
            如果 $A$ 为无限集, $|B| = m$ ,则 $A \cup B$ 中的基数大于等于 $A$ 的基数,也为无限集。
        \end{practices}

        %46.
        \begin{practices}
            $A, B, C$ 的基数相加,减去同时属于 $A, B$ 、 $A, C$ 和 $B, C$ 的。
            同时属于 $A, B, C$ 的加了三次,又被减了三次,所以再次加上。
        \end{practices}

        %47.
        \begin{practices}
            \begin{enumerate}[A.]
                \item $A_i$
                \item $A_1$
            \end{enumerate}
        \end{practices}

        %48.
        \begin{practices}
            \begin{enumerate}[A.]
                \item $A_i$
                \item $A_1$
            \end{enumerate}
        \end{practices}

        %49.
        \begin{practices}
            \begin{enumerate}[A.]
                \item $\{1, x, 3, \cdots , i\}$
                \item $\varnothing$
            \end{enumerate}
        \end{practices}

        %50.
        \begin{practices}
            \begin{enumerate}[A.]
                \item $Z^+$; $\varnothing$
                \item $A_i$; $\{0\}$
                \item $\{x | x > 0\}$; $A_1$
                \item $\{x | x > 1\}$; $\varnoting$
            \end{enumerate}
        \end{practices}

        %51.
        \begin{practices}
            \begin{enumerate}[A.]
                \item $Z$; $A_1$
                \item $Z^*$; $\varnothing$
                \item $R$; $A_1$
                \item $A_1$; $\varnothing$
            \end{enumerate}
        \end{practices}

        %52.
        \begin{practices}
            \begin{enumerate}[A.]
                \item $0011 1000 00$
                \item $1010 0100 01$
                \item $0111 0011 10$
            \end{enumerate}
        \end{practices}

        %53.
        \begin{practices}
            \begin{enumerate}[A.]
                \item $\{1, 2, 3, 4, 7, 8, 9, 10\}$
                \item $\{2, 4, 5, 6, 7\}$
                \item $\{1, 10\}$
            \end{enumerate}
        \end{practices}

        %54.
        \begin{practices}
            \begin{enumerate}[A.]
                \item $\varnothing$
                \item $Z^+$
            \end{enumerate}
        \end{practices}

        %55.
        \begin{practices}
            前一个集合为 $1$ 并且后一个为 $0$ 时,为 $1$ ,否则为 $0$ 。
        \end{practices}

        %56.
        \begin{practices}
            异或。
        \end{practices}

        %57.
        \begin{practices}
            \begin{enumerate}[A.]
                \item $\{a, b, c, d, e, g, p, t, v\}$
                \item $\{b, c, d\}$
                \item $\{b, c, d, e, i, o, t, u, x, y\}$
                \item $\{a, b, c, d, e, g, h, i, n, o, p, t, u, v, x, y, z\}$
            \end{enumerate}
        \end{practices}

        %58.
        \begin{practices}
            与运算和或运算。
        \end{practices}

        %59.
        \begin{practices}
            \begin{enumerate}[A.]
                \item $\{1, 2, 3, \{1, 2, 3\}\}$
                \item $\{\varnothing\}$
                \item $
            \end{enumerate}
        \end{practices}
    }
}
