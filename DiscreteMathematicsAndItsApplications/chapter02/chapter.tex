%%
%% Author: Clay
%% 2019/12/13
%%

\chapter{基本结构:集合、函数、序列、求和与矩阵}
{
    离散数学的许多内容主要研究离散结构,用以表示离散对象。
    许多重要的离散结构是用集合来构建的,集合就是对象的汇集。
    由集合构建的离散结构包括:

    \begin{description}
        \item[组合] 无序对象汇集,广泛用于计数
        \item[关系] 序偶的集合用于表示对象之间的关系
        \item[图] 结点和连接结点的边的集合
        \item[有限状态机] 为计算机器建模
    \end{description}

    函数的概念在离散数学中是非常重要的。
    函数给第一个集合中的每一个元素指派第二个集合中的恰好一个元素。
    可以用于表示算法的计算复杂度,研究的集合的大小,计算对象的数量,以及无数的其他应用方式。
    序列和字符串这样非常有用的结构就是特殊类型的函数。

    通过引入一个集合的大小或基数的概念就可以研究无限集合的相对大小问题。
    当一个集合是有限的或者与正整数的集合具有一样的大小,我们说这个集合是可数的。

    矩阵在离散数学中可用于表示很多种离散结构。

    %%
%% Author: Clay
%% 2020/12/5
%%

\section{死锁的原理}
{
    死锁是指一组进程因为竞争系统资源或互相等待消息,而永远无法向前推进的状态。

    \subsection{可重用资源}
    {
        资源可以分为两个大类:
        可重用资源与消耗性资源。

        可重用资源指的是一次只供一个进程使用,但用后又能被别的进程使用的资源。

        从系统的角度出发,能够解决死锁问题的方法之一是对应用申请系统资源的顺序做出限定。
    }
}

}

\cleardoublepage

\endinput
