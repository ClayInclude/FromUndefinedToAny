%%
%% Author: Clay
%% 2019/12/13
%%

\chapter{基本结构:集合、函数、序列、求和与矩阵}
{
    离散数学的许多内容主要研究离散结构,用以表示离散对象。
    许多重要的离散结构是用集合来构建的,集合就是对象的汇集。
    由集合构建的离散结构包括:

    \begin{description}
        \item[组合] 无序对象汇集,广泛用于计数
        \item[关系] 序偶的集合用于表示对象之间的关系
        \item[图] 结点和连接结点的边的集合
        \item[有限状态机] 为计算机器建模
    \end{description}

    函数的概念在离散数学中是非常重要的。
    函数给第一个集合中的每一个元素指派第二个集合中的恰好一个元素。
    可以用于表示算法的计算复杂度,研究的集合的大小,计算对象的数量,以及无数的其他应用方式。
    序列和字符串这样非常有用的结构就是特殊类型的函数。

    通过引入一个集合的大小或基数的概念就可以研究无限集合的相对大小问题。
    当一个集合是有限的或者与正整数的集合具有一样的大小,我们说这个集合是可数的。

    矩阵在离散数学中可用于表示很多种离散结构。

    %%
%% Author: Clay
%% 2020/2/21
%%

\section{信息存储}
{
    大多数计算机使用8位的块,或者\emreg{字节(byte)},作为最小的可寻址的内存单位,而不是访问内存中单独的位。
    机器级程序将内存视为一个非常大的字节数组,称为\emreg{虚拟内存(virtual memory)}。
    内存的每个字节都由一个唯一的数字来标识,称为它的\emreg{地址(address)},所有可能地址的集合就称为\emreg{虚拟地址空间(virtual address space)}。

    编译器和运行时系统将存储器空间划分为更可管理的单元,来存放不同的\emreg{程序对象(program object)}。

    \subsection{十六进制表示法}
    {
        \emreg{十六进制(hexadecimal, hex)}使用数字 $0 \sim 9$ 和字符 $A \sim F$ 来表示16个可能的值。

        通过展开每个十六进制数字,将它转换为二进制格式。

        如果给定一个二进制数字,可以通过把它分为每4位一组来转换为十六进制。
        如果位总数不是4的倍数,最左边的一组可以少于4位,前面用0补足。

        %1.
        \begin{practicec}
            \begin{enumerate}[A.]
                \item $11 1001 1010 0111 1111 1000$
                \item $0xC97A$
                \item $1101 0101 1110 0100 1100$
                \item $0x26E7B5$
            \end{enumerate}
        \end{practicec}

        当值 $x = 2^n$ 时, $x$ 的二进制表示就是 $1$ 后面跟 $n$ 个 $0$ 。
        十六进制数字 $0$ 代表 4个二进制 $0$ 。
        所以当 $n$ 表示成 $i + 4j$ 的形式,其中 $0 \leq i \leq 3$ ,可以把 $x$ 写为开头的十六进制数字为 $1(i = 0), 2(i = 1), 4(i = 2), 8(i = 3)$ ,后面紧跟 $j$ 个十六进制的 $0$ 。

        %2.
        \begin{practicec}
            \begin{table}[H]
                \[
                    \begin{array}{|c|c|c|}
                        \hline
                        n & 2^n \text{(十进制)} & 2^n \text{(十六进制)} \\
                        \hline
                        9 & 512 & 0x200 \\
                        \hline
                        19 & 524288 & 0x80000 \\
                        \hline
                        14 & 16384 & 0x4000 \\
                        \hline
                        16 & 65536 & 0x10000 \\
                        \hline
                        17 & 131072 & 0x20000 \\
                        \hline
                        5 & 32 & 0x20 \\
                        \hline
                        7 & 128 & 0x80 \\
                        \hline
                    \end{array}
                \]
            \end{table}
        \end{practicec}

        将一个十进制数字 $x$ 转换为十六进制,可以反复地用 $16$ 除 $x$ ,得到一个商 $q$ 和一个余数 $r$ ,也就是 $x = q \cdot 16 + r$ 。
        然后用十六进制数字表示的 $r$ 作为最低为数字,并通过对 $q$ 反复进行这个过程得到剩下的数字。

        将一个十六进制数字转换为十进制数字,可以用相应的 $16$ 的幂乘以每个十六进制数字。

        %3.
        \begin{practicec}
            \begin{table}[H]
                \[
                    \begin{array}{|c|c|c|}
                        \hline
                        \text{十进制} & \text{二进制} & \text{十六进制} \\
                        \hline
                        0 & 0000 0000 & 0x00 \\
                        \hline
                        167 & 1010 0111 & A7 \\
                        \hline
                        62 & 0011 1110 & 3E \\
                        \hline
                        188 & 1011 1100 & BC \\
                        \hline
                        55 & 0011 0111 & 37 \\
                        \hline
                        136 & 1000 1000 & 88 \\
                        \hline
                        243 & 1111 0011 & F3 \\
                        \hline
                        82 & 0101 0010 & 0x52 \\
                        \hline
                        172 & 1010 1100 & 0xAC \\
                        \hline
                        231 & 1110 0111 & 0xE7 \\
                        \hline
                    \end{array}
                \]
            \end{table}
        \end{practicec}

        %4.
        \begin{practicec}
            \begin{enumerate}[A.]
                \item $0x5044$
                \item $0x4FFC$
                \item $0x50A0$
                \item $0x9E$
            \end{enumerate}
        \end{practicec}
    }

    \subsection{字数据大小}
    {
        每台计算机都有一个\emreg{字长(word size)},指明指针数据的\emreg{标称大小(nominal size)}。
        对于一个字长为 $w$ 位的机器而言,虚拟地址的范围为 $0 \sim 2^w - 1$ ,程序最多访问 $2^w$ 个字节。

        整数或者为\emreg{有符号的},即可以表示负数、零和正数;
        或者为\emreg{无符号的},即只能表示非负数。
    }

    \subsection{寻址和字节顺序}
    {
        在几乎所有的机器上,多字节对象都被存储为连续的字节序列,对象的地址为所使用字节中最小的地址。

        考虑一个 $w$ 位的整数,其表示为 $[x_{w - 1}, x_{w - 2}, \cdots, x_1, x_0]$ ,其中 $x_{w - 1}$ 是最高有效位,而 $x_0$ 是最低有效位。
        假设 $w$ 是 $8$ 的倍数,这样这些位就能被分组成为字节。
        某些机器选择在内存中按照从最低有效字节到最高有效字节的顺序存储对象,而另一些机器则按照从最高有效字节到最低有效字节的顺序存储。
        前一种规则称为\emreg{小端法(little endian)},后一种规则称为\emreg{大端法(big endian)}。

        有时候,字节序会成为问题。

        \begin{enumerate}
            \item
            {
                首先是在不同类型的机器之间通过网络传送二进制数据时,一个常见的问题是当小端法机器产生的数据被发送到大端机器或者反过来时,接收程序会发现,字里的字节成了反序的。
            }
            \item
            {
                第二种情况时,当阅读表示整数数据的字节序列时字节顺序也很重要。
            }
            \item
            {
                第三种情况是当编写规避正常的类型系统的程序时。
            }
        \end{enumerate}
    }
}

    %%
%% Author: Clay
%% 2020/12/15
%%

\section{静态链接}
{
    \emreg{静态链接器(static linker)}以一组可重定位目标文件和命令行参数作为输入,生成一个完全链接的、可以加载和运行的可执行目标文件作为输出。
    输入的可重定位目标文件由各种不同的代码和数据\emreg{节(section)}组成,每一节都是一个连续的字节序列。

    为了构造可执行文件,链接器必须完成两个任务:

    \begin{description}
        \item[符号解析(symbol resolution)]
        {
            目标文件定义和引用\emspe{符号},每个符号对应一个函数、一个全局变量或一个\emspe{静态变量}(即C语言中任何以\emcode{static}属性声明的变量)。
            符号解析的目的是将每个符号\emspe{引用}正好和一个符号\emspe{定义}关联起来。
        }
        \item[重定位(relocation)]
        {
            编译器和汇编器生成从地址0开始的代码和数据节。
            链接器通过把每个符号定义与一个内存位置关联起来,从而\emspe{重定位}这些节,然后修改所有对这些符号的引用,使得它们指向这个内存位置。
            链接器使用汇编器产生的\emspe{重定位条目(relocation entry)}的详细指令。
        }
    \end{description}

    目标文件纯粹是字节块的集合。
    这些块中,有些包含程序代码,有些包含程序数据,而其他的则包含引导链接器和加载器的数据结构。
    链接器将这些块连接起来,确定被连接块的运行时位置,并且修改代码和数据块中的各种位置。
    链接器对目标机器了解甚少,产生目标文件的编译器和汇编器已经完成了大部分工作。
}

    %%
%% Author: Clay
%% 2020/2/29
%%

\section{整数运算}
{
    \subsection{无符号加法}
    {
        要想完整的表示算术运算的结果,不能对字长做任何限制。
        \emcode{List}实际上就支持\emreg{无限精度}的运算,允许任意的(要在机器的内存限制之内)整数运算。
        更常见的是,编程语言支持固定精度的运算。

        \begin{defines}[无符号数加法]
            对满足 $0 \leq x, y < 2^w$ 的 $x, y$ 有:

            \begin{align}
                x +_w^u y =
                \begin{cases}
                    x + y, x + y < 2^w
                    \\
                    x + y - 2^w, 2^w \leq x + y < 2^{w + 1}
                \end{cases}
            \end{align}
        \end{defines}

        说一个算数\emreg{溢出},是指完整的整数结果不能放到数据类型的字长限制中去。

        \begin{defines}[检测无符号数加法中的溢出]
            对在范围 $0 \leq x, y \leq UMax_w$ 中的 $x, y$ ,令 $s = x +_w^u y$ 。
            则对计算 $s$ ,当且仅当 $s < x$ (或者等价地 $s < y$ )时,发生了溢出。
        \end{defines}

        %27.
        \begin{practicec}

        \end{practicec}

        模数加法形成了一种数学结构,称为\emreg{阿贝尔群(Abelian group)}。
        也就是说,它是可交换的和可结合的。
        它有一个单位元 $0$ ,并且每个元素有一个加法逆元。

        \begin{defines}[无符号数求反]
            对满足 $0 \leq x < 2^w$ 的任意 $x$ ,其 $w$ 位的无符号逆元 $-_w^ux$ 由下式给出:

            \begin{align}
                -_w^u x =
                \begin{cases}
                    x, x = 0
                    \\
                    2^w - x, x > 0
                \end{cases}
            \end{align}
        \end{defines}

        %28.
        \begin{practicec}
            \begin{table}[H]
                \[
                    \begin{array}{|c|c|c|c|}
                        \hline
                        \multicolumn{2}{|c|}{x} & \multicolumn{2}{c|}{-_4^ux} \\
                        \hline
                        \text{十六进制} & \text{十进制} & \text{十进制} & \text{十六进制} \\
                        \hline
                        0 & 0 & 0 & 0 \\
                        \hline
                        5 & 5 & 11 & B \\
                        \hline
                        8 & 8 & 8 & 8 \\
                        \hline
                        D & 13 & 3 & 3 \\
                        \hline
                        F & 15 & 1 & 1 \\
                        \hline
                    \end{array}
                \]
            \end{table}
        \end{practicec}
    }

    \subsection{补码加法}
    {
        \begin{defines}[补码加法]
            对满足 $-2^{w - 1} \leq x, y \leq 2^{w - 1} - 1$ 的整数 $x, y$ ,有:

            \begin{align}
                x+_w^ty =
                \begin{cases}
                    x + y - 2^w, 2^{w - 1} \leq x + y
                    \\
                    x + y, -2^{w - 1} \leq x + y < 2{w - 1}
                    \\
                    x + y + 2^w, x + y < -2^{w - 1}
                \end{cases}
            \end{align}
        \end{defines}

        \begin{defines}[检测补码加法中的溢出]
            对满足 $TMin_w \leq x, y \leq TMax_w$ 的 $x, y$ ,令 $s = x +_w^t y$ 。
            当且仅当 $x > 0, y > 0$ ,但 $s \leq 0$ 时,计算发生了\emspe{正溢出(positive overflow)}。
            当且仅当 $x < 0, y < 0$ ,但 $s \geq 0$ 时,计算发生了\emspe{负溢出(negative overflow)}。
        \end{defines}

        %29.
        \begin{practicec}
            \begin{table}[H]
                \[
                    \begin{array}{|c|c|c|c|c|}
                        \hline
                        x & y & x + y & x +_5^t y & \text{情况} \\
                        \hline
                        10100 & 10001 & -27 & 5 & \text{负溢出} \\
                        \hline
                        11000 & 11000 & -16 & -16 & \text{正常} \\
                        \hline
                        10111 & 01000 & -1 & -1 & \text{正常} \\
                        \hline
                        00010 & 00101 & 7 & 7 & \text{正常} \\
                        \hline
                        01100 & 00100 & 16 & -16 & \text{正溢出} \\
                        \hline
                    \end{array}
                \]
            \end{table}
        \end{practicec}

        %30.
        \begin{practicec}

        \end{practicec}

        %31.
        \begin{practicec}
            如果有 $sum = x + y$ ,由于补码加法形成了一个阿贝尔群,满足交换律,则有 $sum - x = x + y - x$ ,得出 $sum - x = y$ 。
            无论溢出都会得到相同的结果。
        \end{practicec}

        %32.
        \begin{practicec}
            $y$ 为 $TMin$ 时。
        \end{practicec}
    }

    \subsection{补码的非}
    {
        \begin{defines}[补码的非]
            对满足 $TMin_w \leq x \leq TMax_w$ 的 $x$ ,其补码的非 $-_w^tx$ 由下式给出

            \begin{align}
                -_w^tx =

                \begin{cases}
                    TMin_w, x = TMin_w
                    \\
                    -x, x ? TMin_w
                \end{cases}
            \end{align}
        \end{defines}

        %33.
        \begin{practicec}
            \begin{table}[H]
                \[
                    \begin{array}{|c|c|c|c|}
                        \hline
                        \multicolumn{2}{|c|}{x} & \multicolumn{2}{c|}{-_4^tx} \\
                        \hline
                        \text{十六进制} & \text{十进制} & \text{十进制} & \text{十六进制} \\
                        \hline
                        0 & 0 & 0 & 0 \\
                        \hline
                        5 & 5 & -5 & A \\
                        \hline
                        8 & -8 & -8 & 8 \\
                        \hline
                        D & -3 & 3 & 3 \\
                        \hline
                        F & -1 & 1 & 1 \\
                        \hline
                    \end{array}
                \]
            \end{table}
        \end{practicec}
    }

    \subsection{无符号乘法}
    {
        \begin{defines}[无符号数乘法]
            对满足 $0 \leq x, y \leq < UMax_w$ 的 $x, y$ 有:

            \begin{align}

            \end{align}
        \end{defines}
    }
}

}

\cleardoublepage

\endinput
