%%
%% Author: Clay
%% 2019/12/13
%%

\section{集合}
{
    \subsection{引言}
    {
        所有其他离散结构都建立于集合之上。
        集合可用于把对象聚集在一起。
        通常,一个集合中的对象都有相似的性质,但也不绝对。
        集合语言是以有组织的方式来研究这些集合的工具。

        \begin{defines}
            集合是对象的一个无序的聚集,对象也称为集合的元素(element)或成员(member)。
            集合包含(contain)它的元素。
            我们用 $a \in A$ 来表示 $a$ 是集合 $A$ 中一个元素。
            而记号 $a \notin A$ 表示 $a$ 不是集合 $A$ 中的一个元素。
        \end{defines}

        通常用大写字母来表示集合。
        用小写字母表示集合中的元素。

        描述集合有多种方式。
        一种方式是在可能的情况下一一列出集合中的元素。
        采用在花括号之间列出所有元素的方法。
        这种描述集合的方法也是称为\emreg{花名册方法(roster method)}。

        描述集合的另一种方式是使用\emreg{集合构造器(set builder)}符号。
        通过描述作为集合的成员必须具有的性质来刻画集合中的那些元素。
        当不可能列出集合中所有元素时常用这类记法来描述集合。

        这些集合通常用黑体表示:

        \begin{align*}
            N &= {0, 1, 2, 3, \cdots}, \text{自然数集} \\
            Z &= {\cdots, -2, -1, 0, 1, 2, \cdots}, \text{整数集} \\
            Z^+ &= {1, 2, 3, \cdots}, \text{正整数集} \\
            Q &= {p / q | p \in Z, q \in Z, \text{且} q \neq 0}, \text{有理数集} \\
            R, &\text{实数集} \\
            R^+, &\text{正实数集} \\
            C, &\text{复数集}
        \end{align*}

        回顾表示实数的\emreg{区间}记号。
        当 $a$ 和 $b$ 是实数且 $a < b$ 时,可以写:

        \begin{align*}
            [a, b] &= \{x | a \leq x \leq b\} \\
            [a, b) &= \{x | a \leq x < b\} \\
            (a, b] &= \{x | a < x \leq b\} \\
            (a, b) &= \{x | a < x < b\}
        \end{align*}

        \begin{defines}
            计算机科学中的数据类型或类型的概念是建立在集合这一概念上。
            特别地,数据类型或类型是一个集合的名称。
            连同可以作用在集合对象上的一组操作的集合。
        \end{defines}

        \begin{defines}
            两个集合相等当且仅当它们拥有同样的元素。
            所以,如果 $A$ 和 $B$ 是集合,则 $A$ 和 $B$ 是相等的当且仅当 $\forall x (x \in A \leftrightarrow x \in B)$ 。
            如果 $A$ 和 $B$ 是相等的集合,就记为 $A = B$ 。
        \end{defines}

        集合中元素的排列顺序无关紧要,同一个元素被列出来不止一次也没关系。

        \paraph{空集}
        {
            有一个特殊的不含任何元素的集合。
            这个集合称为\emreg{空集(empty set, null set)},并用 $\varnothing$ 表示。
            空集也可以用 $\{\}$ 表示。
            经常具有一定性质的元素组成的集合其实就是空集。

            只有一个元素的集合叫做\emreg{单元素集(singleton set)}。
            要给常见的错误是混淆空集 $\varnothing$ 和 单元素集合 $\{\varnothing\}$ 。
            集合 $\{\varnothing\}$ 唯一元素是空集本身。
        }

        \paraph{朴素集合论}
        {
            集合定义中用到的术语\emreg{对象},而没有指定一个对象是什么。
            基于对象的直觉概念基础上,将集合描述为对象的聚集最先是由德国数学家乔治·康托于1895年提出的。
            由集合的直觉定义以及无论什么性质都存在一个恰好由具有该性质的对象组成的集合这种直觉概念的使用所产生的理论导致\emreg{悖论(paradox)}或逻辑不一致性。
            由英国哲学家伯特兰·罗素在1902年所证实。
            这些逻辑不一致性可以通过由公理出发构造集合论来避免。
            康托集合论的原始版本,即所谓的\emreg{朴素集合论(naive set theory)}。
        }
    }

    \subsection{文氏图}
    {
        集合可以用文氏图形象的表示。
        在文氏图中\emreg{全集(Universal set)} $U$ ,包含所考虑的全部对象,用矩形框来表示。
        全集随着所关注的对象会有所不同。
        在矩形框内部,圆形或其他几何图形用于表示集合。
        有时候用来表示集合中特定的元素。
        文氏图常用于表示集合之间的关系。
    }

    \subsection{子集}
    {
        \begin{defines}
            集合 $A$ 是集合 $B$ 的子集当且仅当 $A$ 的每个元素也是 $B$ 的元素。
            用记号 $A \subseteq B$ 表示集合 $A$ 是集合 $B$ 的子集。

            $A \subseteq B$ 当且仅当量化式 $\forall x (x \in A \rightarrow x \in B)$ 为真。
            要证明 $A$ 不是 $B$ 的子集,只需要找到一个元素 $x \in A$ 但 $x \notin B$ 。
        \end{defines}

        \begin{defines}
            对于任意集合 $S$ , $(\rmnum 1) \varnothing \subseteq S, (\rmnum 2) S \subseteq S$ 。
        \end{defines}

        我们要强调集合 $A$ 是集合 $B$ 的子集但是 $A \neq B$ 时,就写成 $A \subset B$ 并说 $A$ 是 $B$ 的\emreg{真子集}。
    }
}
