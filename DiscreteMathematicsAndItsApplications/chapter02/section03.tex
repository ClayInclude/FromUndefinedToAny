%%
%% Author: Clay
%% 2020/01/09
%%

\section{函数}
{
    \subsection{引言}
    {
        \begin{defines}
            令 $A$ 和 $B$ 为非空集合。
            从 $A$ 到 $B$ 的函数 $f$ 是对元素的一种指派,对 $A$ 的每个元素恰好指派 $B$ 的一个元素。
            如果 $B$ 中元素 $b$ 是唯一由函数 $f$ 指派给 $A$ 中元素 $a$ 的,则我们就写成 $f(a) = b$ 。
            如果 $f$ 是从 $A$ 到 $B$ 的函数,就写成 $f: A \rightarrow B$ 。
        \end{defines}

        \begin{defines}
            函数有时也称为\emspe{映射(mapping)}或者\emspe{变换(transformation)}。
        \end{defines}

        $A$ 到 $B$ 的关系就是集合 $A \times B$ 的子集。
        对于 $A$ 到 $B$ 的关系,如果对每一个元素 $a \in A$ 都有且仅有一个序偶 $(a, b)$ ,则它就定义了 $A$ 到 $B$ 的一个函数 $f$ 。
        这个函数通过指派 $f(a) = b$ 来定义,其中 $(a, b)$ 是关系中唯一以 $a$ 为第一个元素的序偶。

        \begin{defines}
            如果 $f$ 是 $A$ 到 $B$ 的函数,我们说 $A$ 是 $f$ 的\emspe{定义域(domain)},而 $B$ 是 $f$ 的\emspe{陪域(codomain)}。
            如果 $f(a) = b$ ,我们说 $b$ 是 $a$ 的\emspe{像(image)},而 $a$ 是 $b$ 的\emspe{原像(preimage)}。
            $f$ 的\emspe{值域(range)}或像是 $A$ 中元素的所有像的集合。
            如果 $f$ 是 $A$ 到 $B$ 的函数,我们说 $f$ 把 $A$ \emspe{映射(map)}到 $B$ 。
        \end{defines}

        当两个函数有相同的定义域、陪域,定义域中的每个元素映射到陪域中相同的元素时,这两个函数是\emreg{相等的}。

        一个函数称为是\emreg{实值函数}如果其陪域是实数集合。
        如果其陪域是整数集合则称为\emreg{整数值函数}。

        \begin{defines}
            令 $f_1$ 和 $f_2$ 是从 $A$ 到 $R$ 的函数,那么 $f_1 + f_2$ 和 $f_1f_2$ 也是从 $A$ 到 $R$ 的函数,其定义为对于任意 $x \in A$

            \begin{align*}
                (f_1 + f_2)(x) &= f_1(x) + f_2(x)
                (f_1f_2)(x) &= f_1(x)f_2(x)
            \end{align*}
        \end{defines}

        \begin{defines}
            令 $f$ 为从 $A$ 到 $B$ 的函数, $S$ 为 $A$ 的一个子集。
            $S$ 在函数 $f$ 下的像是由 $S$ 中元素的像组成的 $B$ 的子集。
            我们用 $f(S)$ 表示 $S$ 的像,于是

            \begin{align*}
                f(S) = {t | \exists s \in S (t = f(s))}
            \end{align*}

            也用简写 $\{f(s) | s \in S\}$ 来表示这个集合。
        \end{defines}

        \begin{defines}
            用 $f(S)$ 表示集合 $S$ 在函数 $f$ 下的像可能会有潜在的二义性。
            这里, $f(S)$ 表示一个集合,而不是函数 $f$ 在集合 $S$ 处的值。
        \end{defines}
    }

    \subsection{一对一和映上函数}
    {
        \begin{defines}
            函数 $f$ 是\emspe{一对一(one-to-one)}或\emspe{单射函数(injection)},当且仅当对于 $f$ 的定义域中的所有 $a$ 和 $b$ 有 $f(a) = f(b)$ 蕴含 $a = b$ 。
            一个函数如果是一对一的 ,就称为是单射的(injective)。
        \end{defines}

        \begin{defines}
            我们可以用量词来表达 $f$ 是一对一的,如 $\forall a \forall b (f(a) = f(b) \rightarrow a = b)$ 或等价地 $\forall a \forall b (a \neq b \rightarrow f(a) \neq f(b))$ ,其中论域是函数的定义域。
        \end{defines}

        \begin{defines}
            定义域和陪域都是实数集子集的函数 $f$ 称为是递增的,如果对 $f$ 的定义域中的 $x$ 和 $y$ ,当 $x < y$ 时有 $f(x) \leq f(y)$ ;称为是严格递增的,如果当 $x < y$ 时有 $f(x) < f(y)$ 。
            类似地, $f$ 称为是递减的,如果对 $f$ 的定义域中的 $x$ 和 $y$ ,当 $x < y$ 时有 $f(x) \geq f(y)$ ;称为是严格递减的,当 $x < y$ 时有 $f(x) > f(y)$ (定义中严格一词意味着严格不等式)。
        \end{defines}

        \begin{defines}
            一个从 $A$ 到 $B$ 的函数 $f$ 称为\emspe{映上(onto)}或\emspe{满射(surjection)}函数,当且仅当对每个 $b \in B$ 有元素 $a \in A$ 使得 $f(a) = b$ 。一个函数如果是映上的就称为是\emspe{满射的(surjective)}。
        \end{defines}

        \begin{defines}
            函数 $f$ 是\emspe{一一对应(one-to-one correspondance)}或\emspe{双射(bijection)}函数,如果它既是一对一又是映上的。
            这样的函数称为是\emspe{双射的(bijective)}。
        \end{defines}

        恒等函数是这样的函数,它给每个元素指派到自身.
        函数 $\iota_A$ 是一对一和映上的,所以他是双射函数。

        假设 $f : A \rightarrow B$ 。

        \begin{description}
            \item[要证明 $f$ 是单射的:] 证明对于任意 $x, y \in A$ ,如果 $f(x) = f(y)$ ,则 $x = y$ 。
            \item[要证明 $f$ 不是单射的:] 找到特定的 $x, y \in A$ ,使得 $x \neq y \wedge f(x) = f(y)$ 。
            \item[要证明 $f$ 是满射的:] 考虑任意元素 $y \in B$ ,并找到一个元素 $x \in A$ 使得 $f(x) = y$ 。
            \item[要证明 $f$ 不是满射的:] 找到一个特定的 $y \in B$ ,使得对于任意元素 $x \in A$ 有 $f(x) \neq y$ 。   
        \end{description}
    }

    \subsection{反函数和函数组合}
    {
        \begin{defines}
            令 $f$ 为从集合 $A$ 到 $B$ 的一一对应。
            $f$ 的反函数(或逆函数)是这样的函数,它指派给 $B$ 中元素 $b$ 的是 $A$ 中使得 $f(a) = b$ 的唯一元素 $a$ 。
            $f$ 的反函数用 $f^{-1}$ 表示。
            于是,当 $f(a) = b$ 是 $f^{-1}(b) = a$ 。
        \end{defines}

        如果函数不是一一对应的,就无法定义反函数。
        一一对应关系称为\emreg{可逆的(invertible)},否则称为\emreg{不可逆的(not invertible)}。

        \begin{defines}
            令 $g$ 为从集合 $A$ 到集合 $B$ 的函数, $f$ 是从集合 $B$ 到集合 $C$ 的函数,函数 $f$ 和 $g$ 的\emspe{合成(composition)},记作 $f \circ g$ ,定义为对任意 $a \in A$
            $$(f \circ g)(a) = f(g(a))$$
        \end{defines}

        \begin{defines}
            $f \circ g$ 和 $g \circ f$ 并不相等。
            换言之,对函数的合成而言交换律不成立。
        \end{defines}

         在构造函数和它的反函数的合成时,不论以什么次序合成,得到的都是恒等函数。
         因此 $f^{-1} \circ f = \iota_A$ 和 $f \circ f^{-1} = \iota_B$ 。
         这就是说, $(f^{-1})^{-1} = f$ 。
    }

    \subsection{函数的图}
    {
        \begin{defines}
            令 $f$ 为从集合 $A$ 到集合 $B$ 的函数,函数 $f$ 的\emspe{图像(graph)}是序偶集合 $\{(a, b) | a \in A \wedge f(a) = b\}$ 。
        \end{defines}
    }

    \subsection{一些重要的函数}
    {
        \begin{defines}
            下取整函数(floor)指派给实数 $x$ 的是小于等于 $x$ 的最大整数,用 $\lfloor x \rfloor$ 表示。
            上取整函数(ceil)指派给实数 $x$ 的是大于等于 $x$ 的最小整数,用 $\lceil x \rceil$ 表示。
        \end{defines}

        \begin{defines}
            下取整函数也常称为最大整数函数,这时经常用 $[x]$ 表示。
        \end{defines}

        \begin{table}[htb]
            \[
                \begin{array}{l}
                    \hline
                    \lfloor x \rfloor \leftrightarrow n \leq x < n + 1 \\
                    \lceil x \rceil \leftrightarrow n - 1 < x \leq x \\
                    \lfloor x \rfloor \leftrightarrow x - 1 < n \leq x \\
                    \lceil x \rceil \leftrightarrow x \leq n < x + 1 \\
                    \hline
                    x - 1 < \lfloor x \rfloor \leq x \leq \lceil x \rceil < x + 1 \\
                    \hline
                    \lfloor -x \rfloor = - \lceil x \rceil \\
                    \lceil -x \rceil = - \lfloor x \rfloor \\
                    \hline
                    \lfloor x + n \rfloor = \lfloor x \rfloor + n \\
                    \lceil x + n \rceil = \lceil x \rceil + n \\
                    \hline
                \end{array}
            \]
        \end{table}

        常用的另一个函数是\emreg{阶乘函数 $f : N \rightarrow Z^+$},几位 $f(n) = n!$ 。
    }

    \subsection{部分函数}
    {
        在抽象的数学里,也常讨论那些只在实数的一个子集上有定义的函数。

        \begin{defines}
            一个从集合 $A$ 到集合 $B$ 的\emspe{部分函数(partial function)} $f$ 是给 $A$ 的一个子集(称为 $f$ 的\emspe{定义域(domain of definition)}中的每个元素 $a$ 指派唯一的一个 $B$ 中的元素 $b$ 。
            集合 $A$ 和 $B$ 分别称为 $f$ 的域和陪域。
            我们说 $f$ 对于 $A$ 中但不在 $f$ 的定义域中的元素\emspe{无定义(undefined)}。
            当 $f$ 的定义域等于 $A$ 时,就说 $f$ 是\emspe{全函数(total function)}。
        \end{defines}
    }

    \subsection{练习}
    {
        %1.
        \begin{practices}
            \begin{enumerate}[A.]
                \item $f$ 对于 $x = 0$ 无定义。
                \item $f$ 对于 $x < 0$ 无定义。
                \item 不是函数。
            \end{enumerate}
        \end{practices}

        %2.
        \begin{practices}
            \begin{enumerate}[A.]
                \item 不是函数。
                \item 是。
                \item $f$ 对于 $n = 2$ 无定义。
            \end{enumerate}
        \end{practices}

        %3.
        \begin{practices}
            \begin{enumerate}[A.]
                \item 不是函数。
                \item 是函数。
                \item 不是函数。
            \end{enumerate}
        \end{practices}

        %4.
        \begin{practices}
            \begin{enumerate}[A.]
                \item 定义域:非负整数,值域: $\{0, 1, 2, 3, 4, 5, 6, 7, 8, 9\}$ 。
                \item 定义域:正整数,值域:非负整数。
                \item 定义域:所有位串,值域:非负整数。
                \item 定义域:所有位串,值域:非负整数。
            \end{enumerate}
        \end{practices}

        %5.
        \begin{practices}
            \begin{enumerate}[A.]
                \item 定义域:所有位串,值域:非负整数。
                \item 定义域:所有位串,值域:非负偶数。
                \item 定义域:所有位串,值域: $\{0, 1, 2, 3, 4, 5, 6, 7\}$ 。
                \item 定义域:正整数,值域:所有最大完全平方数。
            \end{enumerate}
        \end{practices}

        %6.
        \begin{practices}
            \begin{enumerate}[A.]
                \item 定义域:正整数序偶,值域:正整数。
                \item 定义域:正整数,值域: $\{0, 1, 2, 3, 4, 5, 6, 7, 8, 9\}$ 。
                \item 定义域:所有位串,值域:非负整数。
                \item 定义域:正整数,值域:正整数。
                \item 定义域:所有位串,值域: $\{\varnothing, 1, 11, 111, 1111, \cdots\}$ 。
            \end{enumerate}
        \end{practices}

        %7.
        \begin{practices}
            \begin{enumerate}[A.]
                \item 定义域:正整数序偶,值域:正整数。
                \item 定义域:正整数,值域: $\{0, 1, 2, 3, 4, 5, ,6 ,7, 8, 9, 10\}$ 。
                \item 定义域:所有位串,值域:非负整数。
                \item 定义域:所有位串,值域:非负整数。
            \end{enumerate}
        \end{practices}

        %8.
        \begin{practices}
            \begin{enumerate}[A.]
                \item $1$
                \item $2$
                \item $-1$
                \item $0$
                \item $3$
                \item $-2$
                \item $1$
                \item $2$
            \end{enumerate}
        \end{practices}

        %9.
        \begin{practices}
            \begin{enumerate}[A.]
                \item $1$
                \item $0$
                \item $0$
                \item $-1$
                \item $3$
                \item $-1$
                \item $2$
                \item $1$
            \end{enumerate}
        \end{practices}

        %10.
        \begin{practices}
            \begin{enumerate}[A.]
                \item 是
                \item 不是
                \item 不是
            \end{enumerate}
        \end{practices}

        %11.
        \begin{practices}
            A
        \end{practices}

        %12.
        \begin{practices}
            \begin{enumerate}[A.]
                \item 是
                \item 不是
                \item 是
                \item 不是
            \end{enumerate}
        \end{practices}

        %13.
        \begin{practices}
            A
        \end{practices}

        %14.
        \begin{practices}
            \begin{enumerate}[A.]
                \item 是
                \item 不是
                \item 是
                \item 是
                \item 不是
            \end{enumerate}
        \end{practices}

        %15.
        \begin{practices}
            \begin{enumerate}[A.]
                \item 是
                \item 不是
                \item 是
                \item 不是
                \item 是
            \end{enumerate}
        \end{practices}

        %16.
        \begin{practices}
            \begin{enumerate}[A.]
                \item 是
                \item 是
                \item 不是
                \item 不是
            \end{enumerate}
        \end{practices}

        %17.
        \begin{practices}
            \begin{enumerate}[A.]
                \item 不是
                \item 不是
                \item 不是
                \item 是
            \end{enumerate}
        \end{practices}

        %18.
        \begin{practices}
            \begin{enumerate}[A.]
                \item 班上所有学生的电话号码的集合。
                \item 班上所有学生学号的集合。
                \item 班上所有学生最后得分的集合。
                \item 班上所有学生家乡的集合。
            \end{enumerate}
        \end{practices}

        %19.
        \begin{practices}
            \begin{enumerate}[A.]
                \item 所有老师办公室的集合。
                \item 所有老师制定的巴士的集合。
                \item 所有老师薪水的集合。
                \item 所有老师社会保险号的集合。
            \end{enumerate}
        \end{practices}

        %20.
        \begin{practices}
            \begin{enumerate}[A.]
                \item $f(n) = n + 1$
                \item $f(n) = \lceil n / 2 \rceil$
                \item $f(n) = -n$
                \item $f(n) = n^2$
            \end{enumerate}
        \end{practices}

        %21.
        \begin{practices}
            \begin{enumerate}[A.]
                \item $f(n) = |n| + 1$
                \item $f(n) = |n|$
                \item 当 $n \geq 0$ 时, $f(n) = 2x + 1$ ,当 $n < 0$ 时, $f(n) = -2x$
                \item $f(n) = n^2$
            \end{enumerate}
        \end{practices}

        %22.
        \begin{practices}
            \begin{enumerate}[A.]
                \item 是
                \item 不是
                \item 不是
                \item 是
            \end{enumerate}
        \end{practices}

        %23.
        \begin{practices}
            \begin{enumerate}[A.]
                \item 是
                \item 不是
                \item 是
                \item 不是
            \end{enumerate}
        \end{practices}

        %24.
        \begin{practices}
            若 $f(x)$ 是严格递增的,则令 $a < b$ ,有 $f(a) < f(b)$ ,由 $g(x) = 1 / f(x)$ 可得 $g(a) > g(b)$ ,则 $g(x)$ 为严格递减。
            若 $g(x)$ 是严格递减的,则令 $a < b$ ,有 $g(a) > g(b)$ ,由 $g(x) = 1 / f(x)$ 可得 $f(a) < f(b)$ ,则 $f(x)$ 为严格递增。
        \end{practices}

        %25.
        \begin{practices}
            同上。
        \end{practices}

        %26.
        \begin{practices}
            \begin{enumerate}[A.]
                \item
                {
                    若 $f(x)$ 是从 $R$ 到自身的严格递增函数,则不失一般性 $f(a) < f(b)$ ,则 $a < b$ 。
                    故 $(f(a) \neq f(b)) \rightarrow a \neq b$ ,等价于 $(a = b) \leftrightarrow (f(a) = f(b))$ 。
                    根据定义, $f(x)$ 是一对一的。
                }
                \item
                {
                    $f(x) = 0$
                }
            \end{enumerate}
        \end{practices}

        %27.
        \begin{practices}
            同上。
        \end{practices}

        %28.
        \begin{practices}
            因为 $e^x$ 始终大于 $0$ ,故其不是满射。
        \end{practices}

        %29.
        \begin{practices}
            因为 $x \neq 0$ 时, $|x| = |-x|$ ,故其不是单射。
        \end{practices}

        %30.
        \begin{practices}
            \begin{enumerate}[A.]
                \item $\{1\}$
                \item $\{-1, 1, 5, 8, 15\}$
                \item $\{0, 1, 2\}$
                \item $\{0, 1, 5, 16\}$
            \end{enumerate}
        \end{practices}

        %31.
        \begin{practices}
            \begin{enumerate}[A.]
                \item $\{1, 0, 3\}$
                \item $\{0, 1, 3, 5, 8\}$
                \item $\{0, 8, 16, 40\}$
                \item $\{1, 12, 33, 35\}$
            \end{enumerate}
        \end{practices}

        %32.
        \begin{practices}
            \begin{enumerate}[A.]
                \item $Z$
                \item $N$
                \item $R$
            \end{enumerate}
        \end{practices}

        %33.
        \begin{practices}
            \begin{enumerate}[A.]
                \item
                {
                    令 $a \neq b$ ,又有 $f(x)$ 和 $g(x)$ 都为单射函数,故 $g(a) \neq g(b) \wedge f(g(a)) \neq f(g(b))$ 。
                    所以 $(f \circ g)(a) \neq (f \circ g)(b)$ , $f \circ g$ 为单射函数。
                }
                \item
                {
                    令 $c \in C$ ,又有 $f(x)$ 和 $g(x)$ 都为满射函数,故存在 $b \in B$ 和 $a \in A$ 使得  $f(b) = c \wedge g(a) = b$ 。
                    所以 $(f \circ g)(a) = c$ , $f \circ g$ 为满射函数。
                }
            \end{enumerate}
        \end{practices}

        %34.
        \begin{practices}
            不能。
        \end{practices}

        %35.
        \begin{practices}
            不能。
        \end{practices}

        %36.
        \begin{practices}
            \begin{align*}
                f \circ g
                &= f(g(x)) \\
                &= f((x + 2)) \\
                &= (x + 2)^2 + 1 \\
                &= x^2 + 4x + 5
            \end{align*}

            \begin{align*}
                g \circ f
                &= g(f(x)) \\
                &= g((x^2 + 1)) \\
                &= x^2 + 3
            \end{align*}
        \end{practices}

        %37.
        \begin{practices}
            \begin{align*}
                f + g
                &= f(x) + g(x) \\
                &= x^2 + 1 + x + 2 \\
                &= x^2 + x + 3
            \end{align*}

            \begin{align*}
                fg
                &= f(x)g(x) \\
                &= (x^2 + 1)(x + 2) \\
                &= x^3 + 2x^2 + x + 2
            \end{align*}
        \end{practices}

        %38.
        \begin{practices}
            根据定义可得, $f(x)$ 和 $g(x)$ 互为反函数。
            $(f^{-1}(ax + b) = x) \rightarrow (f^{-1}(x) = x / a - b / a)$ 。
            故 $c = 1 / a, d = - b / a$ 。
        \end{practices}

        %39.
        \begin{practices}
            令 $x_1 \neq x_2$ ,则 $ax_1 + b \neq ax_2 + b$ 。
            故 $f(x)$ 为单射。
            任何 $y \in R$ 都有 $x \in R$ 使得 $f(x) = y$ 成立,其中 $x = y / a - b / a$ 。
            故 $f(x)$ 为满射。
            则 $f(x)$ 为双射, $f^{-1}(x) = x / a - b / a$ 。
        \end{practices}

        %40.
        \begin{practices}
            \begin{enumerate}[A.]
                \item
                {
                    令 $s \in S, t \in T$ ,则 $y_s \in f(S \cup T) \wedge y_t \in f(S \cup T) \wedge y_s \in f(S) \wedge y_t \in f(T)$ ,故等式成立。
                }
                \item
                {
                    若 $x \in S \cap T$ ,则 $f(x) \in f(S \cap T) \wedge f(x) \in f(S) \wedge f(x) \in f(T)$ ,故等式成立。
                }
            \end{enumerate}
        \end{practices}

        %41.
        \begin{practices}
            \begin{enumerate}[A.]
                \item
                {
                    $f(x) = |x|, S = \{x | x < 0\}, T = \{x | x >= 0\}$
                }
                \item
                {
                    令 $s \in S, t \in T$ ,若相等,则 $f(S) \cap f(T) \subseteq f(S \cap T)$ 。
                    若不相等,则 $f(s) \neq f(t)$ ,其像不会出现在值中,也有 $f(S) \cap f(T) \subseteq f(S \cap T)$ 。
                    故左右两集合相等。
                }
            \end{enumerate}
        \end{practices}

        %42.
        \begin{practices}
            \begin{enumerate}[A.]
                \item $\{-1, 1\}$
                \item $\{x | -1 < x < 0 \vee 0 < x < 1\}$
                \item $\{x | x < -2 \vee x > 2\}$
            \end{enumerate}
        \end{practices}

        %43.
        \begin{practices}
            \begin{enumerate}[A.]
                \item $\{x | 0 \leq x < 1\}$
                \item $\{x | -1 \leq x < 2\}$
                \item $\varnothing$
            \end{enumerate}
        \end{practices}

        %44.
        \begin{practices}
            因为函数不会出现一对多,所以类似练习40可证明。
        \end{practices}

        %45.
        \begin{practices}
            同上。
        \end{practices}

        %46.
        \begin{practices}
            令 $x = n + m$ ,其中 $n$ 为整数部分, $0 \leq m < 1$ 。
            当 $m < 0.5$ 时, $\lfloor n + m + 0.5 \rfloor = n$ ;
            当 $m >= 0.5$ 时, $\lfloor n + m + 0.5 \rfloor = n + 1$ 。
            证毕。
        \end{practices}

        %47.
        \begin{practices}
            同上。
        \end{practices}

        %48.
        \begin{practices}
            当 $x$ 为整数时, $\lfloor x \rfloor = \lceil x \rceil = x$ 。
            当 $x$ 不为整数时,令 $x = n + m$ ,其中 $n$ 为整数部分, $0 \leq m < 1$ 。 $\lceil n + m \rceil = n + 1, \lfloor n + m \rfloor = n$ 。
            证毕。
        \end{practices}

        %49.
        \begin{practices}
            令 $x = n + m$ ,其中 $n$ 为整数部分, $0 \leq m < 1$ 。
            $\lfloor n + m \rfloor = n, \lceil n + m \rceil = n + 1$ 。
            由不等式性质可得出 $x - 1 < \lfloor x \rfloor \leq x \leq \lceil x \rceil x + 1$ 。
            证毕。
        \end{practices}

        %50.
        \begin{practices}
            令 $x = n + l$ ,其中 $n$ 为整数部分, $0 \leq l < 1$ 。
            $\lceil n + l + m \rceil = n + m + 1, \lceil n + l \rceil + m = n + 1 + m$ 。
            证毕。
        \end{practices}

        %51.
        \begin{practices}
            令 $x = m + l$ ,其中 $m$ 为整数部分, $0 \leq l < 1$ 。

            \begin{enumerate}[A.]
                \item
                {
                    如果 $m + l < n$ ,那么 $\lfloor m + l \rfloor = m < n$ 。
                    如果 $\lfloor m + l \rfloor = m < n, x < n$ 。
                }
                \item
                {
                    如果 $n < m + l$ ,那么 $n < m + 1 = \lceil m + l \rceil$ 。
                    如果 $n < \lceil m + l \rceil$ ,那么 $ n \leq m + 1, n < m + l$ 。
                }
            \end{enumerate}
        \end{practices}

        %52.
        \begin{practices}
            \begin{enumerate}[A.]
                \item
                {
                    如果 $m + l \leq n$ ,$\lceil m + l \rceil < m + 1$ ,又因为 $n$ 为整数,所以 $\lceil m + l \rceil \leq n$ 。
                    如果 $\lceil m + l \rceil \leq n$ ,那么 $m + 1 \leq n$ ,所以 $x \leq n$ 。
                }
                \item
                {
                    如果 $n \leq m + l$ ,那么 $n \leq m < \lfloor x \rfloor$ 。
                    如果 $n \leq \lfloor x \rfloor$ ,那么 $n \leq m \leq m + l$ 。
                }
            \end{enumerate}
        \end{practices}

        %53.
        \begin{practices}
            如果 $n$ 为偶数,则有 $n = 2k$ , $\lfloor 2k / 2 \rfloor = k = n / 2$ 。
            如果 $n$ 为奇数,则有 $n = 2k + 1$ , $\lfloor k + 0.5 \rfloor = k = (n - 1) / 2$ 。
        \end{practices}

        %54.
        \begin{practices}
            令 $x = n + m$ ,其中 $n$ 为整数部分, $0 \leq m < 1$ 。
            则 $-x = -n - m = -n - 1 + (1 - m)$ ,故可证。
        \end{practices}

        %55.
        \begin{practices}
            同上。
        \end{practices}

        %56.
        \begin{practices}
            $\lfloor b \rfloor - \lceil a \rceil + 1$
        \end{practices}

        %57.
        \begin{practices}
            $\lceil b \rceil - \lfloor a \rfloor - 1$
        \end{practices}

        %58.
        \begin{practices}
            \begin{enumerate}[A.]
                \item 1
                \item 2
                \item 63
                \item 375
            \end{enumerate}
        \end{practices}

        %59.
        \begin{practices}
            \begin{enumerate}[A.]
                \item 1
                \item 3
                \item 126
                \item 3600
            \end{enumerate}
        \end{practices}

        %60.
        \begin{practices}
            \begin{enumerate}[A.]
                \item 3018
                \item 7075
                \item 23584
            \end{enumerate}
        \end{practices}

        %61.
        \begin{practices}
            \begin{enumerate}[A.]
                \item 100
                \item 256
                \item 1030
                \item 30200
            \end{enumerate}
        \end{practices}

        %62.
        \begin{practices}

        \end{practices}

        %63.
        \begin{practices}

        \end{practices}

        %64.
        \begin{practices}

        \end{practices}

        %65.
        \begin{practices}

        \end{practices}

        %66.
        \begin{practices}

        \end{practices}

        %67.
        \begin{practices}

        \end{practices}

        %68.
        \begin{practices}

        \end{practices}

        %69.
        \begin{practices}
            \begin{align*}
                f^{-1}(x^3 + 1) &= x \\
                f^{-1}(x) = \sqrt[3]{x - 1}
            \end{align*}
        \end{practices}

        %70.
        \begin{practices}
            \begin{align*}
                ((f \circ g) \circ (g^{-1} \circ f^{-1}))(a)
                &= f(g(g^{-1}(f^{-1}(a)))) \\
                &= f(f^{-1}(a)) \\
                &= a
            \end{align*}

            \begin{align*}
                ((g^{-1} \circ f^{-1}) \circ (f \circ g))(b)
                &= f^{-1}(g^{-1}(g(f(b)))) \\
                &= f^{-1}(f(b)) \\
                &= b
            \end{align*}
        \end{practices}

        %71.
        \begin{practices}
            \begin{enumerate}[A.]
                \item $x \in (A \cap B) \leftrightarrow x \in A \wedge x \in B$
                \item $x \in (A \cup B) \leftrightarrow x \in A \vee x \in B$
                \item $x \in \bar{A} \leftrightarrow x \notin A$
                \item $x \in (A \oplus B) \leftrightarrow (x \in A \wedge x \notin B) \vee (x \notin A \wedge x \in B)$
            \end{enumerate}
        \end{practices}

        %72.
        \begin{practices}
            如果函数是单射,但不是满射,那么基数不会相等。
            如果函数是满射,但不是单射,那么基数不会相等。
            证毕。
        \end{practices}

        %73.
        \begin{practices}
            \begin{enumerate}[A.]
                \item T
                \item F
                \item T
                \item F
                \item F
            \end{enumerate}
        \end{practices}

        %74.
        \begin{practices}
            \begin{enumerate}[A.]
                \item T
                \item F
                \item T
                \item F
                \item T
            \end{enumerate}
        \end{practices}

        %75.
        \begin{practices}
            \begin{enumerate}[A.]
                \item
                {
                    令 $x = n^2 + m + \epsilon$ , $n^2$ 为小于 $x$ 的最大平方数。
                    $\sqrt{\lfloor x \rfloor}$ 和 $\sqrt{\lfloor n^2 + m \rfloor}$ 都处于 $[n, n+1)$ 之间,故可证。
                }
                \item
                {
                    同上。
                }
            \end{enumerate}
        \end{practices}

        %76.
        \begin{practices}
            令 $x = m + n$ 。
            当 $n < 1 / 3$ 时,左右都等于 $3m$ 。
            当 $1 / 3 \leq n < 2 / 3$ 时,左右都等于 $3m + 1$ 。
            当 $2 / 3 \leq n < 1$ 时,左右都等于 $3m + 2$ 。
        \end{practices}

        %77.
        \begin{practices}
            \begin{enumerate}[A.]
                \item $n \neq 0$
                \item T
                \item $n \neq 0$
                \item T
                \item $m > n$
            \end{enumerate}
        \end{practices}

        %78.
        \begin{practices}
            \begin{enumerate}[A.]
                \item
                {
                    良定义。
                }
                \item
                {
                    见例77。
                }
            \end{enumerate}
        \end{practices}

        %79.
        \begin{practices}
            \begin{enumerate}[A.]
                \item
                {
                    根据题意, $S$ 具有 $m$ 个元素,将第一个元素指派到 $1$ ,第二个指派到 $2$ , $...$ 第 $m$ 个元素指派到 $m$ 。
                }
                \item
                {
                    同上。
                }
            \end{enumerate}
        \end{practices}

        %80.
        \begin{practices}
            如果 $S$ 为有限集,则真子集 $A$ 与 $S$ 基数不同,不可能存在一一对应关系。
        \end{practices}
    }
}
