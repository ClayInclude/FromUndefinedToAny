%%
%% Author: Clay
%% 2019/5/15
%%

\section{命题逻辑的应用}
{
    \subsection{引言}
    {
        逻辑在数学、计算机科学和其他许多学科有着重要的应用。
        数学、自然科学以及自然语言中的语句通常不太准确,甚至有歧义。
        为了使其精确表达,可以将它们翻译成逻辑语言。
        逻辑可用于软件和硬件的\emreg{规范(specification)}描述。
        命题逻辑及其规则可用于涉及计算机电路、构造计算机程序、验证程序的正确性以及构造专家系统。
        逻辑可用于分析和求解许多熟悉的谜题。
        基于逻辑规则的软件系统也已经开发出来,它能够自动构造某种类型的证明。
    }

    \subsection{语句翻译}
    {
        自然语言通常有二义性。
        把语句翻译成复合命题可以消除歧义。
        这样翻译时也许需要根据语句的含义做一些合理的假设。
        一旦完成了从语句到逻辑表达式的翻译,就可以分析这些逻辑表达式以确定它们的真值,对他们进行操作。
    }

    \subsection{系统规范说明}
    {
        在描述硬件系统和软件系统时,将自然语言语句翻译成逻辑表达式是很重要的一部分。
        系统和软件工程师根据自然语言描述的需求,生成精确而无二义性的规范说明,这些规范说明可作为系统开发的基础。

        系统规范说明应该是\emreg{一致的},也就是说,系统规范说明不应该包含可能导致矛盾的相互冲突的需求。
    }

    \subsection{布尔搜索}
    {
        由于搜索采用命题逻辑的计数,所以称为\emreg{布尔搜索}。

        在布尔搜索中,联结词\emcode{AND}用于匹配同时包含两个搜索项的记录,联结词\emcode{OR}用于匹配两个搜索项之一或两项均匹配的记录,而联结词\emcode{NOT}用于排除某个特定的搜索项。
    }

    \subsection{逻辑谜题}
    {
        可以用逻辑推理解决的谜题称为\emreg{逻辑谜题}。
    }

    \subsection{逻辑电路}
    {
        逻辑命题可应用于计算机硬件的设计。

        \emreg{逻辑电路(数字电路)}接受输入信号 $p_1, p_2, \cdots, p_n$ ,每个信号1位,产生输出信号 $s_1, s_2, \cdots, s_n$ ,每个1位。

        复杂的数字电路可以由从三种简单的基本电路(\emreg{门电路(gate)})构造而来。
        \emreg{非门(NOT gate)}接受一个输入位 $p$ ,产生 $\neg p$ 作为输出。
        \emreg{或门(OR gate)}接受两个输入信号 $p$ 和 $q$ ,每个一位,产生信号 $p \vee q$ 作为输出。
        \emreg{与门(AND gate)}接受两个输入信号 $p$ 和 $q$ ,每个一位,产生信号 $p \wedge q$ 作为输出。
    }
}
