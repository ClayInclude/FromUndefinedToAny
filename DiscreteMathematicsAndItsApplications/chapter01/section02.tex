%%
%% Author: Clay
%% 2019/5/15
%%

\section{命题逻辑的应用}
{
    \subsection{引言}
    {
        逻辑在数学、计算机科学和其他许多学科有着重要的应用。
        数学、自然科学以及自然语言中的语句通常不太准确,甚至有歧义。
        为了使其精确表达,可以将它们翻译成逻辑语言。
        逻辑可用于软件和硬件的\emreg{规范(specification)}描述。
        命题逻辑及其规则可用于涉及计算机电路、构造计算机程序、验证程序的正确性以及构造专家系统。
        逻辑可用于分析和求解许多熟悉的谜题。
        基于逻辑规则的软件系统也已经开发出来,它能够自动构造某种类型的证明。
    }

    \subsection{语句翻译}
    {
        自然语言通常有二义性。
        把语句翻译成复合命题可以消除歧义。
        这样翻译时也许需要根据语句的含义做一些合理的假设。
        一旦完成了从语句到逻辑表达式的翻译,就可以分析这些逻辑表达式以确定它们的真值,对他们进行操作。
    }

    \subsection{系统规范说明}
    {
        在描述硬件系统和软件系统时,将自然语言语句翻译成逻辑表达式是很重要的一部分。
        系统和软件工程师根据自然语言描述的需求,生成精确而无二义性的规范说明,这些规范说明可作为系统开发的基础。

        系统规范说明应该是\emreg{一致的},也就是说,系统规范说明不应该包含可能导致矛盾的相互冲突的需求。
    }

    \subsection{布尔搜索}
    {
        由于搜索采用命题逻辑的计数,所以称为\emreg{布尔搜索}。

        在布尔搜索中,联结词\emcode{AND}用于匹配同时包含两个搜索项的记录,联结词\emcode{OR}用于匹配两个搜索项之一或两项均匹配的记录,而联结词\emcode{NOT}用于排除某个特定的搜索项。
    }

    \subsection{逻辑谜题}
    {
        可以用逻辑推理解决的谜题称为\emreg{逻辑谜题}。
    }

    \subsection{逻辑电路}
    {
        逻辑命题可应用于计算机硬件的设计。

        \emreg{逻辑电路(数字电路)}接受输入信号 $p_1, p_2, \cdots, p_n$ ,每个信号1位,产生输出信号 $s_1, s_2, \cdots, s_n$ ,每个1位。

        复杂的数字电路可以由从三种简单的基本电路(\emreg{门电路(gate)})构造而来。
        \emreg{非门(NOT gate)}接受一个输入位 $p$ ,产生 $\neg p$ 作为输出。
        \emreg{或门(OR gate)}接受两个输入信号 $p$ 和 $q$ ,每个一位,产生信号 $p \vee q$ 作为输出。
        \emreg{与门(AND gate)}接受两个输入信号 $p$ 和 $q$ ,每个一位,产生信号 $p \wedge q$ 作为输出。
    }

    \subsection{练习}
    {
        %1.
        \begin{practices}
            $\neg a \rightarrow e$
        \end{practices}

        %2.
        \begin{practices}
            $m \rightarrow (e \vee p)$
        \end{practices}

        %3.
        \begin{practices}
            $g \rightarrow (\neg m \wedge \neg b \wedge r)$
        \end{practices}

        %4.
        \begin{practices}
            $w \rightarrow (d \vee s)$
        \end{practices}

        %5.
        \begin{practices}
            $e \rightarrow (a \wedge (b \vee p) \wedge r)$
        \end{practices}

        %6.
        \begin{practices}
            $u \rightarrow (b_{32} \wedge g_{1} \wedge r_{1} \wedge h_{16}) \vee (b_{64} \wedge g_{2} \wedge r_{2} \wedge h_{32})$
        \end{practices}

        %7.
        \begin{practices}
            \begin{enumerate}[A.]
                \item $q \rightarrow p$
                \item $q \wedge \neg p$
                \item $q \rightarrow p$
                \item $\neg q \rightarrow \neg p$
            \end{enumerate}
        \end{practices}

        %8.
        \begin{practices}
            \begin{enumerate}[A.]
                \item $r \wedge \neg p$
                \item $(p \wedge r) \rightarrow q$
                \item $\neg r \rightarrow \neg q$
                \item $\neg p \wedge r \rightarrow q$
            \end{enumerate}
        \end{practices}

        %9.
        \begin{practices}
            令 $p$ 为:系统处于多用户态, $q$ 为:系统运行正常, $r$ 为:核心程序起作用。

            则系统规范说明为: $p \leftrightarrow q$ , $q \rightarrow r$ , $\neg q \vee \neg p$ , $\neg p \rightarrow \neg q$ , $p$ 。

            \begin{table}[H]
                \[
                    \begin{array}{c|c|c|c|c|c|c|c}
                        \hline
                        p & q & r & p \leftrightarrow q & q \rightarrow r & \neg r \vee \neg p & \neg p \rightarrow \neg p & p \\
                        \hline
                        0 & 0 & 0 & 1 & 1 & 1 & 1 & 0 \\
                        0 & 0 & 1 & 1 & 1 & 1 & 1 & 0 \\
                        0 & 1 & 0 & 0 & 0 & 1 & 1 & 0 \\
                        0 & 1 & 1 & 0 & 1 & 1 & 1 & 0 \\
                        1 & 0 & 0 & 0 & 1 & 1 & 1 & 1 \\
                        1 & 0 & 1 & 0 & 1 & 0 & 1 & 1 \\
                        1 & 1 & 0 & 1 & 0 & 1 & 1 & 1 \\
                        1 & 1 & 1 & 1 & 1 & 0 & 1 & 1 \\
                        \hline
                    \end{array}
                \]
            \end{table}

            不一致。
        \end{practices}

        %10.
        \begin{practices}
            令 $p$ 为:系统对软件进行了升级, $q$ 为:用户能访问文件系统, $r$ 为:用户能保存新文件。

            则系统规范说明为: $\neg q \rightarrow p$ , $q \rightarrow r$ , $\neg r \rightarrow \neg p$ 。

            \begin{table}[H]
                \[
                    \begin{array}{c|c|c|c|c|c}
                        \hline
                        p & q & r & \neg q \rightarrow p & q \rightarrow r & \neg r \rightarrow \neg p \\
                        \hline
                        0 & 0 & 0 & 0 & 1 & 1 \\
                        0 & 0 & 1 & 0 & 1 & 1 \\
                        0 & 1 & 0 & 1 & 0 & 1 \\
                        0 & 1 & 1 & 1 & 1 & 1 \\
                        1 & 0 & 0 & 1 & 1 & 0 \\
                        1 & 0 & 1 & 1 & 1 & 1 \\
                        1 & 1 & 0 & 1 & 0 & 0 \\
                        1 & 1 & 1 & 1 & 1 & 1 \\
                        \hline
                    \end{array}
                \]
            \end{table}

            一致。
        \end{practices}

        %11.
        \begin{practices}
            令 $p$ 为:路由器能向边缘系统发送分组, $q$ 为:路由器支持新的地址空间, $r$ 为:路由器要安装新的版本文件。

            则系统规范说明为: $p \rightarrow q$ , $q \rightarrow r$ , $r \rightarrow p$ , $\neg q$ 。

            \begin{table}[H]
                \[
                    \begin{array}{c|c|c|c|c|c|c}
                        \hline
                        p & q & r & p \rightarrow q & q \rightarrow r & r \rightarrow p & \neg q \\
                        \hline
                        0 & 0 & 0 & 1 & 1 & 1 & 1 \\
                        0 & 0 & 1 & 1 & 1 & 0 & 1 \\
                        0 & 1 & 0 & 1 & 0 & 1 & 0 \\
                        0 & 1 & 1 & 1 & 1 & 0 & 0 \\
                        1 & 0 & 0 & 0 & 1 & 1 & 1 \\
                        1 & 0 & 1 & 0 & 1 & 1 & 1 \\
                        1 & 1 & 0 & 1 & 0 & 1 & 0 \\
                        1 & 1 & 1 & 1 & 1 & 1 & 0 \\
                        \hline
                    \end{array}
                \]
            \end{table}

            一致。
        \end{practices}

        %12.
        \begin{practices}
            令 $p$ 为:系统未加锁, $q$ 为:新消息被排队, $r$ 为:系统运行正常。

            则系统规范说明为: $p \rightarrow q$ , $p \leftrightarrow r$ , $\neg q \rightarrow \neg q$ , $\neg p \rightarrow \neg q$ , $q$ 。

            \begin{table}[H]
                \[
                    \begin{array}{c|c|c|c|c|c|c|c}
                        \hline
                        p & q & r & p \rightarrow q & p \leftrightarrow r & \neg q \rightarrow \neg q & \neg p \rightarrow \neg q & q \\
                        \hline
                        0 & 0 & 0 & 1 & 1 & 1 & 1 & 0 \\
                        0 & 0 & 1 & 1 & 0 & 1 & 1 & 0 \\
                        0 & 1 & 0 & 1 & 1 & 1 & 0 & 1 \\
                        0 & 1 & 1 & 1 & 0 & 1 & 0 & 1 \\
                        1 & 0 & 0 & 0 & 0 & 1 & 1 & 0 \\
                        1 & 0 & 1 & 0 & 1 & 1 & 1 & 0 \\
                        1 & 1 & 0 & 1 & 0 & 1 & 1 & 1 \\
                        1 & 1 & 1 & 1 & 1 & 1 & 1 & 1 \\
                        \hline
                    \end{array}
                \]
            \end{table}

            一致。
        \end{practices}

        %13.
        \begin{practices}
            \begin{enumerate}[A.]
                \item New AND Jersey AND beach
                \item (the AND isle AND of AND Jersey) NOT (New AND Jersey) AND beach
            \end{enumerate}
        \end{practices}

        %14.
        \begin{practices}
            \begin{enumerate}[A.]
                \item West AND Virginia AND hiking
                \item (NOT West AND Virginia) AND hiking
            \end{enumerate}
        \end{practices}

        %15.
        \begin{practices}
            如果我问你向右边的岔路口是否通向遗址,你会说是吗?
        \end{practices}

        %16.
        \begin{practices}
            \begin{enumerate}[A.]
                \item 无论是说真话的还是说假话的,都会回答``不''。
                \item 如果我问你你是否说真话,你会回答是吗?
            \end{enumerate}
        \end{practices}

        %17.
        \begin{practices}
            第一位和第二位教授想要咖啡。
        \end{practices}

        %18.
        \begin{practices}
            Jasmine和Kanti;Jasmine;谁都不邀请。
        \end{practices}

        %19.
        \begin{practices}
            A是骑士。B是无赖。

            \begin{table}[H]
                \[
                    \begin{array}{c|c|c}
                        \hline
                        A & B & (\neg A \vee \neg B) \otimes A \\
                        \hline
                        0 & 0 & 0 \\
                        0 & 1 & 0 \\
                        1 & 0 & 1 \\
                        1 & 1 & 0 \\
                        \hline
                   \end{array}
               \]
            \end{table}
        \end{practices}

        %20.
        \begin{practices}
            A是无赖,B是骑士。

            \begin{table}[H]
                \[
                    \begin{array}{c|c|c|c}
                        \hline
                        A & B & (A \wedge B) \otimes A & (\neg A) \otimes B \\
                        \hline
                        0 & 0 & 1 & 0 \\
                        0 & 1 & 1 & 1 \\
                        1 & 0 & 0 & 1 \\
                        1 & 1 & 1 & 0 \\
                        \hline
                   \end{array}
               \]
            \end{table}
        \end{practices}

        %21.
        \begin{practices}
            A和B都是骑士。

            \begin{table}[H]
                \[
                    \begin{array}{c|c|c}
                        \hline
                        A & B & (\neg A \vee B) \otimes A \\
                        \hline
                        0 & 0 & 0 \\
                        0 & 1 & 0 \\
                        1 & 0 & 0 \\
                        1 & 1 & 1 \\
                        \hline
                   \end{array}
               \]
            \end{table}
        \end{practices}

        %22.
        \begin{practices}
            都有可能。

            \begin{table}[H]
                \[
                    \begin{array}{c|c|c|c}
                        \hline
                        A & B & A \otimes A & B \otimes B \\
                        \hline
                        0 & 0 & 1 & 1 \\
                        0 & 1 & 1 & 1 \\
                        1 & 0 & 1 & 1 \\
                        1 & 1 & 1 & 1 \\
                        \hline
                   \end{array}
               \]
            \end{table}
        \end{practices}

        %23.
        \begin{practices}
            A是无赖,B是骑士。

            \begin{table}[H]
                \[
                    \begin{array}{c|c|c}
                        \hline
                        A & B & (\neg A \wedge \neg B) \otimes A \\
                        \hline
                        0 & 0 & 0 \\
                        0 & 1 & 1 \\
                        1 & 0 & 0 \\
                        1 & 1 & 0 \\
                        \hline
                   \end{array}
               \]
            \end{table}
        \end{practices}

        %24.
        \begin{practices}
            A是骑士,B是间谍,C是无赖。

            \begin{table}[H]
                \[
                    \begin{array}{c|c|c|c|c|c}
                        \hline
                        A & B & C & (\neg C) \otimes A & A \otimes B & \neg C \\
                        \hline
                        0 & 0 & 0 & 0 & 1 & 1 \\
                        0 & 0 & 1 & 1 & 1 & 0 \\
                        0 & 1 & 0 & 0 & 0 & 1 \\
                        0 & 1 & 1 & 1 & 0 & 0 \\
                        1 & 0 & 0 & 1 & 0 & 1 \\
                        1 & 0 & 1 & 0 & 0 & 0 \\
                        1 & 1 & 0 & 1 & 1 & 1 \\
                        1 & 1 & 1 & 0 & 1 & 0 \\
                        \hline
                   \end{array}
               \]
            \end{table}
        \end{practices}

        %25.
        \begin{practices}
            A是骑士,B是间谍,C是无赖。

            \begin{table}[H]
                \[
                    \begin{array}{c|c|c|c|c|c}
                        \hline
                        A & B & C & A \otimes A & \neg B & B \otimes C \\
                        \hline
                        0 & 0 & 0 & 1 & 1 & 1 \\
                        0 & 0 & 1 & 1 & 1 & 0 \\
                        0 & 1 & 0 & 1 & 0 & 0 \\
                        0 & 1 & 1 & 1 & 0 & 1 \\
                        1 & 0 & 0 & 1 & 1 & 1 \\
                        1 & 0 & 1 & 1 & 1 & 0 \\
                        1 & 1 & 0 & 1 & 0 & 0 \\
                        1 & 1 & 1 & 1 & 0 & 1 \\
                        \hline
                   \end{array}
               \]
            \end{table}
        \end{practices}

        %26.
        \begin{practices}
            无解。

            \begin{table}[H]
                \[
                    \begin{array}{c|c|c|c|c|c}
                        \hline
                        A & B & C & \neg A & \neg B & \neg C \\
                        \hline
                        0 & 0 & 0 & 1 & 1 & 1 \\
                        0 & 0 & 1 & 1 & 1 & 0 \\
                        0 & 1 & 0 & 1 & 0 & 1 \\
                        0 & 1 & 1 & 1 & 0 & 0 \\
                        1 & 0 & 0 & 0 & 1 & 1 \\
                        1 & 0 & 1 & 0 & 1 & 0 \\
                        1 & 1 & 0 & 0 & 0 & 1 \\
                        1 & 1 & 1 & 0 & 0 & 0 \\
                        \hline
                   \end{array}
               \]
            \end{table}
        \end{practices}

        %27.
        \begin{practices}
            A是骑士,B是间谍,C是无赖。

            \begin{table}[H]
                \[
                    \begin{array}{c|c|c|c|c|c}
                        \hline
                        A & B & C & A \otimes A & A \otimes B & \neg C \\
                        \hline
                        0 & 0 & 0 & 1 & 1 & 1 \\
                        0 & 0 & 1 & 1 & 1 & 0 \\
                        0 & 1 & 0 & 1 & 0 & 1 \\
                        0 & 1 & 1 & 1 & 0 & 0 \\
                        1 & 0 & 0 & 1 & 0 & 1 \\
                        1 & 0 & 1 & 1 & 0 & 0 \\
                        1 & 1 & 0 & 1 & 1 & 1 \\
                        1 & 1 & 1 & 1 & 1 & 0 \\
                        \hline
                   \end{array}
               \]
            \end{table}
        \end{practices}

        %28.
        \begin{practices}
            无解。

            \begin{table}[H]
                \[
                    \begin{array}{c|c|c|c|c|c}
                        \hline
                        A & B & C & A \otimes A & A \otimes B & B \otimes C \\
                        \hline
                        0 & 0 & 0 & 1 & 1 & 1 \\
                        0 & 0 & 1 & 1 & 1 & 0 \\
                        0 & 1 & 0 & 1 & 0 & 0 \\
                        0 & 1 & 1 & 1 & 0 & 1 \\
                        1 & 0 & 0 & 1 & 0 & 1 \\
                        1 & 0 & 1 & 1 & 0 & 0 \\
                        1 & 1 & 0 & 1 & 1 & 0 \\
                        1 & 1 & 1 & 1 & 1 & 1 \\
                        \hline
                   \end{array}
               \]
            \end{table}
        \end{practices}

        %29.
        \begin{practices}
            任何一种情况都有可能。

            \begin{table}[H]
                \[
                    \begin{array}{c|c|c|c|c|c}
                        \hline
                        A & B & C & A \otimes A & B \otimes B & C \otimes C \\
                        \hline
                        0 & 0 & 0 & 1 & 1 & 1 \\
                        0 & 0 & 1 & 1 & 1 & 1 \\
                        0 & 1 & 0 & 1 & 1 & 1 \\
                        0 & 1 & 1 & 1 & 1 & 1 \\
                        1 & 0 & 0 & 1 & 1 & 1 \\
                        1 & 0 & 1 & 1 & 1 & 1 \\
                        1 & 1 & 0 & 1 & 1 & 1 \\
                        1 & 1 & 1 & 1 & 1 & 1 \\
                        \hline
                   \end{array}
               \]
            \end{table}
        \end{practices}

        %30.
        \begin{practices}
            A是间谍,B是无赖,C是骑士。
            A是骑士,B是间谍,C是无赖。

            \begin{table}[H]
                \[
                    \begin{array}{c|c|c|c|c|c}
                        \hline
                        A & B & C & A \otimes A & B \otimes B & \neg A \otimes C \\
                        \hline
                        0 & 0 & 0 & 1 & 1 & 0 \\
                        0 & 0 & 1 & 1 & 1 & 1 \\
                        0 & 1 & 0 & 1 & 1 & 0 \\
                        0 & 1 & 1 & 1 & 1 & 1 \\
                        1 & 0 & 0 & 1 & 1 & 1 \\
                        1 & 0 & 1 & 1 & 1 & 0 \\
                        1 & 1 & 0 & 1 & 1 & 1 \\
                        1 & 1 & 1 & 1 & 1 & 0 \\
                        \hline
                   \end{array}
               \]
            \end{table}
        \end{practices}

        %31.
        \begin{practices}
            无解。

            \begin{table}[H]
                \[
                    \begin{array}{c|c|c|c|c|c}
                        \hline
                        A & B & C & A \otimes A & B \otimes B & C \otimes C \\
                        \hline
                        0 & 0 & 0 & 1 & 1 & 0 \\
                        0 & 0 & 1 & 1 & 1 & 1 \\
                        0 & 1 & 0 & 1 & 1 & 0 \\
                        0 & 1 & 1 & 1 & 1 & 1 \\
                        1 & 0 & 0 & 1 & 1 & 1 \\
                        1 & 0 & 1 & 1 & 1 & 0 \\
                        1 & 1 & 0 & 1 & 1 & 1 \\
                        1 & 1 & 1 & 1 & 1 & 0 \\
                        \hline
                   \end{array}
               \]
            \end{table}
        \end{practices}

        %32.
        \begin{practices}
            \begin{enumerate}[A.]
                \item Jones是凶手。
                \item 三人都是凶手。
            \end{enumerate}
        \end{practices}

        %33.
        \begin{practices}
            不能确定。

            \begin{table}[H]
                \[
                    \begin{array}{c|c|c|c|c}
                        \hline
                        F & M & J & \neg F \rightarrow J & J \rightarrow M \\
                        \hline
                        0 & 0 & 0 & 0 & 1 \\
                        0 & 0 & 1 & 1 & 0 \\
                        0 & 1 & 0 & 0 & 1 \\
                        0 & 1 & 1 & 1 & 1 \\
                        1 & 0 & 0 & 1 & 1 \\
                        1 & 0 & 1 & 1 & 0 \\
                        1 & 1 & 0 & 1 & 1 \\
                        1 & 1 & 1 & 1 & 1 \\
                        \hline
                   \end{array}
               \]
            \end{table}
        \end{practices}

        %34.
        \begin{practices}
            Kevin和Vijay在聊天。

            \begin{table}[H]
                \[
                    \begin{array}{c|c|c|c|c|c|c|c|c|c}
                        \hline
                        K & H & R & V & A & K \vee H & R \oplus V & A \rightarrow R & V \otimes K & H \rightarrow (A \wedge K) \\
                        \hline
                        0 & 0 & 0 & 0 & 0 & 0 & 0 & 1 & 1 & 1 \\
                        0 & 0 & 0 & 0 & 1 & 0 & 0 & 0 & 1 & 1 \\
                        0 & 0 & 0 & 1 & 0 & 0 & 1 & 1 & 0 & 1 \\
                        0 & 0 & 0 & 1 & 1 & 0 & 1 & 0 & 0 & 1 \\
                        0 & 0 & 1 & 0 & 0 & 0 & 1 & 1 & 1 & 1 \\
                        0 & 0 & 1 & 0 & 1 & 0 & 1 & 1 & 1 & 1 \\
                        0 & 0 & 1 & 1 & 0 & 0 & 0 & 1 & 0 & 1 \\
                        0 & 0 & 1 & 1 & 1 & 0 & 0 & 1 & 0 & 1 \\
                        0 & 1 & 0 & 0 & 0 & 1 & 0 & 1 & 1 & 0 \\
                        0 & 1 & 0 & 0 & 1 & 1 & 0 & 0 & 1 & 0 \\
                        0 & 1 & 0 & 1 & 0 & 1 & 1 & 1 & 0 & 0 \\
                        0 & 1 & 0 & 1 & 1 & 1 & 1 & 0 & 0 & 0 \\
                        0 & 1 & 1 & 0 & 0 & 1 & 1 & 1 & 1 & 0 \\
                        0 & 1 & 1 & 0 & 1 & 1 & 1 & 1 & 1 & 0 \\
                        0 & 1 & 1 & 1 & 0 & 1 & 0 & 1 & 0 & 0 \\
                        0 & 1 & 1 & 1 & 1 & 1 & 0 & 1 & 0 & 0 \\
                        1 & 0 & 0 & 0 & 0 & 1 & 0 & 1 & 0 & 1 \\
                        1 & 0 & 0 & 0 & 1 & 1 & 0 & 0 & 0 & 1 \\
                        1 & 0 & 0 & 1 & 0 & 1 & 1 & 1 & 1 & 1 \\
                        1 & 0 & 0 & 1 & 1 & 1 & 1 & 0 & 1 & 1 \\
                        1 & 0 & 1 & 0 & 0 & 1 & 1 & 1 & 0 & 1 \\
                        1 & 0 & 1 & 0 & 1 & 1 & 1 & 1 & 0 & 1 \\
                        1 & 0 & 1 & 1 & 0 & 1 & 0 & 1 & 1 & 1 \\
                        1 & 0 & 1 & 1 & 1 & 1 & 0 & 1 & 1 & 1 \\
                        1 & 1 & 0 & 0 & 0 & 1 & 0 & 1 & 0 & 0 \\
                        1 & 1 & 0 & 0 & 1 & 1 & 0 & 0 & 0 & 1 \\
                        1 & 1 & 0 & 1 & 0 & 1 & 1 & 1 & 1 & 0 \\
                        1 & 1 & 0 & 1 & 1 & 1 & 1 & 0 & 1 & 1 \\
                        1 & 1 & 1 & 0 & 0 & 1 & 1 & 1 & 0 & 0 \\
                        1 & 1 & 1 & 0 & 1 & 1 & 1 & 1 & 0 & 1 \\
                        1 & 1 & 1 & 1 & 0 & 1 & 0 & 1 & 1 & 0 \\
                        1 & 1 & 1 & 1 & 1 & 1 & 0 & 1 & 1 & 1 \\
                        \hline
                   \end{array}
               \]
            \end{table}
        \end{practices}

        %35.
        \begin{practices}
            男管家为A,厨师为B,园丁为C,杂役为D。
            则男管家和厨师一定在说谎。

            \begin{table}[H]
                \[
                    \begin{array}{c|c|c|c|c|c|c|c}
                        \hline
                        A & B & C & D & A \rightarrow B & \neg (B \wedge C) & C \wedge D & D \rightarrow \neg B \\
                        \hline
                        0 & 0 & 0 & 0 & 1 & 1 & 0 & 1 \\
                        0 & 0 & 0 & 1 & 1 & 1 & 1 & 1 \\
                        0 & 0 & 1 & 0 & 1 & 1 & 1 & 1 \\
                        0 & 0 & 1 & 1 & 1 & 1 & 1 & 1 \\
                        0 & 1 & 0 & 0 & 1 & 1 & 0 & 1 \\
                        0 & 1 & 0 & 1 & 1 & 1 & 1 & 0 \\
                        0 & 1 & 1 & 0 & 1 & 0 & 1 & 1 \\
                        0 & 1 & 1 & 1 & 1 & 0 & 1 & 0 \\
                        1 & 0 & 0 & 0 & 0 & 1 & 0 & 1 \\
                        1 & 0 & 0 & 1 & 0 & 1 & 1 & 1 \\
                        1 & 0 & 1 & 0 & 0 & 1 & 1 & 1 \\
                        1 & 0 & 1 & 1 & 0 & 1 & 1 & 1 \\
                        1 & 1 & 0 & 0 & 1 & 1 & 0 & 1 \\
                        1 & 1 & 0 & 1 & 1 & 1 & 1 & 0 \\
                        1 & 1 & 1 & 0 & 1 & 0 & 1 & 1 \\
                        1 & 1 & 1 & 1 & 1 & 0 & 1 & 0 \\
                        \hline
                   \end{array}
               \]
            \end{table}
        \end{practices}

        %36.
        \begin{practices}
            \begin{table}[H]
                \[
                    \begin{array}{c|c|c|c|c|c|c|c}
                        \hline
                        A & C & D & J & C & \neg J & D & \neg D \\
                        \hline
                        0 & 0 & 0 & 0 & 0 & 1 & 0 & 1 \\
                        0 & 0 & 0 & 1 & 0 & 0 & 0 & 1 \\
                        0 & 0 & 1 & 0 & 0 & 1 & 1 & 0 \\
                        0 & 0 & 1 & 1 & 0 & 0 & 1 & 0 \\
                        0 & 1 & 0 & 0 & 1 & 1 & 0 & 1 \\
                        0 & 1 & 0 & 1 & 1 & 0 & 0 & 1 \\
                        0 & 1 & 1 & 0 & 1 & 1 & 1 & 0 \\
                        0 & 1 & 1 & 1 & 1 & 0 & 1 & 0 \\
                        1 & 0 & 0 & 0 & 0 & 1 & 0 & 1 \\
                        1 & 0 & 0 & 1 & 0 & 0 & 0 & 1 \\
                        1 & 0 & 1 & 0 & 0 & 1 & 1 & 0 \\
                        1 & 0 & 1 & 1 & 0 & 0 & 1 & 0 \\
                        1 & 1 & 0 & 0 & 1 & 1 & 0 & 1 \\
                        1 & 1 & 0 & 1 & 1 & 0 & 0 & 1 \\
                        1 & 1 & 1 & 0 & 1 & 1 & 1 & 0 \\
                        1 & 1 & 1 & 1 & 1 & 0 & 1 & 0 \\
                        \hline
                   \end{array}
               \]
            \end{table}

            \begin{enumerate}[A.]
                \item John。
                \item Carlos
            \end{enumerate}
        \end{practices}

        %37.
        \begin{practices}
            第一扇门的提示为假,第二扇门为真。
            第二扇门后面。
        \end{practices}

        %38.
        \begin{practices}
            日本人养斑马,挪威人喜欢喝水。

            \begin{table}[H]
                \begin{tabular}{c|c|c|c|c|c}
                    \hline
                    & 1 & 2 & 3 & 4 & 5 \\
                    \hline
                    国籍 & 挪威 & 意大利 & 英国 & 西班牙 & 日本 \\
                    颜色 & 黄色 & 蓝色 & 红色 & 白色 & 绿色 \\
                    工种 & 外交官 & 医师 & 摄影师 & 小提琴家 & 油漆工 \\
                    宠物 & 狐狸 & 马 & 蜗牛 & 狗 & 斑马 \\
                    饮料 & 水 & 茶 & 牛奶 & 橘子汁 & 咖啡 \\
                    \hline
                \end{tabular}
            \end{table}
        \end{practices}

        %39.
        \begin{practices}
            只有一名是诚实的。
        \end{practices}

        %40.
        \begin{practices}
            \begin{enumerate}[A.]
                \item $\neg p \vee \neg q$
                \item $\neg (p \vee (\neg p \wedge q))$
            \end{enumerate}
        \end{practices}

        %41.
        \begin{practices}
            \begin{enumerate}[A.]
                \item $\neg (p \wedge (q \vee \neg r))$
                \item $(\neg p \wedge \neg q) \vee (p \wedge r)$
            \end{enumerate}
        \end{practices}

        %42.
        \begin{practices}
            \begin{tikzpicture}[circuit logic US]
                \matrix[column sep=7mm]
                {
                    \node (p) {p};  &                                                           &                           \\
                                    & \node [and gate, inputs = {normal, inverted}] (a1) {};    &                           \\
                    \node (r) {r};  &                                                           & \node [or gate] (o) {};   \\
                                    & \node [and gate, inputs = {normal, inverted}] (a2) {};    &                           \\
                    \node (q) {q};  &                                                           &                           \\
                };
                \draw (p.east) -- ++(right:3mm) |- (a1.input 1);
                \draw (r.east) -- ++(right:3mm) |- (a1.input 2);
                \draw (r.east) -- ++(right:3mm) |- (a2.input 1);
                \draw (q.east) -- ++(right:3mm) |- (a2.input 2);
                \draw (a1.output) -- ++(right:3mm) |- (o.input 1);
                \draw (a2.output) -- ++(right:3mm) |- (o.input 2);
                \draw (o.output) -- ++(right:3mm);
            \end{tikzpicture}
        \end{practices}

        %43.
        \begin{practices}
            \begin{tikzpicture}[circuit logic US]
                \matrix[column sep=9mm]
                {
                    \node (q) {q};  & & \node [and gate, inputs = {inverted, normal}] (a1) {};  & & \\
                                    & \node [or gate, inputs = {inverted, inverted}] (o1) {};   & & & \\
                    \node (p) {p};  & & & \node [or gate] (o3) {};                                & \\
                                    & \node [or gate] (o2) {};                                  & & & \\
                    \node (r) {r};  & & \node [and gate, inputs = {normal, inverted}] (a2) {};  & & \\
                };
                \draw (q.east) -- ++(right:1mm) |- (a1.input 1);
                \draw (q.east) -- ++(right:2mm) |- (o2.input 2);
                \draw (p.east) -- ++(right:4mm) |- (o1.input 1);
                \draw (p.east) -- ++(right:5mm) |- (a2.input 2);
                \draw (r.east) -- ++(right:7mm) |- (o1.input 2);
                \draw (r.east) -- ++(right:8mm) |- (o2.input 1);
                \draw (o1.output) -- ++(right:3mm) |- (a1.input 2);
                \draw (o2.output) -- ++(right:3mm) |- (a2.input 1);
                \draw (a1.output) -- ++(right:3mm) |- (o3.input 1);
                \draw (a2.output) -- ++(right:3mm) |- (o3.input 2);
                \draw (o3.output) -- ++(right:3mm);
            \end{tikzpicture}
        \end{practices}
    }
}
