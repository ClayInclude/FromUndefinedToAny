%%
%% Author: Clay
%% 2019/12/2
%%

\section{证明导论}
{
    \subsection{引言}
    {
        一个证明是建立数学语句真实性的有效论证。
        证明可以使用定理的假设,假定为真的公理以及之前已经被证明的定理。
        使用这些以及推理规则,证明的最后一步是建立呗证命题的真实性。

        在讨论中,将从定理的形式化证明转向\emreg{非形式化证明}。
        形式化证明提供了所有步骤,并给出论证中每一步所用到的规则。
        然而,许多有用的定理形式化证明会非常长切难以理解。
        实际上,为方便人们阅读,定理证明几乎都是\emreg{非形式化证明(informal proof)},其中每个步骤会用到多于一条的推理规则,有些步骤会被省略,不会显示地列出所用到的假设公理和推理规则。
    }

    \subsection{一些专用术语}
    {
        一个\emreg{定理(theorem)}是一个能够被证明是真的语句。
        定理一词通常是用来专指那些被认为至少有些重要的语句。
        不太重要的定理有时称为\emreg{命题}(定理也可以称为\emreg{事实(fact)}或\emreg{结论(result)})。
        一个定理可以是带一个或多个前提及一个结论的条件语句的全称量化式。
        我们用一个\emreg{证明(proof)}来展示一个定理是真的。
        证明就是建立定理真实性的一个有效论证。
        证明中用到的语句可以包括\emreg{公理(axiom)}(或\emreg{假设(postulate)}),这些是我们假定为真的语句、定理的前提和以前已经被证明的公理。公理可以采用无需定义的原始术语来陈述,而在定理和证明中所用的所有其他术语都必须有定义的。
        推理规则和其术语的定义一起用于从其他的断言推出结论,并绑定在证明中的每个步骤。
        实际上,一个证明的最后一步通常恰好是定理的结论。

        一个不太重要但有助于证明其他结论的定理称为\emreg{引理(lemma)}。
        当用一系列引理来进行复杂的证明时通常比较容易理解,其中每一个引理都被独立证明。
        \emreg{推论(corollary)}是从一个已经被证明的定理可以直接建立起来的一个定理。
        \emreg{猜想(conjecture)}是一个被提出认为是真的命题,通常是基于部分证据、启发式论证或者专家的直觉。
        当猜想的一个证明被发现时,猜想就变成了定理。
    }

    \subsection{理解定理是如何陈述的}
    {
        许多定理断言一个性质相对于论域中的所有元素都成立。
        虽然这些定理的准确陈述需要包含全称量词,但是数学里的标准约定是省略全称量词。
        当证明这种类型的定理时,证明的第一步通常涉及选择论域里的一个一般性元素。
        随后的步骤是证明这个元素具有所考虑的性质。
        最后,全称引入蕴含着定理对论域里所有元素都成立。
    }

    \subsection{证明定理的方法}
    {
        要构造证明,我们需要所有可用的手段,包括不同证明方法的强大的工具库。
        这些方法提供了证明的总体思路和策略。
        理解这些方法是学习如何阅读并构造数学证明的关键所在。
        一旦选定了一种证明方法,使用公理、术语的定义、先前证明的结论和推理规则来完成证明。
    }

    \subsection{直接证明法}
    {
        条件语句 $p \rightarrow q$ 的\emreg{直接证明法}的构造:
        第一步假设 $p$ 为真;
        第二步用推理规则构造,而第三步表明 $q$ 必须也为真。
        直接证明法是通过证明如果 $p$ 为真,那么 $q$ 也肯定为真,这样 $p$ 为真且 $q$ 为假的情况永远不会发生从而证明条件语句 $p \rightarrow q$ 为真。

        直接证明法有时候需要特殊的洞察力并且可能是相当棘手的。

        \begin{defines}
            整数 $n$ 是偶数,如果存在一个整数 $k$ 使得 $n = 2k$ ;
            整数 $n$ 是奇数,如果存在一个整数 $k$ 使得 $n = 2k + 1$ 。
            两个整数当同为偶数或同为奇数时具有相同的奇偶性;
            当一个是偶数而另一个是奇数时具有相反的奇偶性。
        \end{defines}
    }

    \subsection{反证法}
    {
        不采用直接证明法,即不从前提开始以结论结束来证明这类定理的方法叫做\emreg{间接证明法}。

        一类非常有用的间接证明法称为\emreg{反证法(proof by contraposition)}。
        反证法利用了这样一个事实:条件语句 $p \rightarrow q$ 等价于它的逆否命题 $\neg q \rightarrow \neg p$ 。
        用反证法证明 $p \rightarrow q$ 时,我们将 $\neg q$ 作为前提,再用公理、定义和前面证明过的定理,以及推理规则,证明 $\neg q$ 必须成立。

        \paraph{空证明和平凡证明}
        {
            当我们知道 $p$ 为假时,能够很快证明条件语句 $p \rightarrow q$ 为真。
            如果能证明 $p$ 为假,那么我们就有一个 $p \rightarrow q$ 的证明方法,称为\emreg{空证明(vacuous proof)}。

            用 $q$ 为真的事实来证明 $p \rightarrow q$ 的方法叫做\emreg{平凡证明(trivial proof)}。
        }

        \paraph{证明的小策略}
        {
            当想要证明形如 $\forall x (P(x) \rightarrow Q(x))$ 的命题时,首先评估直接证明法是否可行。
            可以通过展开前提中的定义开始。
            然后使用这些前提,加上公理和可用的定理进行推理。
            如果直接证明法得不到什么结果,尝试反证法。
        }

        \begin{defines}
            实数 $r$ 是有理数,如果存在整数 $p$ 和 $q (q \neq 0)$ 使得 $r = p / q$ 。
            不是有理数的实数称为无理数。
        \end{defines}
    }

    \subsection{归谬证明法}
    {
        假设我们要证明命题 $p$ 是真的。
        再假定我们能找到一个矛盾式 $q$ 使得 $\neg p \rightarrow q$ 为真。
        因为 $q$ 是假的,而 $\neg p \rightarrow q$ 是真的,所以能够得出结论 $\neg p$ 为假,这意味着 $p$ 为真。

        因为无论 $r$ 是什么,命题 $r \wedge \neg r$ 就是矛盾式。
        所以如果能够证明对某个命题 $r$ , $\neg p \rightarrow (r \wedge \neg r)$ 为真,就能证明 $p$ 是真的。
        这种类型的证明称为\emreg{归谬证明法(proof by contradiction)}。
        由于归谬证明法不是直接证明结论,所以它是另一种间接证明法。

        归谬证明法可以用于证明条件语句。
        在证明中,我们首先假设结论的否定为真。
        然后采用定理的前提和结论的否定来得到一个矛盾式。
        这样证明是有效的原因是基于 $p \rightarrow q$ 与 $(p \wedge \neg q) \rightarrow F$ 是逻辑等价的。

        可以把一个条件语句的反证改写成归谬证明。
        在 $p \rightarrow q$ 的反证里,假定 $\neg q$ 为真。
        然后证明 $\neg p$ 也必为真。
        为了把 $p \rightarrow q$ 的反证改写成归谬证明,假定 $p$ 和 $\neg q$ 都为真。
        然后利用 $\neg q \rightarrow \neg p$ 的证明步骤来证明 $\neg p$ 也必然为真。
        这样导出矛盾式 $p \wedge \neg p$ ,从而完成归谬证明。

        也可以用归谬法证明 $p \rightarrow q$ 是真的。
        通过假设 $p$ 和 $\neg q$ 都为真来证明 $q$ 也一定为真。
        这蕴含着 $q$ 和 $\neg q$ 都为真,导致矛盾。
        这一点告诉我们,可以将一个直接证明转变为一个归谬证明。

        \paraph{等价证明法}
        {
            为了证明一个双条件命题的定理,即形如 $p \leftrightarrow q$ 的语句,我们证明 $p \rightarrow q$ 和 $q \rightarrow p$ 都是真的。
            这个方法的有效性是建立在重言式的基础上:
            $$(p \leftrightarrow q) \leftrightarrow (p \rightarrow q) \wedge (q \rightarrow p)$$

            有时候一个定理会阐述多个命题都是等价的。
            这样的定理阐述命题 $p_1, p_2, \cdots , p_n$ 都是等价的,这样可以写成
            $$p_1 \leftrightarrow p_2 \cdots \leftrightarrow p_n$$
            这就是说,所有 $n$ 个命题都有相同的真值。
            证明这些命题互相等价的一种方式是使用永真式
            $$(p_1 \leftrightarrow p_2 \cdots \leftrightarrow p_n) \leftrightarrow (p_1 \rightarrow p_2) \wedge (p_2 \cdots \rightarrow \cdots p_n) \wedge (p_n \rightarrow p_1)$$
        }

        \paraph{反例证明法}
        {
            要证明形如 $\forall x P(x)$ 的语句为假,只要能寻找一个反例,即存在一个例子 $x$ 是 $P(x)$ 为假。
        }
    }

    \subsection{证明中的错误}
    {
        数学证明的每一步都应当是正确的,并且结论必须从之前的步骤中逻辑地导出。

        许多不正确的论证都基于一种称为\emreg{窃取论题}的谬误。
        当证明的一个或多个步骤基于待证明命题的真实性时,就会发生这样的谬误。
        换句话说,当命题使用自身或等价于自身的命题来进行证明时会产生这种谬误。
        所以这种谬误也称为\emreg{循环推理}。

        \emreg{在证明中犯错是学习过程的一部分。}
    }

    \subsection{良好的开端}
    {
        有许多定理通过直接利用前提和定理中的名词的定义很容易构造其证明。
        不过,要是不借助于灵活地利用反证法或归谬证明,或其他的证明技术,证明一个定理通常还是很困难的。
        构造证明是一种只能通过体验来学习的艺术,这体验包括写证明、让他人评论你的证明,以及阅读和分析其他证明。
    }

    \subsection{练习}
    {
        %1.
        \begin{practices}
            若 $m = 2k + 1, n = 2l + 1$ ,则 $m$ 和 $n$ 之和 $m + n = 2k + 1 + 2l + 1 = 2(k + l + 1)$ 是偶数。
        \end{practices}

        %2.
        \begin{practices}
            若 $m = 2k, n = 2l$ ,则 $m$ 和 $n$ 之和 $m + n = 2k + 2l = 2(k + l)$ 是偶数。
        \end{practices}

        %3.
        \begin{practices}
            若 $m = 2k$ ,则 $m$ 的平方 $m^2 = (2k)^2 = 4k^2 = 2(2k^2)$ 是偶数。
        \end{practices}

        %4.
        \begin{practices}
            若 $m = 2k$ ,则 $m$ 的相反数 $n = -m = -2k = 2(-k)$ 是偶数。
        \end{practices}

        %5.
        \begin{practices}
            若 $m + n$ 和 $n + p$ 是偶数,则和也为偶数: $m + n + n + p = 2k$ 。
            又有: $m + p = 2k - 2n = 2(k -n)$ 也为偶数。
        \end{practices}

        %6.
        \begin{practices}
            若 $m = 2k + 1, n = 2l + 1$ ,则 $m$ 和 $n$ 之积 $mn = (2k + 1)(2l + 1) = 4kl + 2k + 2l + 1 = 2(2kl + k +l) + 1$ 为奇数。
        \end{practices}

        %7.
        \begin{practices}
            若 $m = 2k+ 1$ ,则有 $(k + 1)^2 - k^2 = k^2 + 2k + 1 - k^2 = 2k + 1 = m$ 。
        \end{practices}

        %8.
        \begin{practices}
            若 $m = n^2$ ,
            当 $n = 0$ , $m + 2 = 2$ 不是完全平方数。
            当 $n \geq 1$ ,则比 $m$ 大的最小完全平方数为 $(n + 1)^2 = n^2 + 2n + 1 > n^2 + 2 = m + 2$,故 $m + 2$ 不是完全平方数。
        \end{practices}

        %9.
        \begin{practices}
            令 $s$ 是有理数, $i$ 是无理数,和为 $u = s + i$ 为有理数。
            则 $u + (-s) = i$ 为有理数,故得出矛盾,所以和为无理数。
        \end{practices}

        %10.
        \begin{practices}
            令 $m = a / b$ , $n = c / d$ ,则 $mn = ac / bd$ 。
            因为 $b$ 和 $d$ 都不为 $0$ ,故 $bd$ 不为 $0$ , $mn$ 为有理数。
        \end{practices}

        %11.
        \begin{practices}
            $\sqrt{2} \times \sqrt{2} = 2$ ,故两个无理数之积不一定是无理数。
        \end{practices}

        %12.
        \begin{practices}
            若 $m$ 为有理数, $n$ 为无理数,其积 $l$ 为有理数,则 $l = mn$ , $n = l / m = l \times \frac{1}{m}$ 。
            显然 $\frac{1}{m}$ 也为有理数,根据练习10可得, $n$ 也为有理数,故矛盾。
            所有有理数和无理数的乘积也为无理数。
        \end{practices}

        %13.
        \begin{practices}
            若 $\frac{1}{x}$ 为有理数,则 $x$ 也为有理数,故矛盾。
            所以如果 $x$ 为无理数, $\frac{1}{x}$ 也为无理数。
        \end{practices}

        %14.
        \begin{practices}
            若 $x = \frac{p}{q}$ ,且 $p \neq 0$ ,则 $\frac{1}{x} = \frac{q}{p}$ ,也为有理数。
        \end{practices}

        %15.
        \begin{practices}
            假设 $x < 1$ 且 $y < 1$ ,则 $x + y < 2$ 。
            故可得,如果 $x + y \geq 2$ ,则 $x \geq 1$ 或 $y \geq 1$ 。
        \end{practices}

        %16.
        \begin{practices}
            假设 $m$ 和 $n$ 都为奇数,根据练习6可得 $mn$ 也为奇数。
            故可得,如果 $mn$ 为偶数,则 $m$ 和 $n$ 之中至少有一个为偶数。
        \end{practices}

        %17.
        \begin{practices}
            若 $n = 2k + 1$ ,则 $(2k + 1)^3 + 5 = (2k)^3 + 6(2k)^2 + 1 + 5 = 2(4k^3 + 3(2k)^2 + 3)$ 为偶数。
            故 $n$ 为偶数。
        \end{practices}

        %18.
        \begin{practices}
            若 $n = 2k + 1$ ,则 $3(2k + 1) +2 = 6k + 5 = 2(3k + 2) + 1$ 为奇数。
            故 $n$ 为偶数。 
        \end{practices}

        %19.
        \begin{practices}
            空证明。
        \end{practices}

        %20.
        \begin{practices}
            平凡证明。
        \end{practices}

        %21.
        \begin{practices}
            平凡证明。
        \end{practices}

        %22.
        \begin{practices}
            假设不能拿到一双颜色一样的袜子,则最多能够选择两只,与前提矛盾。
            故一定能得到一双蓝袜子或一双黑袜子。
        \end{practices}

        %23.
        \begin{practices}
            假设没有10天出现在每星期的同一天里,则最多只能选63天,与前提矛盾。
            故在任意64天中至少有10天在每星期的同一天里。
        \end{practices}

        %24.
        \begin{practices}
            假设至多有两天出现在同一个月份里,则最多只能选24天,与前提矛盾。
            故任意25天中至少有3天在同一个月份。
        \end{practices}

        %25.
        \begin{practices}
            若有 $r = a / b$ ,则 $(a / b)^3 + (a / b) + 1 = 0$ ,
            两边同时乘以 $b^3$ :
            $a^3 + ab^2 + b^3 = 0$
            若 $a$ 和 $b$ 之中有一个为奇数,则 $a^3 + ab^2 + b^3$ 为奇数,得出矛盾。
            故 $a$ 和 $b$ 都为偶数。
            因为 $a / b$ 是既约分数,这里得出 $a$ 和 $b$ 都能被 $2$ 整除,得出矛盾。
            故没有有理数 $r$ 使得 $r^3 + r + 1 = 0$ 。
        \end{practices}

        %26.
        \begin{practices}
            若 $n = 2k$ ,则 $7n + 4 = 14k + 4 = 2(7k + 2)$ 也为偶数。
            若 $7n + 4 = 2k$ ,则 $n = 2k - 4 - 6n = 2(k - 2 - 3n)$ 也为偶数。

            证毕。
        \end{practices}

        %27.
        \begin{practices}
            若 $n = 2k + 1$ ,则 $5(2k + 1) + 6 = 10k + 11 = 2(5k + 5) + 1$ 为奇数。
            若 $5n + 6 = 2k + 1$ ,则 $n = 2k - 5 - 4n = 2(k - 2n - 3) + 1$ 也为奇数。

            证毕。
        \end{practices}

        %28.
        \begin{practices}
            若 $m^2 = n^2$ ,同时开方得 $m = n$ 或 $m = -n$ 。
            若 $m = n$ 或 $m = -n$ ,则 $m^2 = n^2$ 。

            证毕。
        \end{practices}

        %29.
        \begin{practices}
            若 $m \neq 1$ 且 $m \neq -1$ ,则 $mn$ 有一个大于 $1$ 的因子。
            但是 $mn = 1$ , $1$ 没有这个因子,故 $m$ 一定等于 $1$ 或者 $-1$ ,可得出 $n$ 也等于 $1$ 或 $-1$ 。
            若 $m = 1, n = 1$ 或 $m = -1, n = -1$ ,则 $mn = 1$ 。

            证毕。
        \end{practices}

        %30.
        \begin{practices}
            \begin{align*}
                a &< b \\
                2a &< a + b \\
                a &< \frac{a + b}{2}
            \end{align*}

            可得 $\rmnum 1 \leftrightarrow \rmnum 2$ 。

            \begin{align*}
                a &< b \\
                a + b &< 2b \\
                \frac{a + b}{2} &< b
            \end{align*}

            可得 $\rmnum 1 \leftrightarrow \rmnum 3$ 。

            证毕。
        \end{practices}

        %31.
        \begin{practices}
            \begin{align*}
                3x + 2 &= 2k \\
                x &= 2k - 2 - 2x \\
                x &= 2(k - 1 - x) \\
                \\
                x + 5 &= 2(k - 1 - x) + 5 \\
                x + 5 &= 2(k + 1 - x) + 1 \\
                \\
                x^2 &= (2(k - 1 - x))^2 \\
                x^2 &= 2(2(k - 1 - x)^2) \\
            \end{align*}

            以上等式每一步都可逆,故 $\rmnum 1 \leftrightarrow \rmnum 2 \leftrightarrow \rmnum 3$ 。

            证毕。
        \end{practices}

        %32.
        \begin{practices}
            \begin{align*}
                x &= p / q \\
                x / 2 &= p / 2q \\
                3x - 1 &= 3(p / q) - 1 \\
                3x - 1 &= (3p - q) / q
            \end{align*}

            以上等式每一步都可逆,故 $\rmnum 1 \leftrightarrow \rmnum 2 \leftrightarrow \rmnum 3$ 。

            证毕。
        \end{practices}

        %33.
        \begin{practices}
            若 $x$ 是无理数,而 $3x + 2$ 是有理数,则

            \begin{align*}
                3x + 2 &= p / q \\
                x &= ((p / q) - 2) / 3 \\
                x &= (p - 2q) / 3q
            \end{align*}

            $x$ 为有理数,得出矛盾。故 $\rmnum 1 \rightarrow \rmnum 2$ 。

            若 $3x + 2$ 是无理数,而 $x / 2$ 是有理数,则

            \begin{align*}
                x / 2 &= p / q \\
                x &= 2p / q \\
                \\
                3x + 2 &= 3(2p / q) + 2 \\
                3x + 2 &= (6p + 2q) / q
            \end{align*}

            $3x + 2$ 是有理数,得出矛盾。故 $\rmnum 2 \rightarrow \rmnum 3$ 。

            若 $x / 2$ 是无理数,而 $x$ 是有理数,则

            \begin{align*}
                x &= p / q \\
                x / 2 &= p / 2q
            \end{align*}

            $x / 2$ 是有理数,得出矛盾。故 $\rmnum 3 \rightarrow \rmnum 1$ 。

            可得 $\rmnum 1 \leftrightarrow \rmnum 2 \leftrightarrow \rmnum 3$ 。

            证毕。
        \end{practices}

        %34.
        \begin{practices}
            不正确,因为推理过程不能验证双条件满足。
        \end{practices}

        %35.
        \begin{practices}
            不正确,因为推理过程不能验证双条件满足。
        \end{practices}

        %36.
        \begin{practices}
            \begin{align*}
                ((p_1 \leftrightarrow p_3) \wedge (p_2 \leftrightarrow p_3)) &\rightarrow (p_1 \leftrightarrow p_2) \\
                ((p_1 \leftrightarrow p_4) \wedge (p_1 \leftrightarrow p_2)) &\rightarrow (p_2 \leftrightarrow p_4) \\
                ((p_2 \leftrightarrow p_4) \wedge (p_2 \leftrightarrow p_3)) &\rightarrow (p_3 \leftrightarrow p_4)
            \end{align*}
        \end{practices}

        %37.
        \begin{practices}
            对 $(p_1 \rightarrow p_4) \wedge (p_4 \rightarrow p_2) \wedge (p_2 \rightarrow p_5) \wedge (p_5 \rightarrow p_3) \wedge (p_3 \rightarrow p_1)$ 多次使用假言三段论即可得出任意一个命题蕴含另一个命题。
        \end{practices}

        %38.
        \begin{practices}
            15
        \end{practices}

        %39.
        \begin{practices}
            反证法。

            若果每个数都小于平均值,则有:

            \begin{align*}
                a_x &< a_{avg} \\
                \sum_{x = 1}^{n} a_x &< n * a_{avg} \\
                \sum_{x = 1}^{n} a_x &< \sum_{x = 1}^{n} a_x
            \end{align*}

            故至少有一个数大于等于平均值。
        \end{practices}

        %40.
        \begin{practices}
            令 $a_x$ 为任意三个相邻位置的整数和。
            \begin{align*}
                \sum_{x = 1}^{n} a_x &= 3 * 55 \\
                &= 165 \\
                a_x &= 16.5
            \end{align*}

            故存在三个相邻位置的整数和大于等于 $17$ 。
        \end{practices}

        %41.
        \begin{practices}
            \begin{align*}
                n &= 2k \\
                \\
                n + 1 &= 2k + 1 \\
                \\
                3n + 1 &= 6k + 1 \\
                \\
                3n &= 6k
            \end{align*}

            以上等式每一步都可逆,故 $\rmnum 1 \leftrightarrow \rmnum 2 \leftrightarrow \rmnum 3 \leftrightarrow \rmnum 4$ 。
        \end{practices}

        %42.
        \begin{practices}
            \begin{align*}
                1 - n &= 2(-k) \\
                n &= 2k + 1 \\
                \\
                n^2 &= (2k + 1)^2 \\
                n^2 &= 4k^2 + 4k + 1 \\
                n^2 &= 2(2k^2 + 2k) + 1 \\
                \\
                n^3 &= 8k^3 + 12k^2 + 6k + 1 \\
                n^3 &= 2(4k^3 + 6k^2 + 3k) + 1 \\
                \\
                n^2 + 1 &= 4k^2 + 4k + 1 + 1 \\
                n^2 + 1 &= 2(2k^2 + 2k + 1)
            \end{align*}

            以上等式每一步都可逆,故 $\rmnum 1 \leftrightarrow \rmnum 2 \leftrightarrow \rmnum 3 \leftrightarrow \rmnum 4$ 。
        \end{practices}
    }
}
