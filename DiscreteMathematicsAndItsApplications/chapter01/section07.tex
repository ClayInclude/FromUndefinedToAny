%%
%% Author: Clay
%% 2019/12/2
%%

\section{证明导论}
{
    \subsection{引言}
    {
        一个证明是建立数学语句真实性的有效论证。
        证明可以使用定理的假设,假定为真的公理以及之前已经被证明的定理。
        使用这些以及推理规则,证明的最后一步是建立呗证命题的真实性。

        在讨论中,将从定理的形式化证明转向\emreg{非形式化证明}。
        形式化证明提供了所有步骤,并给出论证中每一步所用到的规则。
        然而,许多有用的定理形式化证明会非常长切难以理解。
        实际上,为方便人们阅读,定理证明几乎都是\emreg{非形式化证明(informal proof)},其中每个步骤会用到多于一条的推理规则,有些步骤会被省略,不会显示地列出所用到的假设公理和推理规则。
    }

    \subsection{一些专用术语}
    {
        一个\emreg{定理(theorem)}是一个能够被证明是真的语句。
        定理一词通常是用来专指那些被认为至少有些重要的语句。
        不太重要的定理有时称为\emreg{命题}(定理也可以称为\emreg{事实(fact)}或\emreg{结论(result)})。
        一个定理可以是带一个或多个前提及一个结论的条件语句的全称量化式。
        我们用一个\emreg{证明(proof)}来展示一个定理是真的。
        证明就是建立定理真实性的一个有效论证。
        证明中用到的语句可以包括\emreg{公理(axiom)}(或\emreg{假设(postulate)}),这些是我们假定为真的语句、定理的前提和以前已经被证明的公理。公理可以采用无需定义的原始术语来陈述,而在定理和证明中所用的所有其他术语都必须有定义的。
        推理规则和其术语的定义一起用于从其他的断言推出结论,并绑定在证明中的每个步骤。
        实际上,一个证明的最后一步通常恰好是定理的结论。

        一个不太重要但有助于证明其他结论的定理称为\emreg{引理(lemma)}。
        当用一系列引理来进行复杂的证明时通常比较容易理解,其中每一个引理都被独立证明。
        \emreg{推论(corollary)}是从一个已经被证明的定理可以直接建立起来的一个定理。
        \emreg{猜想(conjecture)}是一个被提出认为是真的命题,通常是基于部分证据、启发式论证或者专家的直觉。
        当猜想的一个证明被发现时,猜想就变成了定理。
    }

    \subsection{理解定理是如何陈述的}
    {
        许多定理断言一个性质相对于论域中的所有元素都成立。
        虽然这些定理的准确陈述需要包含全称量词,但是数学里的标准约定是省略全称量词。
        当证明这种类型的定理时,证明的第一步通常涉及选择论域里的一个一般性元素。
        随后的步骤是证明这个元素具有所考虑的性质。
        最后,全称引入蕴含着定理对论域里所有元素都成立。
    }

    \subsection{证明定理的方法}
    {
        要构造证明,我们需要所有可用的手段,包括不同证明方法的强大的工具库。
        这些方法提供了证明的总体思路和策略。
        理解这些方法是学习如何阅读并构造数学证明的关键所在。
        一旦选定了一种证明方法,使用公理、术语的定义、先前证明的结论和推理规则来完成证明。
    }

    \subsection{直接证明法}
    {
        条件语句 $p \rightarrow q$ 的\emreg{直接证明法}的构造:
        第一步假设 $p$ 为真;
        第二步用推理规则构造,而第三步表明 $q$ 必须也为真。
        直接证明法是通过证明如果 $p$ 为真,那么 $q$ 也肯定为真,这样 $p$ 为真且 $q$ 为假的情况永远不会发生从而证明条件语句 $p \rightarrow q$ 为真。

        直接证明法有时候需要特殊的洞察力并且可能是相当棘手的。

        \begin{defines}
            整数 $n$ 是偶数,如果存在一个整数 $k$ 使得 $n = 2k$ ;
            整数 $n$ 是奇数,如果存在一个整数 $k$ 使得 $n = 2k + 1$ 。
            两个整数当同为偶数或同为奇数时具有相同的奇偶性;
            当一个是偶数而另一个是奇数时具有相反的奇偶性。
        \end{defines}
    }

    \subsection{反证法}
    {
        不采用直接证明法,即不从前提开始以结论结束来证明这类定理的方法叫做\emreg{间接证明法}。

        一类非常有用的间接证明法称为\emreg{反证法(proof by contraposition)}。
        反证法利用了这样一个事实:条件语句 $p \rightarrow q$ 等价于它的逆否命题 $\neg q \rightarrow \neg p$ 。
        用反证法证明 $p \rightarrow q$ 时,我们将 $\neg q$ 作为前提,再用公理、定义和前面证明过的定理,以及推理规则,证明 $\neg q$ 必须成立。

        \paraph{空证明和平凡证明}
        {
            当我们知道 $p$ 为假时,能够很快证明条件语句 $p \rightarrow q$ 为真。
            如果能证明 $p$ 为假,那么我们就有一个 $p \rightarrow q$ 的证明方法,称为\emreg{空证明(vacuous proof)}。

            用 $q$ 为真的事实来证明 $p \rightarrow q$ 的方法叫做\emreg{平凡证明(trivial proof)}。
        }

        \paraph{证明的小策略}
        {
            当想要证明形如 $\forall x (P(x) \rightarrow Q(x))$ 的命题时,首先评估直接证明法是否可行。
            可以通过展开前提中的定义开始。
            然后使用这些前提,加上公理和可用的定理进行推理。
            如果直接证明法得不到什么结果,尝试反证法。
        }

        \begin{defines}
            实数 $r$ 是有理数,如果存在整数 $p$ 和 $q (q \neq 0)$ 使得 $r = p / q$ 。
            不是有理数的实数称为无理数。
        \end{defines}
    }
}