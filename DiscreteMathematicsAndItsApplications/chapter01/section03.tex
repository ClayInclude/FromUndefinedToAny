%%
%% Author: Clay
%% 2019/6/10
%%

\section{命题等价式}
{
    \subsection{引言}
    {
        数学证明中使用的一个重要步骤就是用真值相同的一条语句替换另一条语句。
        因此,从给定复合命题生成具有相同真值命题的方法广泛使用于数学证明的构造。

        \begin{defines}
            一个真值永远是真的复合命题,称为永真式(tautology),也成为重言式。
            一个真值永远为假的复合命题,称为矛盾式(contradicition)。
            既不是永真式又不是矛盾式的复合命题称为可能式(contingency)。
        \end{defines}

        \begin{table}[htb]
            \centering

            \[
                \begin{array}{c|c|c|c}
                    \hline
                    p & \neg p & p \vee \neg p & p \wedge \neg p \\
                    \hline
                    T & F & T & F \\
                    F & T & T & F \\
                    \hline
                \end{array}
            \]

            \caption{永真式和矛盾式的例子}
        \end{table}
    }

    \subsection{逻辑等价式}
    {
        \begin{wraptable}{r}{.3333\textwidth{}}
            \centering

            \[
                \begin{array}{c}
                    \hline
                    \neg (p \wedge q) \equiv \neg p \vee \neg q \\
                    \neg (p \vee q) \equiv \neg p \wedge \neg q \\
                    \hline
                \end{array}
            \]

            \caption{德·摩根律}
        \end{wraptable}

        在所有可能的情况下都有相同真值的两个复合命题称为\emreg{逻辑等价}的。

        \begin{defines}
            如果 $p \leftrightarrow q$ 是,永真式,则复合命题 $p$ 和 $q$ 是逻辑等价的。
            用记号 $p \equiv q$ 表示 $p$ 和 $q$ 是逻辑等价的。
        \end{defines}

        判断两个复合命题是否等价的方法之一是使用真值表。
        特别地,复合命题 $p$ 和 $q$ 是等价的,当且仅当对应它们真值的两列完全一致。

        德·摩根律可以扩展为:
        \begin{align}
            \neg (p_1 \vee p_2 \vee \cdots \vee p_n) \equiv (\neg p_1 \wedge \neg p_2 \wedge \cdots \wedge \neg p_n)
        \end{align}

        和
        \begin{align}
            \neg (p_1 \wedge p_2 \wedge \cdots \wedge p_n) \equiv (\neg p_1 \vee \neg p_2 \vee \cdots \vee \neg p_n)
        \end{align}

        有时用符号 $\bigvee\limits_{j = 1}^n p_j$ 来表示 $p_1 \vee p_2 \vee \cdots \vee p_n$ ,用 $\bigwedge\limits_{j = 1}^n p_j$ 来表示 $p_1 \wedge p_2 \wedge \cdots \wedge p_n$ 。
        采用这种记法扩展的德·摩根律就可以简洁地写成 $\neg (\bigvee\limits_{j = 1}^n p_j) \equiv \bigwedge\limits_{j = 1}^n \neg p_j$ 和 $\neg (\bigwedge\limits_{j = 1}^n p_j) \equiv \bigvee\limits_{j = 1}^n \neg p_j$ 。

        \begin{minipage}[c]{\textwidth{}}
            \begin{minipage}[c]{.6\textwidth{}}
                \begin{table}[H]
                    \centering

                    \begin{tabular}{l|c}
                        \hline
                        \multicolumn{1}{c|}{等价式} & 名称 \\
                        \hline
                        $p \wedge T \equiv p$ & \multirow{2}*{恒等律} \\
                        $p \vee F \equiv p$ & \\
                        \hline
                        $p \vee T \equiv T$ & \multirow{2}*{支配律} \\
                        $p \wedge F \equiv F$ & \\
                        \hline
                        $p \vee p \equiv p$ & \multirow{2}*{幂等律} \\
                        $p \wedge p \equiv p$ & \\
                        \hline
                        $\neg (\neg q) \equiv q$ & 双重否定律 \\
                        \hline
                        $p \vee q \equiv q \vee p$ & \multirow{2}*{交换律} \\
                        $p \wedge q \equiv q \wedge p$ & \\
                        \hline
                        $(p \vee q) \vee r \equiv p \vee (q \vee r)$ & \multirow{2}*{结合律} \\
                        $(p \wedge q) \wedge r \equiv p \wedge (q \wedge r)$ & \\
                        \hline
                        $p \vee (q \wedge r) \equiv (p \vee q) \wedge (p \vee r)$ & \multirow{2}*{分配律} \\
                        $p \wedge (q \vee r) \equiv (p \wedge q) \vee (p \wedge r)$ & \\
                        \hline
                        $\neg (p \wedge q) \equiv \neg p \vee \neg q$ & \multirow{2}*{德·摩根律} \\
                        $\neg (p \vee q) \equiv \neg p \wedge \neg q$ & \\
                        \hline
                        $p \vee (p \wedge q) \equiv p$ & \multirow{2}*{吸收律} \\
                        $p \wedge (p \vee q) \equiv p$ & \\
                        \hline
                        $p \vee \neg p \equiv T$ & \multirow{2}*{否定律} \\
                        $p \wedge \neg p \equiv F$ & \\
                        \hline
                    \end{tabular}

                    \caption{逻辑等价式}
                \end{table}
            \end{minipage}%
            \begin{minipage}[c]{.4\textwidth{}}
                \begin{minipage}[c]{\textwidth{}}
                    \begin{table}[H]
                        \centering

                        \[
                            \begin{array}{l}
                                \hline
                                p \rightarrow q \equiv \neg p \vee q \\
                                p \rightarrow q \equiv \neg q \rightarrow \neg p \\
                                p \vee q \equiv \neg p \rightarrow q \\
                                p \wedge q \equiv \neg (p \rightarrow \neg q) \\
                                \neg (p \rightarrow q) \equiv p \wedge \neg q \\
                                (p \rightarrow q) \wedge (p \rightarrow r) \equiv p \rightarrow (q \wedge r) \\
                                (p \rightarrow r) \wedge (q \rightarrow r) \equiv (p \vee q) \rightarrow r \\
                                (p \rightarrow q) \vee (p \rightarrow r) \equiv p \rightarrow (q \vee r) \\
                                (p \rightarrow r) \vee (q \rightarrow r) \equiv (p \wedge q) \rightarrow r \\
                                \hline
                            \end{array}
                        \]

                        \caption{条件命题的逻辑等价式}
                    \end{table}
                \end{minipage}%

                \vfill

                \begin{minipage}[c]{\textwidth{}}
                    \begin{table}[H]
                        \centering

                        \[
                            \begin{array}{l}
                                \hline
                                p \leftrightarrow q \equiv (p \rightarrow q) \wedge (q \rightarrow p) \\
                                p \leftrightarrow q \equiv \neg p \leftrightarrow \neg q \\
                                p \leftrightarrow q \equiv (p \wedge q) \vee (\neg p \wedge \neg q) \\
                                \neg (p \leftrightarrow q) \equiv p \leftrightarrow \neg q \\
                                \hline
                            \end{array}
                        \]

                        \caption{双条件命题的逻辑等价式}
                    \end{table}
                \end{minipage}
            \end{minipage}
        \end{minipage}
    }

    \subsection{德·摩根律的运用}
    {
        $\neg (p \wedge q) \equiv \neg p \vee \neg q$ 说明一个析取式的否定是由各分命题否定的合取式组成的。
        $\neg (p \vee q) \equiv \neg p \wedge \neg q$说明一个合取式的否定是由各分命题否定的析取式组成的。
    }

    \subsection{构造新的逻辑等价式}
    {
        复合命题中的一个命题可以用与它逻辑等价的复合命题替换,而不改变原复合命题的真值。
        如果 $p$ 和 $q$ 是逻辑等价的, $q$ 和 $r$ 是逻辑等价的,那么 $p$ 和 $r$ 也是逻辑等价的。
    }

    \subsection{命题的可满足性}
    {
        一个复合命题称为是\emreg{可满足的},如果存在一个对其变元的真值赋值使其为真。
        当不存在这样的赋值时,即当复合命题对所有变元的真值赋值都是假的,则复合命题是\emreg{不可满足的}。
        注意一个复合命题是不可满足的当且仅当它的否定对所有变元的真值赋值都是真的,也就是说,当且仅当它的否定是永真式。

        当找到一个特定的使得复合命题为真的真值赋值时,就证明了它是可满足的。
        这样的一个赋值成为这个特定的可满足性问题的一个\emreg{解}。
        要证明一个复合命题是不可满足的,需要证明每一组变元的真值赋值都使其为假。
    }

    \subsection{可满足性的应用}
    {
        在不同领域中的许多问题都可以用命题可满足性来建立模型。
    }

    \subsection{练习}
    {
        %1.
        \begin{practices}
            \begin{enumerate}[A.]
                \item
                {
                    \begin{table}[H]
                        \[
                            \begin{array}{c|c|c}
                                \hline
                                p & p \wedge T & p \\
                                \hline
                                0 & 0 & 0 \\
                                1 & 1 & 1 \\
                            \end{array}
                        \]
                    \end{table}
                }
                \item
                {
                    \begin{table}[H]
                        \[
                            \begin{array}{c|c|c}
                                \hline
                                p & p \vee F & p \\
                                \hline
                                0 & 0 & 0 \\
                                1 & 1 & 1 \\
                            \end{array}
                        \]
                    \end{table}
                }
                \item
                {
                    \begin{table}[H]
                        \[
                            \begin{array}{c|c|c}
                                \hline
                                p & p \wedge F & F \\
                                \hline
                                0 & 0 & 0 \\
                                1 & 0 & 0 \\
                            \end{array}
                        \]
                    \end{table}
                }
                \item
                {
                    \begin{table}[H]
                        \[
                            \begin{array}{c|c|c}
                                \hline
                                p & p \vee T & T \\
                                \hline
                                0 & 1 & 1 \\
                                1 & 1 & 1 \\
                            \end{array}
                        \]
                    \end{table}
                }
                \item
                {
                    \begin{table}[H]
                        \[
                            \begin{array}{c|c|c}
                                \hline
                                p & p \vee p & p \\
                                \hline
                                0 & 0 & 0 \\
                                1 & 1 & 1 \\
                            \end{array}
                        \]
                    \end{table}
                }
                \item
                {
                    \begin{table}[H]
                        \[
                            \begin{array}{c|c|c}
                                \hline
                                p & p \wedge p & p \\
                                \hline
                                0 & 0 & 0 \\
                                1 & 1 & 1 \\
                            \end{array}
                        \]
                    \end{table}
                }
            \end{enumerate}
        \end{practices}

        %2.
        \begin{practices}
            \begin{table}[H]
                \[
                    \begin{array}{c|c|c}
                        \hline
                        p & \neg \neg p & p \\
                        \hline
                        0 & 0 & 0 \\
                        1 & 1 & 1 \\
                    \end{array}
                \]
            \end{table}
        \end{practices}

        %3.
        \begin{practices}
            \begin{enumerate}[A.]
                \item
                {
                    \begin{table}[H]
                        \[
                            \begin{array}{c|c|c|c}
                                \hline
                                p & q & p \vee q & q \vee p \\
                                \hline
                                0 & 0 & 0 & 0 \\
                                0 & 1 & 1 & 1 \\
                                1 & 0 & 1 & 1 \\
                                1 & 1 & 1 & 1 \\
                            \end{array}
                        \]
                    \end{table}
                }
                \item
                {
                    \begin{table}[H]
                        \[
                            \begin{array}{c|c|c|c}
                                \hline
                                p & q & p \wedge q & q \wedge p \\
                                \hline
                                0 & 0 & 0 & 0 \\
                                0 & 1 & 0 & 0 \\
                                1 & 0 & 0 & 0 \\
                                1 & 1 & 1 & 1 \\
                            \end{array}
                        \]
                    \end{table}
                }
            \end{enumerate}
        \end{practices}

        %4.
        \begin{practices}
            \begin{enumerate}[A.]
                \item
                {
                    \begin{table}[H]
                        \[
                            \begin{array}{c|c|c|c|c}
                                \hline
                                p & q & r & (p \vee q) \vee r & q \vee (p \vee r) \\
                                \hline
                                0 & 0 & 0 & 0 & 0 \\
                                0 & 0 & 1 & 1 & 1 \\
                                0 & 1 & 0 & 1 & 1 \\
                                0 & 1 & 1 & 1 & 1 \\
                                1 & 0 & 0 & 1 & 1 \\
                                1 & 0 & 1 & 1 & 1 \\
                                1 & 1 & 0 & 1 & 1 \\
                                1 & 1 & 1 & 1 & 1 \\
                            \end{array}
                        \]
                    \end{table}
                }
                \item
                {
                    \begin{table}[H]
                        \[
                            \begin{array}{c|c|c|c|c}
                                \hline
                                p & q & r & (p \wedge q) \wedge r & q \wedge (p \wedge r) \\
                                \hline
                                0 & 0 & 0 & 0 & 0 \\
                                0 & 0 & 1 & 0 & 0 \\
                                0 & 1 & 0 & 0 & 0 \\
                                0 & 1 & 1 & 0 & 0 \\
                                1 & 0 & 0 & 0 & 0 \\
                                1 & 0 & 1 & 0 & 0 \\
                                1 & 1 & 0 & 0 & 0 \\
                                1 & 1 & 1 & 1 & 1 \\
                            \end{array}
                        \]
                    \end{table}
                }
            \end{enumerate}
        \end{practices}

        %5.
        \begin{practices}
            \begin{table}[H]
                \[
                    \begin{array}{c|c|c|c|c|c|c|c}
                        \hline
                        p & q & r & q \vee r & p \wedge (q \vee r) & p \wedge q & p \wedge r & (p \wedge q) \vee (p \wedge r) \\
                        \hline
                        0 & 0 & 0 & 0 & 0 & 0 & 0 & 0 \\
                        0 & 0 & 1 & 1 & 0 & 0 & 0 & 0 \\
                        0 & 1 & 0 & 1 & 0 & 0 & 0 & 0 \\
                        0 & 1 & 1 & 1 & 0 & 0 & 0 & 0 \\
                        1 & 0 & 0 & 0 & 0 & 0 & 0 & 0 \\
                        1 & 0 & 1 & 1 & 1 & 0 & 1 & 1 \\
                        1 & 1 & 0 & 1 & 1 & 1 & 0 & 1 \\
                        1 & 1 & 1 & 1 & 1 & 1 & 1 & 1 \\
                    \end{array}
                \]
            \end{table}
        \end{practices}

        %6.
        \begin{practices}
            \begin{table}[H]
                \[
                    \begin{array}{c|c|c|c}
                        \hline
                        p & q & \neg (p \wedge q) & \neg p \vee \neg q \\
                        \hline
                        0 & 0 & 1 & 1 \\
                        0 & 1 & 1 & 1 \\
                        1 & 0 & 1 & 1 \\
                        1 & 1 & 0 & 0 \\
                    \end{array}
                \]
            \end{table}
        \end{practices}

        %7.
        \begin{practices}
            \begin{enumerate}[A.]
                \item Jane不富裕,或者不快乐。
                \item Carlos明天既不骑自行车,也不跑步。
                \item Mei不是步行也不乘公共汽车去上课。
                \item Ibrahim不聪明或者不用功。
            \end{enumerate}
        \end{practices}

        %8.
        \begin{practices}
            \begin{enumerate}[A.]
                \item Kwame不会在工业界找工作并且也不会去研究生院读书。
                \item Yoshiko不会掌握Java或者不会掌握微积分。
                \item James不年轻或者不强壮。
                \item Rita不会搬到俄勒冈州也不会搬到华盛顿去。
            \end{enumerate}
        \end{practices}

        %9.
        \begin{practices}
            \begin{enumerate}[A.]
                \item
                {
                    \begin{table}[H]
                        \[
                            \begin{array}{c|c|c|c}
                                \hline
                                p & q & p \wedge q & (p \wedge q) \rightarrow p \\
                                \hline
                                0 & 0 & 0 & 1 \\
                                0 & 1 & 0 & 1 \\
                                1 & 0 & 0 & 1 \\
                                1 & 1 & 1 & 1 \\
                            \end{array}
                        \]
                    \end{table}
                }
                \item
                {
                    \begin{table}[H]
                        \[
                            \begin{array}{c|c|c|c}
                                \hline
                                p & q & p \vee q & p \rightarrow (p \vee q) \\
                                \hline
                                0 & 0 & 0 & 1 \\
                                0 & 1 & 1 & 1 \\
                                1 & 0 & 1 & 1 \\
                                1 & 1 & 1 & 1 \\
                            \end{array}
                        \]
                    \end{table}
                }
                \item
                {
                    \begin{table}[H]
                        \[
                            \begin{array}{c|c|c|c}
                                \hline
                                p & q & p \rightarrow q & \neg p \rightarrow (p \rightarrow q) \\
                                \hline
                                0 & 0 & 1 & 1 \\
                                0 & 1 & 1 & 1 \\
                                1 & 0 & 0 & 1 \\
                                1 & 1 & 1 & 1 \\
                            \end{array}
                        \]
                    \end{table}
                }
                \item
                {
                    \begin{table}[H]
                        \[
                            \begin{array}{c|c|c|c|c}
                                \hline
                                p & q & p \wedge q & p \rightarrow q & (p \wedge q) \rightarrow (p \rightarrow q) \\
                                \hline
                                0 & 0 & 0 & 1 & 1 \\
                                0 & 1 & 0 & 1 & 1 \\
                                1 & 0 & 0 & 0 & 1 \\
                                1 & 1 & 1 & 1 & 1 \\
                            \end{array}
                        \]
                    \end{table}
                }
                \item
                {
                    \begin{table}[H]
                        \[
                            \begin{array}{c|c|c|c}
                                \hline
                                p & q & \neg (p \rightarrow q) & \neg (p \rightarrow q) \rightarrow p \\
                                \hline
                                0 & 0 & 0 & 1 \\
                                0 & 1 & 0 & 1 \\
                                1 & 0 & 1 & 1 \\
                                1 & 1 & 0 & 1 \\
                            \end{array}
                        \]
                    \end{table}
                }
                \item
                {
                    \begin{table}[H]
                        \[
                            \begin{array}{c|c|c|c}
                                \hline
                                p & q & \neg (p \rightarrow q) & \neg (p \rightarrow q) \rightarrow \neg q \\
                                \hline
                                0 & 0 & 0 & 1 \\
                                0 & 1 & 0 & 1 \\
                                1 & 0 & 1 & 1 \\
                                1 & 1 & 0 & 1 \\
                            \end{array}
                        \]
                    \end{table}
                }
            \end{enumerate}
        \end{practices}

        %10.
        \begin{practices}
            \begin{enumerate}[A.]
                \item
                {
                    \begin{table}[H]
                        \[
                            \begin{array}{c|c|c|c|c}
                                \hline
                                p & q & p \vee q & \neg p \wedge (p \vee q) & [\neg p \wedge (p \vee q)] \rightarrow q \\
                                \hline
                                0 & 0 & 0 & 0 & 1 \\
                                0 & 1 & 0 & 0 & 1 \\
                                1 & 0 & 0 & 0 & 1 \\
                                1 & 1 & 1 & 0 & 1 \\
                            \end{array}
                        \]
                    \end{table}
                }
                \item
                {
                    \begin{table}[H]
                        \[
                            \begin{array}{c|c|c|c|c|c|c|c}
                                \hline
                                p & q & r & p \rightarrow q & q \rightarrow r & (p \rightarrow q) \wedge (q \rightarrow r) & p \rightarrow r & [(p \rightarrow q) \wedge (q \rightarrow r)] \rightarrow (p \rightarrow r) \\
                                \hline
                                0 & 0 & 0 & 1 & 1 & 1 & 1 & 1 \\
                                0 & 0 & 1 & 1 & 1 & 1 & 1 & 1 \\
                                0 & 1 & 0 & 1 & 0 & 0 & 1 & 1 \\
                                0 & 1 & 1 & 1 & 1 & 1 & 1 & 1 \\
                                1 & 0 & 0 & 0 & 1 & 0 & 0 & 1 \\
                                1 & 0 & 1 & 0 & 1 & 0 & 1 & 1 \\
                                1 & 1 & 0 & 1 & 0 & 0 & 0 & 1 \\
                                1 & 1 & 1 & 1 & 1 & 1 & 1 & 1 \\
                            \end{array}
                        \]
                    \end{table}
                }
                \item
                {
                    \begin{table}[H]
                        \[
                            \begin{array}{c|c|c|c|c}
                                \hline
                                p & q & p \rightarrow q & p \wedge (p \rightarrow q) & [p \wedge (p \rightarrow q)] \rightarrow q \\
                                \hline
                                0 & 0 & 1 & 0 & 1 \\
                                0 & 1 & 1 & 0 & 1 \\
                                1 & 0 & 0 & 0 & 1 \\
                                1 & 1 & 1 & 1 & 1 \\
                            \end{array}
                        \]
                    \end{table}
                }
                \item
                {
                    \begin{table}[H]
                        \[
                            \begin{array}{c|c|c|c|c|c|c|c}
                                \hline
                                p & q & r & p \vee q & p \rightarrow r & q \rightarrow r & (p \vee q) \wedge (p \rightarrow r) \wedge (q \rightarrow r) & [(p \vee q) \wedge (p \rightarrow r) \wedge (q \rightarrow r)] \rightarrow r \\
                                \hline
                                0 & 0 & 0 & 0 & 1 & 1 & 0 & 1 \\
                                0 & 0 & 1 & 0 & 1 & 1 & 0 & 1 \\
                                0 & 1 & 0 & 1 & 1 & 0 & 0 & 1 \\
                                0 & 1 & 1 & 1 & 1 & 1 & 1 & 1 \\
                                1 & 0 & 0 & 1 & 0 & 1 & 0 & 1 \\
                                1 & 0 & 1 & 1 & 1 & 1 & 1 & 1 \\
                                1 & 1 & 0 & 1 & 0 & 0 & 0 & 1 \\
                                1 & 1 & 1 & 1 & 1 & 1 & 1 & 1 \\
                            \end{array}
                        \]
                    \end{table}
                }
            \end{enumerate}
        \end{practices}

        %11.
        \begin{practices}
            \begin{enumerate}[A.]
                \item
                {
                    \begin{align*}
                        (p \wedge q) \rightarrow p
                        &\equiv \neg (p \wedge q) \vee q \\
                        &\equiv (\neg p \vee \neg q) \vee q \\
                        &\equiv \neg p \vee (\neg q \vee q) \\
                        &\equiv \neg p \vee T \\
                        &\equiv T
                    \end{align*}
                }
                \item
                {
                    \begin{align*}
                        p \rightarrow (p \vee q)
                        &\equiv \neg p \vee (p \vee q) \\
                        &\equiv (neg p \vee p) \vee q \\
                        &\equiv T \vee q \\
                        &\equiv T
                    \end{align*}
                }
                \item
                {
                    \begin{align*}
                        \neg p \rightarrow (p \rightarrow q)
                        &\equiv p \vee (\neg p \vee q) \\
                        &\equiv T \vee q \\
                        &\equiv T
                    \end{align*}
                }
                \item
                {
                    \begin{align*}
                        (p \wedge q) \rightarrow (p \rightarrow q)
                        &\equiv (\neg p \vee \neg q) \vee (\neg p \vee q) \\
                        &\equiv \neg p \vee T \\
                        &\equiv T
                    \end{align*}
                }
                \item
                {
                    \begin{align*}
                        \neg (p \rightarrow q) \rightarrow p
                        &\equiv (\neg p \vee q) \vee p \\
                        &\equiv T \vee q \\
                        &\equiv T
                    \end{align*}
                }
                \item
                {
                    \begin{align*}
                        \neg (p \rightarrow q) \rightarrow \neg q
                        &\equiv (\neg p \vee q) \vee \neg q \\
                        &\equiv p \vee T \\
                        &\equiv T
                    \end{align*}
                }
            \end{enumerate}
        \end{practices}

        %12.
        \begin{practices}
            \begin{enumerate}[A.]
                \item
                {
                    \begin{align*}
                        [\neg p \wedge (p \vee q)] \rightarrow q
                        &\equiv p \vee \neg (p \vee q) \vee q \\
                        &\equiv p \vee (\neg p \wedge \neg q) \vee q \\
                        &\equiv (p \vee \neg p) \wedge (p \vee \neg q) \vee q \\
                        &\equiv [T \wedge (p \vee \neg q)] \ vee q \\
                        &\equiv (T \vee q) \wedge (q \vee p \vee \neg q) \\
                        &\equiv T \wedge (T \vee p) \\
                        &\equiv T
                    \end{align*}
                }
                \item
                {
                    \begin{align*}
                        [(p \rightarrow q) \wedge (q \rightarrow r)] \rightarrow (p \rightarrow r)
                        &\equiv [(\neg p \vee q) \wedge (\neg q \vee r)] \rightarrow (\neg p \vee r) \\
                        &\equiv \neg [(\neg p \vee q) \wedge (\neg q \vee r)] \vee (\neg p \vee r) \\
                        &\equiv (p \wedge \neg q) \vee (q \wedge \neg r) \vee \neg p \vee r \\
                        &\equiv [(\neg p \vee p) \wedge (\neg p \vee \neg q)] \vee [(r \vee q) \wedge (r \vee \neg r)] \\
                        &\equiv [T \wedge (\neg p \vee \neg q)] \vee [(r \vee q) \wedge T] \\
                        &\equiv \neg p \vee \neg q \vee r \vee q \\
                        &\equiv \neg p \vee T \vee r \\
                        &\equiv T
                    \end{align*}
                }
                \item
                {
                    \begin{align*}
                        [p \wedge (p \rightarrow q)] \rightarrow q
                        &\equiv [p \wedge (\neg p \vee q)] \rightarrow q \\
                        &\equiv \neg [p \wedge (\neg p \vee q)] \vee q \\
                        &\equiv \neg p \vee (p \wedge \neg q) \vee q \\
                        &\equiv \neg p \vee \neg q \vee q \\
                        &\equiv T
                    \end{align*}
                }
                \item
                {
                    \begin{align*}
                        [(p \vee q) \wedge (p \rightarrow r) \wedge (q \rightarrow r)] \rightarrow r
                        &\equiv \neg [(p \vee q) \wedge (\neg p \vee r) \wedge (\neg q \vee r)] \vee r \\
                        &\equiv (\neg p \wedge \neg q) \vee (p \wedge \neg r) \vee (q \wedge \neg r) \vee r \\
                        &\equiv (\neg p \wedge \neg q) \vee p \vee q \vee r \\
                        &\equiv \neg q \vee p \vee q \vee r \\
                        &\equiv T
                    \end{align*}
                }
            \end{enumerate}
        \end{practices}

        %13.
        \begin{practices}
            \begin{enumerate}[A.]
                \item
                {
                    \begin{table}[H]
                        \[
                            \begin{array}{c|c|c|c}
                                \hline
                                p & q & p \wedge q & p \vee (p \wedge q) \\
                                \hline
                                0 & 0 & 0 & 0 \\
                                0 & 1 & 0 & 0 \\
                                1 & 0 & 0 & 1 \\
                                1 & 1 & 1 & 1 \\
                                \hline
                           \end{array}
                       \]
                    \end{table}
                }
                \item
                {
                    \begin{table}[H]
                        \[
                            \begin{array}{c|c|c|c}
                                \hline
                                p & q & p \vee q & p \wedge (p \vee q) \\
                                \hline
                                0 & 0 & 0 & 0 \\
                                0 & 1 & 1 & 0 \\
                                1 & 0 & 1 & 1 \\
                                1 & 1 & 1 & 1 \\
                                \hline
                           \end{array}
                       \]
                    \end{table}
                }
            \end{enumerate}
        \end{practices}

        %14.
        \begin{practices}
            \begin{align*}
                [\neg p \wedge (p \rightarrow q)] \rightarrow \neg q
                &\equiv \neg [\neg p \wedge (\neg p \vee q)] \vee \neg q \\
                &\equiv p \vee (p \wedge \neg q) \vee \neg q \\
                &\equiv p \vee \neg q
            \end{align*}

            不是永真式。
        \end{practices}

        %15.
        \begin{practices}
            \begin{align*}
                [\neg q \wedge (p \rightarrow q)] \rightarrow \neg p
                &\equiv \neg [\neg q \wedge (\neg p \vee q)] \vee \neg p \\
                &\equiv q \vee (p \wedge \neg q) \vee \neg p \\
                &\equiv q \vee p \vee \neg p \\
                &\equiv T
            \end{align*}

            是永真式。
        \end{practices}

        %16.
        \begin{practices}
            \begin{table}[H]
                \[
                    \begin{array}{c|c|c|c|c|c}
                        \hline
                        p & q & p \wedge q & \neg p \wedge \neg q & (p \wedge q) \vee (\neg p \wedge \neg q) & p \leftrightarrow q \\
                        \hline
                        0 & 0 & 0 & 1 & 1 & 1 \\
                        0 & 1 & 0 & 0 & 0 & 0 \\
                        1 & 0 & 0 & 0 & 0 & 0 \\
                        1 & 1 & 1 & 0 & 1 & 1 \\
                        \hline
                   \end{array}
               \]
            \end{table}
        \end{practices}

        %17.
        \begin{practices}
            \begin{align*}
                \neg (p \leftrightarrow q)
                &\equiv \neg [(p \wedge q) \vee (\neg p \wedge \neg q)] \\
                &\equiv (\neg p \vee \neg q) \wedge (p \vee q) \\
                &\equiv [\neg p \wedge (p \vee q)] \vee [\neg q \wedge (p \vee q)] \\
                &\equiv [(\neg p \wedge p) \vee (\neg p \wedge q)] \vee [(\neg q \wedge p) \vee (\neg q \wedge q)] \\
                &\equiv [F \vee (\neg p \wedge q)] \vee [(\neg q \wedge p) \vee F] \\
                &\equiv (\neg p \wedge q) \vee (\neg q \wedge p) \\
                &\equiv p \leftrightarrow \neg q
            \end{align*}
        \end{practices}

        %18.
        \begin{practices}
            \begin{align*}
                p \rightarrow q
                &\equiv \neg p \vee q \\
                &\equiv q \vee \neg p \\
                &\equiv \neg q \rightarrow \neg p
            \end{align*}
        \end{practices}

        %19.
        \begin{practices}
            \begin{align*}
                \neg p \leftrightarrow q
                &\equiv (\neg p \wedge q) \vee (p \wedge \neg q) \\
                &\equiv p \leftrightarrow \neg q
            \end{align*}
        \end{practices}

        %20.
        \begin{practices}
            \begin{table}[H]
                \[
                    \begin{array}{c|c|c|c}
                        \hline
                        p & q & p \leftrightarrow q & \neg (p \oplus q)\\
                        \hline
                        0 & 0 & 1 & 1 \\
                        0 & 1 & 0 & 0 \\
                        1 & 0 & 0 & 0 \\
                        1 & 1 & 1 & 1 \\
                        \hline
                   \end{array}
               \]
            \end{table}
        \end{practices}

        %21.
        \begin{practices}
            \begin{align*}
                \neg (p \leftrightarrow q)
                &\equiv \neg [(p \wedge q) \vee (\neg p \wedge \neg q)] \\
                &\equiv (\neg p \vee \neg q) \wedge (p \vee q) \\
                &\equiv [\neg p \wedge (p \vee q)] \vee [\neg q \wedge (p \vee q)] \\
                &\equiv [(\neg p \wedge p) \vee (\neg p \wedge q)] \vee [(\neg q \wedge p) \vee (\neg q \wedge q)] \\
                &\equiv [F \vee (\neg p \wedge q)] \vee [(\neg q \wedge p) \vee F] \\
                &\equiv (\neg p \wedge q) \vee (\neg q \wedge p) \\
                &\equiv \neg p \leftrightarrow q
            \end{align*}
        \end{practices}

        %22.
        \begin{practices}
            \begin{align*}
                (p \rightarrow q) \wedge (p \rightarrow r)
                &\equiv (\neg p \vee q) \wedge (\neg p \vee r) \\
                &\equiv \neg p \vee (q \wedge r) \\
                &\equiv p \rightarrow (q \wedge r)
            \end{align*}
        \end{practices}

        %23.
        \begin{practices}
            \begin{align*}
                (p \rightarrow r) \wedge (q \rightarrow r)
                &\equiv (\neg p \vee r) \wedge (\neg q \vee r) \\
                &\equiv (\neg p \wedge \neg q) \vee r \\
                &\equiv \neg (p \vee q) \vee r \\
                &\equiv (p \vee q) \rightarrow r
            \end{align*}
        \end{practices}

        %24.
        \begin{practices}
            \begin{align*}
                (p \rightarrow q) \vee (p \rightarrow r)
                &\equiv (\neg p \vee q) \vee (\neg p \vee r) \\
                &\equiv \neg p \vee (q \vee r) \\
                &\equiv p \rightarrow (q \vee r)
            \end{align*}
        \end{practices}

        %25.
        \begin{practices}
            \begin{align*}
                (p \rightarrow r) \vee (q \rightarrow r)
                &\equiv (\neg p \vee r) \vee (\neg q \vee r) \\
                &\equiv (\neg p \vee \neg q) \vee r \\
                &\equiv \neg (p \wedge q) \vee r \\
                &\equiv (p \wedge q) \rightarrow r
            \end{align*}
        \end{practices}

        %26.
        \begin{practices}
            \begin{align*}
                \neg p \rightarrow (q \rightarrow r)
                &\equiv p \vee \neg q \vee r \\
                &\equiv \neg q \vee (p \vee r) \\
                &\equiv q \rightarrow (p \vee r)
            \end{align*}
        \end{practices}

        %27.
        \begin{practices}
            \begin{align*}
                p \leftrightarrow q
                &\equiv (p \wedge q) \vee (\neg p \wedge \neg q) \\
                &\equiv [(p \wedge q) \vee (p \wedge \neg p)] \vee [(\neg p \wedge \neg q) \vee (q \wedge \neg q)] \\
                &\equiv [p \wedge (q \vee \neg p)] \vee [\neg q \wedge (\neg p \vee q)] \\
                &\equiv (\neg p \vee q) \wedge (p \vee \neg q) \\
                &\equiv (p \rightarrow q) \wedge (q \rightarrow p)
            \end{align*}
        \end{practices}

        %28.
        \begin{practices}
            \begin{align*}
                p \leftrightarrow q
                &\equiv (p \wedge q) \vee (\neg p \wedge \neg q) \\
                &\equiv \neg p \leftrightarrow \neg q
            \end{align*}
        \end{practices}

        %29.
        \begin{practices}
            \begin{align*}
                (p \rightarrow q) \wedge (q \rightarrow r) \rightarrow (p \rightarrow r)
                &\equiv \neg [(\neg p \vee q) \wedge (\neg q \vee r)] \vee (\neg p \vee r) \\
                &\equiv [(p \wedge \neg q) \vee (q \wedge \neg r)] \vee (\neg p \vee r) \\
                &\equiv \neg q \vee q \vee \neg p \vee r \\
                &\equiv T \vee \neg p \vee r \\
                &\equiv T
            \end{align*}
        \end{practices}

        %30.
        \begin{practices}
            \begin{align*}
                (p \vee q) \wedge (\neg p \vee r) \rightarrow (q \vee r)
                &\equiv [(\neg p \wedge \neg q) \vee (p \wedge \neg r)] \vee q \vee r \\
                &\equiv \neg p \vee p \vee q \vee r \\
                &\equiv T \vee q \vee r \\
                &\equiv T
            \end{align*}
        \end{practices}

        %31.
        \begin{practices}
            \begin{table}[H]
                \[
                    \begin{array}{c|c|c|c|c|c|c}
                        \hline
                        p & q & r & p \rightarrow q & q \rightarrow r & (p \rightarrow q) \rightarrow r & p \rightarrow (q \rightarrow r) \\
                        \hline
                        0 & 0 & 0 & 1 & 1 & 0 & 1 \\
                        0 & 0 & 1 & 1 & 1 & 1 & 1 \\
                        0 & 1 & 0 & 1 & 0 & 0 & 1 \\
                        0 & 1 & 1 & 1 & 1 & 1 & 1 \\
                        1 & 0 & 0 & 0 & 1 & 1 & 1 \\
                        1 & 0 & 1 & 0 & 1 & 1 & 1 \\
                        1 & 1 & 0 & 1 & 0 & 0 & 0 \\
                        1 & 1 & 1 & 1 & 1 & 1 & 1 \\
                        \hline
                   \end{array}
               \]
            \end{table}
        \end{practices}

        %32.
        \begin{practices}
            \begin{table}[H]
                \[
                    \begin{array}{c|c|c|c|c|c|c|c}
                        \hline
                        p & q & r & p \wedge q & p \rightarrow r & q \rightarrow r & (p \wedge q) \rightarrow r & (p \rightarrow r) \wedge (q \rightarrow r) \\
                        \hline
                        0 & 0 & 0 & 0 & 1 & 1 & 1 & 1 \\
                        0 & 0 & 1 & 0 & 1 & 1 & 1 & 1 \\
                        0 & 1 & 0 & 0 & 1 & 0 & 1 & 0 \\
                        0 & 1 & 1 & 0 & 1 & 1 & 1 & 1 \\
                        1 & 0 & 0 & 0 & 0 & 1 & 1 & 0 \\
                        1 & 0 & 1 & 0 & 1 & 1 & 1 & 1 \\
                        1 & 1 & 0 & 1 & 0 & 0 & 0 & 0 \\
                        1 & 1 & 1 & 1 & 1 & 1 & 1 & 1 \\
                        \hline
                   \end{array}
               \]
            \end{table}
        \end{practices}

        %33.
        \begin{practices}
            \begin{table}[H]
                \[
                    \begin{array}{c|c|c|c|c|c|c|c|c|c}
                        \hline
                        p & q & r & s & p \rightarrow q & r \rightarrow s & (p \rightarrow q) \rightarrow (r \rightarrow s) & p \rightarrow r & q \rightarrow s & (p \rightarrow r) \rightarrow (q \rightarrow s) \\
                        \hline
                        0 & 0 & 0 & 0 & 1 & 1 & 1 & 1 & 1 & 1 \\
                        0 & 0 & 0 & 1 & 1 & 1 & 1 & 1 & 1 & 1 \\
                        0 & 0 & 1 & 0 & 1 & 0 & 0 & 1 & 1 & 1 \\
                        0 & 0 & 1 & 1 & 1 & 1 & 1 & 1 & 1 & 1 \\
                        0 & 1 & 0 & 0 & 1 & 1 & 1 & 1 & 0 & 0 \\
                        0 & 1 & 0 & 1 & 1 & 1 & 1 & 1 & 1 & 1 \\
                        0 & 1 & 1 & 0 & 1 & 0 & 0 & 1 & 0 & 0 \\
                        0 & 1 & 1 & 1 & 1 & 1 & 1 & 1 & 1 & 1 \\
                        1 & 0 & 0 & 0 & 0 & 1 & 1 & 0 & 1 & 1 \\
                        1 & 0 & 0 & 1 & 0 & 1 & 1 & 0 & 1 & 1 \\
                        1 & 0 & 1 & 0 & 0 & 0 & 1 & 1 & 1 & 1 \\
                        1 & 0 & 1 & 1 & 0 & 1 & 1 & 1 & 1 & 1 \\
                        1 & 1 & 0 & 0 & 1 & 1 & 1 & 0 & 0 & 1 \\
                        1 & 1 & 0 & 1 & 1 & 1 & 1 & 0 & 1 & 1 \\
                        1 & 1 & 1 & 0 & 1 & 0 & 0 & 1 & 0 & 0 \\
                        1 & 1 & 1 & 1 & 1 & 1 & 1 & 1 & 1 & 1 \\
                        \hline
                   \end{array}
               \]
            \end{table}
        \end{practices}

        %34.
        \begin{practices}
            \begin{enumerate}[A.]
                \item $p \wedge \neg q$
                \item $p \vee (q \wedge (r \vee F))$
                \item $(p \vee \neg q) \wedge (q \vee T)$
            \end{enumerate}
        \end{practices}

        %35.
        \begin{practices}
            \begin{enumerate}[A.]
                \item $p \vee \neg q \vee \neg r$
                \item $(p \vee q \vee r) \wedge s$
                \item $(p \wedge T) \vee (q \wedge F)$
            \end{enumerate}
        \end{practices}

        %36.
        \begin{practices}
            当 $s$ 是一个足够简单的命题时,不包含 $\vee \wedge F T$时。
        \end{practices}

        %37.
        \begin{practices}
            当经过两次对换,所有的符号又变回原命题,所以相等。
        \end{practices}

        %38.
        \begin{practices}
            根据对偶式的定义,进行替换后相等。
        \end{practices}

        %39.
        \begin{practices}
            根据德·摩根定律, $\neg p$ 和 $p^{*}$ 是相同的,只不过其中原子命题 $p_i$ 被其否定代替。
        \end{practices}

        %40.
        \begin{practices}
            $p \wedge q \wedge \neg r$
        \end{practices}

        %41.
        \begin{practices}
            多个命题的异或,若含奇数个真命题,则结果为真;
            若含偶数个真命题,则结果为假。

            多个命题的同或,若含奇数个假命题,则结果为假;
            若含偶数个假命题,则结果为真。

            $\neg (p \otimes q \otimes r)$
        \end{practices}

        %42.
        \begin{practices}
            写出每一行的合取。
            若值为真,则不变,若值为假,则取反。
            将所有项做析取,得到析取范式。
        \end{practices}

        %43.
        \begin{practices}
            写出真值表,根据42题可得,其复合命题可以仅使用 $\neg \wedge \vee$ 表达。
            故 $\neg \wedge \vee$ 是功能完备的。
        \end{practices}

        %44.
        \begin{practices}
            \begin{align*}
                \neg (\neg p \wedge \neg q)
                &\equiv \neg \neg p \vee \neg \neg q \\
                &\equiv p \vee q
            \end{align*}

            根据43题,其复合命题中所有的 $p \vee q$ 都可以用 $\neg (\neg p \wedge \neg q)$ 代替。
            故 $\neg \wedge$ 是功能完备的。
        \end{practices}

        %45.
        \begin{practices}
            \begin{align*}
                \neg (\neg p \vee \neg q)
                &\equiv \neg \neg p \wedge \neg \neg q \\
                &\equiv p \wedge q
            \end{align*}

            根据43题,其复合命题中所有的 $p \wedge q$ 都可以用 $\neg (\neg p \vee \neg q)$ 代替。
            故 $\neg \vee$ 是功能完备的。
        \end{practices}

        %46.
        \begin{practices}
            \begin{table}[H]
                \[
                    \begin{array}{c|c|c}
                        \hline
                        p & q & p | q \\
                        \hline
                        0 & 0 & 1 \\
                        0 & 1 & 1 \\
                        0 & 1 & 1 \\
                        1 & 1 & 0 \\
                        \hline
                   \end{array}
               \]
            \end{table}
        \end{practices}

        %47.
        \begin{practices}
            \begin{table}[H]
                \[
                    \begin{array}{c|c|c|c}
                        \hline
                        p & q & p | q & \neg (p \wedge q) \\
                        \hline
                        0 & 0 & 1 & 1 \\
                        0 & 1 & 1 & 1 \\
                        0 & 1 & 1 & 1 \\
                        1 & 1 & 0 & 0 \\
                        \hline
                   \end{array}
               \]
            \end{table}
        \end{practices}

        %48.
        \begin{practices}
            \begin{table}[H]
                \[
                    \begin{array}{c|c|c}
                        \hline
                        p & q & p \downarrow q \\
                        \hline
                        0 & 0 & 1 \\
                        0 & 1 & 0 \\
                        0 & 1 & 0 \\
                        1 & 1 & 0 \\
                        \hline
                   \end{array}
               \]
            \end{table}
        \end{practices}

        %49.
        \begin{practices}
            \begin{table}[H]
                \[
                    \begin{array}{c|c|c|c}
                        \hline
                        p & q & p \downarrow q & \neg (p \vee q) \\
                        \hline
                        0 & 0 & 1 & 1 \\
                        0 & 1 & 0 & 0 \\
                        0 & 1 & 0 & 0 \\
                        1 & 1 & 0 & 0 \\
                        \hline
                   \end{array}
               \]
            \end{table}
        \end{practices}

        %50.
        \begin{practices}
            \begin{enumerate}[A.]
                \item
                {
                    \begin{table}[H]
                        \[
                            \begin{array}{c|c|c}
                                \hline
                                p & p \downarrow p & \neg p \\
                                \hline
                                0 & 1 & 1 \\
                                1 & 0 & 0 \\
                                \hline
                           \end{array}
                       \]
                    \end{table}
                }
                \item
                {
                    \begin{align*}
                        (p \downarrow q) \downarrow (p \downarrow q)
                        &\equiv \neg (p \downarrow q) \\
                        &\equiv \neg (\neg (p \vee q)) \\
                        &\equiv p \vee q
                    \end{align*}
                }
                \item
                {
                    根据45题, $\{\downarrow\}$ 是功能完备集。
                }
            \end{enumerate}
        \end{practices}

        %51.
        \begin{practices}
            \begin{align*}
                p \rightarrow q
                &\equiv (\neg p) \vee q \\
                &\equiv (p \downarrow p) \vee q \\
                &\equiv [(p\downarrow p) \downarrow q] \downarrow [(p\downarrow p) \downarrow q]
            \end{align*}
        \end{practices}

        %52.
        \begin{practices}
            \begin{enumerate}[A.]
                \item
                {
                    \begin{table}[H]
                        \[
                            \begin{array}{c|c|c}
                                \hline
                                p & p | p & \neg p \\
                                \hline
                                0 & 1 & 1 \\
                                1 & 0 & 0 \\
                                \hline
                           \end{array}
                       \]
                    \end{table}
                }
                \item
                {
                    \begin{align*}
                        (p | q) | (p | q)
                        &\equiv \neg (p | q) \\
                        &\equiv \neg (\neg (p \wedge q)) \\
                        &\equiv p \wedge q
                    \end{align*}
                }
                \item
                {
                    根据44题, $\{\downarrow\}$ 是功能完备集。
                }
            \end{enumerate}
        \end{practices}

        %53.
        \begin{practices}
            \begin{align*}
                p | q
                &\equiv \neg (p \wedge q) \\
                &\equiv \neg (q \wedge p) \\
                &\equiv q | p
            \end{align*}
        \end{practices}

        %54.
        \begin{practices}
            \begin{table}[H]
                \[
                    \begin{array}{c|c|c|c|c|c|c}
                        \hline
                        p & q & r & (q | r) & (p | q) & p | (q | r) & (p | q) | r \\
                        \hline
                        0 & 0 & 0 & 1 & 1 & 1 & 1 \\
                        0 & 0 & 1 & 1 & 1 & 1 & 0 \\
                        0 & 1 & 0 & 1 & 1 & 1 & 1 \\
                        0 & 1 & 1 & 0 & 1 & 1 & 0 \\
                        1 & 0 & 0 & 1 & 1 & 0 & 1 \\
                        1 & 0 & 1 & 1 & 1 & 0 & 0 \\
                        1 & 1 & 0 & 1 & 0 & 0 & 1 \\
                        1 & 1 & 1 & 0 & 0 & 1 & 1 \\
                        \hline
                   \end{array}
               \]
            \end{table}
        \end{practices}

        %55.
        \begin{practices}
            16种。
        \end{practices}

        %56.
        \begin{practices}
            若 $p, q$ 逻辑等价,则其拥有相同的真值表, $q, r$ 逻辑等价,则也有相同的真值表。
            所以 $p, q, r$ 真值表相同,即 $p, q, r$ 三者互为逻辑等价。
        \end{practices}

        %57.
        \begin{practices}
            目录数据库没有打开,或者系统在初始状态,或者监控程序被置于关闭状态。
        \end{practices}

        %58.
        \begin{practices}
            五个同时为真。
        \end{practices}

        %59.
        \begin{practices}
            九个同时为真。
        \end{practices}

        %60.
        \begin{practices}
            根据定义,对于每一种真值,不可满足式的否定真值都为真,所以不可满足的复合命题的否定是永真式。
            对于每一种真值,永真式的否定真值都为假,所以永真式的否定是不可满足的。
        \end{practices}

        %61.
        \begin{practices}
            \begin{enumerate}[A.]
                \item
                {
                    当 $p = False, q = False$ 时,命题为真。
                }
                \item
                {
                    不可满足。
                }
                \item
                {
                    不可满足。
                }
            \end{enumerate}
        \end{practices}

        %62.
        \begin{practices}
            根据最大项之积, $4$ 个命题变元,只要少于 $2^4 = 16$ 个最大项,都是可满足的。

            \begin{enumerate}[A.]
                \item
                {
                    最大项最多有10个。
                }
                \item
                {
                    最大项最多有12个。
                }
                \item
                {
                    $p \vee \neg q \vee \neg s$ 最大项为 $p \vee \neg q \vee r \vee \neg s, p \vee \neg q \vee \neg r \vee \neg s$ , $p \vee \neg q \vee \neg r$ 最大项为 $p \vee \neg q \vee \neg r \vee s, p \vee \neg q \vee \neg r \vee \neg s$ 。
                    其中 $p \vee \neg q \vee \neg r \vee \neg s$ 重复,即只有三个最大项。其余六项最多有12个最大项,最大项不超过15个。
                }
            \end{enumerate}

            三个命题均为可满足的。
        \end{practices}

        %63.
        \begin{practices}
            \begin{enumerate}[A.]
                \item
                {
                    对于已知数的每个单元,当第 $i$ 行第 $j$ 列的单元是已知数 $n$ 时,断言 $p(i, j, n)$ 。
                }
                \item
                {
                    断言每一行包含了每一个数:

                    \[
                        \bigwedge^{4}_{i = 1}\bigwedge^{4}_{n = 1}\bigvee^{4}_{j = 1} p(i, j, n)
                    \]
                }
                \item
                {
                    断言每一列包含了每一个数:

                    \[
                        \bigwedge^{4}_{j = 1}\bigwedge^{4}_{n = 1}\bigvee^{4}_{i = 1} p(i, j, n)
                    \]
                }
                \item
                {
                    断言每一个四宫格包含了每一个数:

                    \[
                        \bigwedge^{1}_{r = 0}\bigwedge^{1}_{s = 0}\bigwedge^{4}_{n = 1}\bigvee^{2}_{i = 1}\bigvee^{2}_{j = 1} p(3r + i, 3s + j, n)
                    \]
                }
                \item
                {
                    断言没有一个单元包含多于一个数,对所有可能的 $p(i, i, n) \rightarrow \neg p(i, j, n^{\prime})$ 取合取,其中 $n, n^{\prime}, i, j$的取值范围是 $1 \sim 4$ 并且 $n \neq n^{\prime}$ 。
                }
            \end{enumerate}
        \end{practices}

        %64.
        \begin{practices}
            \[
                \bigwedge^{9}_{i = 1} \bigwedge^{9}_{j = 1} \bigvee^{9}_{n = 1} p(i, j, n)
            \]
        \end{practices}

        %65.
        \begin{practices}
            要断言第 $j$ 列包含数 $n$ ,构成 $\bigvee\limits^{9}_{i = 1} p(i, j, n)$ 。
            要断言第 $j$ 列包含所有的 $n$ 个数,将 $n$ 的所有九个可能值的析取式做合取,得到 $\bigwedge\limits^{9}_{n = 1} \bigvee\limits^{9}_{i = 1} p(i, j, n)$ 。
            要断言每一列包含了每一个数,将九列做合取,得到了: $\bigwedge\limits^{9}_{j = 1} \bigwedge\limits^{9}_{n = 1} \bigvee\limits^{9}_{i = 1} p(i, j, n)$ 。
        \end{practices}

        %66.
        \begin{practices}
            要断言某个九宫格包含数 $n$ ,构成 $\bigvee\limits^{3}_{i = 1}\bigvee\limits^{3}_{j = 1} p(3r + i, 3s + j, n)$ 。
            要断言某个九宫格包含所有的 $n$ 个数,将 $n$ 的所有九个可能值的析取式做合取,得到 $\bigwedge\limits^{9}_{n = 1}\bigvee\limits^{3}_{i = 1}\bigvee\limits^{3}_{j = 1} p(3r + i, 3s + j, n)$ 。
            要断言每一个九宫格包含了每一个数,将九个九宫格做合取,得到了: $\bigwedge\limits^{2}_{r = 0}\bigwedge\limits^{2}_{s = 0}\bigwedge\limits^{9}_{n = 1}\bigvee\limits^{3}_{i = 1}\bigvee\limits^{3}_{j = 1} p(3r + i, 3s + j, n)$ 。
        \end{practices}
    }
}
