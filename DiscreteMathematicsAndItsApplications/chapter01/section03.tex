%%
%% Author: Clay
%% 2019/6/10
%%

\section{命题等价式}
{
    \subsection{引言}
    {
        数学证明中使用的一个重要步骤就是用真值相同的一条语句替换另一条语句。
        因此,从给定复合命题生成具有相同真值命题的方法广泛使用于数学证明的构造。

        \begin{defines}
            一个真值永远是真的复合命题,称为永真式(tautology),也成为重言式。
            一个真值永远为假的复合命题,称为矛盾式(contradicition)。
            既不是永真式又不是矛盾式的复合命题称为可能式(contingency)。
        \end{defines}

        \begin{table}[htb]
            \centering

            \[
                \begin{array}{c|c|c|c}
                    \hline
                    p & \neg p & p \vee \neg p & p \wedge \neg p \\
                    \hline
                    T & F & T & F \\
                    F & T & T & F \\
                    \hline
                \end{array}
            \]

            \caption{永真式和矛盾式的例子}
        \end{table}
    }

    \subsection{逻辑等价式}
    {
        \begin{wraptable}{r}{.3333\textwidth{}}
            \centering

            \[
                \begin{array}{c}
                    \hline
                    \neg (p \wedge q) \equiv \neg p \vee \neg q \\
                    \neg (p \vee q) \equiv \neg p \wedge \neg q \\
                    \hline
                \end{array}
            \]

            \caption{德 $\cdot$ 摩根律}
        \end{wraptable}

        在所有可能的情况下都有相同真值的两个复合命题称为\emreg{逻辑等价}的。

        \begin{defines}
            如果 $p \leftrightarrow q$ 是,永真式,则复合命题 $p$ 和 $q$ 是逻辑等价的。
            用记号 $p \equiv q$ 表示 $p$ 和 $q$ 是逻辑等价的。
        \end{defines}

        判断两个复合命题是否等价的方法之一是使用真值表。
        特别地,复合命题 $p$ 和 $q$ 是等价的,当且仅当对应它们真值的两列完全一致。

        德 $\cdot$ 摩根律可以扩展为:
        \begin{align}
            \neg (p_1 \vee p_2 \vee \cdots \vee p_n) \equiv (\neg p_1 \wedge \neg p_2 \wedge \cdots \wedge \neg p_n)
        \end{align}

        和
        \begin{align}
            \neg (p_1 \wedge p_2 \wedge \cdots \wedge p_n) \equiv (\neg p_1 \vee \neg p_2 \vee \cdots \vee \neg p_n)
        \end{align}

        有时用符号 $\bigvee\limits_{j = 1}^n p_j$ 来表示 $p_1 \vee p_2 \vee \cdots \vee p_n$ ,用 $\bigwedge\limits_{j = 1}^n p_j$ 来表示 $p_1 \wedge p_2 \wedge \cdots \wedge p_n$ 。
        采用这种记法扩展的德 $\cdot$ 摩根律就可以简洁地写成 $\neg (\bigvee\limits_{j = 1}^n p_j) \equiv \bigwedge\limits_{j = 1}^n \neg p_j$ 和 $\neg (\bigwedge\limits_{j = 1}^n p_j) \equiv \bigvee\limits_{j = 1}^n \neg p_j$ 。

        \begin{minipage}[c]{\textwidth{}}
            \begin{minipage}[c]{.6\textwidth{}}
                \begin{table}[H]
                    \centering

                    \begin{tabular}{l|c}
                        \hline
                        \multicolumn{1}{c|}{等价式} & 名称 \\
                        \hline
                        $p \wedge T \equiv p$ & \multirow{2}*{恒等律} \\
                        $p \vee F \equiv p$ & \\
                        \hline
                        $p \vee T \equiv T$ & \multirow{2}*{支配律} \\
                        $p \wedge F \equiv F$ & \\
                        \hline
                        $p \vee p \equiv p$ & \multirow{2}*{幂等律} \\
                        $p \wedge p \equiv p$ & \\
                        \hline
                        $\neg (\neg q) \equiv q$ & 双重否定律 \\
                        \hline
                        $p \vee q \equiv q \vee p$ & \multirow{2}*{交换律} \\
                        $p \wedge q \equiv q \wedge p$ & \\
                        \hline
                        $(p \vee q) \vee r \equiv p \vee (q \vee r)$ & \multirow{2}*{结合律} \\
                        $(p \wedge q) \wedge r \equiv p \wedge (q \wedge r)$ & \\
                        \hline
                        $p \vee (q \wedge r) \equiv (p \vee q) \wedge (p \vee r)$ & \multirow{2}*{分配律} \\
                        $p \wedge (q \vee r) \equiv (p \wedge q) \vee (p \wedge r)$ & \\
                        \hline
                        $\neg (p \wedge q) \equiv \neg p \vee \neg q$ & \multirow{2}*{德 $\cdot$ 摩根律} \\
                        $\neg (p \vee q) \equiv \neg p \wedge \neg q$ & \\
                        \hline
                        $p \vee (p \wedge q) \equiv p$ & \multirow{2}*{吸收律} \\
                        $p \wedge (p \vee q) \equiv p$ & \\
                        \hline
                        $p \vee \neg p \equiv T$ & \multirow{2}*{否定律} \\
                        $p \wedge \neg p \equiv F$ & \\
                        \hline
                    \end{tabular}

                    \caption{逻辑等价式}
                \end{table}
            \end{minipage}%
            \begin{minipage}[c]{.4\textwidth{}}
                \begin{minipage}[c]{\textwidth{}}
                    \begin{table}[H]
                        \centering

                        \[
                            \begin{array}{l}
                                \hline
                                p \rightarrow q \equiv \neg p \vee q \\
                                p \rightarrow q \equiv \neg q \rightarrow \neg p \\
                                p \vee q \equiv \neg p \rightarrow q \\
                                p \wedge q \equiv \neg (p \rightarrow \neg q) \\
                                \neg (p \rightarrow q) \equiv p \wedge \neg q \\
                                (p \rightarrow q) \wedge (p \rightarrow r) \equiv p \rightarrow (q \wedge r) \\
                                (p \rightarrow r) \wedge (q \rightarrow r) \equiv (p \vee q) \rightarrow r \\
                                (p \rightarrow q) \vee (p \rightarrow r) \equiv p \rightarrow (q \vee r) \\
                                (p \rightarrow r) \vee (q \rightarrow r) \equiv (p \wedge q) \rightarrow r \\
                                \hline
                            \end{array}
                        \]

                        \caption{条件命题的逻辑等价式}
                    \end{table}
                \end{minipage}%

                \vfill

                \begin{minipage}[c]{\textwidth{}}
                    \begin{table}[H]
                        \centering

                        \[
                            \begin{array}{l}
                                \hline
                                p \leftrightarrow q \equiv (p \rightarrow q) \wedge (q \rightarrow p) \\
                                p \leftrightarrow q \equiv \neg p \leftrightarrow \neg q \\
                                p \leftrightarrow q \equiv (p \wedge q) \vee (\neg p \wedge \neg q) \\
                                \neg (p \leftrightarrow q) \equiv p \leftrightarrow \neg q \\
                                \hline
                            \end{array}
                        \]

                        \caption{双条件命题的逻辑等价式}
                    \end{table}
                \end{minipage}
            \end{minipage}
        \end{minipage}
    }

    \subsection{德 $\cdot$ 摩根律的运用}
    {
        $\neg (p \wedge q) \equiv \neg p \vee \neg q$ 说明一个析取式的否定是由各分命题否定的合取式组成的。
        $\neg (p \vee q) \equiv \neg p \wedge \neg q$说明一个合取式的否定是由各分命题否定的析取式组成的。
    }

    \subsection{构造新的逻辑等价式}
    {
        复合命题中的一个命题可以用与它逻辑等价的复合命题替换,而不改变原复合命题的真值。
        如果 $p$ 和 $q$ 是逻辑等价的, $q$ 和 $r$ 是逻辑等价的,那么 $p$ 和 $r$ 也是逻辑等价的。
    }

    \subsection{命题的可满足性}
    {
        一个复合命题称为是\emreg{可满足的},如果存在一个对其变元的真值赋值使其为真。
        当不存在这样的赋值时,即当复合命题对所有变元的真值赋值都是假的,则复合命题是\emreg{不可满足的}。
        注意一个复合命题是不可满足的当且仅当它的否定对所有变元的真值赋值都是真的,也就是说,当且仅当它的否定是永真式。

        当找到一个特定的使得复合命题为真的真值赋值时,就证明了它是可满足的。
        这样的一个赋值成为这个特定的可满足性问题的一个\emreg{解}。
        要证明一个复合命题是不可满足的,需要证明每一组变元的真值赋值都使其为假。
    }

    \subsection{可满足性的应用}
    {
        在不同领域中的许多问题都可以用命题可满足性来建立模型。
    }

    %1.
    \begin{practices}
        \begin{enumerate}[A.]
            \item
            {
                \begin{table}[H]
                    \[
                        \begin{array}{c|c|c}
                            \hline
                            p & p \wedge T & p \\
                            \hline
                            0 & 0 & 0 \\
                            1 & 1 & 1 \\
                        \end{array}
                    \]
                \end{table}
            }
            \item
            {
                \begin{table}[H]
                    \[
                        \begin{array}{c|c|c}
                            \hline
                            p & p \vee F & p \\
                            \hline
                            0 & 0 & 0 \\
                            1 & 1 & 1 \\
                        \end{array}
                    \]
                \end{table}
            }
            \item
            {
                \begin{table}[H]
                    \[
                        \begin{array}{c|c|c}
                            \hline
                            p & p \wedge F & F \\
                            \hline
                            0 & 0 & 0 \\
                            1 & 0 & 0 \\
                        \end{array}
                    \]
                \end{table}
            }
            \item
            {
                \begin{table}[H]
                    \[
                        \begin{array}{c|c|c}
                            \hline
                            p & p \vee T & T \\
                            \hline
                            0 & 1 & 1 \\
                            1 & 1 & 1 \\
                        \end{array}
                    \]
                \end{table}
            }
            \item
            {
                \begin{table}[H]
                    \[
                        \begin{array}{c|c|c}
                            \hline
                            p & p \vee p & p \\
                            \hline
                            0 & 0 & 0 \\
                            1 & 1 & 1 \\
                        \end{array}
                    \]
                \end{table}
            }
            \item
            {
                \begin{table}[H]
                    \[
                        \begin{array}{c|c|c}
                            \hline
                            p & p \wedge p & p \\
                            \hline
                            0 & 0 & 0 \\
                            1 & 1 & 1 \\
                        \end{array}
                    \]
                \end{table}
            }
        \end{enumerate}
    \end{practices}

    %2.
    \begin{practices}
        \begin{table}[H]
            \[
                \begin{array}{c|c|c}
                    \hline
                    p & \neg \neg p & p \\
                    \hline
                    0 & 0 & 0 \\
                    1 & 1 & 1 \\
                \end{array}
            \]
        \end{table}
    \end{practices}

    %3.
    \begin{practices}
        \begin{enumerate}[A.]
            \item
            {
                \begin{table}[H]
                    \[
                        \begin{array}{c|c|c|c}
                            \hline
                            p & q & p \vee q & q \vee p \\
                            \hline
                            0 & 0 & 0 & 0 \\
                            0 & 1 & 1 & 1 \\
                            1 & 0 & 1 & 1 \\
                            1 & 1 & 1 & 1 \\
                        \end{array}
                    \]
                \end{table}
            }
            \item
            {
                \begin{table}[H]
                    \[
                        \begin{array}{c|c|c|c}
                            \hline
                            p & q & p \wedge q & q \wedge p \\
                            \hline
                            0 & 0 & 0 & 0 \\
                            0 & 1 & 0 & 0 \\
                            1 & 0 & 0 & 0 \\
                            1 & 1 & 1 & 1 \\
                        \end{array}
                    \]
                \end{table}
            }
        \end{enumerate}
    \end{practices}

    %4.
    \begin{practices}
        \begin{enumerate}[A.]
            \item
            {
                \begin{table}[H]
                    \[
                        \begin{array}{c|c|c|c|c}
                            \hline
                            p & q & r & (p \vee q) \ vee r & q \vee (p \vee r) \\
                            \hline
                            0 & 0 & 0 & 0 & 0 \\
                            0 & 0 & 1 & 1 & 1 \\
                            0 & 1 & 0 & 1 & 1 \\
                            0 & 1 & 1 & 1 & 1 \\
                            1 & 0 & 0 & 1 & 1 \\
                            1 & 0 & 1 & 1 & 1 \\
                            1 & 1 & 0 & 1 & 1 \\
                            1 & 1 & 1 & 1 & 1 \\
                        \end{array}
                    \]
                \end{table}
            }
            \item
            {
                \begin{table}[H]
                    \[
                        \begin{array}{c|c|c|c|c}
                            \hline
                            p & q & r & (p \wedge q) \wedge r & q \wedge (p \wedge r) \\
                            \hline
                            0 & 0 & 0 & 0 & 0 \\
                            0 & 0 & 1 & 0 & 0 \\
                            0 & 1 & 0 & 0 & 0 \\
                            0 & 1 & 1 & 0 & 0 \\
                            1 & 0 & 0 & 0 & 0 \\
                            1 & 0 & 1 & 0 & 0 \\
                            1 & 1 & 0 & 0 & 0 \\
                            1 & 1 & 1 & 1 & 1 \\
                        \end{array}
                    \]
                \end{table}
            }
        \end{enumerate}
    \end{practices}

    %5.
    \begin{practices}
        \begin{table}[H]
            \[
                \begin{array}{c|c|c|c|c|c|c|c}
                    \hline
                    p & q & r & q \vee r & p \wedge (q \vee r) & p \wedge q & p \wedge r & (p \wedge q) \vee (p \wedge r) \\
                    \hline
                    0 & 0 & 0 & 0 & 0 & 0 & 0 & 0 \\
                    0 & 0 & 1 & 1 & 0 & 0 & 0 & 0 \\
                    0 & 1 & 0 & 1 & 0 & 0 & 0 & 0 \\
                    0 & 1 & 1 & 1 & 0 & 0 & 0 & 0 \\
                    1 & 0 & 0 & 0 & 0 & 0 & 0 & 0 \\
                    1 & 0 & 1 & 1 & 1 & 0 & 1 & 1 \\
                    1 & 1 & 0 & 1 & 1 & 1 & 0 & 1 \\
                    1 & 1 & 1 & 1 & 1 & 1 & 1 & 1 \\
                \end{array}
            \]
        \end{table}
    \end{practices}

    %6.
    \begin{practices}
        \begin{table}[H]
            \[
                \begin{array}{c|c|c|c}
                    \hline
                    p & q & \neg (p \wedge q) &\neg p \vee \neg q \\
                    \hline
                    0 & 0 & 1 & 1 \\
                    0 & 1 & 1 & 1 \\
                    1 & 0 & 1 & 1 \\
                    1 & 1 & 0 & 0 \\
                \end{array}
            \]
        \end{table}
    \end{practices}

    %7.
    \begin{practices}
        \begin{enumerate}[A.]
            \item Jane不富裕,或者不快乐。
            \item Carlos明天既不骑自行车,也不跑步。
            \item Mei不是步行也不乘公共汽车去上课。
            \item Ibrahim不聪明或者不用功。
        \end{enumerate}
    \end{practices}

    %8.
    \begin{practices}
        \begin{enumerate}[A.]
            \item Kwame不会在工业界找工作并且也不会去研究生院读书。
            \item Yoshiko不会掌握Java或者不会掌握微积分。
            \item James不年轻或者不强壮。
            \item Rita不会搬到俄勒冈州也不会搬到华盛顿去。
        \end{enumerate}
    \end{practices}
}
