%%
%% Author: Clay
%% 2019/12/5
%%

\section{证明的方法和策略}
{
    \subsection{引言}
    {
        数学家工作时,他们拟定猜测并试图证明或推翻之。
    }

    \subsection{穷举证明法和分情形证明法}
    {
        有时候采用单一的论证不能在定理的所有可能情况下都成立,故不能证明该定理。
        现在介绍一种通过分别考虑不同的情况的来证明定理的方法。
        该方法是基于现在要介绍的一个推理规则。
        为了证明如下的条件语句
        $$(p_1 \vee p_2 \vee \cdots \vee p_n) \rightarrow q$$
        可以用永真式
        $$((p_1 \vee p_2 \vee \cdots \vee p_n) \rightarrow q) \leftrightarrow ((p_1 \rightarrow q) \wedge (p_2 \rightarrow q) \wedge \cdots \wedge (p_n \rightarrow q))$$
        作为推理规则。
        这个推理规则说明可以通过分别证明每个条件语句 $p_i \rightarrow q(i = 1, 2, \cdots , n)$ 来证明由命题 $p_1, p_2, \cdots , p_n$ 的析取式组成前提的原条件语句。
        这种论证称为\emreg{分情形证明法(proof by cases)}。

        \paraph{穷举证明法}
        {
            有些定理可以通过检验相对少量的例子来证明。
            这样的证明叫做\emreg{穷举证明法(exhaustive proof, proof by exhaustion)},因为这些证明是要穷尽所有可能性的。
            一个穷举证明法是分情形证明的特例,这里每一种情形检验一个例子。

            注意当不可能列出所有要检查的情形时,即使是计算机也不能检查所有情形。
        }

        \paraph{分情形证明法}
        {
            分情形证明一定要覆盖定理中出现的所有可能的情况。
        }

        \paraph{充分利用分情形证明法}
        {
            当一个证明不可能同时顾及所有情形时,应该考虑采用分情形证明法。
            一般地,当没有明显的思路开始一个证明,而每一种情形的额外信息又能推进证明时,可以寻求分情形证明法。

            在分情形证明中,有时我们能消除几乎全部而只留下少量情形。
        }

        \paraph{不失一般性}
        {
            一般地,当证明中用到\emreg{不失一般性(without loss of generality, WLOG)}一词时,我们断言通过证明定理的一种情形,不需要额外的论证来证明其他特定的情形。
            也就是说,其他的一系列情形论证可以通过对论证做一些简单的改变,或者通过补充一些简单的初始步骤来完成。
            当然,不正确地应用这个原理会导致不幸的错误发生,有时候所做的会导致失去一般性。
            这类假设通常是由于忽略了一个情形可能与其他情形有着巨大的差异。
            这样会导致一个不完整的或许不可补救的证明。
        }

        \paraph{穷举证明法和分情形证明法中的常见错误}
        {
            推理中的一种常见错误是从个例中得出不正确的结论。
            不管考虑了多少不同的个例,都不能从个例来证明定理,除非每一种可能情况都覆盖了。
        }
    }

    \subsection{存在性证明}
    {
        许多定理是断言特定类型对象的存在性。
        这种类型的定理是形如 $\exists x P(x)$ 的命题,其中 $P$ 是谓词。
        $\exists x P(x)$ 这类命题的证明称为\emreg{存在性证明(existence proof)}。
        通过找出一个使得 $P(a)$ 为真的元素 $a$ (称为一个物证)来给出 $\exists x P(x)$ 的存在性证明。
        这样的存在性证明称为是\emreg{构造性的(constructive)}。
        也可以给出一种\emreg{非构造性的}存在性证明,即不是找出使 $P(a)$ 为真的元素 $a$ ,而是以某种其他方式来证明 $\exists x P(x)$ 为真。
        给出非构造性证明的常用方法是使用归谬证明,证明该存在量化式的否定蕴含一个矛盾。
    }

    \subsection{唯一性证明}
    {
        某些定理断言具有特定性质的元素唯一存在。
        这些定理断言恰好只有一个元素具有这个性质。
        要证明这类语句,需要证明存在一个具有此性质的元素,以及没有其他元素具有此性质。
        \emreg{唯一性证明(uniqueness proof)}的两个部分如下:

        \begin{description}
            \item[存在性:] 证明存在某个元素 $x$ 具有期望的性质。
            \item[唯一性:] 证明如果 $y \neq x$ ,则 $y$ 不具有期望的性质。
        \end{description}

        我们也可以等价地证明如果 $x$ 和 $y$ 都具有期望的性质,则 $x = y$ 。

        \begin{defines}
            证明存在唯一元素 $x$ 使得 $P(x)$ 为真等同于证明语句 $\exists x (P(x) \wedge \forall y (y \neq x \rightarrow \neg P(y)))$ 。
        \end{defines}
    }

    \subsection{证明策略}
    {
        寻找证明是一项富于挑战性的工作。
        当你面对待证命题时,应该先把术语替换成其定义,再仔细分析前提结论的含义。
        这样做之后,用一种可用的证明方法去尝试证明结论。
        一般情况下,如果语句是条件语句,就应该首先尝试直接证明法;
        如果这样不行,就尝试间接证明法。
        如果这些方法都不行,就尝试归谬证明法。

        \paraph{正向和反向推理}
        {
            无论选择什么证明方法,都需要为证明找一个起点。
            条件语句的直接证明就从前提开始。
            利用这些前提以及公理和已知定理,用导向结论的一系列步骤来构造证明。
            这类推理称为\emreg{正向推理(forward reasoning)},是用来证明相对简单结论的一类最常见推理方式。
            同样,要开始间接证明,就从结论的否定开始,用一系列步骤来得出前提的否定。

            但是,正向推理常常难以用来证明更复杂的结论,因为得出想要的结论所需要的推理可能并不明显。
            在这种情况下使用\emreg{反向推理(backward reasoning)}可能会有所帮助。
            要反向推理命题 $q$ ,我们就寻找一个命题 $p$ 并可证明其具有性质 $p \rightarrow q$ 。
            寻找一个命题 $r$ 并能证明 $q \rightarrow r$ 不会有所帮助,因为从 $q \rightarrow r$ 和 $r$ 得出 $q$ 为真是一种肯定结论的谬误。
        }

        \paraph{改变现有证明}
        {
            在寻找可用于证明语句方法时,一个很好的思路是利用类似结论现有的证明。
            一个现有的证明通常可以改编用于证明其他结论。
            即使不是这样,现有证明中的一些想法也会有所帮助。
        }
    }

    \subsection{寻找反例}
    {
        当面对一个猜想时,首先可以试图去证明这个猜想,如果你的尝试没有成功,你可以试图寻找一个反例。
        如果你不能找到反例,可以再试图证明这个语句。
    }

    \subsection{证明策略实践}
    {
        以探索概念和例子开始,提出问题,形成猜想,并企图通过证明或者通过反例来解决这些猜想。
        这些就是数学家的日常活动。

        人们基于各种可能证据来拟定猜想。
        对特殊情形的考察可能够导致一个猜想,就像识别一些可能的模式。
        对已知定理的假设和结论稍作改变也能导致可信的猜想。
        有些时候,猜想的建立是基于直觉或者甚至认为结果成立的信念。
        无论猜想是怎样产生的,一旦它被形式化描述,目标就是证明或者驳斥它。
    }

    \subsection{拼接}
    {
        通过对棋盘拼接游戏的简要研究研究能够解释证明策略的各个方面。
        研究棋盘的拼接游戏是一种能快速发现多种结论并用各种证明方法来构造其证明的很有效方法。
        在这个领域几乎创造了无穷多的猜想及其研究。

        一个\emreg{棋盘}是一个由水平和垂直分割成同样大小方格组成的矩形。
        8行和8列的棋盘称为\emreg{标准棋盘(standard checkerboard)}。
        在这一节我们用术语\emreg{拼板(board)}指任意大小的矩形棋盘,以及删除一个或多个方格剩下的棋盘的组成。
        一个\emreg{骨牌(domino)}是一块一乘二的方格组成的矩形。
        当一个拼板的所有方格由不重叠的骨牌覆盖并且没有骨牌悬空时,我们就说一个拼板由骨牌所\emreg{拼接(tiled)}。

        用同样的方格沿边粘连起来构成的相同形状的板块而非骨牌来做拼接游戏。
        这样的板块称为是\emreg{多联骨牌(polyomino)}。
        一种三联骨牌是\emreg{直三联骨牌(straight triomino)},另一种是\emreg{直角三联骨牌(right triomino)}。
    }

    \subsection{开放问题的作用}
    {
        数学中的许多进展是人们在试图解决著名的悬而未决的问题时而做出的。
    }

    \subsection{其他证明方法}
    {
        本章介绍了证明中使用的基本方法。
        同时描述了如何利用这些方法来证明各种结论。
        后续章节中将会用到这些证明方法。
    }

    \subsection{练习}
    {
        %1.
        \begin{practices}
            \begin{enumerate}[i.]
                \item 当 $n = 1$ , $1^2 + 1 \geq 2^1$
                \item 当 $n = 2$ , $2^2 + 1 \geq 2^2$
                \item 当 $n = 3$ , $3^2 + 1 \geq 2^3$
                \item 当 $n = 4$ , $4^2 + 1 \geq 2^4$
            \end{enumerate}

            故当 $n$ 是 $1 \leq n \leq 4$ 的正整数时,有 $n^2 + 1 \geq 2^n$ 。
            证毕。
        \end{practices}

        %2.
        \begin{practices}
            \begin{enumerate}[i.]
                \item 当 $n = 1$ ,不存在。
                \item 当 $n = 8$ ,不存在。
                \item 当 $n = 27$ ,不存在。
                \item 当 $n = 64$ ,不存在。
                \item 当 $n = 125$ ,不存在。
                \item 当 $n = 216$ ,不存在。
                \item 当 $n = 343$ ,不存在。
                \item 当 $n = 512$ ,不存在。
                \item 当 $n = 729$ ,不存在。
                \item 当 $n = 1000$ ,不存在。
            \end{enumerate}
        \end{practices}

        %3.
        \begin{practices}
            \begin{enumerate}[i.]
                \item 当 $x \geq y$ 时, $max(x, y) + min(x, y) \equiv x + y$ 。
                \item 当 $x < y$ 时, $max(x, y) + min(x, y) \equiv x + y$ 。
            \end{enumerate}
        \end{practices}

        %4.
        \begin{practices}
            \begin{enumerate}[i.]
                \item 当 $a$ 是三个数中最小的数, $min(a, min(b, c)) = a = min(min(a, b), c)$ 。
                \item 当 $b$ 是三个数中最小的数, $min(a, min(b, c)) = b = min(min(a, b), c)$ 。
                \item 当 $c$ 是三个数中最小的数, $min(a, min(b, c)) = c = min(min(a, b), c)$ 。
            \end{enumerate}
        \end{practices}

        %5.
        \begin{practices}
            \begin{enumerate}[i.]
                \item 当 $x = y$ 时, $min(x, y) = x = (2x / 2) = (x + y - 0) / 2$ , $max(x, y) = x = (2x / 2) = (x + y - 0) / 2$ 。
                \item 不失一般性,当 $x > y$ 时, $min(x, y) = y = (x + y - x + y) / 2$ , $max(x, y) = x = (x + y - y + 2) / 2$ 。
            \end{enumerate}
        \end{practices}

        %6.
        \begin{practices}
            不失一般性,当 $x$ 为偶数 $y$ 为奇数时,$x = 2k, y = 2l + 1$ 。
            $5x + 5y = 10k + 10l +5 = 2(5k + 5l + 2) + 1$ ,故为奇数。
        \end{practices}

        %7.
        \begin{practices}
            \begin{enumerate}[i.]
                \item
                {
                    当 $x$ 和 $y$ 都大于 $0$ 时,

                    \begin{align*}
                        |x| + |y| &\geq |x + y| \\
                        x + y &\geq x + y
                    \end{align*}
                }
                \item
                {
                    当 $x$ 和 $y$ 都小于 $0$ 时,

                    \begin{align*}
                        |x| + |y| &\geq |x + y| \\
                        -x - y &\geq -x - y
                    \end{align*}
                }
                \item
                {
                    不失一般性,当 $x$ 大于 $0$ , $y$ 小于 $0$ 时。

                    \begin{align*}
                        |x| + |y| &\geq |x + y| \\
                        x - y &\geq -x - y \\
                        \text{或} x - y &\geq x + y \\
                        2x &\geq 2y \\
                        \text{或} 0 &\geq 2y
                    \end{align*}
                }

                证毕。
            \end{enumerate}
        \end{practices}

        %8.
        \begin{practices}
            1。
            构造性的。
        \end{practices}

        %9.
        \begin{practices}
            $100^2 = 10000, 101^2 = 10201$ ,这中间超过100个连续的正整数不是完全平方数。
            构造性的。
        \end{practices}

        %10.
        \begin{practices}
            令 $n = 2 \times 10^{500} + 15, n + 1 = 2 \times 10^{500} + 16$ ,若 $n$ 和 $n + 1$ 都为完全平方数,则 $n = 0, n + 1 = 1$ 。
            $n$ 显然大于 $0$ 。
            故得出矛盾,则两数中有一个必然不是完全平方数。
            非构造性的。
        \end{practices}

        %11.
        \begin{practices}
            $0$ 和 $1$ 。
        \end{practices}

        %12.
        \begin{practices}
            至少有两个数字符号相同,所以存在两个数字乘积为正。
        \end{practices}

        %13.
        \begin{practices}
            令 $x = 2, y = \sqrt{2}$ ,若 $2^{\sqrt{2}}$ 为无理数,则得证。
            若为有理数,则令 $x = 2^{\sqrt{2}}, y = \sqrt{2} / 4$ ,则 $(2^{\sqrt{2}})^{\sqrt{2} / 4} = 2^{\sqrt{2} \times \sqrt{2} / 4} = 2^{2 / 4} = 2^{1 / 2} = \sqrt{2}$ 。
        \end{practices}

        %14.
        \begin{practices}
            令 $a = 2, b = 1 / 2$ ,则 $2^{1 / 2} = \sqrt{2}$ 为无理数。
        \end{practices}

        %15.
        \begin{practices}
            \begin{enumerate}[A.]
                \item 当 $y$ 使得 $P(y)$ 为真当且仅当 $x = y$ 。
                \item $x$ 使得 $P(x)$ 为真,并且如果 $x$ 和 $y$ 同时使得 $P(x)$ 和 $P(y)$ 为真,那么 $x = y$ 。
                \item 存在 $x$ 使得 $P(x)$ 为真,并且如果 $y$ 使得 $P(y)$ 为真那么 $x = y$ 。
            \end{enumerate}
        \end{practices}

        %16.
        \begin{practices}
            如果 $ax + b = c$ , 那么 $x = (c - b) / a$ 。
            如果有 $r$ 为方程解,那么 $ar + b = c$ ,得出 $r = (c - b) / a = x$ 。
        \end{practices}

        %17.
        \begin{practices}
            \begin{enumerate}[i.]
                \item
                {
                    如果 $a > c, b > c$ ,那么 $a - c = b - c$ ,得出 $a = b$ ,得出矛盾。
                }
                \item
                {
                    如果 $a < c, b < c$ ,那么 $c - a = c - b$ ,得出 $a = b$ ,得出矛盾。
                }
                \item
                {
                    不失一般性,假设 $a < c, b > c$ ,那么 $c - a = b - c$ ,得出 $c = (a + b) / 2$ 。
                    如果有 $r$ 满足该式,则 $r - a = b - r$ ,得出 $r = (a + b) / 2 = c$ 。
                }
            \end{enumerate}
        \end{practices}

        %18.
        \begin{practices}
            如果 $n > r$ ,那么 $|n - r| < 1 / 2$ ,有 $n < r + 1 / 2$ 。
            如果 $r > n$ ,那么 $|r - n| < 1 / 2$ ,有 $n > r - 1 / 2$ 。

            故 $r - 1 / 2 < n < r + 1 / 2$ 。
            如果有 $m$ 满足条件,则 $m$ 和 $n$ 的距离不超过 $r + 1 / 2 - (r - 1 / 2) = 1$ ,且都为整数,故 $m = n$ 。
        \end{practices}

        %19.
        \begin{practices}
            如果 $n$ 是奇数,那么 $n = 2k + 1 = k - 2 + k + 3$ , $k = (n - 1) / 2$ 。
            如果有 $l$ 满足条件,则 $n = l - 2 + l + 3$ , $l = (n - 1) / 2 = k$ 。
        \end{practices}

        %20.
        \begin{practices}
            令 $n = x - \epsilon$ ,则 $n$ 是 $x$ 向下取整,此时 $n$ 和 $\epsilon$ 都是唯一。
        \end{practices}

        %21.
        \begin{practices}
            令 $n = x + \epsilon$ ,则 $n$ 是 $x$ 向上取整,此时 $n$ 和 $\epsilon$ 都是唯一。
        \end{practices}

        %22.
        \begin{practices}
            \begin{align*}
                (x - 1 / x)^2 &\geq 0 \\
                x^2 - 2 + 1 / x^2 &\geq 0 \\
                x^2 + 1 / x^2 \geq 2
            \end{align*}
        \end{practices}

        %23.
        \begin{practices}
            \begin{align*}
                2xy / (x + y) &< \sqrt{xy} \\
                (2xy / (x + y)) \times ((x + y) / 2\sqrt{xy}) &< \sqrt{xy} \times ((x + y) / 2\sqrt{xy}) \\
                \sqrt{xy} &< (x + y) / 2
            \end{align*}
        \end{practices}

        %24.
        \begin{practices}
            \begin{align*}
                \sqrt{\frac{x^2 + y^2}{2}} &\geq \frac{x + y}{2} \\
                \frac{x^2 + y^2}{2} &\geq \frac{(x + y)^2}{4} \\
                2x^2 + 2y^2 &\geq x^2 + 2xy + y^2 \\
                x^2 - 2xy + y^2 &\geq 0 \\
                (x - y)^2 &\geq 0
            \end{align*}
        \end{practices}

        %25.
        \begin{practices}
            每次总和增加 $|j - k|$ ,同时减少 $j + k$ 。
            保持奇偶性不变,则最后一个数必定是奇数。
        \end{practices}

        %26.
        \begin{practices}
            如果要得到9个0,则上一步需要时9个0或者9个1。
            而之前每一步中,每对相邻的位必然不同。
            故得出矛盾。
        \end{practices}

        %27.
        \begin{practices}
            由例5可知,2次幂的末尾只有 $0, 1, 4, 5, 6, 9$ 。
            故 $x^4 = (x^2)^2$ 。
            所以4次幂的末尾只有 $1, 6, 5, 0$ 。
        \end{practices}

        %28.
        \begin{practices}
            末尾两位数字都由 $00, 01, 04, 09, 16, 25, 36, 49, 64, 81, 21, 44, 69, 96, 56, 89, 24, 61, 41, 84, 29, 76$ 组成。
            令 $n = 100a + b, n^2 = (100a + b)^2 = 10000a^2 + 200ab + b^2 = 100(100a^2 + 2ab) + b^2$ ,故可认为 $b^2$ 和 $n^2$ 末尾两数字相同。
            分情形证明即可。
        \end{practices}

        %29.
        \begin{practices}
            \begin{enumerate}[i.]
                \item $n = 1$ , $1 + 1 \neq 100$ 。
                \item $n = 2$ , $4 + 8 \neq 100$ 。
                \item $n = 3$ , $9 + 27 \neq 100$ 。
                \item $n = 4$ , $16 + 64 \neq 100$ 。
                \item $n \geq 5$ 时, $n^2 + n^3 \geq 100$ 。
            \end{enumerate}

            故不存在整数解。
        \end{practices}

        %30.
        \begin{practices}
            $x$ 为 $-2, -1, 0, 1, 2$ , $y$ 为 $-1, 0, 1$ 。
            都不满足。
        \end{practices}

        %31.
        \begin{practices}
            $x = 5, y = 0$ 时有解。
        \end{practices}

        %32.
        \begin{practices}
            令 $x = m^2 - n^2, y = 2mn, z = m^2 + n^2$ ,则 $x^2 + y^2 = (m^2 - n^2)^2 + (2mn)^2 = m^4 - 2m^2n^2 + n^4 + 4m^2n^2 = (m^2 + n^2)^2 = z^2$ ,故由无数个解。
        \end{practices}

        %33.
        \begin{practices}
            如果 $a > \sqrt[3]{n}, b > \sqrt[3]{n}, c > \sqrt[3]{n}$ ,则 $abc > (\sqrt[3]{n})^3 = n$ ,得出矛盾。
        \end{practices}

        %34.
        \begin{practices}
            令 $\sqrt[3]{2} = a / b$ ,其中 $a$ 和 $b$ 不可约分,则 $2 = a^3 / b^3$ , $a^3 = 2b^3$ ,由于奇偶性不变,此时 $a$ 为偶数。
            假设为 $a = 2s$ ,则 $8s^3 = 2b^3$ , $b$ 也为偶数。
            得出矛盾。
        \end{practices}

        %35.
        \begin{practices}
            找到一个特定的公分母,令给定的有理数为 $a / b, c / b$ ,其中 $b$ 为正整数, $a, c$ 为整数, $a < c$ ,同时有 $(a + 1) / b \leq c / b$ 。
            又有 $x = (a + \sqrt{2} / 2) / b$ 在两数之间,且为无理数。
        \end{practices}

        %36.
        \begin{practices}
            令 $a$ 为有理数, $b$ 为无理数,则 $(a + b) / 2$ 为无理数且在两数之间。
        \end{practices}

        %37.
        \begin{practices}
            如果执行交换,则在总和上加上了 $x_iy_j + x_jy_i$ ,同时减去了 $x_iy_i + x_jy_j$ ,净效果是加上了 $x_iy_j + x_jy_i - x_iy_i - x_jy_j = (x_j - x_i)(y_i - y_j)$ ,由排序规则可求出值。
        \end{practices}

        %38.
        \begin{practices}
            \begin{enumerate}[1.]
                \item (8, 0, 0)
                \item (3, 2, 3)
                \item (6, 2, 0)
                \item (6, 0, 2)
                \item (1, 5, 2)
                \item (1, 4, 3)
            \end{enumerate}
        \end{practices}

        %39.
        \begin{practices}
            \begin{enumerate}[A.]
                \item
                {
                    \begin{enumerate}[1.]
                        \item 6
                        \item 3
                        \item 10
                        \item 5
                        \item 16
                        \item 8
                        \item 4
                        \item 2
                        \item 1
                    \end{enumerate}
                }
                \item
                {
                    \begin{enumerate}[1.]
                        \item 7
                        \item 22
                        \item 11
                        \item 34
                        \item 17
                        \item 52
                        \item 26
                        \item 13
                        \item 40
                        \item 20
                        \item 10
                        \item 5
                        \item 16
                        \item 8
                        \item 4
                        \item 2
                        \item 1
                    \end{enumerate}
                }
                \item
                {
                    \begin{enumerate}[1.]
                        \item 17
                        \item 52
                        \item 26
                        \item 13
                        \item 40
                        \item 20
                        \item 10
                        \item 5
                        \item 16
                        \item 8
                        \item 4
                        \item 2
                        \item 1
                    \end{enumerate}
                }
                \item
                {
                    \begin{enumerate}[1.]
                        \item 21
                        \item 64
                        \item 32
                        \item 16
                        \item 8
                        \item 4
                        \item 2
                        \item 1
                    \end{enumerate}
                }
            \end{enumerate}
        \end{practices}

        %40.
        \begin{practices}
            \begin{enumerate}[A.]
                \item
                {
                    \begin{enumerate}[1.]
                        \item 16
                        \item 8
                        \item 4
                        \item 2
                        \item 1
                    \end{enumerate}
                }
                \item
                {
                    \begin{enumerate}[1.]
                        \item 11
                        \item 34
                        \item 17
                        \item 52
                        \item 26
                        \item 13
                        \item 40
                        \item 20
                        \item 10
                        \item 5
                        \item 16
                        \item 8
                        \item 4
                        \item 2
                        \item 1
                    \end{enumerate}
                }
                \item
                {
                    \begin{enumerate}[1.]
                        \item 35
                        \item 106
                        \item 53
                        \item 160
                        \item 80
                        \item 40
                        \item 20
                        \item 10
                        \item 5
                        \item 16
                        \item 8
                        \item 4
                        \item 2
                        \item 1
                    \end{enumerate}
                }
                \item
                {
                    \begin{enumerate}[1.]
                        \item 113
                        \item 340
                        \item 170
                        \item 85
                        \item 256
                        \item 128
                        \item 64
                        \item 32
                        \item 16
                        \item 8
                        \item 4
                        \item 2
                        \item 1
                    \end{enumerate}
                }
            \end{enumerate}
        \end{practices}

        %41.
        \begin{practices}
            不失一般性,假设左上角和右上角被去除了。
            则可以用三个骨牌填充最上面一行,下面七行都用水平骨牌填充。
        \end{practices}

        %42.
        \begin{practices}
            分别用三个骨牌填充最上面和最下面一行,则中间六行都可以用水平骨牌填充。
        \end{practices}

        %43.
        \begin{practices}
            因为有偶数格数,则行为偶数或列为偶数。
            前者可以用垂直骨牌拼接,后者可以用水平骨牌拼接。
        \end{practices}

        %44.
        \begin{practices}
            染色后,不失一般性,有12个白色和13个黑色格子,而四个角都是黑色格子,去掉之后剩下12个白色格子和10个黑色格子。
        \end{practices}

        %45.
        \begin{practices}
            粗略计算,最多有 $8!$ 种情况,没法用穷举法。
        \end{practices}

        %46.
        \begin{practices}
            去掉一黑一白后,一条通路就变成了两条,每条通路从不同的颜色开始,以另一种颜色结尾,是偶数长度。
        \end{practices}

        %47.
        \begin{practices}
            去掉一个白色角相邻的两个黑色方格,则没有骨牌可以覆盖到这个角。
        \end{practices}

        %48.
        \begin{practices}
            $(3, 3), (3, 6) , (6, 3), (6, 6)$
        \end{practices}
    }
}