%%
%% Author: Clay
%% 2019/12/5
%%

\section{证明的方法和策略}
{
    \subsection{引言}
    {
        数学家工作时,他们拟定猜测并试图证明或推翻之。
    }

    \subsection{穷举证明法和分情形证明法}
    {
        有时候采用单一的论证不能在定理的所有可能情况下都成立,故不能证明该定理。
        现在介绍一种通过分别考虑不同的情况的来证明定理的方法。
        该方法是基于现在要介绍的一个推理规则。
        为了证明如下的条件语句
        $$(p_1 \vee p_2 \vee \cdots \vee p_n) \rightarrow q$$
        可以用永真式
        $$((p_1 \vee p_2 \vee \cdots \vee p_n) \rightarrow q) \leftrightarrow ((p_1 \rightarrow q) \wedge (p_2 \rightarrow q) \wedge \cdots \wedge (p_n \rightarrow q))$$
        作为推理规则。
        这个推理规则说明可以通过分别证明每个条件语句 $p_i \rightarrow q(i = 1, 2, \cdots , n)$ 来证明由命题 $p_1, p_2, \cdots , p_n$ 的析取式组成前提的原条件语句。
        这种论证称为\emreg{分情形证明法(proof by cases)}。

        \paraph{穷举证明法}
        {
            有些定理可以通过检验相对少量的例子来证明。
            这样的证明叫做\emreg{穷举证明法(exhaustive proof, proof by exhaustion)},因为这些证明是要穷尽所有可能性的。
            一个穷举证明法是分情形证明的特例,这里每一种情形检验一个例子。

            注意当不可能列出所有要检查的情形时,即使是计算机也不能检查所有情形。
        }

        \paraph{分情形证明法}
        {
            分情形证明一定要覆盖定理中出现的所有可能的情况。
        }

        \paraph{充分利用分情形证明法}
        {
            当一个证明不可能同时顾及所有情形时,应该考虑采用分情形证明法。
            一般地,当没有明显的思路开始一个证明,而每一种情形的额外信息又能推进证明时,可以寻求分情形证明法。

            在分情形证明中,有时我们能消除几乎全部而只留下少量情形。
        }

        \paraph{不失一般性}
        {
            一般地,当证明中用到\emreg{不失一般性(without loss of generality, WLOG)}一词时,我们断言通过证明定理的一种情形,不需要额外的论证来证明其他特定的情形。
            也就是说,其他的一系列情形论证可以通过对论证做一些简单的改变,或者通过补充一些简单的初始步骤来完成。
            当然,不正确地应用这个原理会导致不幸的错误发生,有时候所做的会导致失去一般性。
        }
    }
}