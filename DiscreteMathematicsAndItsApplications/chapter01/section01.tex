%%
%% Author: Clay
%% 2019/5/15
%%

\section{命题逻辑}
{
    \subsection{引言}
    {
        逻辑规则给出数学语句的准确含义,这些规则用来区分有效和无效的数学论证。
    }

    \subsection{命题}
    {
        首先介绍逻辑的基本构件---命题。
        \emreg{命题}是一个陈述语句,它或真或假,但不能既真又假。

        用字母来表示\emreg{命题变元},它是代表命题的变量。
        如果一个命题是真命题,它的\emreg{真值}为真,用\emcode{T}表示;
        如果它是假命题,其真值为假,用\emcode{F}表示。

        涉及命题的逻辑领域称为\emreg{命题演算}或\emreg{命题逻辑}。

        称为\emreg{复合命题}的新命题是由已知命题用逻辑运算符组合而来。

        \begin{defines}
            令 $p$ 为一命题,则 $p$ 的否定记作 $\neg p$ (也可记作 $\overline{p}$ ),指\emspe{不是 $p$ 所指的情形}。

            命题 $\neg p$ 读作\emspe{非 $p$ }。
            $p$ 的否定( $\neg p$ )的真值和 $p$ 的真值相反。
        \end{defines}
 
        \begin{wraptable}{r}{.3333\textwidth{}}
            \centering

            \begin{tabular}{c|c}
                \hline
                $p$ & $\neg p$ \\
                \hline
                \emcode{T} & \emcode{F} \\
                \emcode{F} & \emcode{T} \\
                \hline
            \end{tabular}

            \caption{命题之否定的真值表}
        \end{wraptable}

        命题的否定也可以看作\emreg{否定运算符}作用在命题上的结果。
        否定运算符从一个已知命题构造出一个新命题。
        现在引入从两个或多个已知命题构造新命题的逻辑运算符。
        这些逻辑运算符也称为\emreg{联结词}。

        \begin{defines}
            令 $p$ 和 $q$ 为命题, $p$ 和 $q$ 的合取即命题\emspe{ $p$ 并且 $q$ },记作 $p \wedge q$ 。
            当 $p$ 和 $q$ 都是真时, $p \wedge q$ 命题为真,否则为假。
        \end{defines}

        在合取逻辑中,有时用到\emreg{但是}一词,而非\emreg{并且(and)}一词。
        
        \begin{defines}
            令 $p$ 和 $q$ 为命题, $p$ 和 $q$ 的析取即命题\emspe{ $p$ 或 $q$ },记作 $p \vee q$ 。
            当 $p$ 和 $q$ 均为假时,析取命题 $p \vee q$ 为假, 否则为真。
        \end{defines}

        在析取中使用的联结词\emreg{或(or)}对应于词\emreg{或}在自然语言中所使用的两种情况之一,即\emreg{兼或(inclusive or)}。
        
        \begin{defines}
            令 $p$ 和 $q$ 为命题, $p$ 和 $q$ 的\emspe{异或(exclusive or)}(记作 $p \oplus q$ )是这样一个命题,当 $p$ 和 $q$ 中恰好只有一个为真时命题为真,否则为假。
        \end{defines}

        \begin{minipage}[c]{\textwidth{}}
            \begin{minipage}[c]{.5\textwidth{}}
                \begin{table}[H]
                    \centering

                    \begin{tabular}{cc|c}
                        \hline
                        $p$ & $q$ & $p \wedge q$ \\
                        \hline
                        \emcode{T} & \emcode{T} & \emcode{T} \\
                        \emcode{T} & \emcode{F} & \emcode{F} \\
                        \emcode{F} & \emcode{T} & \emcode{F} \\
                        \emcode{F} & \emcode{F} & \emcode{F} \\
                        \hline
                    \end{tabular}

                    \caption{两命题合取的真值表}
                \end{table}
            \end{minipage}%
            \begin{minipage}[c]{.5\textwidth{}}
                \begin{table}[H]
                    \centering

                    \begin{tabular}{cc|c}
                        \hline
                        $p$ & $q$ & $p \vee q$ \\
                        \hline
                        \emcode{T} & \emcode{T} & \emcode{T} \\
                        \emcode{T} & \emcode{F} & \emcode{T} \\
                        \emcode{F} & \emcode{T} & \emcode{T} \\
                        \emcode{F} & \emcode{F} & \emcode{F} \\
                        \hline
                    \end{tabular}

                    \caption{两命题析取的真值表}
                \end{table}
            \end{minipage}%
        \end{minipage}
    }

    \subsection{条件语句}
    {
        \begin{defines}
            令 $p$ 和 $q$ 为命题。
            条件语句 $p \rightarrow q$ 是命题\emspe{如果 $p$ ,则 $q$ }。
            当 $p$ 为真而 $q$ 为假时,条件语句 $p \rightarrow q$ 为假,否则为真。
            在条件语句 $p \rightarrow q$ 中, $p$ 称为假设(前件、前提), $q$ 称为结论(后件)。
        \end{defines}

        语句 $p \rightarrow q$ 称为条件语句,因为 $p \rightarrow q$ 可以断定在条件 $p$ 成立的时候 $q$ 为真。
        条件语句也称为\emreg{蕴含}。

        \begin{minienv}
            \begin{minipage}[c]{.5\textwidth{}}
                \begin{table}[H]
                    \centering

                    \begin{tabular}{cc|c}
                        \hline
                        $p$ & $q$ & $p \oplus q$ \\
                        \hline
                        \emcode{T} & \emcode{T} & \emcode{F} \\
                        \emcode{T} & \emcode{F} & \emcode{T} \\
                        \emcode{F} & \emcode{T} & \emcode{T} \\
                        \emcode{F} & \emcode{F} & \emcode{F} \\
                        \hline
                    \end{tabular}

                    \caption{两命题异或的真值表}
                \end{table}
            \end{minipage}%
            \begin{minipage}[c]{.5\textwidth{}}
                \begin{table}[H]
                    \centering

                    \begin{tabular}{cc|c}
                        \hline
                        $p$ & $q$ & $p \rightarrow q$ \\
                        \hline
                        \emcode{T} & \emcode{T} & \emcode{T} \\
                        \emcode{T} & \emcode{F} & \emcode{F} \\
                        \emcode{F} & \emcode{T} & \emcode{T} \\
                        \emcode{F} & \emcode{F} & \emcode{T} \\
                        \hline
                    \end{tabular}

                    \caption{条件命题 $p \rightarrow q$ 的真值表}
                \end{table}
            \end{minipage}%
        \end{minienv}

        \paraph{逆命题、逆否命题与反命题}
        {
            命题 $q \rightarrow p$ 称为 $p \rightarrow q$ 的逆命题,而 $p \rightarrow q$ 的逆否命题是 $\neg q \rightarrow \neg p$ 。
            命题 $\neg p \rightarrow \neg q$ 称为 $p \rightarrow q$ 的反命题。
            只有逆否命题总是和原命题具有相同的真值。

            当两个复合命题总是具有相同真值时,称它们是\emreg{等价的}
            一个条件语句与它的逆否命题是等价的。
            条件语句的逆与反也是等价的。
        }
                
        \begin{wraptable}{r}{.3333\textwidth{}}
            \centering

            \begin{tabular}{cc|c}
                \hline
                $p$ & $q$ & $p \leftrightarrow q$ \\
                \hline
                \emcode{T} & \emcode{T} & \emcode{T} \\
                \emcode{T} & \emcode{F} & \emcode{F} \\
                \emcode{F} & \emcode{T} & \emcode{F} \\
                \emcode{F} & \emcode{F} & \emcode{T} \\
                \hline
            \end{tabular}

            \caption{双条件语句 $p \leftrightarrow q$ 的真值表}
        \end{wraptable}

        \begin{defines}
            令 $p$ 和 $q$ 为命题。
            双条件语句 $p \leftrightarrow q$ 是命题\emspe{ $p$ 当且仅当 $q$}。
            当 $p$ 和 $q$ 具有同样的真值时,双条件语句为真,否则为假。
            双条件语句也称为双向蕴含。
        \end{defines}

        双条件语句可以用缩写符号\emcode{iff}代替\emreg{当且仅当(if and only if)}。
        $p \leftrightarrow q$ 与 $p \rightarrow q \wedge q \rightarrow p$ 有完全相同的真值。

        \paraph{双条件的隐式使用}
        {
            在自然语言中双条件并不总是显式地使用。
            特别是在自然语言中很少使用双条件中的\emreg{当且仅当}结构。
            通常用\emreg{如果,那么}或\emreg{仅当}结构来表示双蕴含。
            \emreg{当且仅当}的另一部分则是隐含的。
            也就是逆命题是蕴含的而没有明说出来。
        }
    }

    \subsection{复合命题的真值表}
    {
        可以用联结词来构造含有一些命题变元的结构复杂的复合命题。
        可以用真值表来决定这些复合命题的真值。
    }

    \subsection{逻辑运算符的优先级}
    {
        通常使用括号来规定复合命题中的逻辑运算符的作用顺序。
        为了减少括号的数量,规定否定运算符优先于其他所有逻辑运算符。

        另一个常用的优先级规则是合取运算符优先于析取运算符。

        最后,一个已被接受的规则是条件运算符和双条件运算符的优先级低于合取和析取运算符。

        \begin{table}[htb]
            \centering

            \begin{tabular}{c|c}
                \hline
                运算符 & 优先级 \\
                \hline
                $\neg$ & 1 \\
                \hline
                $\wedge$ & 2 \\
                $\vee$ & 3 \\
                \hline
                $\rightarrow$ & 4 \\
                $\leftrightarrow$ & 5 \\
                \hline
            \end{tabular}

            \caption{逻辑运算符的优先级}
        \end{table}
    }

    \subsection{逻辑运算和位运算}
    {
        \begin{wraptable}{r}{.3333\textwidth{}}
            \centering

            \begin{tabular}{c|c}
                \hline
                真值 & 位 \\
                \hline
                \emspe{T} & $1$ \\
                \emspe{F} & $0$ \\
                \hline
            \end{tabular}
        \end{wraptable}

        位一词的含义来自\emreg{二进制数字(binary digit)}。
        如果一个变量的值或为真或为假,则此变量就称为\emreg{布尔变量}。

        计算机的\emreg{位运算}对应于逻辑联结词。
        
        \begin{table}[htb]
            \centering

            \[
                \begin{array}{c|c|c|c|c}
                    \hline
                    x & y & x \vee y & x \wedge y & x \oplus y \\
                    \hline
                    0 & 0 & 0 & 0 & 0 \\
                    0 & 1 & 1 & 0 & 1 \\
                    1 & 0 & 1 & 0 & 1 \\
                    1 & 1 & 1 & 1 & 0 \\
                    \hline
                \end{array}
            \]

            \caption{位运算符的真值表}
        \end{table}
    }

    \subsection{练习}
    {
        %1.
        \begin{practices}
            \begin{enumerate}[A.]
                \item 是命题;真。
                \item 是命题;假。
                \item 是命题;真。
                \item 是命题;假。
                \item 不是命题。
                \item 不是命题。
            \end{enumerate}
        \end{practices}

        %2.
        \begin{practices}
            \begin{enumerate}[A.]
                \item 不是命题。
                \item 不是命题。
                \item 是命题;假。
                \item 不是命题。
                \item 是命题;假。
                \item 不是命题。
            \end{enumerate}
        \end{practices}

        %3.
        \begin{practices}
            \begin{enumerate}[A.]
                \item Mei没有MP3播放器。
                \item 新泽西被污染了。
                \item $2 + 1 \neq 3$ 。
                \item 缅因州的夏天并不是又热又晒。
            \end{enumerate}
        \end{practices}

        %4.
        \begin{practices}
            \begin{enumerate}[A.]
                \item Jennifer和Teja不是朋友。
                \item 面包师没有说一打有13个。
                \item Abby没有每天发送100多条文本信息。
                \item 121不是一个完全平方数。
            \end{enumerate}
        \end{practices}

        %5.
        \begin{practices}
            \begin{enumerate}[A.]
                \item Steve的笔记本有小于等于100GB的空闲磁盘空间。
                \item Zach没有阻止来自Jennifer的邮件或短信。
                \item $7 \times 11 \times 13 \neq 999$ 。
                \item Diane周日没有骑自行车骑了100英里。
            \end{enumerate}
        \end{practices}

        %6.
        \begin{practices}
            \begin{enumerate}[A.]
                \item 真。
                \item 真。
                \item 假。
                \item 假。
                \item 假。
            \end{enumerate}
        \end{practices}

        %7.
        \begin{practices}
            \begin{enumerate}[A.]
                \item 假。
                \item 真。
                \item 真。
                \item 真。
                \item 真。
            \end{enumerate}
        \end{practices}

        %8.
        \begin{practices}
            \begin{enumerate}[A.]
                \item 本周我没有买彩票。
                \item 本周我买了一张彩票或者我赢得了百万美元大奖。
                \item 如果本周我买了一张彩票,那么我就赢得了百万美元大奖。
                \item 本周我买了一张彩票并且赢得了百万美元大奖。
                \item 本周我买了一张彩票当且仅当我赢了百万美元大奖。
                \item 如果我本周没有买彩票,那么我没有赢得百万美元大奖。
                \item 我本周没有买彩票并且也没有赢得百万美元大奖。
                \item 我本周没有买彩票或者我本周买了彩票并且中了百万美元大奖。
            \end{enumerate}
        \end{practices}

        %9.
        \begin{practices}
            \begin{enumerate}[A.]
                \item 在新泽西海岸附近没有发现过鲨鱼。
                \item 在新泽西海岸游泳是允许的并且在海岸附近发现过鲨鱼。
                \item 在新泽西海岸有用是不允许的,或者在海岸附近发现过鲨鱼。
                \item 如果在新泽西海岸游泳是允许的,那么在海岸附近没有发现过鲨鱼。
                \item 如果在海岸附近没有发现过鲨鱼,那么在新泽西海岸游泳是允许的。
                \item 如果在新泽西海岸游泳是不允许的,那么在海岸附近没有发现过鲨鱼。
                \item 在新泽西海岸游泳是允许的当且仅当在海岸附近没有发现过鲨鱼。
                \item 在新泽西海岸游泳是不允许的并且在海岸附近没有发现过鲨鱼。
            \end{enumerate}
        \end{practices}

        %10.
        \begin{practices}
            \begin{enumerate}[A.]
                \item 选举还没有结果。
                \item 选举已经有了结果或选票已经计数完毕。
                \item 选举还没有结果并且选票已计数完毕。
                \item 如果选举已经有了结果,那么选票已经计数完毕。
                \item 如果选票还没有计数完毕,那么选举没有结果。
                \item 如果选举还没有结果,那么选票还没有计数完毕。
                \item 选举已经有了结果当且仅当选票已经计数完毕。
                \item 选票还没有计数完毕,或者,选举还没有结果并且选票已经计数完毕。
            \end{enumerate}
        \end{practices}

        %11.
        \begin{practices}
            \begin{enumerate}[A.]
                \item $p \wedge q$
                \item $p \wedge \neg q$
                \item $\neg p \wedge \neg q$
                \item $p \vee q$
                \item $p \rightarrow q$
                \item $\neg(p \vee q)$
                \item $p \leftrightarrow q$
            \end{enumerate}
        \end{practices}

        %12.
        \begin{practices}
            \begin{enumerate}[A.]
                \item 如果你得流感了,那么你将错过期终考试。
                \item 这门课你及格了当且仅当你没有错过期终考试。
                \item 如果你错过了期终考试,那么这门课你不会及格。
                \item 你得流感了或者你错过了期终考试或者这门课你及格了。
                \item 如果你得了流感或者你错过了期终考试,那么这门课你不及格。
                \item 你得流感了并且你错过了期末考试,或者你没有错过期末考试并且这门课你及格了。
            \end{enumerate}
        \end{practices}

        %13.
        \begin{practices}
            \begin{enumerate}[A.]
                \item $\neg p$
                \item $p \wedge \neg q$
                \item $p \rightarrow q$
                \item $\neg p \rightarrow \neg q$
                \item $p \rightarrow q$
                \item $\neg p \wedge q$
                \item $q \rightarrow p$
            \end{enumerate}
        \end{practices}

        %14.
        \begin{practices}
            \begin{enumerate}[A.]
                \item $r \wedge \neg q$
                \item $p \wedge q \wedge r$
                \item $r \rightarrow p$
                \item $p \wedge \neg q \wedge r$
                \item $(p \wedge q) \rightarrow r$
                \item $(p \vee q) \leftrightarrow r$
            \end{enumerate}
        \end{practices}

        %15.
        \begin{practices}
            \begin{enumerate}[A.]
                \item $\neg p \wedge r$
                \item $\neg p \wedge q \wedge r$
                \item $r \rightarrow (q \leftrightarrow \neg p)$
                \item $\neg q \wedge \neg p \wedge r$
                \item $(q \rightarrow (\neg r \wedge \neg p)) \wedge \neg ((\neg r \wedge \neg p) \rightarrow q)$
                \item $(p \wedge r) \rightarrow \neg q$
            \end{enumerate}
        \end{practices}

        %16.
        \begin{practices}
            \begin{enumerate}[A.]
                \item 真
                \item 假
                \item 真
                \item 假
            \end{enumerate}
        \end{practices}

        %17.
        \begin{practices}
            \begin{enumerate}[A.]
                \item 假
                \item 真
                \item 真
                \item 真
            \end{enumerate}
        \end{practices}

        %18.
        \begin{practices}
            \begin{enumerate}[A.]
                \item 真
                \item 真
                \item 假
                \item 真
            \end{enumerate}
        \end{practices}

        %19.
        \begin{practices}
            \begin{enumerate}[A.]
                \item 异或
                \item 兼或
                \item 兼或
                \item 异或
            \end{enumerate}
        \end{practices}

        %20.
        \begin{practices}
            \begin{enumerate}[A.]
                \item 兼或
                \item 异或
                \item 兼或
                \item 异或
            \end{enumerate}
        \end{practices}

        %21.
        \begin{practices}
            \begin{enumerate}[A.]
                \item 兼或
                \item 异或
                \item 异或
                \item 兼或
            \end{enumerate}
        \end{practices}

        %22.
        \begin{practices}
            \begin{enumerate}[A.]
                \item 如果想要晋升,那么帮老板洗车是很有必要的。
                \item 如果吹了南风,则预示着春天要来了。
                \item 如果你的计算机购买时间不超过一年,则保修单有效。
                \item 如果Willy行骗,那么就会被抓住。
                \item 如果你想访问网站,那么你必须支付订阅费。
                \item 如果你了解合适的人群,那么你会当选。
                \item 如果Carol坐船,那么他就会晕船。
            \end{enumerate}
        \end{practices}

        %23.
        \begin{practices}
            \begin{enumerate}[A.]
                \item 如果吹了东北风,那么就会下雪。
                \item 如果天暖持续一周,那么苹果树就会开花。
                \item 如果活塞队赢得了冠军,那么久意味着他们打败了湖人队。
                \item 如果要到达达朗斯峰的顶峰,那么必须走8英里。
                \item 如果能世界闻名,那么就能得到终身教授职位。
                \item 如果你驾车超过400英里,那么就需要买汽油了。
                \item 如果你的保单是有效的,那么你购买CD机不足90天。
                \item 如果水不是太凉,那么Jan要去游泳。
            \end{enumerate}
        \end{practices}

        %24.
        \begin{practices}
            \begin{enumerate}[A.]
                \item 如果我把地址发给你,那么你一定给我发了一封电子邮件。
                \item 如果你出生在美国,那么你可以成为美国公民。
                \item 如果你保存好课本,那么它会是你未来其他课程有用的参考书。
                \item 如果红翼队的守门员表现出色,那么红翼队将赢得斯坦利杯。
                \item 如果你获得这一职位,那么表明你有最好的信誉。
                \item 如果有风暴,那么沙滩会受到侵蚀。
                \item 如果你能登录到服务器,那么你有一个有效的口令。
                \item 如果你不是太晚才开始爬山,那么你能登顶。
            \end{enumerate}
        \end{practices}

        %25.
        \begin{practices}
            \begin{enumerate}[A.]
                \item 你买了冰激凌蛋卷当且仅当外边热。
                \item 你赢得竞赛当且仅当你有唯一的获胜券。
                \item 你能得到提拔当且仅当你有关系网。
                \item 你看电视当且仅当你心智衰退。
                \item 火车晚点当且仅当我乘坐的那些日子。
            \end{enumerate}
        \end{practices}

        %26.
        \begin{practices}
            \begin{enumerate}[A.]
                \item 你能在这门课得到A当且仅当你学会解离散数学问题。
                \item 你每天看报当且仅当你了解情况。
                \item 这天是周末当且仅当天在下雨。
                \item 巫师不在家当且仅当你能看到巫师。
            \end{enumerate}
        \end{practices}

        %27.
        \begin{practices}
            \begin{enumerate}[A.]
                \item
                {
                    \begin{enumerate}[1)]
                        \item 如果我明天去滑雪,那么今天会下雪。
                        \item 如果我明天不去滑雪,那么今天没有下雪。
                        \item 如果今天没有下雪,那么我明天不去滑雪。
                    \end{enumerate}
                }
                \item
                {
                    \begin{enumerate}[1)]
                        \item 如果我来上课,那么有测验。
                        \item 如果我没来上课,那么没有测验。
                        \item 如果没有测验,那么我没来上课。
                    \end{enumerate}
                }
                \item
                {
                    \begin{enumerate}[1)]
                        \item 如果一个正整数是素数,那么它没有1和自身以外的因子。
                        \item 如果一个正整数不是素数,那么它有1和自身以外的因子。
                        \item 如果一个正整数有1和自身以外的因子,那么它不是素数。
                    \end{enumerate}
                }
            \end{enumerate}
        \end{practices}

        %28.
        \begin{practices}
            \begin{enumerate}[A.]
                \item
                {
                    \begin{enumerate}[1)]
                        \item 如果我今晚呆在家里,那么今晚将下雪。
                        \item 如果我今晚没有呆在家里,那么今晚将不会下雪。
                        \item 如果今晚没有下雪,那么我将不会呆在家里。
                    \end{enumerate}
                }
                \item
                {
                    \begin{enumerate}[1)]
                        \item 如果我去了海滩,那么是阳光充足的夏天。
                        \item 如果我没有去海滩,那么不是阳光充足的夏天。
                        \item 如果不是阳光充足的夏天,那么我不会去海滩。
                    \end{enumerate}
                }
                \item
                {
                    \begin{enumerate}[1)]
                        \item 如果我睡到中午,那么我工作到很晚。
                        \item 如果我没有睡到中午,那么我没有工作到很晚。
                        \item 如果我没有工作到很晚,那么我没必要睡到中午。
                    \end{enumerate}
                }
            \end{enumerate}
        \end{practices}

        %29.
        \begin{practices}
            \begin{enumerate}[A.]
                \item $2$
                \item $16$
                \item $64$
                \item $16$
            \end{enumerate}
        \end{practices}
        
        %30.
        \begin{practices}
            \begin{enumerate}[A.]
                \item $4$
                \item $8$
                \item $64$
                \item $32$
            \end{enumerate}
        \end{practices}
        
        %31.
        \begin{practices}
            \begin{enumerate}[A.]
                \item 
                {
                    \begin{table}[H]
                        \[
                            \begin{array}{c|c|c}
                                \hline
                                p & \neg p & p \wedge \neg p \\
                                \hline
                                0 & 1 & 0 \\
                                1 & 0 & 0 \\
                                \hline
                            \end{array}
                        \]
                    \end{table}
                }
                \item
                {
                    \begin{table}[H]
                        \[
                            \begin{array}{c|c|c}
                                \hline
                                p & \neg p & p \vee \neg p \\
                                \hline
                                0 & 1 & 1 \\
                                1 & 0 & 1 \\
                                \hline
                            \end{array}
                        \]
                    \end{table}
                }
                \item
                {
                    \begin{table}[H]
                        \[
                            \begin{array}{c|c|c|c|c}
                                \hline
                                p & q & \neg q & p \vee \neg q & p \vee \neg q \rightarrow q \\
                                \hline
                                0 & 0 & 1 & 1 & 0 \\
                                0 & 1 & 0 & 0 & 1 \\
                                1 & 0 & 1 & 1 & 0 \\
                                1 & 1 & 0 & 1 & 1 \\
                                \hline
                            \end{array}
                        \]
                    \end{table}
                }
                \item
                {
                    \begin{table}[H]
                        \[
                            \begin{array}{c|c|c|c|c}
                                \hline
                                p & q & p \vee q & p \wedge q & (p \vee q) \rightarrow (p \wedge q) \\
                                \hline
                                0 & 0 & 0 & 0 & 1 \\
                                0 & 1 & 1 & 0 & 0 \\
                                1 & 0 & 1 & 0 & 0 \\
                                1 & 1 & 1 & 1 & 1 \\
                                \hline
                            \end{array}
                        \]
                    \end{table}
                }
                \item
                {
                    \begin{table}[H]
                        \[
                            \begin{array}{c|c|c|c|c}
                                \hline
                                p & q & p \rightarrow q & \neg q \rightarrow \neg p & (p \rightarrow q) \leftrightarrow (\neg q \rightarrow \neg p) \\
                                \hline
                                0 & 0 & 1 & 1 & 1 \\
                                0 & 1 & 1 & 1 & 1 \\
                                1 & 0 & 0 & 0 & 1 \\
                                1 & 1 & 1 & 1 & 1 \\
                                \hline
                            \end{array}
                        \]
                    \end{table}
                }
                \item
                {
                    \begin{table}[H]
                        \[
                            \begin{array}{c|c|c|c|c}
                                \hline
                                p & q & p \rightarrow q & q \rightarrow p & (p \rightarrow q) \rightarrow (q \rightarrow p) \\
                                \hline
                                0 & 0 & 1 & 1 & 1 \\
                                0 & 1 & 1 & 0 & 0 \\
                                1 & 0 & 0 & 1 & 1 \\
                                1 & 1 & 1 & 1 & 1 \\
                                \hline
                            \end{array}
                        \]
                    \end{table}
                }
            \end{enumerate}
        \end{practices}

        %32.
        \begin{practices}
            \begin{enumerate}[A.]
                \item 
                {
                    \begin{table}[H]
                        \[
                            \begin{array}{c|c|c}
                                \hline
                                p & \neg p & p \rightarrow \neg p \\
                                \hline
                                0 & 1 & 1 \\
                                1 & 0 & 0 \\
                                \hline
                            \end{array}
                        \]
                    \end{table}
                }
                \item 
                {
                    \begin{table}[H]
                        \[
                            \begin{array}{c|c|c}
                                \hline
                                p & \neg p & p \leftrightarrow \neg p \\
                                \hline
                                0 & 1 & 0 \\
                                1 & 0 & 0 \\
                                \hline
                            \end{array}
                        \]
                    \end{table}
                }
                \item 
                {
                    \begin{table}[H]
                        \[
                            \begin{array}{c|c|c|c}
                                \hline
                                p & q & p \vee q & p \oplus (p \vee q) \\
                                \hline
                                0 & 0 & 0 & 0 \\
                                0 & 1 & 1 & 1 \\
                                1 & 0 & 1 & 0 \\
                                1 & 1 & 1 & 0 \\
                                \hline
                            \end{array}
                        \]
                    \end{table}
                }
                \item
                {
                    \begin{table}[H]
                        \[
                            \begin{array}{c|c|c|c|c}
                                \hline
                                p & q & p \wedge q & p \vee q & (p \wedge q) \rightarrow (p \vee q) \\
                                \hline
                                0 & 0 & 0 & 0 & 1 \\
                                0 & 1 & 0 & 1 & 1 \\
                                1 & 0 & 0 & 1 & 1 \\
                                1 & 1 & 1 & 1 & 1 \\
                                \hline
                            \end{array}
                        \]
                    \end{table}
                }
                \item
                {
                    \begin{table}[H]
                        \[
                            \begin{array}{c|c|c|c|c}
                                \hline
                                p & q & q \rightarrow \neg p & p \leftrightarrow q & (q \rightarrow \neg p) \leftrightarrow (p \leftrightarrow q) \\
                                \hline
                                0 & 0 & 1 & 1 & 1 \\
                                0 & 1 & 1 & 0 & 0 \\
                                1 & 0 & 1 & 0 & 0 \\
                                1 & 1 & 0 & 1 & 0 \\
                                \hline
                            \end{array}
                        \]
                    \end{table}
                }
                \item
                {
                    \begin{table}[H]
                        \[
                            \begin{array}{c|c|c|c|c}
                                \hline
                                p & q & p \leftrightarrow q & p \leftrightarrow \neg q & (p \leftrightarrow q) \oplus (p \leftrightarrow \neg q) \\
                                \hline
                                0 & 0 & 1 & 0 & 1 \\
                                0 & 1 & 0 & 1 & 1 \\
                                1 & 0 & 0 & 1 & 1 \\
                                1 & 1 & 1 & 0 & 1 \\
                                \hline
                            \end{array}
                        \]
                    \end{table}
                }
            \end{enumerate}
        \end{practices}

        %33.
        \begin{practices}
            \begin{enumerate}[A.]
                \item
                {
                    \begin{table}[H]
                        \[
                            \begin{array}{c|c|c|c|c}
                                \hline
                                p & q & p \vee q & p \oplus q & (p \vee q) \rightarrow (p \oplus q) \\
                                \hline
                                0 & 0 & 0 & 0 & 1 \\
                                0 & 1 & 1 & 1 & 1 \\
                                1 & 0 & 1 & 1 & 1 \\
                                1 & 1 & 1 & 0 & 0 \\
                                \hline
                            \end{array}
                        \]
                    \end{table}
                }
                \item
                {
                    \begin{table}[H]
                        \[
                            \begin{array}{c|c|c|c|c}
                                \hline
                                p & q & p \oplus q & p \wedge q & (p \oplus q) \rightarrow (p \wedge q) \\
                                \hline
                                0 & 0 & 0 & 0 & 1 \\
                                0 & 1 & 1 & 0 & 0 \\
                                1 & 0 & 1 & 0 & 0 \\
                                1 & 1 & 0 & 1 & 1 \\
                                \hline
                            \end{array}
                        \]
                    \end{table}
                }
                \item
                {
                    \begin{table}[H]
                        \[
                            \begin{array}{c|c|c|c|c}
                                \hline
                                p & q & p \vee q & p \wedge q & (p \vee q) \oplus (p \wedge q) \\
                                \hline
                                0 & 0 & 0 & 0 & 0 \\
                                0 & 1 & 1 & 0 & 1 \\
                                1 & 0 & 1 & 0 & 1 \\
                                1 & 1 & 1 & 1 & 0 \\
                                \hline
                            \end{array}
                        \]
                    \end{table}
                }
                \item
                {
                    \begin{table}[H]
                        \[
                            \begin{array}{c|c|c|c|c}
                                \hline
                                p & q & p \leftrightarrow q & \neg p \leftrightarrow q & (p \leftrightarrow q) \oplus (\neg p \leftrightarrow q) \\
                                \hline
                                0 & 0 & 1 & 0 & 1 \\
                                0 & 1 & 0 & 1 & 1 \\
                                1 & 0 & 0 & 1 & 1 \\
                                1 & 1 & 1 & 0 & 1 \\
                                \hline
                            \end{array}
                        \]
                    \end{table}
                }
                \item
                {
                    \begin{table}[H]
                        \[
                            \begin{array}{c|c|c|c|c|c}
                                \hline
                                p & q & r & p \leftrightarrow q & \neg p \leftrightarrow \neg r & (p \leftrightarrow q) \oplus (\neg p \leftrightarrow \neg r) \\
                                \hline
                                0 & 0 & 0 & 1 & 1 & 0 \\
                                0 & 0 & 1 & 1 & 0 & 1 \\
                                0 & 1 & 0 & 0 & 1 & 1 \\
                                0 & 1 & 1 & 0 & 0 & 0 \\
                                1 & 0 & 0 & 0 & 0 & 0 \\
                                1 & 0 & 1 & 0 & 1 & 1 \\
                                1 & 1 & 0 & 1 & 0 & 1 \\
                                1 & 1 & 1 & 1 & 1 & 0 \\
                                \hline
                            \end{array}
                        \]
                    \end{table}
                }
                \item
                {
                    \begin{table}[H]
                        \[
                            \begin{array}{c|c|c|c|c}
                                \hline
                                p & q & p \oplus q & p \oplus \neg q & (p \oplus q) \rightarrow (p \oplus \neg q) \\
                                \hline
                                0 & 0 & 0 & 1 & 1 \\
                                0 & 1 & 1 & 0 & 0 \\
                                1 & 0 & 1 & 0 & 0 \\
                                1 & 1 & 0 & 1 & 1 \\
                                \hline
                            \end{array}
                        \]
                    \end{table}
                }
            \end{enumerate}
        \end{practices}

        %34.
        \begin{practices}
            \begin{enumerate}[A.]
                \item 
                {
                    \begin{table}[H]
                        \[
                            \begin{array}{c|c}
                                \hline
                                p & p \oplus p \\
                                \hline
                                0 & 0 \\
                                1 & 0 \\
                                \hline
                            \end{array}
                        \]
                    \end{table}
                }
                \item 
                {
                    \begin{table}[H]
                        \[
                            \begin{array}{c|c|c}
                                \hline
                                p & \neg p & p \oplus \neg p \\
                                \hline
                                0 & 1 & 1 \\
                                1 & 0 & 1 \\
                                \hline
                            \end{array}
                        \]
                    \end{table}
                }
                \item 
                {
                    \begin{table}[H]
                        \[
                            \begin{array}{c|c|c}
                                \hline
                                p & q & p \oplus \neg q \\
                                \hline
                                0 & 0 & 1 \\
                                0 & 1 & 0 \\
                                1 & 0 & 0 \\
                                1 & 1 & 1 \\
                                \hline
                            \end{array}
                        \]
                    \end{table}
                }
                \item 
                {
                    \begin{table}[H]
                        \[
                            \begin{array}{c|c|c}
                                \hline
                                p & q & \neg p \oplus \neg q \\
                                \hline
                                0 & 0 & 0 \\
                                0 & 1 & 1 \\
                                1 & 0 & 1 \\
                                1 & 1 & 0 \\
                                \hline
                            \end{array}
                        \]
                    \end{table}
                }
                \item
                {
                    \begin{table}[H]
                        \[
                            \begin{array}{c|c|c|c|c}
                                \hline
                                p & q & p \oplus q & p \oplus \neg q & (p \oplus q) \vee (p \oplus \neg q) \\
                                \hline
                                0 & 0 & 0 & 1 & 1 \\
                                0 & 1 & 1 & 0 & 1 \\
                                1 & 0 & 1 & 0 & 1 \\
                                1 & 1 & 0 & 1 & 1 \\
                                \hline
                            \end{array}
                        \]
                    \end{table}
                }
                \item
                {
                    \begin{table}[H]
                        \[
                            \begin{array}{c|c|c|c|c}
                                \hline
                                p & q & p \oplus q & p \oplus \neg q & (p \oplus q) \wedge (p \oplus \neg q) \\
                                \hline
                                0 & 0 & 0 & 1 & 0 \\
                                0 & 1 & 1 & 0 & 0 \\
                                1 & 0 & 1 & 0 & 0 \\
                                1 & 1 & 0 & 1 & 0 \\
                                \hline
                            \end{array}
                        \]
                    \end{table}
                }
            \end{enumerate}
        \end{practices}

        %35.
        \begin{practices}
            \begin{enumerate}[A.]
                \item 
                {
                    \begin{table}[H]
                        \[
                            \begin{array}{c|c|c}
                                \hline
                                p & q & p \rightarrow \neg q \\
                                \hline
                                0 & 0 & 1 \\
                                0 & 1 & 1 \\
                                1 & 0 & 1 \\
                                1 & 1 & 0 \\
                                \hline
                            \end{array}
                        \]
                    \end{table}
                }
                \item 
                {
                    \begin{table}[H]
                        \[
                            \begin{array}{c|c|c}
                                \hline
                                p & q & \neg p \leftrightarrow q \\
                                \hline
                                0 & 0 & 0 \\
                                0 & 1 & 1 \\
                                1 & 0 & 1 \\
                                1 & 1 & 0 \\
                                \hline
                            \end{array}
                        \]
                    \end{table}
                }
                \item
                {
                    \begin{table}[H]
                        \[
                            \begin{array}{c|c|c|c|c}
                                \hline
                                p & q & p \rightarrow q & \neg p \rightarrow q & (p \rightarrow q) \vee (\neg p \rightarrow q) \\
                                \hline
                                0 & 0 & 1 & 0 & 1 \\
                                0 & 1 & 1 & 1 & 1 \\
                                1 & 0 & 0 & 1 & 1 \\
                                1 & 1 & 1 & 1 & 1 \\
                                \hline
                            \end{array}
                        \]
                    \end{table}
                }
                \item
                {
                    \begin{table}[H]
                        \[
                            \begin{array}{c|c|c|c|c}
                                \hline
                                p & q & p \rightarrow q & \neg p \rightarrow q & (p \rightarrow q) \wedge (\neg p \rightarrow q) \\
                                \hline
                                0 & 0 & 1 & 0 & 0 \\
                                0 & 1 & 1 & 1 & 1 \\
                                1 & 0 & 0 & 1 & 0 \\
                                1 & 1 & 1 & 1 & 1 \\
                                \hline
                            \end{array}
                        \]
                    \end{table}
                }
                \item
                {
                    \begin{table}[H]
                        \[
                            \begin{array}{c|c|c|c|c}
                                \hline
                                p & q & p \leftrightarrow q & \neg p \leftrightarrow q & (p \leftrightarrow q) \vee (\neg p \leftrightarrow q) \\
                                \hline
                                0 & 0 & 1 & 0 & 1 \\
                                0 & 1 & 0 & 1 & 1 \\
                                1 & 0 & 0 & 1 & 1 \\
                                1 & 1 & 1 & 0 & 1 \\
                                \hline
                            \end{array}
                        \]
                    \end{table}
                }
                \item
                {
                    \begin{table}[H]
                        \[
                            \begin{array}{c|c|c|c|c}
                                \hline
                                p & q & \neg p \leftrightarrow \neg q & p \leftrightarrow q & (\neg p \leftrightarrow \neg q) \leftrightarrow (p \leftrightarrow q) \\
                                \hline
                                0 & 0 & 1 & 1 & 1 \\
                                0 & 1 & 0 & 0 & 1 \\
                                1 & 0 & 0 & 0 & 1 \\
                                1 & 1 & 1 & 1 & 1 \\
                                \hline
                            \end{array}
                        \]
                    \end{table}
                }
            \end{enumerate}
        \end{practices}

        %36.
        \begin{practices}
            \begin{enumerate}[A.]
                \item
                {
                    \begin{table}[H]
                        \[
                            \begin{array}{c|c|c|c|c}
                                \hline
                                p & q & r & p \vee q & (p \vee q) \vee r \\
                                \hline
                                0 & 0 & 0 & 0 & 0 \\
                                0 & 0 & 1 & 0 & 1 \\
                                0 & 1 & 0 & 1 & 1 \\
                                0 & 1 & 1 & 1 & 1 \\
                                1 & 0 & 0 & 1 & 1 \\
                                1 & 0 & 1 & 1 & 1 \\
                                1 & 1 & 0 & 1 & 1 \\
                                1 & 1 & 1 & 1 & 1 \\
                                \hline
                            \end{array}
                        \]
                    \end{table}
                }
                \item
                {
                    \begin{table}[H]
                        \[
                            \begin{array}{c|c|c|c|c}
                                \hline
                                p & q & r & p \vee q & (p \vee q) \wedge r \\
                                \hline
                                0 & 0 & 0 & 0 & 0 \\
                                0 & 0 & 1 & 0 & 0 \\
                                0 & 1 & 0 & 1 & 0 \\
                                0 & 1 & 1 & 1 & 1 \\
                                1 & 0 & 0 & 1 & 0 \\
                                1 & 0 & 1 & 1 & 1 \\
                                1 & 1 & 0 & 1 & 0 \\
                                1 & 1 & 1 & 1 & 1 \\
                                \hline
                            \end{array}
                        \]
                    \end{table}
                }
                \item
                {
                    \begin{table}[H]
                        \[
                            \begin{array}{c|c|c|c|c}
                                \hline
                                p & q & r & p \wedge q & (p \wedge q) \vee r \\
                                \hline
                                0 & 0 & 0 & 0 & 0 \\
                                0 & 0 & 1 & 0 & 1 \\
                                0 & 1 & 0 & 0 & 0 \\
                                0 & 1 & 1 & 0 & 1 \\
                                1 & 0 & 0 & 0 & 0 \\
                                1 & 0 & 1 & 0 & 1 \\
                                1 & 1 & 0 & 1 & 1 \\
                                1 & 1 & 1 & 1 & 1 \\
                                \hline
                            \end{array}
                        \]
                    \end{table}
                }
                \item
                {
                    \begin{table}[H]
                        \[
                            \begin{array}{c|c|c|c|c}
                                \hline
                                p & q & r & p \wedge q & (p \wedge q) \wedge r \\
                                \hline
                                0 & 0 & 0 & 0 & 0 \\
                                0 & 0 & 1 & 0 & 0 \\
                                0 & 1 & 0 & 0 & 0 \\
                                0 & 1 & 1 & 0 & 0 \\
                                1 & 0 & 0 & 0 & 0 \\
                                1 & 0 & 1 & 0 & 0 \\
                                1 & 1 & 0 & 1 & 0 \\
                                1 & 1 & 1 & 1 & 1 \\
                                \hline
                            \end{array}
                        \]
                    \end{table}
                }
                \item
                {
                    \begin{table}[H]
                        \[
                            \begin{array}{c|c|c|c|c}
                                \hline
                                p & q & r & p \vee q & (p \vee q) \wedge \neg r \\
                                \hline
                                0 & 0 & 0 & 0 & 0 \\
                                0 & 0 & 1 & 0 & 0 \\
                                0 & 1 & 0 & 1 & 1 \\
                                0 & 1 & 1 & 1 & 0 \\
                                1 & 0 & 0 & 1 & 1 \\
                                1 & 0 & 1 & 1 & 0 \\
                                1 & 1 & 0 & 1 & 1 \\
                                1 & 1 & 1 & 1 & 0 \\
                                \hline
                            \end{array}
                        \]
                    \end{table}
                }
                \item
                {
                    \begin{table}[H]
                        \[
                            \begin{array}{c|c|c|c|c}
                                \hline
                                p & q & r & p \wedge q & (p \wedge q) \vee \neg r \\
                                \hline
                                0 & 0 & 0 & 0 & 1 \\
                                0 & 0 & 1 & 0 & 0 \\
                                0 & 1 & 0 & 0 & 1 \\
                                0 & 1 & 1 & 0 & 0 \\
                                1 & 0 & 0 & 0 & 1 \\
                                1 & 0 & 1 & 0 & 0 \\
                                1 & 1 & 0 & 1 & 1 \\
                                1 & 1 & 1 & 1 & 1 \\
                                \hline
                            \end{array}
                        \]
                    \end{table}
                }
            \end{enumerate}
        \end{practices}

        %37.
        \begin{practices}
            \begin{enumerate}[A.]
                \item
                {
                    \begin{table}[H]
                        \[
                            \begin{array}{c|c|c|c|c}
                                \hline
                                p & q & r & \neg q \vee r & p \rightarrow (\neg q \vee r) \\
                                \hline
                                0 & 0 & 0 & 1 & 1 \\
                                0 & 0 & 1 & 1 & 1 \\
                                0 & 1 & 0 & 0 & 1 \\
                                0 & 1 & 1 & 1 & 1 \\
                                1 & 0 & 0 & 1 & 1 \\
                                1 & 0 & 1 & 1 & 1 \\
                                1 & 1 & 0 & 0 & 0 \\
                                1 & 1 & 1 & 1 & 1 \\
                                \hline
                            \end{array}
                        \]
                    \end{table}
                }
                \item
                {
                    \begin{table}[H]
                        \[
                            \begin{array}{c|c|c|c|c}
                                \hline
                                p & q & r & q \rightarrow r & \neg p \rightarrow (q \rightarrow r) \\
                                \hline
                                0 & 0 & 0 & 1 & 1 \\
                                0 & 0 & 1 & 1 & 1 \\
                                0 & 1 & 0 & 0 & 0 \\
                                0 & 1 & 1 & 1 & 1 \\
                                1 & 0 & 0 & 1 & 1 \\
                                1 & 0 & 1 & 1 & 1 \\
                                1 & 1 & 0 & 0 & 1 \\
                                1 & 1 & 1 & 1 & 1 \\
                                \hline
                            \end{array}
                        \]
                    \end{table}
                }
                \item
                {
                    \begin{table}[H]
                        \[
                            \begin{array}{c|c|c|c|c|c}
                                \hline
                                p & q & r & p \rightarrow q & \neg p \rightarrow r & (p \rightarrow q) \vee (\neg p \rightarrow r) \\
                                \hline
                                0 & 0 & 0 & 1 & 0 & 1 \\
                                0 & 0 & 1 & 1 & 1 & 1 \\
                                0 & 1 & 0 & 1 & 0 & 1 \\
                                0 & 1 & 1 & 1 & 1 & 1 \\
                                1 & 0 & 0 & 0 & 1 & 1 \\
                                1 & 0 & 1 & 0 & 1 & 1 \\
                                1 & 1 & 0 & 1 & 1 & 1 \\
                                1 & 1 & 1 & 1 & 1 & 1 \\
                                \hline
                            \end{array}
                        \]
                    \end{table}
                }
                \item
                {
                    \begin{table}[H]
                        \[
                            \begin{array}{c|c|c|c|c|c}
                                \hline
                                p & q & r & p \rightarrow q & \neg p \rightarrow r & (p \rightarrow q) \wedge (\neg p \rightarrow r) \\
                                \hline
                                0 & 0 & 0 & 1 & 0 & 0 \\
                                0 & 0 & 1 & 1 & 1 & 1 \\
                                0 & 1 & 0 & 1 & 0 & 0 \\
                                0 & 1 & 1 & 1 & 1 & 1 \\
                                1 & 0 & 0 & 0 & 1 & 0 \\
                                1 & 0 & 1 & 0 & 1 & 0 \\
                                1 & 1 & 0 & 1 & 1 & 1 \\
                                1 & 1 & 1 & 1 & 1 & 1 \\
                                \hline
                            \end{array}
                        \]
                    \end{table}
                }
                \item
                {
                    \begin{table}[H]
                        \[
                            \begin{array}{c|c|c|c|c|c}
                                \hline
                                p & q & r & p \leftrightarrow q & \neg q \leftrightarrow r & (p \leftrightarrow q) \vee (\neg q \leftrightarrow r) \\
                                \hline
                                0 & 0 & 0 & 1 & 0 & 1 \\
                                0 & 0 & 1 & 1 & 1 & 1 \\
                                0 & 1 & 0 & 0 & 1 & 1 \\
                                0 & 1 & 1 & 0 & 0 & 0 \\
                                1 & 0 & 0 & 0 & 0 & 0 \\
                                1 & 0 & 1 & 0 & 1 & 1 \\
                                1 & 1 & 0 & 1 & 1 & 1 \\
                                1 & 1 & 1 & 1 & 0 & 1 \\
                                \hline
                            \end{array}
                        \]
                    \end{table}
                }
                \item
                {
                    \begin{table}[H]
                        \[
                            \begin{array}{c|c|c|c|c|c}
                                \hline
                                p & q & r & \neg p \leftrightarrow \neg q & p \leftrightarrow r & (\neg p \leftrightarrow \neg q) \vee (p \leftrightarrow r) \\
                                \hline
                                0 & 0 & 0 & 1 & 1 & 1 \\
                                0 & 0 & 1 & 1 & 0 & 0 \\
                                0 & 1 & 0 & 0 & 0 & 1 \\
                                0 & 1 & 1 & 0 & 1 & 0 \\
                                1 & 0 & 0 & 0 & 1 & 0 \\
                                1 & 0 & 1 & 0 & 0 & 1 \\
                                1 & 1 & 0 & 1 & 0 & 0 \\
                                1 & 1 & 1 & 1 & 1 & 1 \\
                                \hline
                            \end{array}
                        \]
                    \end{table}
                }
            \end{enumerate}
        \end{practices}

        %38.
        \begin{practices}
            \begin{table}[H]
                \[
                    \begin{array}{c|c|c|c|c|c|c}
                        \hline
                        p & q & r & s & p \rightarrow q & (p \rightarrow q) \rightarrow r & ((p \rightarrow q) \rightarrow r) \rightarrow s \\
                        \hline
                        0 & 0 & 0 & 0 & 1 & 0 & 1 \\
                        0 & 0 & 0 & 1 & 1 & 0 & 1 \\
                        0 & 0 & 1 & 0 & 1 & 1 & 0 \\
                        0 & 0 & 1 & 1 & 1 & 1 & 1 \\
                        0 & 1 & 0 & 0 & 1 & 0 & 1 \\
                        0 & 1 & 0 & 1 & 1 & 0 & 1 \\
                        0 & 1 & 1 & 0 & 1 & 1 & 0 \\
                        0 & 1 & 1 & 1 & 1 & 1 & 1 \\
                        1 & 0 & 0 & 0 & 0 & 1 & 0 \\
                        1 & 0 & 0 & 1 & 0 & 1 & 1 \\
                        1 & 0 & 1 & 0 & 0 & 1 & 0 \\
                        1 & 0 & 1 & 1 & 0 & 1 & 1 \\
                        1 & 1 & 0 & 0 & 1 & 0 & 1 \\
                        1 & 1 & 0 & 1 & 1 & 0 & 1 \\
                        1 & 1 & 1 & 0 & 1 & 1 & 0 \\
                        1 & 1 & 1 & 1 & 1 & 1 & 1 \\
                        \hline
                    \end{array}
                \]
            \end{table}
        \end{practices}

        %39.
        \begin{practices}
            \begin{table}[H]
                \[
                    \begin{array}{c|c|c|c|c|c|c}
                        \hline
                        p & q & r & s & p \leftrightarrow q & r \leftrightarrow s & (p \leftrightarrow q) \leftrightarrow (r \leftrightarrow s) \\
                        \hline
                        0 & 0 & 0 & 0 & 1 & 1 & 1 \\
                        0 & 0 & 0 & 1 & 1 & 0 & 0 \\
                        0 & 0 & 1 & 0 & 1 & 0 & 0 \\
                        0 & 0 & 1 & 1 & 1 & 1 & 1 \\
                        0 & 1 & 0 & 0 & 0 & 1 & 0 \\
                        0 & 1 & 0 & 1 & 0 & 0 & 1 \\
                        0 & 1 & 1 & 0 & 0 & 0 & 1 \\
                        0 & 1 & 1 & 1 & 0 & 1 & 0 \\
                        1 & 0 & 0 & 0 & 0 & 1 & 0 \\
                        1 & 0 & 0 & 1 & 0 & 0 & 1 \\
                        1 & 0 & 1 & 0 & 0 & 0 & 1 \\
                        1 & 0 & 1 & 1 & 0 & 1 & 0 \\
                        1 & 1 & 0 & 0 & 1 & 1 & 1 \\
                        1 & 1 & 0 & 1 & 1 & 0 & 0 \\
                        1 & 1 & 1 & 0 & 1 & 0 & 0 \\
                        1 & 1 & 1 & 1 & 1 & 1 & 1 \\
                        \hline
                    \end{array}
                \]
            \end{table}
        \end{practices}

        %40.
        \begin{practices}
            当 $p, q, r$ 真值相同时,若其都为真,则 $p \vee \neg q \equiv T \vee F \equiv T$ ,同理 $q \vee \neg r \equiv T, r \vee \neg p \equiv T$ ,故 $(p \vee \neg q) \wedge (q \vee \neg r) \wedge (r \vee \neg p) \equiv T \wedge T \wedge T \equiv T$ 。

            若其都为假,同样能推出 $(p \vee \neg q) \wedge (q \vee \neg r) \wedge (r \vee \neg p) \equiv T \wedge T \wedge T \equiv T$ 。

            当真值不全相同时,假设 $p, q, r$ 其中 $p$ 为真,若此时 $q$ 也为真,则 $r$ 为假, 于是 $q \vee \neg r$ 为假,整个表达式为假。
            若 $p$ 为真时, $q$ 为假,则 $p \vee \neg q$ 为假,整个表达式为假。

            故当 $p, q, r$ 真值相同时, $(p \vee \neg q) \wedge (q \vee \neg r) \wedge (r \vee \neg p)$ 为真,而其他情况为假。
        \end{practices}

        %41.
        \begin{practices}
            当 $p, q, r$ 真值相同时,若其都为真, 则 $(p \vee q \vee r) \wedge (\neg p \vee \neg q \vee \neg r) \equiv T \wedge F \equiv F$ ,若其都为假, 则  $(p \vee q \vee r) \wedge (\neg p \vee \neg q \vee \neg r) \equiv F \wedge t \equiv F$ 。

            当其有真有假时, $p \vee q \vee r$ 为真, $\neg p \vee \neg q \vee \neg r$ 也为真,则 $T \wedge T \equiv T$ 。
            
            故当 $p, q, r$ 有真有假时,表达式真值为真,否则为假。
        \end{practices}

        %42.
        \begin{practices}
            \begin{enumerate}[A.]
                \item $2$
                \item $1$
                \item $2$
                \item $1$
                \item $2$
            \end{enumerate}
        \end{practices}

        %43.
        \begin{practices}
            \begin{enumerate}[A.]
                \item
                \begin{enumerate}[1)]
                    \item $111 1111$
                    \item $000 0000$
                    \item $111 1111$
                \end{enumerate}
                \item
                \begin{enumerate}[1)]
                    \item $1111 1010$
                    \item $1010 0000$
                    \item $0101 1010$
                \end{enumerate}
                \item
                \begin{enumerate}[1)]
                    \item $10 0111 1001$
                    \item $00 0100 0000$
                    \item $10 0011 1001$
                \end{enumerate}
                \item
                \begin{enumerate}[1)]
                    \item $11 1111 1111$
                    \item $00 0000 0000$
                    \item $11 1111 1111$
                \end{enumerate}
            \end{enumerate}
        \end{practices}

        %44.
        \begin{practices}
            \begin{enumerate}[A.]
                \item $1 1000$
                \item $0 1101$
                \item $1 1001$
                \item $1 1011$
            \end{enumerate}
        \end{practices}

        %45.
        \begin{practices}
            \begin{enumerate}[A.]
                \item $0.2$
                \item $0.6$
            \end{enumerate}
        \end{practices}

        %46.
        \begin{practices}
            \begin{enumerate}[A.]
                \item $0.4$
                \item $0.2$
            \end{enumerate}
        \end{practices}

        %47.
        \begin{practices}
            \begin{enumerate}[A.]
                \item $0.8$
                \item $0.6$
            \end{enumerate}
        \end{practices}

        %48.
        \begin{practices}
            不是。这是一个悖论,无法给出一个真值。
        \end{practices}

        %49.
        \begin{practices}
            \begin{enumerate}[A.]
                \item 列表中第99条为真,其余为假。
                \item 1-50条为真,50-99为假。
                \item 不可能发生。
            \end{enumerate}
        \end{practices}

        %50.
        \begin{practices}
            \begin{enumerate}[A.]
                \item 不存在。这是个一悖论。
            \end{enumerate}
        \end{practices}
    }
}
