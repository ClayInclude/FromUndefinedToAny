%%
%% Author: Clay
%% 2019/5/15
%%

\section{命题逻辑}
{
    \subsection{引言}
    {
        逻辑规则给出数学语句的准确含义,这些规则用来区分有效和无效的数学论证。
    }

    \subsection{命题}
    {
        首先介绍逻辑的基本构件---命题。
        \emreg{命题}是一个陈述语句,它或真或假,但不能既真又假。

        用字母来表示\emreg{命题变元},它是代表命题的变量。
        如果一个命题是真命题,它的\emreg{真值}为真,用\emcode{T}表示;
        如果它是假命题,其真值为假,用\emcode{F}表示。

        涉及命题的逻辑领域称为\emreg{命题演算}或\emreg{命题逻辑}。

        称为\emreg{复合命题}的新命题是由已知命题用逻辑运算符组合而来。

        \begin{defines}
            令 $p$ 为一命题,则 $p$ 的否定记作 $\neg p$ (也可记作 $\bar p$ ),指``不是 $p$ 所指的情形。''

            命题 $\neg p$ 读作``非 $p$ ''。
            $p$ 的否定( $\neg p$ )的真值和 $p$ 的真值相反。
        \end{defines}
 
        \begin{wraptable}{r}{.3333\textwidth{}}
            \centering

            \begin{tabular}{c|c}
                \hline
                $p$ & $\neg p$ \\
                \hline
                \emcode{T} & \emcode{F} \\
                \emcode{F} & \emcode{T} \\
                \hline
            \end{tabular}

            \caption{命题之否定的真值表}
            \label{c01s01t01}
        \end{wraptable}

        表\ref{c01s01t01}是命题 $p$ 及其否定的\emreg{真值表}。

        命题的否定也可以看作\emreg{否定运算符}作用在命题上的结果。
        否定运算符从一个已知命题构造出一个新命题。
        现在引入从两个或多个已知命题构造新命题的逻辑运算符。
        这些逻辑运算符也称为\emreg{联结词}。

        \begin{defines}
            令 $p$ 和 $q$ 为命题, $p$ 和 $q$ 的合取即命题`` $p$ 并且 $q$ '',记作 $p \wedge q$ 。
            当 $p$ 和 $q$ 都是真时, $p \wedge q$ 命题为真,否则为假。
        \end{defines}

        在合取逻辑中,有时用到\emreg{但是}一词,而非\emreg{并且(and)}一词。
        
        \begin{defines}
            令 $p$ 和 $q$ 为命题, $p$ 和 $q$ 的析取即命题`` $p$ 或 $q$'',记作 $p \vee q$ 。
            当 $p$ 和 $q$ 均为假时,析取命题 $p \vee q$ 为假, 否则为真。
        \end{defines}

        在析取中使用的联结词\emreg{或(or)}对应于词\emreg{或}在自然语言中所使用的两种情况之一,即\emreg{兼或(inclusive or)}。
        
        \begin{defines}
            令 $p$ 和 $q$ 为命题, $p$ 和 $q$ 的\emspe{异或(exclusive or)}(记作 $p \oplus q$ )是这样一个命题,当 $p$ 和 $q$ 中恰好只有一个为真时命题为真,否则为假。
        \end{defines}

        \begin{minipage}[c]{\textwidth{}}
            \begin{minipage}[c]{.5\textwidth{}}
                \begin{table}[H]
                    \centering

                    \begin{tabular}{cc|c}
                        \hline
                        $p$ & $q$ & $p \wedge q$ \\
                        \hline
                        \emcode{T} & \emcode{T} & \emcode{T} \\
                        \emcode{T} & \emcode{F} & \emcode{F} \\
                        \emcode{F} & \emcode{T} & \emcode{F} \\
                        \emcode{F} & \emcode{F} & \emcode{F} \\
                        \hline
                    \end{tabular}

                    \caption{两命题合取的真值表}
                \end{table}
            \end{minipage}%
            \begin{minipage}[c]{.5\textwidth{}}
                \begin{table}[H]
                    \centering

                    \begin{tabular}{cc|c}
                        \hline
                        $p$ & $q$ & $p \vee q$ \\
                        \hline
                        \emcode{T} & \emcode{T} & \emcode{T} \\
                        \emcode{T} & \emcode{F} & \emcode{T} \\
                        \emcode{F} & \emcode{T} & \emcode{T} \\
                        \emcode{F} & \emcode{F} & \emcode{F} \\
                        \hline
                    \end{tabular}

                    \caption{两命题析取的真值表}
                \end{table}
            \end{minipage}%
        \end{minipage}
    }

    \subsection{条件语句}
    {
        \begin{defines}
            令 $p$ 和 $q$ 为命题。
            条件语句 $p \rightarrow q$ 是命题``如果 $p$ ,则 $q$ ''。
            当 $p$ 为真而 $q$ 为假时,条件语句 $p \rightarrow q$ 为假,否则为真。
            在条件语句 $p \rightarrow q$ 中, $p$ 称为假设(前件、前提), $q$ 称为结论(后件)。
        \end{defines}

        语句 $p \rightarrow q$ 称为条件语句,因为 $p \rightarrow q$ 可以断定在条件 $p$ 成立的时候 $q$ 为真。
        条件语句也称为\emreg{蕴含}。

        \begin{minipage}[c]{\textwidth{}}
            \begin{minipage}[c]{.5\textwidth{}}
                \begin{table}[H]
                    \centering

                    \begin{tabular}{cc|c}
                        \hline
                        $p$ & $q$ & $p \oplus q$ \\
                        \hline
                        \emcode{T} & \emcode{T} & \emcode{F} \\
                        \emcode{T} & \emcode{F} & \emcode{T} \\
                        \emcode{F} & \emcode{T} & \emcode{T} \\
                        \emcode{F} & \emcode{F} & \emcode{F} \\
                        \hline
                    \end{tabular}

                    \caption{两命题异或的真值表}
                \end{table}
            \end{minipage}%
            \begin{minipage}[c]{.5\textwidth{}}
                \begin{table}[H]
                    \centering

                    \begin{tabular}{cc|c}
                        \hline
                        $p$ & $q$ & $p \rightarrow q$ \\
                        \hline
                        \emcode{T} & \emcode{T} & \emcode{T} \\
                        \emcode{T} & \emcode{F} & \emcode{F} \\
                        \emcode{F} & \emcode{T} & \emcode{T} \\
                        \emcode{F} & \emcode{F} & \emcode{T} \\
                        \hline
                    \end{tabular}

                    \caption{条件命题 $p \rightarrow q$ 的真值表}
                \end{table}
            \end{minipage}%
        \end{minipage}

        \itemcaption{逆命题、逆否命题与反命题}
        {
            命题 $q \rightarrow p$ 称为 $p \rightarrow q$ 的逆命题,而 $p \rightarrow q$ 的逆否命题是 $\neg q \rightarrow \neg p$ 。
            命题 $\neg p \rightarrow \neg q$ 称为 $p \rightarrow q$ 的反命题。
            只有逆否命题总是和原命题具有相同的真值。

            当两个复合命题总是具有相同真值时,称它们是\emreg{等价的}
            一个条件语句与它的逆否命题是等价的。
            条件语句的逆与反也是等价的。
        }

        % \begin{wraptable}{r}{.3333\textwidth{}}
        %     \centering

        %     \begin{tabular}{c|c}
        %         \hline
        %         $p$ & $\neg p$ \\
        %         \hline
        %         \emcode{T} & \emcode{F} \\
        %         \emcode{F} & \emcode{T} \\
        %         \hline
        %     \end{tabular}

        %     \caption{命题之否定的真值表}
        %     \label{c01s01t01}
        % \end{wraptable}
    }
}
