%%
%% Author: Clay
%% 2019/9/23
%%

\section{命题等价式}
{
    \subsection{引言}
    {
        \emreg{谓词逻辑}用来表达数学和计算机科学中各种语句的意义,并允许推理和探索对象之间的关系。
        \emreg{量词}可以对这样的语句进行推理:某一性质对于某一类型的所有对象均成立,存在一个对象使得某一特性成立。
    }

    \subsection{谓词}
    {
        语句``$x$ 大于3'' 有两个部分。
        第一部分即变量 $x$ 是语句的主语。
        第二部分(\emreg{谓词``大于3''})表明语句的主语具有的一个性质。

        语句 $P(x)$ 也可以说成是命题函数 $P$ 在 $x$ 的值。
        一旦给变量 $x$ 赋一个值,语句 $P(x)$ 就成为命题并具有真值。

        一般地,设计 $n$ 个变量 $x_1, x_2, \cdots, x_n$ 的语句可以表示成
        $$P(x_1, x_2, \cdots, x_n)$$
        形式为 $P(x_1, x_2, \cdots, x_n)$ 的语句是 \emreg{命题函数} $P$ 在 $n$ 元组 $(x_1, x_2, \cdots, x_n)$ 的值, $P$ 也称为\emreg{ $n$ 位谓词} 或 \emreg{ $n$ 元谓词}。 

        \paraph{前置和后置条件}
        {
            谓词还可以用来验证计算机程序,也就是证明当给定合法输入时计算机程序总是能产生所期望的输出。(除非建立了程序的正确性,否则无论测试了多少次都不能证明程序对所有输入都产生期望的输出,除非能测试到每个输入值。)
            描述合法输入的语句叫做\emreg{前置条件},而程序运行的输出应该满足的条件称为\emreg{后置条件}。
        }
    }

    \subsection{量词}
    {
        有一种称为\emreg{量化}的重要方式也可以从命题函数生成一个命题。
        量化表示在何种程度上谓词对于一个范围的个体成立。
        这里集中讨论两类量化:
        全称量化,它告诉我们一个谓词在所考虑的范围内对每一个个体都为真;
        存在量化,它告诉我们一个谓词在所考虑范围内的一个或多个个体为真。
        处理谓词和量词的逻辑领域称为\emreg{谓词演算}。

        \paraph{全称量词}
        {
            许多数学命题断言某一性质对于变量在某一特定域内的所有值均为真,这一特定域称为变量的\emreg{论域 (domain of discourse)}(或\emreg{全体域 (universe of discourse)}),时常简称为\emreg{域 (domain)}。
            使用全称量词时必须指定论域,否则语句的\emreg{全称量化}就是无定义的。

            \begin{defines}
                $P(x)$ 的\emspe{全称量化}是语句
                $$P(x)\text{对}x\text{在其论域的所有值为真。}$$
                符号 $\forall xP(x)$ 表示 $P(x)$ 的全称量化,其中 $\forall$ 称为\emspe{全称量词}。
                命题 $\forall xP(x)$ 读做``对所有 $x, P(x)$''或``对每个 $x, P(x)$''。
                一个使 $P(x)$ 为假的个体称为 $\forall xP(x)$ 的\emspe{反例}。
            \end{defines}

            \begin{table}[htb]
                \centering

                \begin{tabular}{c|c|c}
                    \hline
                    命题 & 什么时候为真 & 什么时候为假 \\
                    \hline
                    $\forall xP(x)$ & 对每一个 $x, P(x)$ 都为真 & 有一个 $x$ 使 $P(x)$ 为假 \\
                    $\exists xP(x)$ & 有一个 $x$ 使 $P(x)$ 为真 & 对每一个 $x, P(x)$ 都为假 \\
                    \hline
                \end{tabular}

                \caption{量词}
            \end{table}

            如果论域为空,那么 $\forall xP(x)$ 对任何命题函数 $P(x)$ 都为真。

            要证明当 $x$ 在论域中时 $P(x)$ 不总为真,方法之一就是寻找一个 $\forall xP(x)$ 的反例。
        }

        \paraph{存在量词}
        {
            许多数学定理断言:有一个个体使得某种性质成立。

            \begin{defines}
                $P(x)$ 的存在量化是命题
                $$\text{论域中存在一个个体}x\text{满足}P(x)\text{。}$$
                用符号 $\exists xP(x)$ 表示 $P(x)$ 的存在量化,其中 $\exists$ 称为存在量词。
            \end{defines}

            使用存在量词时必须指定论域,否则语句的\emreg{存在量化}就是无定义的。

            如果论域为空,那么 $\exists xP(x)$ 对任何命题函数 $P(x)$ 都为假。
        }

        \paraph{唯一性量词}
        {
            其他量词中最常见的是\emreg{唯一性量词},用符号 $\exists !$ 或 $\exists_1$ 表示。
        }
    }

    \subsection{约束论域的量词}
    {
        要限定一个量词的论域时,变量必须满足的条件直接放在量词的后面。

        全称量化的约束和一个条件语句的全称量化等价。
        存在量化的约束和一个合取式的存在量化等价。
    }

    \subsection{量词的优先级}
    {
        量词 $\forall$ 和 $\exists$ 比命题演算中的所有逻辑运算符都具有更高的优先级。
    }

    \subsection{变量绑定}
    {
        当量词作用与变量 $x$ 的时候,此变量的这次出现为\emreg{约束的}。
        一个变量的出现被成为是\emreg{自由的},如果没有被量词约束或设置为某一特定值。
        命题函数中的所有变量出现必须是约束的或者被设置未等于某个特定值的,才能把它转变为一个命题。
    }

    \subsection{涉及量词的逻辑等价式}
    {
        \begin{defines}
            涉及谓词和量词的语句是等价的当且仅当无论用什么谓词代入这些语句,也无论为这些命题函数里的变量指定什么论域,它们都有相同的真值。
            用 $S \equiv T$ 表示涉及谓词和量词的两个语句 $S$ 和 $T$ 是逻辑等价的。
        \end{defines}
    }

    \subsection{量化表达式的否定}
    {
        量词的否定规则称为\emreg{量词的德·摩根律}。

        \begin{table}[htb]
            \centering

            \begin{tabular}{c|c|c|c}
                \hline
                否定 & 等价语句 & 何时为真 & 何时为假 \\
                \hline
                $\neg \exists xP(x)$ & $\forall x \neg P(x)$ & 对每个 $x, P(x)$ 为假 & 有个 $x$ ,使 $p(x)$ 为真 \\
                $\neg \forall xP(x)$ & $\exists x \neg P(x)$ & 有个 $x$ ,使 $p(x)$ 为假 & 对每个 $x, P(x)$ 为真 \\
                \hline
            \end{tabular}
        \end{table}
    }

    \subsection{将语句到逻辑表达式的翻译}
    {
        将汉语(或其他自然语言)语句翻译成逻辑表达式,这在数学、逻辑编程、人工智能、软件工程以及许多其他学科中是一项重要的任务。
    }

    \subsection{系统规范说明中量词的使用}
    {
        1.2节使用命题来表示系统规范说明。
        然而,许多系统规范说明涉及谓词和量词。
    }

    \subsection{选自路易斯·卡罗尔的例子}
    {
        前两句称为\emreg{前提(premise)},第三句称为\emreg{结论(conlusion)}。
        合在一起作为一个整体称为是一个\emreg{论证(argument)}。
    }

    \subsection{逻辑程序设计}
    {
        Prolog(Programming in Logic)程序包括一组声明,其中包括两类语句:\emreg{Prolog事实}和\emreg{Prolog规则}。
        Prolog事实通过指定那些满足谓词的元素来定义谓词。
    }

    \subsection{练习}
    {
        %1.
        \begin{practices}
            \begin{enumerate}[A.]
                \item $true$
                \item $true$
                \item $false$
            \end{enumerate}
        \end{practices}

        %2.
        \begin{practices}
            \begin{enumerate}[A.]
                \item $true$
                \item $false$
                \item $false$
                \item $true$
            \end{enumerate}
        \end{practices}

        %3.
        \begin{practices}
            \begin{enumerate}[A.]
                \item $true$
                \item $false$
                \item $false$
                \item $false$
            \end{enumerate}
        \end{practices}

        %4.
        \begin{practices}
            $x = 2$
        \end{practices}

        %5.
        \begin{practices}
            \begin{enumerate}[A.]
                \item 存在一位学生每个工作日都花5个多小时上课。
                \item 每一位学生每个工作日都花5个多小时上课。
                \item 存在一位学生没有每个工作日都花5个多小时上课。
                \item 每一位学生没有每个工作日都花5个多小时上课。
            \end{enumerate}
        \end{practices}

        %6.
        \begin{practices}
            \begin{enumerate}[A.]
                \item 存在一位学生已经去过北达科他。
                \item 每一位学生已经去过北达科他。
                \item 不存在一位学生已经去过北达科他。
                \item 存在一位学生没有已经去过北达科他。
                \item 不是每一位学生已经去过北达科他。
                \item 每一位学生都没有去过北达科他。
            \end{enumerate}
        \end{practices}

        %7.
        \begin{practices}
            \begin{enumerate}[A.]
                \item 对于每一个人,如果他是一个喜剧演员,那么他很有趣。
                \item 对于每一个人,他都是一个喜剧演员,并且他很有趣。
                \item 存在这样的人,如果他是一个喜剧演员,那么他很有趣。
                \item 存在这样的人,他是一个喜剧演员,并且他很有趣。
            \end{enumerate}
        \end{practices}

        %8.
        \begin{practices}
            \begin{enumerate}[A.]
                \item 对于每一只动物,如果它是兔子,那么它会跳跃。
                \item 对于每一只动物,它都是兔子,并且它会跳跃。
                \item 存在一只动物,如果它是兔子,那么它会跳跃。
                \item 存在一只动物,它是兔子,并且它会跳跃。
            \end{enumerate}
        \end{practices}

        %9.
        \begin{practices}
            \begin{enumerate}[A.]
                \item $\exists x (P(x) \wedge Q(x))$
                \item $\exists x (P(x) \wedge \neg Q(x))$
                \item $\forall x (P(x) \vee Q(x))$
                \item $\forall x \neg (P(x) \vee Q(x))$
            \end{enumerate}
        \end{practices}

        %10.
        \begin{practices}
            \begin{enumerate}[A.]
                \item $\exists x (C(x) \wedge D(x) \wedge F(x))$
                \item $\forall x (C(x) \vee D(x) \vee F(x))$
                \item $\exists x (C(x) \wedge D(x) \wedge \neg F(x))$
                \item $\neg \exists x (C(x) \wedge D(x) \wedge F(x))$
                \item $\exists x C(x) \wedge \exists x D(x) \wedge \exists x F(x)$
            \end{enumerate}
        \end{practices}

        %11.
        \begin{practices}
            \begin{enumerate}[A.]
                \item 1
                \item 1
                \item 0
                \item 0
                \item 1
                \item 0
            \end{enumerate}
        \end{practices}

        %12.
        \begin{practices}
            \begin{enumerate}[A.]
                \item 1
                \item 1
                \item 0
                \item 1
                \item 0
                \item 1
                \item 0
            \end{enumerate}
        \end{practices}

        %13.
        \begin{practices}
            \begin{enumerate}[A.]
                \item 1
                \item 1
                \item 1
                \item 0
            \end{enumerate}
        \end{practices}

        %14.
        \begin{practices}
            \begin{enumerate}[A.]
                \item 0
                \item 1
                \item 1
                \item 0
            \end{enumerate}
        \end{practices}

        %15.
        \begin{practices}
            \begin{enumerate}[A.]
                \item 1
                \item 0
                \item 1
                \item 0
            \end{enumerate}
        \end{practices}

        %16.
        \begin{practices}
            \begin{enumerate}[A.]
                \item 1
                \item 0
                \item 1
                \item 0
            \end{enumerate}
        \end{practices}

        %17.
        \begin{practices}
            \begin{enumerate}[A.]
                \item $P(0) \vee P(1) \vee P(2) \vee P(3) \vee P(4)$
                \item $P(0) \wedge P(1) \wedge P(2) \wedge P(3) \wedge P(4)$
                \item $\neg P(0) \vee \neg P(1) \vee \neg P(2) \vee \neg P(3) \vee \neg P(4)$
                \item $\neg P(0) \wedge \neg P(1) \wedge \neg P(2) \wedge \neg P(3) \wedge \neg P(4)$
                \item $\neg (P(0) \vee P(1) \vee P(2) \vee P(3) \vee P(4))$
                \item $\neg (P(0) \wedge P(1) \wedge P(2) \wedge P(3) \wedge P(4))$
            \end{enumerate}
        \end{practices}

        %18.
        \begin{practices}
            \begin{enumerate}[A.]
                \item $P(-2) \vee P(-1) \vee P(0) \vee P(1) \vee P(2)$
                \item $P(-2) \wedge P(-1) \wedge P(0) \wedge P(1) \wedge P(2)$
                \item $\neg P(-2) \vee \neg P(-1) \vee \neg P(0) \vee \neg P(1) \vee \neg P(2)$
                \item $\neg P(-2) \wedge \neg P(-1) \wedge \neg P(0) \wedge \neg P(1) \wedge \neg P(2)$
                \item $\neg (P(-2) \vee P(-1) \vee P(0) \vee P(1) \vee P(2))$
                \item $\neg (P(-2) \wedge P(-1) \wedge P(0) \wedge P(1) \wedge P(2))$
            \end{enumerate}
        \end{practices}

        %19.
        \begin{practices}
            \begin{enumerate}[A.]
                \item $P(1) \vee P(2) \vee P(3) \vee P(4) \vee P(5)$
                \item $P(1) \wedge P(2) \wedge P(3) \wedge P(4) \wedge P(5)$
                \item $\neg (P(1) \vee P(2) \vee P(3) \vee P(4) \vee P(5))$
                \item $\neg (P(1) \wedge P(2) \wedge P(3) \wedge P(4) \wedge P(5))$
                \item $(P(1) \wedge P(2) \wedge P(4) \wedge P(5)) \vee (\neg P(1) \vee \neg P(2) \vee \neg P(3) \vee \neg P(4) \vee \neg P(5))$
            \end{enumerate}
        \end{practices}

        %20.
        \begin{practices}
            \begin{enumerate}[A.]
                \item $P(-5) \vee P(-3) \vee P(-1) \vee P(1) \vee P(3) \vee P(5)$
                \item $P(-5) \wedge P(-3) \wedge P(-1) \wedge P(1) \wedge P(3) \wedge P(5)$
                \item $P(-5) \wedge P(-3) \wedge P(-1) \wedge P(3) \wedge P(5)$
                \item $P(1) \wedge P(3) \wedge P(5)$
                \item $(\neg P(-5) \vee \neg P(-3) \vee \neg P(-1) \vee \neg P(1) \vee \neg P(3) \vee \neg P(5) \wedge (P(-5) \wedge P(-3) \wedge P(-1)))$
            \end{enumerate}
        \end{practices}

        %21.
        \begin{practices}
            \begin{enumerate}[A.]
                \item 学过离散数学的人;没有学过离散数学的人。
                \item 21岁以上的人;21岁以下的人。
                \item 拥有相同母亲的两人;母亲不同的两人。
                \item 祖母不相同的两人;祖母相同的两人。
            \end{enumerate}
        \end{practices}

        %22.
        \begin{practices}
            \begin{enumerate}[A.]
                \item 说印地语的人;不说印地语的人。
                \item 21岁以上的人;21岁以下的人。
                \item 拥有相同名字的两个人;名字不相同的两个人。
                \item 认识两个以上其他人的人;认识两个以下其他人的人。
            \end{enumerate}
        \end{practices}

        %23.
        \begin{practices}
            令 $C(x)$ 为命题函数``$x$ 在你的班上''。
            \begin{enumerate}[A.]
                \item $\exists x H(x)$ ; $\exists x (C(x) \wedge H(x))$ 。
                \item $\forall x H(x)$ ; $\forall x (C(x) \rightarrow H(x))$ 。
                \item $\exists x \neg H(x)$ ; $\exists x (C(x) \wedge \neg H(x))$ 。
                \item $\exists x H(x)$ ; $\exists x (C(x) \wedge H(x))$ 。
                \item $\forall x \neg H(x)$ ; $\exists x (C(x) \rightarrow \neg H(x))$ 。
            \end{enumerate}
        \end{practices}

        %24.
        \begin{practices}
            令 $C(x)$ 为命题函数``$x$ 在你的班上''。
            \begin{enumerate}[A.]
                \item $\forall x H(x)$ ; $\forall x (C(x) \rightarrow H(x))$ 。
                \item $\exists x H(x)$ ; $\exists x (C(x) \wedge H(x))$ 。
                \item $\exists x \neg H(x)$ ; $\exists x (C(x) \wedge \neg H(x))$ 。
                \item $\forall x H(x)$ ; $\forall x (C(x) \rightarrow H(x))$ 。
                \item $\exists x \neg H(x)$ ; $\exists x (C(x) \wedge \neg H(x))$ 。
            \end{enumerate}
        \end{practices}

        %25.
        \begin{practices}
            令 $P(x)$ 为命题函数``$x$ 是完美的'',令 $F(x)$ 为命题函数``$x$ 是你的朋友''。
            \begin{enumerate}[A.]
                \item $\neg \exists x P(x)$
                \item $\exists \neg x P(x)$
                \item $\forall x (F(x) \rightarrow P(x))$
                \item $\exists x (F(x) \wedge P(x))$
                \item $\forall x (F(x) \wedge P(x))$
                \item $\exists x (\neg F(x)) \vee \exists x (\neg P(x))$
            \end{enumerate}
        \end{practices}

        %26.
        \begin{practices}
            令 $C(x)$ 为命题函数``$x$ 在你的班上''。
            \begin{enumerate}[A.]
                \item $\exists x H(x)$
                \item $\forall x (C(x) \wedge H(x))$
                \item $\forall x \neg H(x)$
                \item $\forall x \neg H(x)$
                \item $\forall x H(x)$
            \end{enumerate}
        \end{practices}

        %27.
        \begin{practices}
            令 $C(x)$ 为命题函数``$x$ 在你的班上''。
            \begin{enumerate}[A.]
                \item $\exists x H(x)$
                \item $\exists x \neg H(x)$
                \item $\exists x H(x)$
                \item $\forall x (C(x) \rightarrow H(x))$
                \item $\exists x (C(x) \wedge \neg H(x))$
            \end{enumerate}
        \end{practices}

        %28.
        \begin{practices}
            令 $R(x)$ 为命题函数``$x$ 在正确的位置上'', $E(x)$ 为命题函数``$x$ 状态良好'', $T(x)$ 为命题函数``$x$是(你的)工具''。
            \begin{enumerate}[A.]
                \item $\exists x \not R(x)$
                \item $\forall x (T(x) \rightarrow (R(x) \wedge E(x)))$
                \item $\forall x (R(x) \wedge E(x))$
                \item $\neg \exists x (R(x) \wedge E(x))$
                \item $\exists x (T(x) \wedge (\neg R(x) \wedge E(x)))$
            \end{enumerate}
        \end{practices}

        %29.
        \begin{practices}
            令 $T(x)$ 为命题函数``$x$ 是永真式'', $C(x)$ 为命题函数``$x$ 是矛盾式''。
            \begin{enumerate}[A.]
                \item $\exists x T(x)$
                \item $\forall x (C(x) \rightarrow T(\neg x)$
                \item $\exists x \exists y (\neg C(x) \wedge \neg T(X) \wedge \neg C(y) \wedge \neg T(y) \wedge T(x \vee y))$
                \item $\forall x \forall y ((T(x) \wedge T(y)) \rightarrow T(x \wedge y))$
            \end{enumerate}
        \end{practices}

        %30.
        \begin{practices}
            \begin{enumerate}[A.]
                \item $P(1, 3) \vee P(2, 3) \vee P(3, 3)$
                \item $P(1, 1) \wedge P(1, 2) \wedge P(1, 3)$
                \item $\neg P(2, 1) \vee \neg P(2, 2) \vee \neg P(2, 3)$
                \item $\neg P(1, 2) \wedge \neg P(2, 2) \wedge \neg P(3, 2)$
            \end{enumerate}
        \end{practices}

        %31.
        \begin{practices}
            \begin{enumerate}[A.]
                \item $Q(0, 0, 0) \wedge Q(0, 1, 0)$
                \item $Q(0, 1, 1) \vee Q(1, 1, 1) \vee Q(2, 1, 1)$
                \item $\neg Q(0, 0, 0) \vee \neg Q(0, 0, 1)$
                \item $\neg Q(0, 0, 1) \vee \neg Q(1, 0, 1) \vee \neg Q(2, 0, 1)$
            \end{enumerate}
        \end{practices}

        %32.
        \begin{practices}
            \begin{enumerate}[A.]
                \item 存在一只狗不长跳蚤。
                \item 所有的马都不会做加法。
                \item 有一只考拉不会爬树。
                \item 有一只猴子会说法语。
                \item 所有的猪都不会游泳和捕鱼。
            \end{enumerate}
        \end{practices}

        %33.
        \begin{practices}
            \begin{enumerate}[A.]
                \item 所有的狗都不会学习新的技巧。
                \item 有一只兔子会微积分。
                \item 有一只鸟不会飞。
                \item 有一只狗会说话。
                \item 班上有人会法语和俄语。
            \end{enumerate}
        \end{practices}

        %34.
        \begin{practices}
            \begin{enumerate}[A.]
                \item 所有的司机都遵守驾驶速度限制。
                \item 有些瑞典电影不严肃。
                \item 有些人能保守秘密。
                \item 班上所有的人都有良好的心态。
            \end{enumerate}
        \end{practices}

        %35.
        \begin{practices}
            \begin{enumerate}[A.]
                \item $x = 0$
                \item $x = 0$
                \item $x \neq 1$
            \end{enumerate}
        \end{practices}

        %36.
        \begin{practices}
            \begin{enumerate}[A.]
                \item $x = 1$
                \item $x = \sqrt{2}$
                \item $x = 0$
            \end{enumerate}
        \end{practices}

        %37.
        \begin{practices}
            \begin{enumerate}[A.]
                \item $\forall x ((F(x, 25000) \vee S(x, 25)) \rightarrow V(x))$
                \item $\forall x (((M(x) \wedge T(3)) \vee (\neg M(x) \wedge T(3.5))) \rightarrow C(x))$
                \item $\forall x ((P(x, 60) \vee (P(x, 45) \wedge D(x))) \rightarrow G(x))$
                \item $\exists x ((P(x, 21)) \wedge G(x, A))$
            \end{enumerate}
        \end{practices}

        %38.
        \begin{practices}
            \begin{enumerate}[A.]
                \item 存在一个系统处于开放状态。
                \item 所有的系统都处于故障或诊断状态。
                \item 存在一个系统处于开放状态或存在一个系统处于诊断状态。
                \item 存在一个系统处于不可用状态。
                \item 所有的系统都处于不可工作状态。
            \end{enumerate}
        \end{practices}

        %39.
        \begin{practices}
            \begin{enumerate}[A.]
                \item 如果一台打印机不能提供服务并且很忙,那么有一份打印作业丢了。
                \item 如果所有的打印机都很忙,那么有一份打印作业在队列中。
                \item 如果有一份打印作业在队列中并且丢了,那么有一台打印机不能提供服务。
                \item 如果所有的打印机都很忙,并且所有的打印作业都在队列中,那么会有打印作业丢失。
            \end{enumerate}
        \end{practices}

        %40.
        \begin{practices}
            \begin{enumerate}[A.]
                \item $S(30) \rightarrow \forall x W(x)$
                \item $E \rightarrow \forall x (O(x) \wedge C(x))$
                \item $L \rightarrow B$
                \item $(M(8) \wedge S(56)) \rightarrow P$
            \end{enumerate}
        \end{practices}

        %41.
        \begin{practices}
            \begin{enumerate}[A.]
                \item $S(10) \rightarrow \exists x C(x)$
                \item $\forall x S(x) \rightarrow A$
                \item $\forall x(\neg M(x)) S(x)$
                \item $\forall x(\neg S(x)) C(x)$
            \end{enumerate}
        \end{practices}

        %42.
        \begin{practices}
            \begin{enumerate}[A.]
                \item $\forall x V(x)$
                \item $L \rightarrow \forall x V(x)$
                \item $D(x) \rightarrow D(y)$
                \item $(S(100, 500) \wedge \neg D(x)) \rightarrow \exists x W(x)$
            \end{enumerate}
        \end{practices}

        %43.
        \begin{practices}
            若 $P(x)$ 有真有假, $Q(x)$ 总是为假,则 $\forall x P(x) \rightarrow \forall x Q(x)$ 为真, $\forall (P(x) \rightarrow Q(x))$ 为假。
        \end{practices}

        %44.
        \begin{practices}
            若 $P(x)$ 有真有假, $Q(x)$ 有真有假,则 $\forall x P(x) \leftrightarrow \forall x Q(x)$ 总是为真, $\forall (P(x) \leftrightarrow Q(x))$ 不一定为真。
        \end{practices}

        %45.
        \begin{practices}
            若 $P(x)$ 和 $Q(x)$ 有一个为真,则两条语句都为真。如果都为假,则两条语句都为假。
        \end{practices}

        %46.
        \begin{practices}
            \begin{enumerate}[A.]
                \item 如果 $A$ 为真,则左边为真,右边每一项都为真,整体为真。如果 $A$ 为假,两边都等价于 $\forall x P(x)$ 。
                \item 如果 $A$ 为真,则左边为真,右边每一项都为真,整体为真。如果 $A$ 为假,两边都等价于 $\exists x P(x)$ 。
            \end{enumerate}
        \end{practices}

        %47.
        \begin{practices}
            \begin{enumerate}[A.]
                \item 如果 $A$ 为假,则左边为假,右边每一项都为假,整体为假。如果 $A$ 为真,两边都等价于 $\forall x P(x)$ 。
                \item 如果 $A$ 为假,则左边为假,右边每一项都为假,整体为假。如果 $A$ 为真,两边都等价于 $\exists x P(x)$ 。
            \end{enumerate}
        \end{practices}

        %48.
        \begin{practices}
            \begin{enumerate}[A.]
                \item 如果 $A$ 为假,则左右都为真。如果 $A$ 为真,若 $P(x)$ 有真有假,则两边都为假,若 $P(x)$ 都为真,则两边都为真。
                \item 如果 $A$ 为假,则左右都为真。如果 $A$ 为真,若 $P(x)$ 有真有假,则两边都为真,若 $P(x)$ 都为假,则两边都为假。
            \end{enumerate}
        \end{practices}

        %49.
        \begin{practices}
            \begin{enumerate}[A.]
                \item 如果 $A$ 为真,则左右都为真。如果 $A$ 为假,若 $P(x)$ 有真有假,则两边都为真,若 $P(x)$ 都为真,则两边都为真。
                \item 如果 $A$ 为真,则左右都为真。如果 $A$ 为假,若 $P(x)$ 有真有假,则两边都为真,若 $P(x)$ 都为真,则两边都为假。
            \end{enumerate}
        \end{practices}

        %50.
        \begin{practices}
            $P(x)$ 一半为真, $Q(x)$ 另一半为真。
        \end{practices}

        %51.
        \begin{practices}
            存在一个 $P(x)$ 为真,存在一个 $Q(x)$ 为真,但不同时为真。
        \end{practices}

        %52.
        \begin{practices}
            \begin{enumerate}[A.]
                \item 0
                \item 0
                \item 1
                \item 0
            \end{enumerate}
        \end{practices}

        %53.
        \begin{practices}
            \begin{enumerate}[A.]
                \item 1
                \item 0
                \item 1
            \end{enumerate}
        \end{practices}

        %54.
        \begin{practices}
            $P(1) \wedge \neg P(2) \wedge \neg p(3) \vee \neg P(1) \wedge P(2) \wedge \neg p(3) \vee \neg P(1) \wedge \neg P(2) \wedge p(3)$
        \end{practices}

        %55.
        \begin{practices}
            \begin{enumerate}[A.]
                \item 1
                \item 0
                \item juana; kiko
                \item math273; cs301
                \item juana; kiko
            \end{enumerate}
        \end{practices}

        %56.
        \begin{practices}
            \begin{enumerate}[A.]
                \item 0
                \item 1
                \item cs301
                \item juana; kiko
                \item chan
            \end{enumerate}
        \end{practices}

        %57.
        \begin{practices}
            \begin{codelist}
                \begin{lstlisting}
                    sibling(X, Y) :- mother(M, X), mother(M, Y), father(F, X), father(F, Y).
                \end{lstlisting}
            \end{codelist}
        \end{practices}

        %58.
        \begin{practices}
            \begin{codelist}
                \begin{lstlisting}
                    grandfather(X, Y) :- mother(C, Y), father(X, C); father(C, Y), father(X, Y).
                \end{lstlisting}
            \end{codelist}
        \end{practices}

        %59.
        \begin{practices}
            \begin{enumerate}[A.]
                \item $\forall x (P(x) \rightarrow \neg Q(x))$
                \item $\forall x (Q(x) \rightarrow R(x))$
                \item $\forall x (P(x) \rightarrow R(x))$
            \end{enumerate}

            不能。不无知的人也可能爱慕虚荣。
        \end{practices}
    }
}
