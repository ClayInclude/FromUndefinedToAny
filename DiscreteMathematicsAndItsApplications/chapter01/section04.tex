%%
%% Author: Clay
%% 2019/9/23
%%

\section{命题等价式}
{
    \subsection{引言}
    {
        \emreg{谓词逻辑}用来表达数学和计算机科学中各种语句的意义,并允许推理和探索对象之间的关系。
        \emreg{量词}可以对这样的语句进行推理:某一性质对于某一类型的所有对象均成立,存在一个对象使得某一特性成立。
    }

    \subsection{谓词}
    {
        语句``$x$ 大于3'' 有两个部分。
        第一部分即变量 $x$ 是语句的主语。
        第二部分(\emreg{谓词``大于3''})表明语句的主语具有的一个性质。

        语句 $P(x)$ 也可以说成是命题函数 $P$ 在 $x$ 的值。
        一旦给变量 $x$ 赋一个值,语句 $P(x)$ 就成为命题并具有真值。

        一般地,设计 $n$ 个变量 $x_1, x_2, \cdots, x_n$ 的语句可以表示成
        $$P(x_1, x_2, \cdots, x_n)$$
        形式为 $P(x_1, x_2, \cdots, x_n)$ 的语句是 \emreg{命题函数} $P$ 在 $n$ 元组 $(x_1, x_2, \cdots, x_n)$ 的值, $P$ 也称为\emreg{ $n$ 位谓词} 或 \emreg{ $n$ 元谓词}。 

        \paraph{前置和后置条件}
        {
            谓词还可以用来验证计算机程序,也就是证明当给定合法输入时计算机程序总是能产生所期望的输出。(除非建立了程序的正确性,否则无论测试了多少次都不能证明程序对所有输入都产生期望的输出,除非能测试到每个输入值。)
            描述合法输入的语句叫做\emreg{前置条件},而程序运行的输出应该满足的条件称为\emreg{后置条件}。
        }
    }

    \subsection{量词}
    {
        有一种称为\emreg{量化}的重要方式也可以从命题函数生成一个命题。
        量化表示在何种程度上谓词对于一个范围的个体成立。
        这里集中讨论两类量化:
        全称量化,它告诉我们一个谓词在所考虑的范围内对每一个个体都为真;
        存在量化,它告诉我们一个谓词在所考虑范围内的一个或多个个体为真。
        处理谓词和量词的逻辑领域称为\emreg{谓词演算}。

        \paraph{全称量词}
        {
            许多数学命题断言某一性质对于变量在某一特定域内的所有值均为真,这一特定域称为变量的\emreg{论域 (domain of discourse)}(或\emreg{全体域 (universe of discourse)}),时常简称为\emreg{域 (domain)}。
            使用全称量词时必须指定论域,否则语句的\emreg{全称量化}就是无定义的。

            \begin{defines}
                $P(x)$ 的\emspe{全称量化}是语句
                $$P(x)\text{对}x\text{在其论域的所有值为真。}$$
                符号 $\forall xP(x)$ 表示 $P(x)$ 的全称量化,其中 $\forall$ 称为\emspe{全称量词}。
                命题 $\forall xP(x)$ 读做``对所有 $x, P(x)$''或``对每个 $x, P(x)$''。
                一个使 $P(x)$ 为假的个体称为 $\forall xP(x)$ 的\emspe{反例}。
            \end{defines}

            \begin{table}[htb]
                \centering

                \begin{tabular}{c|c|c}
                    \hline
                    命题 & 什么时候为真 & 什么时候为假 \\
                    \hline
                    $\forall xP(x)$ & 对每一个 $x, P(x)$ 都为真 & 有一个 $x$ 使 $P(x)$ 为假 \\
                    $\exists xP(x)$ & 有一个 $x$ 使 $P(x)$ 为真 & 对每一个 $x, P(x)$ 都为假 \\
                    \hline
                \end{tabular}

                \caption{量词}
            \end{table}

            如果论域为空,那么 $\forall xP(x)$ 对任何命题函数 $P(x)$ 都为真。

            要证明当 $x$ 在论域中时 $P(x)$ 不总为真,方法之一就是寻找一个 $\forall xP(x)$ 的反例。
        }

        \paraph{存在量词}
        {
            许多数学定理断言:有一个个体使得某种性质成立。

            \begin{defines}
                $P(x)$ 的存在量化是命题
                $$\text{论域中存在一个个体}x\text{满足}P(x)\text{。}$$
                用符号 $\exists xP(x)$ 表示 $P(x)$ 的存在量化,其中 $\exists$ 称为存在量词。
            \end{defines}

            使用存在量词时必须指定论域,否则语句的\emreg{存在量化}就是无定义的。

            如果论域为空,那么 $\exists xP(x)$ 对任何命题函数 $P(x)$ 都为假。
        }

        \paraph{唯一性量词}
        {
            其他量词中最常见的是\emreg{唯一性量词},用符号 $\exists !$ 或 $\exists_1$ 表示。
        }
    }

    \subsection{约束论域的量词}
    {
        要限定一个量词的论域时,变量必须满足的条件直接放在量词的后面。

        全称量化的约束和一个条件语句的全称量化等价。
        存在量化的约束和一个合取式的存在量化等价。
    }

    \subsection{量词的优先级}
    {
        量词 $\forall$ 和 $\exsits$ 比命题演算中的所有逻辑运算符都具有更高的优先级。
    }

    \subsection{变量绑定}
    {
        当量词作用与变量 $x$ 的时候,此变量的这次出现为\emreg{约束的}。
        一个变量的出现被成为是\emreg{自由的},如果没有被量词约束或设置为某一特定值。
        命题函数中的所有变量出现必须是约束的或者被设置未等于某个特定值的,才能把它转变为一个命题。
    }

    \subsection{练习}
    {
        %1.
        \begin{practices}
            \begin{enumerate}[A.]
                \item $true$
                \item $true$
                \item $false$
            \end{enumerate}
        \end{practices}

        %2.
        \begin{practices}
            \begin{enumerate}[A.]
                \item $true$
                \item $false$
                \item $false$
                \item $true$
            \end{enumerate}
        \end{practices}

        %3.
        \begin{practices}
            \begin{enumerate}[A.]
                \item $true$
                \item $false$
                \item $false$
                \item $false$
            \end{enumerate}
        \end{practices}

        %4.
        \begin{practices}
            $x = 2$
        \end{practices}
    }
}
