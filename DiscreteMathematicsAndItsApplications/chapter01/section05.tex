%%
%% Author: Clay
%% 2019/10/23
%%

\section{嵌套量词}
{
    \subsection{引言}
    {
        \emreg{嵌套量词}即一个量词出现在另一个量词的作用域内。
    }

    \subsection{理解涉及嵌套量词的语句}
    {
        \paraph{将量化当作循环}
        {
            在处理多个变量的量化式时,有时候借助嵌套循环来思考是有益的。
        }
    }

    \subsection{量词的顺序}
    {
        量词的顺序是很重要的,除非所有量词均为全称量词或均为存在量词。

        \begin{table}[htb]
            \centering

            \begin{tabular}{c|c|c}
                \hline
                语句 & 何时为真 & 何时为假 \\
                \hline
                \makecell{$\forall x \forall y P(x, y)$ \\ $\forall y \forall x P(x, y)$} & 对每一对 $x, y$ , $P(x, y)$ 均为真 & 存在一对 $x, y$ 使得 $P(x, y)$ 为假 \\
                \hline
                $\forall x \exists y P(x, y)$ & 对每个 $x$ ,都存在一个 $y$ 使得 $P(x, y)$ 为真 & 存在一个 $x$ 使得 $P(x, y)$ 对每个 $y$ 均为假 \\
                \hline
                $\exists x \forall y P(x, y)$ & 存在一个 $x$ ,使得 $P(x, y)$ 对每个 $y$ 均为真 & 对每个 $x$ ,存在一个 $y$ 使得 $P(x, y)$ 为假 \\
                \hline
                \makecell{$\forall x \forall y P(x, y)$ \\ $\forall y \forall x P(x, y)$} & 存在一对 $x, y$ ,使得 $P(x, y)$ 为真 & 对每一对 $x, y$ , $P(x, y)$ 均为假 \\
                \hline
            \end{tabular}

            \caption{两个变量的量化式}
        \end{table}
    }

    \subsection{数学语句到嵌套量词语句的翻译}
    {
        用汉语表达的数学语句可以被翻译成逻辑表达式。
    }

    \subsection{嵌套量词到自然语言的翻译}
    {
        用嵌套量词表达汉语语句的表达式可能会相当复杂。
        在翻译这样的表达式时,第一步是写出表达式中量词和谓词的含义,第二步是用简单的句子来表达这个含义。
    }

    \subsection{汉语语句到逻辑表达式的翻译}
    {
        需要使用\emreg{空量化(null quantification)}的规则,当受量词约束的变元没有出现在语句的某一部分时可以使用。
    }

    \subsection{嵌套量词的否定}
    {
        带嵌套量词语句的否定可以通过连续的应用单个量词语句的否定规则得到。
    }

    \subsection{练习}
    {
        %1.
        \begin{practices}
            \begin{enumerate}[A.]
                \item 对于每个 $x$ ,存在一个 $y$ 使得 $x < y$ 成立。
                \item 对于所有 $x, y$ ,如果它们都是非负数,那么它们的积也是非负数。
                \item 对于所有 $x, y$ ,都存在一个 $z$ 等于它们的积。
            \end{enumerate}
        \end{practices}

        %2.
        \begin{practices}
            \begin{enumerate}[A.]
                \item 存在一个 $x$ ,使得所有 $y$ 都有 $xy = y$ 。
                \item 对于所有 $x, y$ ,如果 $x$ 是非负数且 $y$ 是负数,那么 $x - y$ 为正数。
                \item 对于所有 $x, y$ ,都存在一个 $z$ 等于它们的差。
            \end{enumerate}
        \end{practices}

        %3.
        \begin{practices}
            \begin{enumerate}[A.]
                \item 存在一个 $x$ 和 存在一个 $y$ ,其中 $x$ 已经发送电子邮件消息给 $y$ 。
                \item 存在一个 $x$ ,已经发送电子邮件消息给班上所有的人。
                \item 班上的每一位学生都已经发送电子邮件消息给至少一位学生。
                \item 存在一位学生 $y$ ,班上每一位学生都给他发送了电子邮件消息。
                \item 班上的每一位学生都至少有一位学生给他发送了电子邮件消息。
                \item 班上所有的学生都互相发送了电子邮件消息。
            \end{enumerate}
        \end{practices}

        %4.
        \begin{practices}
            \begin{enumerate}[A.]
                \item 存在一位学生选修过一门课。
                \item 存在一位学生选修了所有的课。
                \item 所有的学生都至少选修了一门课。
                \item 至少有一门课被所有的学生选修了。
                \item 所有的课都至少被一位学生选修过。
                \item 所有的学生选修了所有的课。
            \end{enumerate}
        \end{practices}

        %5.
        \begin{practices}
            \begin{enumerate}[A.]
                \item Sarah Smith访问过www.att.com。
                \item 至少有一位学生访问过www.imdb.org。
                \item Jose Orez至少访问过一个网站。
                \item 存在一个网站,Ashok Puri和Cindy Yoon都访问过。
                \item 除了David Belcher之外,还有另外一个人也访问过David Belcher访问过的所有网站。
                \item 存在两个不同的人 $x$ 和 $y$ ,他们恰好访问过相同的网站。
            \end{enumerate}
        \end{practices}

        %6.
        \begin{practices}
            \begin{enumerate}[A.]
                \item Randy Goldberg注册了课程CS 252。
                \item 有一位学生注册了Math 695。
                \item Carol Sitea注册了至少一门课程。
                \item 至少一位学生同时注册了Math 222和CS 252。
                \item 有两位学生不同的学生,其中一人注册了另一人所有注册的课程。
                \item 有两位学生恰好注册了相同的课程。
            \end{enumerate}
        \end{practices}

        %7.
        \begin{practices}
            \begin{enumerate}[A.]
                \item Abdallah Hussein不喜欢Japanese。
                \item 存在一位学生喜欢Korean并且所有的学生都喜欢Mexican。
                \item 存在一样食物,Monique Arsenault或Jay Johnson至少一人喜欢。
                \item 存在一样食物,在两位学生中只有一人喜欢。
                \item 存在两位学生,一位学生喜欢所有的食物当且仅当另一位学生也喜欢所有的食物。
                \item 
            \end{enumerate}
        \end{practices}

        %8.
        \begin{practices}
            \begin{enumerate}[A.]
                \item $\exists x \exists y Q(x, y)$
                \item $\neg \exists x \exists y Q(x, y)$
                \item $\exists x (Q(x, Jeopardy) \wedge Q(x, Whiil of Fortune))$
                \item $\forall y \exists x Q(x, y)$
                \item $\exists a \exists b ((a \neq b) \wedge Q(a, Jeopardy) \wedge (b, Jeopardy))$
            \end{enumerate}
        \end{practices}

        %9.
        \begin{practices}
            \begin{enumerate}[A.]
                \item $\forall x L(x, Jerry)$
                \item $\forall x \exists y L(x, y)$
                \item $\exists y \forall x L(x, y)$
                \item $\neg \exists x \forall y L(x, y)$
            \end{enumerate}
        \end{practices}
    }
}
