%%
%% Author: Clay
%% 2019/11/25
%%

\section{推理规则}
{
    \subsection{引言}
    {
        数学中的证明是建立数学命题真实性的有效论证。
        所谓的\emreg{论证(argument)},是指一连串的命题并以结论为最后的命题。
        所谓\emreg{有效性(valid)},是指结论或论证的最后一个命题必须根据论证过程前面的命题或\emreg{前提(premise)}的真实性推出。
        一个论证是有效的当且仅当不可能出现所有前提为真而论证为假的情况。
        为从已知命题中推出新的命题,我们应用推理规则,这是构造有效论证的模板。
        推理规则是建立命题真实性的基本工具。

        \emreg{谬误(fallacy)}直接导致无效论证。
    }

    \subsection{命题逻辑的有效论证}
    {
        在讨论中,为了分析一个论证,我们用命题变量代替命题。
        这将一个论证改变为一个\emreg{论证形式}。
        一个论证的有效性来自于论证形式的有效性。
        用这些关键概念的定义来总结用于讨论论证有效性的术语。

        \begin{defines}
            命题逻辑中的一个论证是一连串的命题。
            除了论证中最后一个命题外都叫做前提,最后那个命题叫作结论。
            一个论证是有效的,如果它的所有前提为真蕴含着结论为真。

            命题逻辑中的论证形式是一连串涉及命题变元的复合命题。
            无论用什么特定命题来替换其中的命题变元,如果前提均真时结论为真,则称该论证形式是有效的。
        \end{defines}

        从有效论证形式的定义可知,当 $(p_1 \wedge P_2 \wedge \cdots \wedge p_n) \rightarrow q$ 是永真式时,带有前提 $p_1, p_2, \cdots , p_n$ 以及结论 $q$ 的论证形式是有效的。

        证明命题逻辑中论证有效性的关键就是要证明它的论证形式的有效性。
    }

    \subsection{命题逻辑的推理规则}
    {
        我们总是可以用一个真值表来证明一个论证形式是有效的。
        通过证明只要前提为真则结论也就肯定为真来做到这一点。

        我们可以先建立一些相对简单的论证形式(称为\emreg{推理规则})的有效性。
        这些推理规则可以作为基本构件用来构造更多复杂的有效论证形式。

        永真式 $(p \wedge (p \rightarrow q)) \rightarrow q$ 是称为\emreg{假言推理(modus ponens)}或\emreg{分离规则(law of detachment)}的推理规则的基础。(拉丁文modus ponens的意思是\emreg{确认模式(mode that affirms)}。)

        \begin{table}[htb]
            \centering

            \begin{tabular}{c|c|c}
                \hline
                推理规则 & 永真式 & 名称 \\
                \hline
            \end{tabular}

            \caption{推理规则}
        \end{table}
    }
}
