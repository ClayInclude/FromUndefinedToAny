%%
%% Author: Clay
%% 2019/11/25
%%

\section{推理规则}
{
    \subsection{引言}
    {
        数学中的证明是建立数学命题真实性的有效论证。
        所谓的\emreg{论证(argument)},是指一连串的命题并以结论为最后的命题。
        所谓\emreg{有效性(valid)},是指结论或论证的最后一个命题必须根据论证过程前面的命题或\emreg{前提(premise)}的真实性推出。
        一个论证是有效的当且仅当不可能出现所有前提为真而论证为假的情况。
        为从已知命题中推出新的命题,我们应用推理规则,这是构造有效论证的模板。
        推理规则是建立命题真实性的基本工具。

        \emreg{谬误(fallacy)}直接导致无效论证。
    }

    \subsection{命题逻辑的有效论证}
    {
        在讨论中,为了分析一个论证,我们用命题变量代替命题。
        这将一个论证改变为一个\emreg{论证形式}。
        一个论证的有效性来自于论证形式的有效性。
        用这些关键概念的定义来总结用于讨论论证有效性的术语。

        \begin{defines}
            命题逻辑中的一个论证是一连串的命题。
            除了论证中最后一个命题外都叫做前提,最后那个命题叫作结论。
            一个论证是有效的,如果它的所有前提为真蕴含着结论为真。

            命题逻辑中的论证形式是一连串涉及命题变元的复合命题。
            无论用什么特定命题来替换其中的命题变元,如果前提均真时结论为真,则称该论证形式是有效的。
        \end{defines}

        从有效论证形式的定义可知,当 $(p_1 \wedge P_2 \wedge \cdots \wedge p_n) \rightarrow q$ 是永真式时,带有前提 $p_1, p_2, \cdots , p_n$ 以及结论 $q$ 的论证形式是有效的。

        证明命题逻辑中论证有效性的关键就是要证明它的论证形式的有效性。
    }

    \subsection{命题逻辑的推理规则}
    {
        我们总是可以用一个真值表来证明一个论证形式是有效的。
        通过证明只要前提为真则结论也就肯定为真来做到这一点。

        我们可以先建立一些相对简单的论证形式(称为\emreg{推理规则})的有效性。
        这些推理规则可以作为基本构件用来构造更多复杂的有效论证形式。

        永真式 $(p \wedge (p \rightarrow q)) \rightarrow q$ 是称为\emreg{假言推理(modus ponens)}或\emreg{分离规则(law of detachment)}的推理规则的基础。(拉丁文modus ponens的意思是\emreg{确认模式(mode that affirms)}。)

        \begin{table}[htb]
            \centering

            \begin{tabular}{c|c|c}
                \hline
                推理规则 & 永真式 & 名称 \\
                \hline
                \makecell{$p$ \\ $\underline{p \rightarrow q}$ \\ $\therefore q$} & $(p \wedge (p \rightarrow q)) \rightarrow q$ & 假言推理 \\
                \hline
                \makecell{$\neg q$ \\ $\underline{p \rightarrow q}$ \\ $\therefore \neg p$} & $(\neg q \wedge (p \rightarrow q)) \rightarrow \neg p$ & 取拒式 \\
                \hline
                \makecell{$p \rightarrow q$ \\ $\underline{q \rightarrow r}$ \\ $\therefore p \rightarrow r$} & $((p \rightarrow q) \wedge (q \rightarrow r)) \rightarrow (p \rightarrow r)$ & 假言三段论 \\
                \hline
                \makecell{$p \vee q$ \\ $\underline{\neg p}$ \\ $\therefore q$} & $((p \vee q) \wedge \neg p) \rightarrow q$ & 析取三段论 \\
                \hline
                \makecell{$\underline{p}$ \\ $\therefore p \vee q$} & $p \rightarrow (p \vee q)$ & 附加律 \\
                \hline
                \makecell{$\underline{p \wedge q}$ \\ $\therefore p$} & $(p \wedge q) \rightarrow p$ & 化简律 \\
                \hline
                \makecell{$p$ \\ $\underline{q}$ \\ $\therefore p \wedge q$} & $(p \wedge q) \rightarrow (p \wedge q)$ & 合取律 \\
                \hline
                \makecell{$p \vee q$ \\ $\underline{\neg p \vee r}$ \\ $\therefore q \vee r$} & $((p \vee q) \wedge (\neg p \vee r)) \rightarrow (q \vee r)$ & 消解律 \\
                \hline
            \end{tabular}

            \caption{推理规则}
        \end{table}
    }

    \subsection{使用推理规则建立论证}
    {
        当有多个前提时,常常需要用到多个推理规则来证明一个论证是有效的。

        也可以用真值表来证明只要四个前提的每一个都为真,那么结论也为真。
    }

    \subsection{消解律}
    {
        已经开发出的计算机程序能够将定理的推理和证明任务自动化。
        许多这类程序利用称为\emreg{消解律(resolution)}的推理规则。
        这个推理规则基于永真式:
        $$((p \vee q) \wedge (\neg p \vee r)) \rightarrow (q \vee r)$$
        消解规则最后的析取式 $q \vee r$ 称为\emreg{消解式(resolvent)}。
        当在此永真式中令 $q = r$ 时,可得 $(p \vee q) \wedge (\neg p \vee q) \rightarrow q$ 。
        而且,当令 $r = F$ 时,可得 $(p \vee q) \wedge \neg p \rightarrow q$ ,这是永真式,析取三段论规则就基于此式。

        消解律在基于逻辑规则的编程语言中扮演着重要的角色,可以用消解律来构建自动定理证明系统。
        要使用消解律作为仅有的推理规则来构造命题逻辑中的逻辑,假设和结论必须表示为\emreg{子句(clause)},这里子句是指变量或其否定的一个析取式。
    }

    \subsection{谬误}
    {
        几种常见的谬误都来源于不正确的论证。
        这些谬误看上去像是推理规则,但是它们是基于可满足式而不是永真式。

        命题 $((p \rightarrow q) \wedge q) \rightarrow p$ 不是永真式,这类不正确的推理称为\emreg{肯定结论的谬误(fallacy of affirming the conclusion)}。

        命题 $((p \rightarrow q )\wedge \neg p) \rightarrow \neg q$ 不是永真式,这类不正确的推理称为\emreg{否定假设的谬误(fallacy of denying the hypothesis)}。
    }

    \subsection{量化命题的推理规则}
    {
        \emreg{全称实例(universal instantiation)}是从给定前提 $\forall x P(x)$ 得出 $P(c)$ 为真的推理规则,其中 $c$ 是论域里的一个特定成员。

        \emreg{全称引入(existential geralization)}是从对论域里所有元素 $c$ 都有 $P(c)$ 为真的前提推出 $\forall x P(x)$ 为真的推理规则。
        所选择的元素 $c$ 必须是论域里一个任意的元素,而不是特定的元素。
        从 $\forall x P(x)$ 断言对于论语中元素 $c$ 的存在性时,我们不能对 $c$ 进行控制,并且除了 $c$ 来自于论域以外不能对 $c$ 做出任何假设。

        \emreg{存在实例(existential instantiation)}是允许从 $\exists x P(x)$ 得出论域中存在一个元素 $c$ 使得 $P(c)$ 为真的推理规则。
        这里不能选择一个任意值的 $c$ ,而必须是使得 $P(c)$ 为真的那个 $c$ 。
        通常不知道 $c$ 是什么,而仅仅知道它存在。

        \emreg{存在引入(existential geralization)}是用来从已知有一特定的 $c$ 是 $P(c)$ 为真时得出 $\exists x P(x)$ 为真的推理规则。

        \begin{table}[htb]
            \centering

            \begin{tabular}{c|c}
                \hline
                推理规则 & 名称 \\
                \hline
                \makecell{$\underline{\forall x P(x)}$ \\ $\therefore P(x)$} & 全称实例 \\
                \hline
                \makecell{$\underline{P(c), \text{任意}c}$ \\ $\therefore \forall x P(x)$} & 全称引入 \\
                \hline
                \makecell{$\underline{\exists x P(x)}$ \\ $\therefore P(x), \text{对某个元素}c$} & 存在实例 \\
                \hline
                \makecell{$\underline{P(x), \text{对某个元素}c}$ \\ $\therefore \exists x P(x)$} & 存在引入 \\
                \hline
            \end{tabular}

            \caption{量化命题的推理规则}
        \end{table}
    }

    \subsection{命题和量化命题推理规则的组合使用}
    {
        由于全称实例和假言推理在一起使用是如此广泛,所以这种规则的组合有时称为\emreg{全称假言推理(universal modus ponens)}。
        这个规则告诉我们:如果 $\forall x (P(x) \rightarrow Q(x))$ 为真,并且如果 $P(a)$ 对在全称量词论域中的一个特定元素 $a$ 为真,那么 $Q(a)$ 也肯定为真。
        
        \begin{align*}
            &\forall x (P(x) \rightarrow Q(x)) \\
            &\underline{P(x), \text{其中a是论域中一个特定的元素}} \\
            \therefore &Q(a)
        \end{align*}

        \emreg{全称取拒式(universal modus tollens)}将全称实例和取拒式组合在一起。

        \begin{align*}
            &\forall x (P(x) \rightarrow Q(x)) \\
            &\underline{\neg Q(a), \text{其中a是论域中一个特定的元素}} \\
            \therefore &\neg P(a)
        \end{align*}
    }

    \subsection{练习}
    {
        %1.
        \begin{practices}
            假言推理,有效。能。
        \end{practices}

        %2.
        \begin{practices}
            取拒式,有效。能。
        \end{practices}

        %3.
        \begin{practices}
            \begin{enumerate}[A.]
                \item 附加律
                \item 化简律
                \item 假言推理
                \item 取拒式
                \item 假言三段论
            \end{enumerate}
        \end{practices}

        %4.
        \begin{practices}
            \begin{enumerate}[A.]
                \item 化简律
                \item 析取三段论
                \item 假言推理
                \item 附加律
                \item 假言三段论
            \end{enumerate}
        \end{practices}

        %5.
        \begin{practices}
            $p$ :Randy很用功。
            $q$ :Randy是一个笨孩子。
            $r$ :Randy不会得到工作。

            \begin{align*}
                &p \rightarrow q &\text{前提引入} \\
                &p &\text{前提引入} \\
                &q &\text{假言推理} \\
                &q \rightarrow r &\text{前提引入} \\
                &r &\text{假言推理}
            \end{align*}
        \end{practices}

        %6.
        \begin{practices}
            $p$ :天不下雨
            $q$ :天不起雾
            $r$ :帆船比赛举行
            $s$ :进行救生表演
            $t$ :颁发奖杯

            \begin{align*}
                &r \rightarrow t &\text{前提引入} \\
                &\neg t &\text{前提引入} \\
                &\neg r &\text{取拒式} \\
                &\neg r \vee \neg s &\text{附加律} \\
                &\neg (r \wedge s) &\text{德·摩根定律} \\
                &(p \vee q) \rightarrow (r \wedge s) &\text{前提引入} \\
                &\neg (p \vee q) &\text{取拒式} \\
                &\neg p \wedge \neg q &\text{德·摩根定律} \\
                &\neg p &\text{化简律}
            \end{align*}
        \end{practices}

        %7.
        \begin{practices}
            全称假言推理。
        \end{practices}

        %8.
        \begin{practices}
            全称取拒式。
        \end{practices}

        %9.
        \begin{practices}
            \begin{enumerate}[A.]
                \item 我在周四休假;我周二没有休假;周四下雨。
                \item 我没有吃辣的食物;我睡觉时没有打雷。
                \item 我聪明。
                \item Ralph没有主修计算机。
                \item 你购买许多东西就是对你有利。
                \item 老鼠啃咬它们的食物;野兔不是啮齿类动物。
            \end{enumerate}
        \end{practices}

        %10.
        \begin{practices}
            \begin{enumerate}[A.]
                \item 我没有打曲棍球。
                \item 我在周一工作;周一是晴天或者半晴;我没有在周二工作;我没有在周五工作。
                \item 蜻蜓有6条腿;蜘蛛不是昆虫。
                \item Homer不是学生。
                \item 豆腐不好吃。
                \item 我看见大象在路上跑。
            \end{enumerate}
        \end{practices}

        %11.
        \begin{practices}
            令 $p \equiv (p_1, p_2, \cdots , p_n)$ 。
            若 $p \wedge q$ 和 $r$ 都为真,则 $p$ 和 $q \rightarrow r$ 也一定为真。
        \end{practices}

        %12.
        \begin{practices}
            \begin{align*}
                &\neg p \vee \neg t \vee r \vee s \\
                &\neg s \\
                &\neg p \vee \neg t \vee r \\
                &\neg t \vee r \vee \neg u \\
                &\neg (t \wedge u) \vee r \\
                &\neg q \vee r
            \end{align*}
        \end{practices}

        %13.
        \begin{practices}
            \begin{enumerate}[A.]
                \item 全称假言推理;存在引入。
                \item 全称假言推理;存在引入。
                \item 全称假言三段论。
                \item 合取律;存在实例。
            \end{enumerate}
        \end{practices}

        %14.
        \begin{practices}
            \begin{enumerate}[A.]
                \item 全称假言推理;存在引入。
                \item 全称假言推理;全称引入。
                \item 全称假言推理;存在引入。
                \item 全称假言推理;存在引入。
            \end{enumerate}
        \end{practices}

        %15.
        \begin{practices}
            \begin{enumerate}[A.]
                \item 全称假言推理。
                \item 肯定结论的谬误。
                \item 否定假设的谬误。
                \item 全称取拒式。
            \end{enumerate}
        \end{practices}

        %16.
        \begin{practices}
            \begin{enumerate}[A.]
                \item 全称取拒式。
                \item 否定假设的谬误。
                \item 肯定结论的谬误。
                \item 全称假言推理。
            \end{enumerate}
        \end{practices}

        %17.
        \begin{practices}
            不能选取任意的 $x$ 。
        \end{practices}

        %18.
        \begin{practices}
            不能选取任意的 $s$ 。
        \end{practices}

        %19.
        \begin{practices}
            \begin{enumerate}[A.]
                \item 肯定结论的谬误。
                \item 取拒式。
                \item 否定假设的谬误。
            \end{enumerate}
        \end{practices}

        %20.
        \begin{practices}
            \begin{enumerate}[A.]
                \item 肯定结论的谬误。
                \item 假言推理。
            \end{enumerate}
        \end{practices}

        %21.
        \begin{practices}
            取拒式。
        \end{practices}

        %22.
        \begin{practices}
            取拒式。
        \end{practices}

        %23.
        \begin{practices}
            化简律错误。
        \end{practices}

        %24.
        \begin{practices}
            化简律错误。
        \end{practices}

        %25.
        \begin{practices}
            \begin{align*}
                &\forall x (P(x) \rightarrow Q(x)) \\
                &P(a) \rightarrow Q(a) \\
                &\neg Q(a) \\
                &\neg P(a)
            \end{align*}
        \end{practices}

        %26.
        \begin{practices}
            通过全称实例,对于任意 $c$ ,都有 $P(c) \rightarrow Q(c)$ 和 $Q(c) \rightarrow R(c)$ 。
            于是对于任意 $c$ ,都有 $P(c) \rightarrow R(c)$ 。
            通过全称引入,有 $\forall x P(x) \rightarrow R(c)$ 。
        \end{practices}

        %27.
        \begin{practices}
            \begin{align*}
                &\forall x (P(x) \wedge R(x)) \\
                &\forall x P(x) \\
                &\forall x (P(x) \rightarrow (Q(x) \wedge S(x))) \\
                &\forall x (Q(x) \wedge S(x)) \\
                &\forall x R(x) \\
                &\forall x S(x) \\
                &\forall x (R(x \wedge S(x)))
            \end{align*}
        \end{practices}

        %28.
        \begin{practices}
            \begin{align*}
                &\forall x (P(x) \vee Q(x)) \\
                &\forall x (P(x) \vee \neg Q(x) \vee R(x)) \\
                &\forall x (P(x) \vee P(x) \vee R(x)) \\
                &\forall x (P(x) \vee R(x)) \\
                &\forall x (\neg R(x) \rightarrow P(x))
            \end{align*}
        \end{practices}

        %29.
        \begin{practices}
            \begin{align*}
                &\forall x (P(x) \vee Q(x)) \\
                &\forall x (\neg Q(x) \vee S(x)) \\
                &\forall x (P(x) \vee S(x)) \\
                &\forall x (\neg R(x) \vee \neg S(x)) \\
                &\forall x (\neg R(x) \vee P(x)) \\
                &\forall x (\neg P(x) \rightarrow \neg R(x)) \\
                &\exists x (\neg P(x) \rightarrow \neg R(x)) \\
                &\exists x (\neg P(x)) \\
                &\exists x (\neg R(x))
            \end{align*}
        \end{practices}

        %30.
        \begin{practices}
            $p$ :Allen是一个好男孩。
            $q$ :Hillary是一个好女孩。
            $r$ :David很开心。

            \begin{align*}
                &\neg p \vee q \\
                &p \vee r \\
                &q \vee r
            \end{align*}
        \end{practices}

        %31.
        \begin{practices}
            $p$ :天没下雨。
            $q$ :Yvette带雨伞了。
            $r$ :Yvette没有被淋湿。

            \begin{align*}
                &p \vee q \\
                &\neg q \vee r \\
                &p \vee r \\
                &\neg p \vee r \\
                &r
            \end{align*}
        \end{practices}

        %32.
        \begin{practices}
            \begin{align*}
                ((p \vee q) \wedge (\neg p \vee r)) \rightarrow (q \vee r)
                &\equiv ((p \vee F) \wedge (\neg p \vee F)) \rightarrow (F \vee F) \\
                &\equiv (p \wedge \neg p) \rightarrow F \\
                \therefore (p \wedge \neg p) \equiv F
            \end{align*}
        \end{practices}

        %33.
        \begin{practices}
            \begin{align*}
                (p \vee q) \wedge (\neg p \vee q) \wedge (p \vee \neg q) \wedge (\neg p \vee \neg q)
                &\equiv (q \vee q) \wedge (\neg q \vee \neg q) \\
                &\equiv q \wedge \neg q \\
                &\equiv F
            \end{align*}
        \end{practices}

        %34.
        \begin{practices}
            $p$ :逻辑是很容易的。
            $q$ :有许多学生喜欢逻辑。
            $r$ :数学是容易的。

            $\neg p \vee \neg q$

            $\neg r \vee p$

            \begin{enumerate}[A.]
                \item $\neg q \vee \neg r$ ;有效。
                \item $r \vee \neg q$ ;无效。
                \item $\neg r \vee \neg p$ ;无效。
                \item $p \vee \neg r$ ;无效。
                \item $q \vee (\neg r \vee p)$ 无效。
            \end{enumerate}
        \end{practices}

        %35.
        \begin{practices}
            有效。
        \end{practices}
    }
}
