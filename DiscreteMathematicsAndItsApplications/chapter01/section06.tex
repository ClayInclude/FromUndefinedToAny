%%
%% Author: Clay
%% 2019/11/25
%%

\section{推理规则}
{
    \subsection{引言}
    {
        数学中的证明是建立数学命题真实性的有效论证。
        所谓的\emreg{论证(argument)},是指一连串的命题并以结论为最后的命题。
        所谓\emreg{有效性(valid)},是指结论或论证的最后一个命题必须根据论证过程前面的命题或\emreg{前提(premise)}的真实性推出。
        一个论证是有效的当且仅当不可能出现所有前提为真而论证为假的情况。
        为从已知命题中推出新的命题,我们应用推理规则,这是构造有效论证的模板。
        推理规则是建立命题真实性的基本工具。
    }
}
