%%
%% Author: Clay
%% 2020/2/18
%%

\section{逻辑门电路}
{
    \subsection{引言}
    {
        本章所讨论的电路中,输出都只与输入有关,而与电路的当前状态无关。
        换句话说,这些电路都没有存储能力,这样的电路叫做\emreg{组合电路}或\emreg{门电路}。

        使用三种元件来构造组合电路,分别是\emreg{反相器}、\emreg{或门(OR gate)}和\emreg{与门(AND gate)}。
    }

    \subsection{门的组合}
    {
        使用反相器、或门和与门的组合可以构造组合电路。
    }

    \subsection{电路的例子}
    {

    }

    \subsection{加法器}
    {
        具有多个输出的电路称为\emreg{多重输出电路}。
    }

    \subsection{练习}
    {
        %1.
        \begin{practices}
            $(x + y)\bar{y}$
        \end{practices}

        %2.
        \begin{practices}
            $\overline{\bar{x}\bar{y}}$
        \end{practices}

        %3.
        \begin{practices}
            $\overline{xy} + (\bar{z} + x)$
        \end{practices}

        %4.
        \begin{practices}
            $\overline{\bar{x}yz}(\bar{x} + y + \bar{z})$
        \end{practices}

        %5.
        \begin{practices}
            $(x + y + z) + (\bar{x} + y + z) + (\bar{x} + \bar{y} + \bar{z})$
        \end{practices}

        %6.
        \begin{practices}

        \end{practices}

        %7.
        \begin{practices}
            $vwx + vwy + vwz + vxy + vxz + vyz + wxy + wxz + wyz + xyz$
        \end{practices}

        %8.
        \begin{practices}
            $w \otimes x \otimes y \otimes z$
        \end{practices}

        %9.
        \begin{practices}
            最低位使用半加器,进位传递给下一位,之后使用全加器。
        \end{practices}

        %10.
        \begin{practices}
            \begin{align*}
                D &= x \oplus y \\
                C &= \bar{x}y
            \end{align*}
        \end{practices}

        %11.
        \begin{practices}
            \begin{align*}
                D &= x \oplus y \oplus c \\
                C &= \bar{x}(y \oplus c) + yc
            \end{align*}
        \end{practices}

        %12.
        \begin{practices}
            同练习19。
        \end{practices}

        %13.
        \begin{practices}
            $\bar{x_1}x_0\bar{y_1}\bar{y_0} + x_1\bar{x_0}\bar{y_1} + x_1x_0\bar{y_1y_0}$
        \end{practices}

        %14.
        \begin{practices}
            \begin{align*}
                s_0 &= x_0y_0 \\
                s_1 &= (x_0y_1 + x_1y_0)\overline{(x_0x_1y_0y_1)} \\
                s_2 &= x_1y_1(\bar{x_0} + \bar{y_0}) \\
                s_3 &= x_0x_1y_0y_1
            \end{align*}
        \end{practices}

        %15.
        \begin{practices}

        \end{practices}

        %16.
        \begin{practices}

        \end{practices}

        %17.
        \begin{practices}

        \end{practices}

        %18.
        \begin{practices}

        \end{practices}

        %19.
        \begin{practices}
            $c_1c_0s_3 + c_1\bar{c_0}s_2 + \bar{c_1}c_0s_1 + \bar{c_1}\bar{c_0}s_0$
        \end{practices}

        %20.
        \begin{practices}
            \begin{enumerate}[A.]
                \item 2
                \item 3
                \item 3
                \item 6
            \end{enumerate}
        \end{practices}
    }
}
