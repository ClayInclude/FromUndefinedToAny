%%
%% Author: Clay
%% 2020/2/18
%%

\section{布尔函数}
{
    \subsection{引言}
    {
        布尔代数提供的是集合 $\{0, 1\}$ 上的运算和规则。

        布尔代数中的恒等式可以直接转换为复合命题中的等价式。
    }

    \subsection{布尔表达式和布尔函数}
    {
        设 $B = \{0, 1\}$ ,则 $B^n = \{(x_1, x_2, \cdots, x_n) | x_i \in B, 1 \leq i \leq n\}$ 是由 $0, 1$ 能构成的所有 $n$ 元组的集合。
        变元 $x$ 如果仅从 $B$ 中取值,则称该变元为\emreg{布尔变元}。
        从 $B^n$ 到 $B$ 的函数称为\emreg{$n$ 元布尔函数}。

        布尔函数也可由变元和布尔运算构成的表达式来表示。
        关于变元 $x_1, x_2, \cdots, x_n$ 的\emreg{布尔表达式}可以递归地定义如下:

        \begin{enumerate}
            \item $0, 1, x_1, x_2, \cdots, x_n$ 是布尔表达式。
            \item 如果 $E_1, E_2$ 是布尔表达式,则 $\bar{E_1}, E_1E_2, E_1 + E_2$ 是布尔表达式。
        \end{enumerate}

        每个布尔表达式表示一个布尔函数,此函数的值是通过在表达式中用 $0, 1$ 替换变元得到的。

        $n$ 个变元的布尔函数 $F$ 和 $G$ 是相等的,当且仅当 $F(b_1, b_2, \cdots, b_n) = G(b_1, b_2, \cdots, b_n)$ ,其中 $b_1, b_2, \cdots, b_n$ 属于 $B$ 。
        表示同一个函数的不同的布尔表达式称为是\emreg{等价的}。
        布尔函数 $F$ 的\emreg{补函数}是 $\bar{F}$ ,其中 $\bar{F}(x_1, \cdots, x_n) = \overline{F(x_1, \cdots, x_n)}$ 。
        设 $F$ 和 $G$ 是 $n$ 元布尔函数,函数的\emreg{布尔和} $F + G$ 与\emreg{布尔积} $FG$ 分别定义为:

        \begin{align*}
            (F + G)(x_1, \cdots, x_n) &= F(x_1, \cdots, x_n) + G(x_1, \cdots, x_n) \\
            (FG)(x_1, \cdots, x_n) &= F(x_1, \cdots, x_n)G(x_1, \cdots, x_n)
        \end{align*}

        乘积法则表明有 $2^{2^n}$ 个不同的 $n$ 元布尔函数。
    }

    \subsection{布尔代数恒等式}
    {
        将布尔恒等式、逻辑等价式和集合恒等式相比较,所有这些都是一个更抽象结构中恒等式集合的特殊情形。
        每个恒等式都可以通过适当的转换得到。
    }

    \subsection{对偶性}
    {
        一个布尔表达式的\emreg{对偶}可以用如下方法得到:交换布尔和与布尔积,并交换 $0$ 和 $1$ 。

        对于由布尔表达式表示的函数的恒等式,当取恒等式两边的函数的对偶时,等式仍然成立,此结果叫\emreg{对偶性}原理。
    }

    \subsection{布尔代数的抽象定义}
    {
        布尔函数和表达式所有的结论都可以转换成命题的结论,也可以转换成关于集合的结论。
    }

    \subsection{练习}
    {
        %1.
        \begin{practices}
            \begin{enumerate}[A.]
                \item 1
                \item 1
                \item 0
                \item 0
            \end{enumerate}
        \end{practices}

        %2.
        \begin{practices}
            \begin{enumerate}[A.]
                \item 0
                \item 0
                \item 0, 1
                \item 不存在
            \end{enumerate}
        \end{practices}

        %3.
        \begin{practices}
            \begin{enumerate}[A.]
                \item
                {
                    \begin{align*}
                        (1 \cdot 1) + (\overline{0 \cdot 1} + 0)
                        &= 1 + (1 + 0) \\
                        &= 1
                    \end{align*}
                }
                \item
                {
                    $(T \wedge T) \vee (\neg (F \wedge T) \vee F) = T$
                }
            \end{enumerate}
        \end{practices}

        %4.
        \begin{practices}
            \begin{enumerate}[A.]
                \item
                {
                    \begin{align*}
                        (\bar{1} \cdot \bar{0}) + (1 \cdot \bar{0})
                        &= 0 + 1 \\
                        &= 1
                    \end{align*}
                }
                \item
                {
                    $(\neg T \wedge \neg F) \vee (T \wedge \neg F) = T$
                }
            \end{enumerate}
        \end{practices}

        %5.
        \begin{practices}
            \begin{enumerate}[A.]
                \item
                {
                    \begin{table}[H]
                        \centering

                        \[
                            \begin{array}{c|c|c|c}
                                \hline
                                x & y & z & \bar{x}y \\
                                \hline
                                0 & 0 & 0 & 0 \\
                                0 & 0 & 1 & 0 \\
                                0 & 1 & 0 & 1 \\
                                0 & 1 & 1 & 1 \\
                                1 & 0 & 0 & 0 \\
                                1 & 0 & 1 & 0 \\
                                1 & 1 & 0 & 0 \\
                                1 & 1 & 1 & 0 \\
                                \hline
                            \end{array}
                        \]
                    \end{table}
                }
                \item
                {
                    \begin{table}[H]
                        \centering

                        \[
                            \begin{array}{c|c|c|c}
                                \hline
                                x & y & z & x + yz \\
                                \hline
                                0 & 0 & 0 & 0 \\
                                0 & 0 & 1 & 0 \\
                                0 & 1 & 0 & 0 \\
                                0 & 1 & 1 & 1 \\
                                1 & 0 & 0 & 1 \\
                                1 & 0 & 1 & 1 \\
                                1 & 1 & 0 & 1 \\
                                1 & 1 & 1 & 1 \\
                                \hline
                            \end{array}
                        \]
                    \end{table}
                }
                \item
                {
                    \begin{table}[H]
                        \centering

                        \[
                            \begin{array}{c|c|c|c}
                                \hline
                                x & y & z & x\bar{y} + \overline{xyz} \\
                                \hline
                                0 & 0 & 0 & 1 \\
                                0 & 0 & 1 & 1 \\
                                0 & 1 & 0 & 1 \\
                                0 & 1 & 1 & 1 \\
                                1 & 0 & 0 & 1 \\
                                1 & 0 & 1 & 1 \\
                                1 & 1 & 0 & 1 \\
                                1 & 1 & 1 & 0 \\
                                \hline
                            \end{array}
                        \]
                    \end{table}
                }
                \item
                {
                    \begin{table}[H]
                        \centering

                        \[
                            \begin{array}{c|c|c|c}
                                \hline
                                x & y & z & x(yz + \bar{y}\bar{z}) \\
                                \hline
                                0 & 0 & 0 & 0 \\
                                0 & 0 & 1 & 0 \\
                                0 & 1 & 0 & 0 \\
                                0 & 1 & 1 & 0 \\
                                1 & 0 & 0 & 1 \\
                                1 & 0 & 1 & 0 \\
                                1 & 1 & 0 & 0 \\
                                1 & 1 & 1 & 1 \\
                                \hline
                            \end{array}
                        \]
                    \end{table}
                }
            \end{enumerate}
        \end{practices}

        %6.
        \begin{practices}
            \begin{enumerate}[A.]
                \item
                {
                    \begin{table}[H]
                        \centering

                        \[
                            \begin{array}{c|c|c|c}
                                \hline
                                x & y & z & \bar{z} \\
                                \hline
                                0 & 0 & 0 & 1 \\
                                0 & 0 & 1 & 0 \\
                                0 & 1 & 0 & 1 \\
                                0 & 1 & 1 & 0 \\
                                1 & 0 & 0 & 1 \\
                                1 & 0 & 1 & 0 \\
                                1 & 1 & 0 & 1 \\
                                1 & 1 & 1 & 0 \\
                                \hline
                            \end{array}
                        \]
                    \end{table}
                }
                \item
                {
                    \begin{table}[H]
                        \centering

                        \[
                            \begin{array}{c|c|c|c}
                                \hline
                                x & y & z & \bar{x}y + \bar{y}z \\
                                \hline
                                0 & 0 & 0 & 0 \\
                                0 & 0 & 1 & 1 \\
                                0 & 1 & 0 & 1 \\
                                0 & 1 & 1 & 1 \\
                                1 & 0 & 0 & 0 \\
                                1 & 0 & 1 & 1 \\
                                1 & 1 & 0 & 0 \\
                                1 & 1 & 1 & 0 \\
                                \hline
                            \end{array}
                        \]
                    \end{table}
                }
                \item
                {
                    \begin{table}[H]
                        \centering

                        \[
                            \begin{array}{c|c|c|c}
                                \hline
                                x & y & z & x\bar{y} + \overline{xyz} \\
                                \hline
                                0 & 0 & 0 & 1 \\
                                0 & 0 & 1 & 1 \\
                                0 & 1 & 0 & 1 \\
                                0 & 1 & 1 & 1 \\
                                1 & 0 & 0 & 1 \\
                                1 & 0 & 1 & 1 \\
                                1 & 1 & 0 & 1 \\
                                1 & 1 & 1 & 0 \\
                                \hline
                            \end{array}
                        \]
                    \end{table}
                }
                \item
                {
                    \begin{table}[H]
                        \centering

                        \[
                            \begin{array}{c|c|c|c}
                                \hline
                                x & y & z & \bar{y}(xz + \bar{x}\bar{z}) \\
                                \hline
                                0 & 0 & 0 & 1 \\
                                0 & 0 & 1 & 0 \\
                                0 & 1 & 0 & 0 \\
                                0 & 1 & 1 & 0 \\
                                1 & 0 & 0 & 0 \\
                                1 & 0 & 1 & 1 \\
                                1 & 1 & 0 & 0 \\
                                1 & 1 & 1 & 0 \\
                                \hline
                            \end{array}
                        \]
                    \end{table}
                }
            \end{enumerate}
        \end{practices}

        %7.
        \begin{practices}

        \end{practices}

        %8.
        \begin{practices}

        \end{practices}

        %9.
        \begin{practices}
            都为 $1$ 或都为 $0$ 。
        \end{practices}

        %10.
        \begin{practices}
            $2^{(2^7)}$
        \end{practices}

        %11.
        \begin{practices}
            \begin{align*}
                x + xy
                &= x \cdot 1 + xy \\
                &= x(1 + y) \\
                &= x \cdot 1 \\
                &= x
            \end{align*}
        \end{practices}

        %12.
        \begin{practices}
            如果 $F$ 取值为 $1$ ,证明 $xy, xz, yz$ 中至少有一个为 $1$ ,则其中至少有两个为 $1$ 。
            如果其中至少有两个为 $1$ ,则 $xy, xz, yz$ 中至少有一个为 $1$ , $F$ 取值为 $1$ 。
        \end{practices}

        %13.
        \begin{practices}
            使用真值表即可。
        \end{practices}

        %14.
        \begin{practices}

        \end{practices}

        %15.
        \begin{practices}

        \end{practices}

        %16.
        \begin{practices}

        \end{practices}

        %17.
        \begin{practices}

        \end{practices}

        %18.
        \begin{practices}

        \end{practices}

        %19.
        \begin{practices}

        \end{practices}

        %20.
        \begin{practices}

        \end{practices}

        %21.
        \begin{practices}

        \end{practices}

        %22.
        \begin{practices}

        \end{practices}

        %23.
        \begin{practices}

        \end{practices}

        %24.
        \begin{practices}
            \begin{enumerate}[A.]
                \item $x$
                \item $\bar{x}$
                \item $0$
                \item $1$
            \end{enumerate}
        \end{practices}

        %25.
        \begin{practices}
            真值表。
        \end{practices}

        %26.
        \begin{practices}
            真值表。
        \end{practices}

        %27.
        \begin{practices}
            \begin{enumerate}[A.]
                \item T
                \item F
                \item F
            \end{enumerate}
        \end{practices}

        %28.
        \begin{practices}
            \begin{enumerate}[A.]
                \item $xy$
                \item $\bar{x} + \bar{y}$
                \item $(x + y + z)(\bar{x}\bar{y}\bar{z})$
                \item $(x + \bar{z})(x + 1)(\bar{x} + 0)$
            \end{enumerate}
        \end{practices}

        %29.
        \begin{practices}
            根据德·摩根律,对一个表达式的求补,除了变元取补外,其余等同于对偶律。
        \end{practices}

        %30.
        \begin{practices}
            由练习29可知, $F(\bar{x_1}, \cdots, \bar{x_n}) = G(\bar{x_1}, \cdots, \bar{x_n})$ 。
            由德摩根律可知,其相等。
        \end{practices}

        %31.
        \begin{practices}
            16
        \end{practices}

        %32.
        \begin{practices}
            ?
        \end{practices}

        %33.
        \begin{practices}

        \end{practices}

        %34.
        \begin{practices}

        \end{practices}

        %35.
        \begin{practices}

        \end{practices}

        %36.
        \begin{practices}

        \end{practices}

        %37.
        \begin{practices}

        \end{practices}

        %38.
        \begin{practices}

        \end{practices}

        %39.
        \begin{practices}

        \end{practices}

        %40.
        \begin{practices}

        \end{practices}

        %41.
        \begin{practices}

        \end{practices}

        %42.
        \begin{practices}

        \end{practices}

        %43.
        \begin{practices}
            ?
        \end{practices}
    }
}
