\chapter{序}
{
    什么是\emreg{什么是离散数学}?
    离散数学是数学中研究离散对象的部分。
    更一般地,每当需要对对象进行计数时,需要研究两个有限(或可数)集合之间的关系时,需要分析涉及有限步骤的过程时,就会用到离散数学。

    \section{为什么要学习离散数学}
    {
        首先,通过这个课程可以培养数学素质,即理解和构造数学论证的能力。
        其次,离散数学为许多计算机科学课程提供数学基础,这些课程包括数据结构、算法、数据库理论、自动机理论、形式语言、编译理论、计算机安全以及操作系统。
        以离散数学中研究的内容为基础的数学课程包括逻辑、集合论、数论、线性代数、抽象代数、组合学、图论及概率论。
        此外,离散数学还包含在运筹学、化学工程学以及生物学等领域问题求解所必需的数学基础。
    }
}
\cleardoublepage

\endinput
