\chapter{前言}
{
    \section{离散数学课程的目标}
    {
        \begin{description}
            \item[数学推理]
            {
                本书以数理逻辑开篇,在后面的论证方法的讨论中,数理逻辑是基础。
                本书描述了构造证明的方法与技巧。
                本书特别强调数学归纳法。
            }
            \item[组合分析]
            {
                一个重要的解题技巧就是计数或枚举对象。
                本书中,对枚举的讨论从基本的技术着手,重点是用组合分析方法来解决计数问题并分析算法。
            }
            \item[离散结构]
            {
                离散数学课程应该教会学生如何处理离散结构,即表示离散对象以及对象之间关系的抽象数学结构。
                离散结构包括集合、置换、关系、图、树和有限状态机等。
            }
            \item[算法思维]
            {
                在清楚地描述算法后,就可以构造一个计算机程序来实现它。
                这一过程中涉及的数学部分包括算法的详细说明、正确性验证以及执行算法所需要的计算机内存和时间的分析等,这些内容在本书中均有介绍。
            }
            \item[应用于建模]
            {
                离散数学几乎在每个可以想象到的研究领域都有应用,本书介绍了其在计算机科学和数据网络中的许多应用,还介绍了在其他各种领域中的应用,如化学、植物学、动物学、语言学、地理学、商业以及因特网等。
            }
        \end{description}
    }
}

\cleardoublepage

\endinput
