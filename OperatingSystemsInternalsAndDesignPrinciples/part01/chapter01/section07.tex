%%
%% Author: Clay
%% 2020/8/20
%%

\section{直接内存访问}
{
    I/O操作有3种可能的技术:可编程I/O、中断驱动I/O和直接内存访问方式。

    当处理器正在执行程序、并遇到一条与I/O相关的指令时,它通过给相应的I/O模块发命令来执行这条指令。
    使用\emreg{可编程I/O}操作时,I/O模块执行请求的动作并设置I/O状态寄存器中相应的位,它不执行其它的操作来通知处理器,尤其是不会中断处理器的执行。
    因此,处理器在执行I/O指令后,需要周期性地检测I/O模块的状态,以确定I/O操作是否完成。

    可编程I/O的问题是,处理器需要长时间的等待,以确定I/O模块是否做好接收或发送更多数据的准备。
    在长时间的等待期间,处理器必须不断地询问I/O模块的状态,其结果是严重影响了整个系统的性能。

    \emreg{中断驱动I/O(Interrupt-Driven I/O)}方式下,处理器向I/O模块发送I/O命令,然后继续处理其他一些有用的工作。
    当I/O模块做好与处理器交换数据的准备时,它将中断处理器的执行,并请求服务。

    尽管中断驱动I/O比简单的可编程I/O更加有效,但处理器仍需主动干预。
    在内存和I/O模块之间进行数据传输,并且任何数据传输都需要通过处理器来完成。
    这两种I/O方式在以下两个方面都存在固有的缺陷:

    \begin{itemize}
        \item I/O传输速度受限于处理器检测设备和提供服务的速度。
        \item 为了管理I/O传输,处理器需要执行较多的指令来完成每次I/O的传输。
    \end{itemize}

    当需要移动大量数据时,需要更有效的技术的支持:\emreg{直接内存访问(Direct Memory Access, DMA)}技术。
    DMA的工作模式为,当处理器尝试读或写一块数据时,会产生一条命令,向DMA模块发送如下信息:

    \begin{itemize}
        \item 是否请求读或写操作
        \item 需要访问的I/O设备的地址
        \item 读/写数据的内存起始地址
        \item 读/写数据的长度
    \end{itemize}

    DMA模块直接与存储器交互,传输整个数据块。
    当传输完成后,DMA模块向处理器发送中断信号。

    DMA模块需要控制总线,以便与存储器进行数据传输。
}
