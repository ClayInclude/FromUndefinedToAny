%%
%% Author: Clay
%% 2020/8/18
%%

\section{基本组成}
{
    计算机包括处理器、存储器和I/O组件,其中每个类型包含一个或多个模块。
    这些组件以某种方式相互连接,实现计算机执行程序的主要功能。

    \begin{description}
        \item[处理器] 控制计算机的操作,并执行数据处理的功能。
        \item[内存] 存储数据和程序。
        \item[I/O模块] 在计算机和外部环境之间交换数据。
        \item[系统总线] 提供处理器、主存,以及I/O模块之间的通信。
    \end{description}

    处理器的一个功能是与内存交换数据。
    它通常使用两个内部寄存器:\emreg{内存地址寄存器(Memory Address Register, MAR)}和\emreg{内存缓冲寄存器(Memory Buffer Register, MBR)}。
    \emreg{I/O地址寄存器(I/O Address Register, I/O AR)}用于指定一个特定的I/O设备。
    \emreg{I/O缓冲寄存器(I/O Buffer Register, I/O BR)}用于I/O模块和处理器之间的数据交换。

    内存由一组顺序编号的、由地址定义的单元组成。
    每个单元都包含一个二进制数,代表一条指令或数据。
}
