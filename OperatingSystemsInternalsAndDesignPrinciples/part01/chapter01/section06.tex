%%
%% Author: Clay
%% 2020/8/20
%%

\section{高速缓存}
{
    \subsection{动机}
    {
        处理器执行指令的速度受限于存储器周期。
        处理器的运行速度和内存的访问速度极度不匹配,解决方法是利用局部性原理在处理器和内存之间提供一个容量虽小、但速度较快的存储器,即高速缓存。
    }

    \subsection{高速缓存的原理}
    {
        高速缓存试图使其速度接近于最快的存储器,而容量也尽可能大,但价格与低廉的半导体存储器相当。
    }

    \subsection{高速缓存的设计}
    {
        在设计虚拟存储器和磁盘高速缓存时,必须解决类似的问题:

        \begin{itemize}
            \item 高速缓存的容量(Cache Size)
            \item 块的大小(Block Size)
            \item 映射函数(Mapping Function)
            \item 置换算法(Replacement Algorithm)
            \item 写策略(Write Policy)
            \item 高速缓存的级数(Number of Cache Levels)
        \end{itemize}

        小容量的高速缓存对性能会产生较大的影响。

        高速缓存和内存间进行数据交换的单元的大小。
        随着块大小的增加,每载入一个数据块,更多的有用数据也被加载到高速缓存中。
        由于局部性原理,数据的命中率也会提高。
        但当数据块增大到一定程度时,如果还继续增大,命中率反而会下降,这是因为预先载入的数据被访问的可能性会低于再次使用那些需要移除高速缓存的数据的可能。

        当一个新的数据块被读入高速缓存时,\emreg{映射函数(Mapping Function)}决定该数据块占用高速缓存的哪个单元。
        映射函数的设计需要考虑两方面的约束:
        置换方法应该尽可能最小化被替换的块近期再次被访问的概率。
        映射函数越灵活,所需的电路就越复杂,因为需要检索高速缓存来确定指定的块是否在高速缓存中。

        一个新的数据块读入高速缓存,而高速缓存中所有的存储\emreg{槽(Slots)}已被其他数据块占满时,\emreg{置换算法(Replacement Algorithm)}用于选择替换到哪一个数据块。
        合理且有效的策略是置换高速缓存中最长时间没有被访问的数据块。
        该策略称为\emreg{最长未使用(Least-Recently-Used, LRU)}算法。

        如果高速缓存中某个数据块的内容被修改,则需要在它被换出高速缓存之前将其写回内存。
        \emreg{写策略(Write Policy)}用于确定写回内存的时间。
        一种极端的情况是,只要数据块被更新,就执行写操作。
        另一种极端情况是,只有在该数据块被替换时才执行写操作。
        后一种策略可以最大限度的减少内存写操作执行的次数。
        但让内存处于一种过时的状态,这会干扰多处理器的操作,以及I/O硬件模块的直接内存访问。

        最常用的方法是设计多级缓存。
    }
}
