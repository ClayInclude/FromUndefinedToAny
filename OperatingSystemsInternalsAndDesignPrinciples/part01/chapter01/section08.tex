%%
%% Author: Clay
%% 2020/8/25
%%

\section{多处理器和多核组织结构}
{
    一般情况下,计算机被视为一台顺序执行的机器。
    在微操作级别,同一时间会产生多个控制信号。
    长期以来,指令流水线技术在一定程度上可以重叠取指和执行操作的执行。

    并行处理进一步提升系统性能,通过复制处理器来提供并行的能力:
    \emreg{对称多处理器(Symmetric Multiprocessors, SMPs)},\emreg{多核计算机(Multicore Computer)}和\emreg{集群(Clusters)}。

    \subsection{对称多处理器}
    {
        一个对称多处理器SMP可以定义为具有特点的、独立的计算机系统:

        \begin{itemize}
            \item 具有两个或两个以上性能相当的处理器
            \item 处理器共享内存和I/O设备,并通过总线或其他内部连接方式互连
            \item 所有处理器通过相同的通道或连接到相同设备的不同通道共享对I/O设备的访问
            \item 所有处理器能执行相同的功能(称之为对称)
            \item 整个系统由操作系统进行控制
        \end{itemize}

        相比于单处理器结构,SMP结构具有很多潜在的优势:

    \begin{itemize}
        \item 性能
        \item 可用性
        \item 增量扩展
        \item 可伸缩性
    \end{itemize}

    SMP的一个显著特性是,多处理器的存在对用户来说是透明的,操作系统负责调度任务到每个处理器上,并负责处理器之间的同步。
    }

    \subsection{组织结构}
    {
        SMP有多个处理器,每个处理器都有自己的控制单元、算术逻辑单元和寄存器。
        每个处理器都可以通过某种形式
    }
}
