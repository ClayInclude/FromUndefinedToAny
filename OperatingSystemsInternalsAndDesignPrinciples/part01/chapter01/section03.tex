%%
%% Author: Clay
%% 2020/8/18
%%

\section{指令的执行}
{
    处理器执行的程序由一组保存在存储器中的指令组成。
    程序的执行就是重复的读取和执行指令的过程。

    一条指令的处理过程称为一个指令周期。
    指令周期可简述成取指阶段和执行阶段。

    每个指令周期开始时,处理器从内存读取一条指令。
    通常,\emreg{程序计数器(Program Counter)}保存下一次需读取的指令地址。

    读取的指令存放在处理器的\emreg{指令寄存器(Instruction Register, IR)}中。
    指令中包含用于指定处理器将要执行的操作的信息,处理器解释该指令并执行对应的操作。

    通常,执行的操作可以分为4种类型:

    \begin{description}
        \item[处理器--内存] 数据从处理器传递到内存,或从内存传递到处理器。
        \item[处理器--I/O] 通过处理器和I/O模块间的数据传输,数据输出到外部设备,或从外部设备输入数据。
        \item[数据处理] 处理器执行数据的算数或逻辑操作。
        \item[控制] 某些指令可以使指令的执行顺序产生变化。
    \end{description}

    大多数现代处理器都具有包含多个地址的指令,因此在一个指令周期中可能会产生多次内存访问。
    此外,除了内存访问外,还可能产生I/O操作的执行。
}