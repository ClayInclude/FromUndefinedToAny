%%
%% Author: Clay
%% 2020/8/18
%%

\section{中断}
{
    所有计算机都提供中断机制,允许其他模块中断处理器的正常运行。

    \begin{table}[htb]
        \begin{tabular}{c|c}
            \hline
            程序中断 & 当某些条件成立时由指令的执行结果产生,如算术溢出、被零除、试图执行非法的机器指令或内存空间访问越界等 \\
            \hline
            时钟中断 & 由处理器内部的计时器触发产生,允许操作系统有规律的执行某些特定操作 \\
            \hline
            I/O中断 & 由I/O控制器产生,通知已完成某项操作或产生某种错误状态 \\
            \hline
            硬件故障中断 & 出现故障时产生,如电源故障、内存奇偶校验错等 \\
            \hline
        \end{tabular}
    \end{table}

    最初,中断是提高处理器效率的一种手段。

    \subsection{中断和指令周期}
    {
        有了中断,处理器可以在I/O操作的执行过程中执行其他指令。

        中断打断了指令序列的正常执行。
        当中断处理完成后,程序将恢复执行。

        为了适应中断,指令周期中需要加入一个中断阶段。
        在中断阶段中,处理器检测是否有中断产生。
        如果没有中断,处理器继续运行,并在取指阶段读取当前程序的下一条指令。
        如果有中断,处理器中断当前程序的执行,并执行相应的中断处理程序。

        这个处理过程存在一定的开销。
        在中断处理程序中,必须通过额外指令的执行来确定中断的类型,并据此选取合适的操作。
    }

    \subsection{中断处理}
    {
        常见的中断处理过程:

        \begin{enumerate}
            \item 该设备发送中断信号给处理器。
            \item 处理器在响应该中断前,结束当前指令的执行。
            \item 处理器评测中断请求,确定存在未响应的中断,并给发送该中断的设备发送确认信息,通知该设备移除该中断信号。
            \item 处理器做好将控制权交给中断程序的准备:在断点保存恢复当前程序执行所需要的信息。
            \item 处理器将响应该中断的中断处理程序入口地址装入程序计数器。
            \item 此时,被中断程序的程序地址和车工内需状态字PSW保存在系统栈中,此外还有其他一些信息需要保存,尤其是处理器寄存器的内容。
            \item 中断处理程序开始处理中断。
            \item 当完成中断处理后,从栈中恢复被保存的寄存器信息,还原至相关寄存器。
            \item 最后一步操作是从栈中恢复程序状态字PSW和程序计数器的值,以便执行位于被中断的程序断点处的下一条指令。
        \end{enumerate}

        中断不是程序执行时调用的一个例程,它可以在用户程序执行的任何一个时刻或任何一个执行点发生,是无法预知的。
    }

    \subsection{多个中断}
    {
        考虑当处理中断时,又产生了一个或多个中断的情况。

        处理多个中断有两种方法。
        第一种方法是中断处理过程中禁止其他的中断。
        禁止中断,是指处理器忽略忽略所有新的中断请求信号。
        如果在此期间发生了中断,通常挂起该中断,只有当处理器允许中断时,再由处理器检测该中断。
        所有中断都严格按顺序处理。

        上述方法的缺点是没有考虑相对优先级和时间限制的要求。

        第二种方法是为所有中断指定优先级,并允许高优先级的中断打断低优先级的中断处理程序的执行。
    }
}
