%%
%% Author: Clay
%% 2020/9/7
%%

\section{复习题}
{
    %1.
    \begin{reviewc}
        处理器、内存、I/O模块和系统总线。
    \end{reviewc}

    %2.
    \begin{reviewc}
        内存地址寄存器(MAR)和内存缓冲寄存器(MBR)。
    \end{reviewc}

    %3.
    \begin{reviewc}
        处理器-内存、处理器-I/O、数据处理和控制。
    \end{reviewc}

    %4.
    \begin{reviewc}
        为所有中断指定优先级,并允许高优先级的中断打断低优先级中断处理程序的执行。
    \end{reviewc}

    %5.
    \begin{reviewc}
        产生中断时,开始执行中断服务例程。
        如果产生了新的高优先级中断,则终止该终端服务例程,执行新的终端服务例程,执行完成之后,处理器恢复之前的状态继续执行之前的中断服务。
        如果低优先级中断,则该中断被挂起,知道当前中断服务例程执行完毕。
    \end{reviewc}

    %6.
    \begin{reviewc}
        \begin{enumerate}[A.]
            \item 单位成本逐层递减。
            \item 容量逐层递增。
            \item 访问时假案逐层递增。
            \item 处理器访问存储器的频率逐层递减。
        \end{enumerate}
    \end{reviewc}

    %7.
    \begin{reviewc}
        \begin{enumerate}[A.]
            \item 高速缓存的容量。
            \item 块的大小。
            \item 映射函数。
            \item 置换算法。
            \item 写策略。
            \item 高速缓存的级数。
        \end{enumerate}
    \end{reviewc}

    %8.
    \begin{reviewc}
        多处理器系统是具有两个或两个以上性能相当的处理器,多核系统是指一块芯片上有多个处理器。
    \end{reviewc}

    %9.
    \begin{reviewc}
        一种情况是只要数据块被更新,就执行写操作。
        另一种情况是,只有在该数据块被替换时才执行写操作。
    \end{reviewc}

    %10.
    \begin{reviewc}
        一旦进入循环或开始执行子程序,就会重复访问某个范围内的指令集合。
        表和数组的操作也会涉及对某些聚集数据的多次访问。
    \end{reviewc}
}
