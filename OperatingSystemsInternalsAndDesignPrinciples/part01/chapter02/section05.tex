%%
%% Author: Clay
%% 2020/11/13
%%

\section{容错}
{
    容错是指系统或组件出现在硬件或软件故障时继续执行常规操作的能力,通常涉及到某种程度的冗余,其目的是提高系统的可靠性。
    通常,增加了容错,就增强了可靠性,但由此带来的是成本增大,包括经济上的或性能上的,甚至两者兼而有之。
    因此,容错措施的采纳程度取决于资源的重要程度。

    \subsection{基本概念}
    {
        衡量系统容错能力的指标有可靠性,平均无故障时间(MTTF)和可用性。

        系统的\emreg{可靠性(Reliability)} $R(t)$ 定义为在运行时间 $t$ 内正常运行的概率。
        对计算机系统和操作系统而言,正常运行是指一套程序的正确执行且确保数据不被非法修改。

        \emreg{平均无故障时间(Mean Time To Failure, MTTF)}定义为

        \begin{align*}
            MTTF = \int^{\infty}_0 R(t)
        \end{align*}

        \emreg{平均修复时间(Mean Time To Repair, MTTR)}是修复或替换故障元素的平均时间:

        \begin{align*}
            MTTR = \int^{\infty}_0 1 - R(t)
        \end{align*}

        可用性是实体在单位时间内,在给定条件下正常运行的能力。
        系统不可用的这段时间称为\emreg{停机时间(Downtime)};
        系统处于可用状态的时间称为\emreg{运行时间(Uptime)}。
        系统可用性 $A$ 可表示为:

        \begin{align*}
            A = \frac{MTTF}{MTTF + MTTR}
        \end{align*}
    }

    \subsection{故障}
    {
        在IEEE标准字典中,\emreg{故障(Fault)}被定义为因部件失效、操作错误、来自环境的物理干扰、设计错误、程序错误或数据结构的错误而导致的不正确的硬件或软件状态。

        故障可表现为:

        \begin{itemize}
            \item 硬件设备或部件上的缺陷,如短路或断线;
            \item 计算机程序中不正确的步骤、过程或数据定义。
        \end{itemize}

        故障可以分为以下几种类型:

        \begin{description}
            \item[永久型(Permanent)] 该类故障产生后将一直存在,直至失效的部件被跟换或修复。
            \item[临时型(Temporary)]
            {
                该类故障并不是在所有操作条件下的任何时刻都会出现。
                临时型故障又可分为两类:

                \begin{description}
                    \item[短暂型(Transient)] 只出现一次的故障,如由脉冲噪声引起的位传输错误、电源干扰、改变内存位的辐射等。
                    \item[间歇型(Intermittent)] 在多个不可预知的时间内发生的故障,如因松散连接引起的故障。
                \end{description}
            }
        \end{description}

        一般情况下,可以在系统中采用增加冗余的方法来实现容错。
        冗余方法包括以下几种:

        \begin{description}
            \item[空间(物理)冗余] 物理冗余包括多个部件的使用,这些部件可以同时执行相同的功能,或者将一个部件配置成另一个部件的备份,这样该部件故障后备份不剪仍然可用。
            \item[时间冗余]
            {
                时间冗余是指在检测到错误时,重复执行该函数或操作。
                该方法对临时型故障非常有效,但不适用于永久型故障。
            }
            \item[信息冗余] 信息冗余通过复制或数据编码来检测和纠正错误,以实现容错。
        \end{description}
    }

    \subsection{操作系统中的机制}
    {
        操作系统软件中使用了很多技术支持容错:

        \begin{description}
            \item[进程隔离]
            {
                多个进程在内存、文件访问和执行流程等方面相互隔离。
                操作系统为管理进程提供的结构在一定程度上对进程进行保护,如果某个进程出错,不会影响其他的进程。
            }
            \item[并发控制] 进程通信或协作时可能出现的一些困难和故障,会讨论保证正确操作和恢复故障所用到的技术。
            \item[虚拟机] 虚拟机提供更大程度的应用隔离。
            \item[检查点和回滚]
            {
                检查点是保存某个存储设备中的应用程序状态的备份,从而可以对故障免疫。
                回滚操作从当前保存的检查点重新开始执行。
                当产生故障时,应用程序的状态将回滚到当前检查点,并从该点重新开始执行。
            }
        \end{description}
    }
}
