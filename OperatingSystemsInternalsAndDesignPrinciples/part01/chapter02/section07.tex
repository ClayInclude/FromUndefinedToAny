%%
%% Author: Clay
%% 2020/11/13
%%

\section{微软Windows简介}
{
    \subsection{背景}
    {
        微软于1985年开始使用Windows这一名称,旨在对MS-DOS操作系统的操作环境进行扩展,并成功应用于早期的个人计算机上。
    }

    \subsection{体系结构}
    {
        和几乎所有的操作系统一样,Windows系统把面向应用的软件和操作系统内核软件分离开来,后者包括运行在内核模式的执行体、核心、设备驱动和硬件抽象层。

        \subsubsection{操作系统组织}
        {
            Windows是高度模块化的系统。
            每个系统函数都由操作系统的一个组件进行管理,操作系统的其余部分和所有应用程序通过相应组件提供的标准接口访问该函数。
            核心的系统数据只能通过相应的函数访问。
            理论上,任何模块都可以移动、升级或替换,而无需重写整个系统或其标准应用编程接口(API)。

            Windows内核态的组件包括以下五个部分:

            \begin{description}
                \item[执行体] 包括操作系统核心服务,如内存管理、进程和线程管理、安全、I/O和进程间通信。
                \item[内核]
                {
                    控制处理器的执行。
                    内核负责管理线程调度、进度切换、异常和中断处理、多核处理器同步。
                    不同于执行体和用户级程序,内核代码不以线程方式运行。
                }
                \item[硬件抽象层(Hardware Abstract Layer, HAL)]
                {
                    在通用的硬件命令、响应、与某特定平台专用的硬件命令和相应之间进行映射.
                    它将操作系统从与平台相关的硬件差异中隔离出来。
                    HAL使每个机器的系统总线、直接存储器访问(DMA)控制器、中断控制器、系统计时器和内存控制器等看起来对执行体和内核组件都是相同的。
                }
                \item[设备驱动]
                {
                    扩展执行程序功能的动态库,包括硬件设备驱动程序,可以将用户I/O函数调用转换成特定硬件设备的I/O请求,同时还包括一些软件组件,用于实现文件系统、网络协议和其他需要运行在内核态的系统扩展功能。
                }
                \item[窗口和图形系统] 实现GUI函数,如处理窗口、用户接口控制和图形绘制等。
            \end{description}
        }
    }
}
