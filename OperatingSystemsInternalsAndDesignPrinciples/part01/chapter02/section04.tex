%%
%% Author: Clay
%% 2020/11/2
%%

\section{现代操作系统的发展}
{
    这些现代操作系统能对新硬件的发展、新应用和新的安全威胁做出响应。
    促使操作系统发展的硬件因素主要有多处理器系统、速度越来越快的处理器、高速网络连接和容量与类型不断增加的存储设备。
    多媒体应用、互联网和Web访问、客户端/服务器计算等应用领域也对操作系统的设计产生了影响。
    在安全性方面,互联网访问方式极大地增加了对计算机的潜在威胁和更加复杂的攻击,如病毒、蠕虫和黑客技术,这些都对操作系统的设计产生了深远的影响。

    操作系统大致可归为以下几类:

    \begin{itemize}
        \item 微内核体系结构
        \item 多线程
        \item 对称多处理
        \item 分布式操作系统
        \item 面向对象的设计
    \end{itemize}

    迄今为止,大多数操作系统都具有\emreg{大单内核(Monolithic Kernel)}特征,操作系统的大多数功能都由这些大内核提供,包括调度、文件系统、网络、设备驱动和内存管理等。
    \emreg{微内核(Micro Kernel)}体系结构只将一些最基本的功能放入内核,包括地址空间,\emreg{进程间通信(Interprocess Communication, IPC)}和基本的调度等。
    其他的操作系统服务由运行在用户态、与其他应用程序类似的进程提供,有时也称为服务。

    \emreg{Multithreading}技术是将一个执行应用程序的进程划分成可以并发执行的多个线程。
    线程和进程的区别主要有:

    \begin{description}
        \item[线程(Thread)]
        {
            线程是可分派的工作单元,包括处理器上下文和堆栈中自己的数据区域。
            线程顺序执行,并且是可中断的,因此处理器可以转到另一个线程。
        }
        \item[进程(Process)]
        {
            进程由一个或多个线程,以及相关的系统资源组成,是程序的一次执行过程。
            将一个应用程序分解成多个线程,程序员可以在很大程度上控制应用程序的模块性和与应用程序相关的事件的执行时机。
        }
    \end{description}

    多线程对本质上相互独立、无需线性串行处理的、以多任务方式执行的应用程序
}
