%%
%% Author: Clay
%% 2020/11/16
%%

\section{复习题}
{
    %1.
    \begin{reviewc}
        便利性、有效性和扩展能力。
    \end{reviewc}

    %2.
    \begin{reviewc}
        用户态和内核态。
    \end{reviewc}

    %3.
    \begin{reviewc}
        当一个作业需要等待I/O时,处理器可以转去执行另一个不需要等待I/O的作业。
    \end{reviewc}

    %4.
    \begin{reviewc}
        \begin{enumerate}
            \item 一个正在执行的程序。
            \item 程序所需的相关数据。
            \item 程序执行的上下文。
        \end{enumerate}
    \end{reviewc}

    %5.
    \begin{reviewc}
        进程表记录每个进程的入口,包括指向存放该进程的内存空间的地址指针,还包括该进程的部分或全部执行上下文。
    \end{reviewc}

    %6.
    \begin{reviewc}
        \begin{enumerate}
            \item 进程隔离
            \item 自动分配和管理
            \item 支持模块化编程
            \item 保护和访问控制
            \item 永久存储
        \end{enumerate}
    \end{reviewc}

    %7.
    \begin{reviewc}
        分配给每个正在运行的进程微观上的一段CPU时间。
    \end{reviewc}

    %8.
    \begin{reviewc}
        在固定的时间间隔内产生中断,处理器恢复控制权,当前用户被抢占,将处理器分配给另一个用户。
    \end{reviewc}

    %9.
    \begin{reviewc}
        单内核:将OS的全部功能做进内核中,并作为一个单一进程运行,具有唯一地址空间。

        微内核:只将OS中最核心的功能加入内核。
        而其他功能作为处于用户态的进程而向外提供某种服务。
    \end{reviewc}

    %10.
    \begin{reviewc}
        将一个执行应用程序的进程划分成可以并发执行的多个线程。
    \end{reviewc}

    %11.
    \begin{reviewc}
        许多独立的,网络连接的,通讯的,并且物理上分离的计算节点的集合。
        每个节点包含全局总操作系统的一个特定的软件子集。
    \end{reviewc}
}
