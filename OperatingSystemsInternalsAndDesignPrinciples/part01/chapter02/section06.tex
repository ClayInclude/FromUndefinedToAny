%%
%% Author: Clay
%% 2020/11/13
%%

\section{多处理器和多核操作系统设计设计要考虑的因素}
{
    \subsection{对称多处理器操作系统设计上的考虑}
    {
        设计时考虑的关键问题主要有一下5个方面:

        \begin{description}
            \item[并发进程或线程]
            {
                内核程序应该是可重入的,以使多个处理器能同时执行同一段内核代码。
                当多个处理器执行内核的相同或不同部分时,需要提供合适的内核表和管理用的数据结构,以防止数据损坏和无效的操作。
            }
            \item[调度]
            {
                任何一个处理器都可以执行调度,这既增加了执行调度策略的复杂性,也增加了确保调度用数据结构不被破坏的难度。
                如果使用内核级多线程方式,就存在将同一进程的多个线程同时调度到多个处理器上的可能。
            }
            \item[同步]
            {
                多个活跃进程有可能会访问共享的地址空间或共享的I/O设备,因此必须认真考虑如何提供有效的同步机制的问题。
                同步用来实现互斥和时间排序。
                在多处理器操作系统中,锁机制是常见的一种同步机制。
            }
            \item[内存管理]
            {
                多处理器上的内存管理需要处理单处理器系统中内存管理相关的所有问题。
                操作系统还需要充分利用硬件提供的并行能力达到性能的最优化。
                不同处理器上的分页机制必须相互协作,以实现多处理器共享页或段时数据的一致性,并完成页面置换。
                物理页面的重复使用是需要关注的最大问题,既必须保证页面在被重新使用之前,页面中原有的内容不会被再次访问。
            }
            \item[可靠性和容错]
            {
                当处理器故障时,操作系统应该尽可能减轻其影响。
                调度器和操作系统的其他部分必须能识别产生故障的处理器,并重构管理表。
            }
        \end{description}
    }

    \subsection{多核操作系统设计上的考虑}
    {
        多核系统的设计需要考虑SMP系统设计时要考虑的所有问题,但会更多地关注于并行规模的问题。

        众核系统的设计挑战是,如何有效利用多和处理能力以及如何智能且有效地管理芯片上的资源。
        关注的焦点是如何将众核系统固有的并行能力与应用程序的性能需求相匹配。
        可以从下面3个层次开发当前多核系统潜在的并行能力:
        首先是每个核内部的硬件并行,即指令级并行;
        其次是每个处理器上并发执行多道程序或多个线程的并行能力;
        最后是应用程序在多核上并发执行多个进程或多个线程的并行能力。

        \subsubsection{应用层并行}
        {
            原则上,大多数应用可细分为能并行执行的多个任务,这些任务可以以多进程或多线程的方式实现。
            其难点在于开发人员必须决定如何将应用程序分割成多个可独立执行的任务。

            支持开发者的最有效的举措之一是多线程优化技术(Grand Central Dispatch, GCD)。

            本质上,GCD是一种线程池机制,操作系统可以将任务映射成代表并发可用程度的线程(以及用于I/O阻塞的线程)。
        }

        \subsection{虚拟机方式}
        {
            另一种方式是必须意识到,随着单芯片上内核数量的不断增长,在内核上尝试多道程序设计以支持多个应用程序,可能会造成资源的错误使用。
            相反,如果为某个进程分配一个或多个内核,并让处理器处理该进程,就能避免过多的任务切换和调度策略的开销。
            这样,多核操作系统就可以作为管理程序,在较高的层次实现调度策略,以确定如何将内核分配给应用程序,而不处理其他的资源分配。

            多道程序设计技术基于进程这一概念,进程是执行环境的抽象。
        }
    }
}
