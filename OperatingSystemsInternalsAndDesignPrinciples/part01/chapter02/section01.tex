%%
%% Author: Clay
%% 2020/9/8
%%

\section{操作系统的目标和功能}
{
    操作系统的设计应遵循以下3个目标。

    \begin{description}
        \item[便利性] 使计算机更易于使用。
        \item[有效性] 以更有效的方式使用系统资源。
        \item[扩展能力] 在构造操作系统时,能在不影响正常服务的前提下进行有效的开发、测试,并引入新的系统功能。
    \end{description}

    \subsection{作为用户与系统交互接口的操作系统}
    {
        操作系统提供了一系列系统程序,其中部分程序称为实用工具或库函数,用于在创建程序、管理文件和控制I/O设备时实现经常使用的功能。
        操作系统为程序员屏蔽了硬件细节,并为其提供了方便的编程接口以使用系统。

        操作系统通常从以下几个方面提供服务:

        \begin{description}
            \item[程序的开发]
            {
                操作系统提供一系列工具和服务,严格来说并不属于操作系统内核的一部分。
                它们由操作系统提供,作为应用程序开发工具提供给用户使用。
            }
            \item[程序的运行]
            {
                运行一个程序需要很多步骤,包括把指令和数据加载到内存,初始化I/O设备和文件,准备好其他将使用的资源等。
                操作系统为用户处理这些调度问题。
            }
            \item[I/O设备的访问]
            {
                每个I/O设备的操作都需要自己特有的指令集和控制信号。
                操作系统提供一个统一的接口。
            }
            \item[系统的访问]
            {
                操作系统对整个系统和某些特殊的系统资源的访问进行控制。
                访问模块必须能够提供对资源和数据的保护,以避免未授权用户的访问,同时还必须解决资源竞争时产生的冲突问题。
            }
            \item[错误检测和响应]
            {
                在计算机系统的运行过程中,可能会产生各种各样的错误,包括来自系统硬件内部或外部的错误,以及各种软件错误。
                无论何种情况,操作系统都必须做出响应,在尽量不影响正在执行的应用程序的情况下清楚该错误。
                响应操作可以是终止引起错误的程序、重试该操作,也可以是将错误报告给应用程序。
            }
            \item[日志]
            {
                优秀的操作系统会收集使用各种资源的统计数据,并监测性能参数。
            }
        \end{description}

        典型计算机系统中的3种主要接口:

        \begin{description}
            \item[指令集架构(Instruction Set Architecture, ISA)]
            {
                ISA定义了计算机遵循的机器语言指令系统,该接口是硬件和软件的分界线。
            }
            \item[应用程序二进制接口(Application Binary Interface, ABI)]
            {
                ABI定义了在程序间进行二进制移植的标准,它定义了操作系统的系统调用接口,以及在系统中通过用户级ISA能使用的硬件资源和服务。
            }
            \item[应用程序编程接口(Application Programming Interface, API)]
            {
                API允许应用程序访问系统的硬件资源和,这些服务由用户级ISA和\emreg{高级语言(High-Level Language, HLL)}库调用来提供。
            }
        \end{description}
    }

    \subsection{作为资源管理器的操作系统}
    {

    }
}
