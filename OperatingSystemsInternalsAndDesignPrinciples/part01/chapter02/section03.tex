%%
%% Author: Clay
%% 2020/9/14
%%

\section{主要成就}
{
    \subsection{进程}
    {
        \emreg{进程(Process)}的概念是操作系统设计的核心。
        有很多关于进程这个术语的定义:

        \begin{itemize}
            \item 一个正在执行的程序。
            \item 程序在计算机上运行的一个实例。
            \item 可以分配给处理器,并由处理器执行的一个实体。
            \item 由一个顺序执行的线程、一个当前状态和一组相关的系统资源所构成的活动单元。
        \end{itemize}

        计算机系统发展的第一条主线是多道程序设计技术让处理器和I/O设备同时处于忙碌状态,以达到最大效率。
        它的关键机制是:
        为响应表示I/O事务完成的信号,操作系统对驻留在内存中的各种程序进行处理器切换。

        计算机系统发展的第二条主线是通用目的的分时,其主要目标是能及时响应单个用户的请求。
        在计算中还必须考虑操作系统的开销因素。

        计算机系统发展的第三条主线是实时事务处理系统。
        这种情况下,大量用都需要查询或修改数据库。
        事务处理系统和分时系统的主要区别在于前者局限于一个或几个应用,而分时系统的用户可以编写程序、执行作业和使用各种应用程序。
        对这两种情况,系统响应时间都是最重要的。

        系统程序员在开发早期的多道程序设计和多用户交互系统时使用的主要工具是中断。
        任何一个作业的活动都可能因某个已定义的事件而暂停执行。
        处理器会保存某些上下文,然后跳转到中断处理程序,确定中断类型并处理该中断,然后恢复用户被中断的作业或其他作业的处理。
    }
}
