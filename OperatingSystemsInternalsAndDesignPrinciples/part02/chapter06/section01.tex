%%
%% Author: Clay
%% 2020/12/5
%%

\section{死锁的原理}
{
    死锁是指一组进程因为竞争系统资源或互相等待消息,而永远无法向前推进的状态。

    \subsection{可重用资源}
    {
        资源可以分为两个大类:
        可重用资源与消耗性资源。

        可重用资源指的是一次只供一个进程使用,但用后又能被别的进程使用的资源。

        从系统的角度出发,能够解决死锁问题的方法之一是对应用申请系统资源的顺序做出限定。
    }

    \subsection{消耗性资源}
    {
        消耗性资源指的是可被创建和销毁的资源。
        一般来说,对这类资源的使用没有数量上的限制。

        不存在解决所有死锁问题的单一有效方案。
    }

    \subsection{资源分配图}
    {
        描述进程对资源占用情况的一个有效工具是资源分配图。
        资源分配图是一个有向图,图中的每个顶点表示一个进程或资源,从进程顶点指向资源顶点的边表示进程申请资源但还未被满足的情况。
        在资源顶点内部的小黑点表示该资源的一个实例。
        资源分配图中由可重用资源顶点中的小黑点指向进程顶点的有向边表示该资源已有进程获得。
        如果资源时消耗性的,则表明进程是该资源的生产者。

        存在资源分配环路,导致了死锁的发生。
    }

    \subsection{死锁发生的条件}
    {
        死锁的发生有以下3个条件:

        \begin{description}
            \item[互斥] 资源一次只能由一个进程使用。
            \item[占用并等待] 进程必须占用分配给它的资源。
            \item[不可剥夺] 如果资源被某进程占用,则不可将资源强行从占用它的进程中剥夺出来。
        \end{description}

        这三个条件是死锁发生的必要条件,而非充分条件。
        死锁的发生,还需要第四个条件:

        \begin{description}
            \item[环路等待] 进程和资源的分配关系在资源分配图上表现为一个封闭的环路,其中,每个进程至少占用一个资源,且该资源同时被环路中的其他进程申请。
        \end{description}

        如果前3个条件成立的话,致命区域才会存在。
        如果环路等待条件也成立,就意味着进程的共同推进路线进入了致命区域。

        解决死锁问题有3个方案:
        其一,\emreg{死锁预防(Prevention)}方案是破除死锁发生的4个条件之一;
        其二,\emreg{死锁避免(Avoidance)}方案是基于资源分配的当前状况,动态地做出资源配决策;
        其三,\emreg{死锁检测(Detection)}方案检测系统中已经发生的死锁,并想办法打破进程间发生的死锁状态。
    }
}
