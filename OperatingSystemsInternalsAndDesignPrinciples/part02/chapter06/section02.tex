%%
%% Author: Clay
%% 2020/12/7
%%

\section{死锁预防(Deadlock Prevention)}
{
    死锁预防方案的思想是设计一个系统,该系统能够去除死锁发生的可能。
    可以将死锁预防的具体方案归纳为两种:一种是直接方案,一种是间接方案。
    间接方案是破除死锁发生条件中前三条中的任意一条,而直接方案是预防环路等待条件的发生。

    \subsection{互斥条件}
    {
        一般来说,互斥条件与具体的资源属性有关,是很难破除的。
        如果对某资源的访问需要互斥,则操作系统必须提供某种互斥机制。
    }

    \subsection{占有并等待条件}
    {
        占有并等待条件可以采用如下方法进行破除:
        要求进程在执行前,一次申请完它在执行过程中可能会用到的所有资源。
        如果系统暂时无法满足它的资源申请要求,则进程必须等待直到系统拥有足够多的资源,并将资源一次性全部分配给它才能开始运行。
        然而,该方案会导致系统的效率下降。
        有两个原因:
        其一,进程可能在获得它的运行所需的所有资源前,需要等待很长时间。
        其二,分配给进程的资源可能长时间得不到使用。

        还有一个现实的问题是进程在它执行前,可能无从得知它将用到哪些资源。
    }

    \subsection{不可抢占条件}
    {
        该条件可以采用以下几个方法破除:
        第一种方法,如果进程占用了一些资源,且在后续的资源请求得不到满足的情况下,它就必须释放已经占用的资源,并在后续的执行中,一次性申请它所需的所有资源。
        另一种方法是,如果一个高优先级进程申请已经被另一个低优先级进程所占用的资源,操作系统将从后者处抢占该资源将其分配给前者,该方法仅适用于任意两个进程都拥有不同优先级的情况。
        该方法进能够用在资源的状态很容易保存和恢复的场合。
    }

    \subsection{环路等待条件}
    {
        可以用有序资源分配的方法来破除环路等待条件:
        先将资源进行线性编序,如果进程获得资源R,那么后续请求的资源编号都只能大于或都小于R。

        破除环路等待条件还是可能会导致系统效率的降低,或者进程执行速度的降低,亦或不必要的资源分配请求拒绝等问题。
    }
}
