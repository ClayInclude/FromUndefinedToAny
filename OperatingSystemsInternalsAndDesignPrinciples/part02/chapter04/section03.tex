%%
%% Author: Clay
%% 2020/12/1
%%

\section{多核和多线程}
{
    \subsection{多核环境下应用的性能}
    {
        应用在多核处理器上所能取得的性能上的好处,实际上取决于它对多处理器资源的利用能力。

        在实际情况下,应用在多处理器系统上运行时,还需要考虑通信、任务分派,以及Cache一致性等开销。
        这些都导致它的性能曲线在不断增加处理器数量的过程中,在达到高峰后,随着处理器数量的增加而开始逐渐下降。
        其原因是处理器数量的增加而导致额外开销的增加。

        在数据库管理系统及数据库应用领域中,多核处理器资源能够得到较好的利用。
        除此之外,很多类型的服务器系统都能较好地发挥多核处理器的处理能力,这是因为服务器程序往往需要处理非常多相对独立的并行事务。

        还有一些类型的应用能够通过增加处理器的个数提高性能:

        \begin{itemize}
            \item 多线程的简单应用。
            \item 多进程应用。
            \item Java应用。
            \item 多实例应用。
        \end{itemize}
    }

    \subsection{应用示例:Valve的游戏软件}
    {
        Valve对线程的粒度进行了以下区分:

        \begin{description}
            \item[粗粒度线程]
            {
                独立的模块被设计为在独立的处理器上执行。
                这种设计非常直接,在本质上每个模块都是单线程的。
            }
            \item[细粒度线程]
            {
                很多相似或相同的任务可被构造成这种粒度的线程,并在执行过程中在多个处理器上并行执行。
            }
            \item[混合粒度线程]
            {
                在设计某些模块时采用细粒度的线程,而在设计别的模块时采用单线程设计。
            }
        \end{description}
    }
}
