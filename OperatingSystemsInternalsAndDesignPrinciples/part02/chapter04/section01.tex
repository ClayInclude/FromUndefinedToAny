%%
%% Author: Clay
%% 2020/11/18
%%

\section{进程和线程}
{
    之前对进程概念的讨论体现了以下两个概念。

    \begin{description}
        \item[资源占用]
        {
            一个进程包含了容纳进程镜像的虚地址空间。
            在执行过程中,进程可以申请得到对系统资源如主存、I/O通道、I/O设备以及文件等的控制权。
            操作系统的任务是对这些系统资源进行保护,防止进程对资源的非法访问。
        }
        \item[调度/执行]
        {
            进程的执行将按一定轨迹(Trace)进行,它的轨迹可能包含一个或多个程序的代码。
            如果系统中有多个进程并发执行,它们的执行轨迹将是交替向前发展的。
            这样,就有了进程的状态,且每个进程都要被赋予调度优先级和参与调度的实体。
        }
    \end{description}

    为了区分这两个概念,把进程的调度实体称为线程或者\emreg{轻量级进程(Lightweight Process)},而在考虑资源的权属上,可以把拥有资源的个体称为进程或者任务。

    \subsection{多线程}
    {
        多线程指的是操作系统在一个进程内支持多个并发执行路径的能力。
        传统的方案为系统中每个进程支持一条执行路径,这种方案称为单线程方案。
        相反的,多线程方案支持在单个进程中的多条并发执行路径。

        在多线程环境下,进程是参与分配资源,以及权限保护的基本单位,它包含:

        \begin{itemize}
            \item 容纳进程镜像的虚地址空间
            \item 对处理器的受保护访问,其他进程的信息,一组文件和I/O资源
        \end{itemize}

        进程可包含一个或多个线程,每个线程包含:

        \begin{itemize}
            \item 线程的状态
            \item 线程的上下文,以及线程在进程代码中的执行位置
            \item 私有的执行栈
            \item 静态存储空间和局部变量
            \item 对系统资源的访问权限
        \end{itemize}

        进程的线程共享它们的父进程的状态,以及它拥有的系统资源,且它们共存于同一地址空间,访问同一组数据。
        如果当其中的一个线程更改了内存中的数据,该进程的其他线程就能看到这一更改。

        线程概念的产生,实际上是对计算机系统的性能进行优化的结果。

        \begin{itemize}
            \item 相比于创建一个新的进程而言,在现有的进程基础上创建一个线程将要快得多。
            \item 相比于结束一个进程而言,结束一个线程要快得多。
            \item 在同一个进程的不同线程间切换,要比在不同的进程间切换要快得多。
            \item 线程间通信要比进程间通信快得多。
        \end{itemize}

        如果某应用需要被设计为多个单元的执行过程,那么将这些单元设计为多个线程要比将它们设计为多个独立的进程要高效的多。

        即使在单个处理器上运行,线程的结构也同样可以用来在一个程序中够造和区分逻辑上独立的功能。

        在支持线程的操作系统中,调度往往基于线程来进行。
        大多数有关执行的状态信息都保存在线程级的数据结构中。
        有一些影响到一个进程所有线程的动作必须在进程的级别处理。
        进程的挂起操作导致它所有的线程被同时挂起。
        进程的结束将导致它所有的线程一起结束。
    }

    \subsection{线程功能}
    {
        \subsubsection{线程状态}
        {
            线程的基本状态是运行态、就绪态以及阻塞态。
            挂起态对现场来说是没有意义的。

            导致线程状态发生变化的四个基本操作如下:

            \begin{description}
                \item[创建(Spawn)]
                {
                    一般来说,进程被创建的同时,对应的一个线程也会被创建。
                    气候,进程可以通过给出入口地址和参数,创建另一个线程。
                }
                \item[阻塞(Block)]
                {
                    当线程等待某事件的发生才能继续执行时,它将进入阻塞态。
                    处理器在其后转而执行下一个处于就绪态的线程。
                }
                \item[唤醒(Unblock)] 如果线程所等待的事件发生,该线程将进入就绪队列。
                \item[结束(Finish)] 当线程执行完毕后,它对应的寄存器上下文及私有堆栈资源将被系统回收。
            \end{description}
        }

        \subsubsection{线程同步}
        {
            一个进程的所有线程共享该进程的地址空间,以及它所拥有的系统资源。
            这样,一个线程对系统资源的任何改变都将影响到属于该进程的所有其他线程运行的环境。
            为了保证程序的正确执行,需要对线程的活动加以协调和同步,使得他们不会相互干扰或者破坏重要的全局数据结构。

            用于线程同步的方法和机制与用于进程同步的方法和机制是一样的。
        }
    }
}
