%%
%% Author: Clay
%% 2020/12/1
%%

\section{复习题}
{
    %1.
    \begin{reviewc}
        \emreg{线程控制块(Thread Control Block, TCB)}是与\emreg{进程控制块(PCB)}相似的子控制块,只是TCB中所保存的线程状态比PCB中保存少而已。

        进程是表示资源分配的基本单位,又是调度运行的基本单位。
        线程是进程中执行的最小单位,即执行处理机调度的基本单位。

        可以将线程控制块理解为进程控制块的组成和附属。
    \end{reviewc}

    %2.
    \begin{reviewc}
        进程切换涉及虚拟地址空间的切换而线程不会。
    \end{reviewc}

    %3.
    \begin{reviewc}
        资源占用和调度/执行。
    \end{reviewc}

    %4.
    \begin{reviewc}

    \end{reviewc}
}
