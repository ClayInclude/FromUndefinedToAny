%%
%% Author: Clay
%% 2020/12/1
%%

\section{复习题}
{
    %1.
    \begin{reviewc}
        \emreg{线程控制块(Thread Control Block, TCB)}是与\emreg{进程控制块(PCB)}相似的子控制块,只是TCB中所保存的线程状态比PCB中保存少而已。

        进程是表示资源分配的基本单位,又是调度运行的基本单位。
        线程是进程中执行的最小单位,即执行处理机调度的基本单位。

        可以将线程控制块理解为进程控制块的组成和附属。
    \end{reviewc}

    %2.
    \begin{reviewc}
        进程切换涉及虚拟地址空间的切换而线程不会。
    \end{reviewc}

    %3.
    \begin{reviewc}
        资源占用和调度/执行。
    \end{reviewc}

    %4.
    \begin{reviewc}

    \end{reviewc}

    %5.
    \begin{reviewc}
        进程是表示资源分配的基本单位,又是调度运行的基本单位。
        线程是进程中执行的最小单位,即执行处理机调度的基本单位。

        线程没有自己的虚拟地址空间。
    \end{reviewc}

    %6.
    \begin{reviewc}
        多线程之间比多进程之间更容易共享数据,线程一般来说都比进程更高效。
    \end{reviewc}

    %7.
    \begin{reviewc}
        操作系统内核管理和维护所有进程及进程中线程的上下文信息。
        操作系统自身也能够采用多线程的设计。
    \end{reviewc}

    %8.
    \begin{reviewc}
        在云环境下,线程是从用户角度看到的一个活动。
        在创建后,线程从一个给定起点开始执行。
        在执行过程中,线程可以从一个地址空间迁移到另一地址空间,这种迁移甚至可以跨越机器的边界。
    \end{reviewc}
}
