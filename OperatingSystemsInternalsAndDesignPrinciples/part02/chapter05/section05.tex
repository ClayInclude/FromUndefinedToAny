%%
%% Author: Clay
%% 2020/12/4
%%

\section{消息通信}
{
    一般来说,进程的交互和协作需要两个基本元素:
    同步和通信。
    同步用于进程的互斥,通信则用于进程间的协作。
    消息通信就能够提供这两个基本元素,而且它的适用范围很广:
    除了能在单处理器和共享内存的多处理器系统外,也适用于分布式系统。

    消息通信的基本形式为以下两个原语:

    \begin{itemize}
        \item \emcode{send(destination, message)}
        \item \emcode{receive(source, message)}
    \end{itemize}

    \subsection{同步(Synchronization)}
    {
        当一个进程调用了发送原语后,有两种可能:
        其一,发送进程阻塞,直到消息到达后才被唤醒;
        其二,发送进程不阻塞。

        对于接收原语,也有量两种可能:
        一,如果进程在调用接收原语前,消息已经到达,则进程继续执行;
        二,如果消息在进程调用接收原语前并未到达,则可能进程阻塞直到消息到达后才被唤醒,或者进程放弃消息接收操作而继续执行。

        在具体系统的设计中,常用的有以下三种组合:

        \begin{description}
            \item[发送端阻塞,接收端阻塞]
            {
                两端都阻塞,消息被接收后再将两端唤醒。
                这一组合又常被称为\emreg{交汇点(rendezvous)}方案,常被用于要求进程进行强同步的场合。
            }
            \item[发送端不阻塞,接收端阻塞]
            {
                发送端在发完消息后继续执行,而接收端必须阻塞。
                它使得进程能够最快地将消息发送给多个目的进程,而接收端必须等待消息到达后才能做有用的工作。
            }
            \item[发送端不阻塞,接收端不阻塞] 两端都不用等待。
        \end{description}

        采用非阻塞发送的潜在危险在于,发送端可能(错误地)重复发送消息。
    }

    \subsection{寻址(Addressing)}
    {
        消息寻址的方案从它所属的类型来说分为两种:
        \emreg{直接(Direct)}寻址和\emreg{间接(Indirect)}寻址。
        采用直接寻址,发送端在发送消息前必须给出一个标识来执行消息的目标进程。
        对于接收端,一种方案是在接收消息前\emreg{显式(Explicit)}地给出消息源,这样就要求接收端必须在接收消息前已经知道哪个进程会给它发消息。
        由于这种特性,显式消息源方案可以用于实现并发进程间的协作。
        并不是所有消息接收端都确切的知道哪些进程要向它发送消息的。
        对这类应用来说,更有效的方案是采用\emreg{隐式(Implicit)}消息源,它可以接收来自任意进程的消息且在接收到一个消息后,消息接收原语\emcode{receive}的\emcode{source}参数会返回该消息的发送者的地址。

        采用间接寻址,消息不再从发送端到接收端,而是由发送端将消息发送到一个共享的数据结构中,该数据结构采用队列结构(又被称为邮箱,Mailbox)来存放消息。

        间接寻址的优势在于消息通信的灵活性,因为通过去耦合,发送端和接收端的关系可以实一对一、多对一、或者多对多。

        进程和邮箱的关系可以是\emreg{静态(Static)}也可以是\emreg{动态(Dynamic)}的。
        \emreg{端口(Port)}的设置往往是静态的,且只跟一个进程绑定。
        \emreg{连接(Connect)}和\emreg{断开(Disconnect)}原语可以用于发送者跟邮箱的动态绑定。

        另外,进程跟邮箱之间还有\emreg{权属关系(Onwership)}。
        端口一般属于创建它的进程。
    }

    \subsection{消息格式}
    {
        消息格式的设计取决于通信机构的设计目标,以及通信机构的运行环境,如单机或分布式环境。
        在某些系统中,设计者为了降低消息的处理和存储开销,而将消息设计为短小、定长的格式。
        如果这种设计来传递大量数据,就可以将数据放到一个文件中,而在消息中存访指向该文件的引用。
        但更为灵活的设计时将消息设置为边长的。

        消息被分为两个部分:
        \emreg{消息头(Header)}和\emreg{消息体(Body)}。
        其中,消息头存访关于消息的描述信息,而消息体存放消息的内容。
    }

    \subsection{队列组织}
    {
        最简单的队列组织算法是先进先出,但该算法不能很好的处理紧急消息。
    }

    \subsection{互斥的实现}
    {
        任何希望进入临界区的进程都必须先获得一个消息,如果邮箱为空,进程就会被阻塞。
        获得消息的进程在执行完临界区后,向邮箱发出一个消息。
    }
}
