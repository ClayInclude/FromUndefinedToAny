%%
%% Author: Clay
%% 2020/12/4
%%

\section{管程(Monitor)}
{
    采用信号灯能够简洁而有效地在进程间实现互斥和协作。
    然而采用信号灯的方式来实现正确的程序往往并不容易。

    管程的提出,就是希望设计一种程序语言,它能够提供跟信号灯等效的功能,且易于控制。

    \subsection{管程和信号}
    {
        管程可以看作是包含了一个或多个函数,一个初始化过程,以及\emreg{本地数据(Local Data)}的一个软件模块。
        它的主要特征如下:

        \begin{itemize}
            \item 本地数据只能被管程的内部函数所访问,外部函数无法访问其本地数据。
            \item 进程只能通过调用管程自己定义的函数来进入管程。
            \item 在管程内执行的进程一次只能有一个。
        \end{itemize}

        为了更好地支持并发处理,管程还定义了一个同步工具。

        管程采用了\emreg{条件变量(Condition Variables)}的方法来实现这种同步工具。
        条件变量是莞城中的特殊数据类型,只能用以下两个函数来访问。

        \begin{description}
            \item[cwait(c)] 将调用该函数将访问条件变量c的进程阻塞。
            \item[csignal(c)] 继续执行之前因调用cwait阻塞的进程。
        \end{description}

        对比信号灯而言,管程的真正优势在于它所有的同步操作都被限制在管程内部。
    }

    \subsection{采用通知和广播的管程模型}
    {

    }
}
