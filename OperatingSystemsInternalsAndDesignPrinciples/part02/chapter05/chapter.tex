%%
%% Author: Clay
%% 2020/8/17
%%

\chapter{并发:互斥与同步}
{
    操作系统的核心设计问题都与进程和线程的管理相关:

    \begin{description}
        \item[多道处理程序] 在单处理器的计算机系统中,管理多个进程。
        \item[多处理器环境] 在多处理器的系统中,管理多个线程。
        \item[分布式处理]
        {
            在拥有多个计算机的系统中,管理多进程。
            当前被广泛采用的集群处理就是这种模式的典型应用。
        }
    \end{description}

    对于这些领域的应用,以及操作系统的设计本身,最基础的问题就是\emreg{并发(Concurrency)}。
    并发本身又涉及很多方面的设计问题,包含进程间通信,资源共享,多进程中活动的同步,以及处理器时间的分配等。

    并发在以下情况下产生:

    \begin{description}
        \item[多应用] 多道程序设计技术的发明,使得可以在一个系统中容纳多个应用程序,且处理器的时间得以动态地分配给这些应用。
        \item[结构化的应用] 模块化和结构化程序设计的引入,使得很多应用能够有效地采用结构化的方法,以及多个并发进程来实现。
        \item[操作系统的结构] 结构化的设计同样也适用于操作系统本身,操作系统往往采用一组进程或线程来完成它自身的功能。
    \end{description}

    %%
%% Author: Clay
%% 2020/12/5
%%

\section{死锁的原理}
{
    死锁是指一组进程因为竞争系统资源或互相等待消息,而永远无法向前推进的状态。

    \subsection{可重用资源}
    {
        资源可以分为两个大类:
        可重用资源与消耗性资源。

        可重用资源指的是一次只供一个进程使用,但用后又能被别的进程使用的资源。

        从系统的角度出发,能够解决死锁问题的方法之一是对应用申请系统资源的顺序做出限定。
    }
}

}

\cleardoublepage

\endinput
