%%
%% Author: Clay
%% 2020/12/5
%%

\section{复习题}
{
    %1.
    \begin{reviewc}
        \begin{enumerate}[A.]
            \item 操作系统必须能够跟踪进程的执行状态。
            \item 操作系统必须能够为进程分配和回收各种系统资源。
            \item 操作系统能够保护进程的数据和已拥有的物理资源,避免它们被其它进程非法访问。
            \item 进程的功能以及它的输出结果,必须与它跟系统中其它并发进程的相对速度无关。
        \end{enumerate}
    \end{reviewc}

    %2.
    \begin{reviewc}

    \end{reviewc}

    %3.
    \begin{reviewc}
        当系统中有多个进程对公共变量不加限制的读写,该变量最终值取决于进程的读写顺序,就称该系统发生了竞态。
    \end{reviewc}

    %.4
    \begin{reviewc}
        \begin{enumerate}[A.]
            \item 无交互
            \item 间接交互
            \item 直接交互
        \end{enumerate}
    \end{reviewc}

    %5.
    \begin{reviewc}
        竞争关系是进程间互相独立、无联系,但是需要竞争同一系统资源。
        协作是指进程间有关联,需要同步数据来完成工作。
    \end{reviewc}

    %6.
    \begin{reviewc}
        \begin{enumerate}[A.]
            \item 互斥
            \item 死锁
            \item 饥饿
        \end{enumerate}
    \end{reviewc}

    %7.
    \begin{reviewc}
        如果系统对资源访问的排队算法不恰当,会导致饥饿。
    \end{reviewc}

    %8.
    \begin{reviewc}
        \begin{enumerate}[A.]
            \item 信号灯的初值为非负整数。
            \item semWait操作将信号灯的值减1,如果之后它的值为负数,进程将被阻塞,否则继续执行。
            \item semSignal操作将型号等的值加1,如果之后它的值大于等于0,则进程唤醒在该信号灯上等待的一个进程,并继续自己的执行。
        \end{enumerate}
    \end{reviewc}

    %9.
    \begin{reviewc}
        二进制信号量取值只能是0和1。
    \end{reviewc}

    %10.
    \begin{reviewc}
        信号量的wait和signal操作可以是不同的进程。
        互斥锁的lock和unlock必须是同一进程。
    \end{reviewc}

    %11.
    \begin{reviewc}
        抽象、封装。
    \end{reviewc}

    %12.
    \begin{reviewc}
        间接寻址的优势在于消息通信的灵活性,因为通过去耦合,发送端和接收端的关系可以实一对一、多对一、或者多对多。
    \end{reviewc}

    %13.
    \begin{reviewc}
        \begin{enumerate}[A.]
            \item 可以有任意多个读者同时读取数据区中的内容。
            \item 一次只有一个写着允许向数据区中写。
            \item 写者在想数据区中写时,不允许读者同时读取。
        \end{enumerate}
    \end{reviewc}
}
