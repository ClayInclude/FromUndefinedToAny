%%
%% Author: Clay
%% 2020/12/3
%%

\section{互斥:硬件的支撑方案}
{
    \subsection{关中断}
    {
        在单处理器系统上,多个并发进程的执行,在微观上不可能同时发生。
        进程如果不是因为发出了系统调用或者系统发生了中断的话,会一直执行下去。
        所以,当进程执行的情况下,避免它被系统中所产生的中断所打断,就能够达到对资源互斥访问的目的。
        对中断的开关可以采用操作系统所提供的原语方式,提供给应用程序所调用。

        因为临界区的执行不可被中断,互斥的目的也就达到了。
        然而这一方案的代价是高昂的:
        因为处理器不能像之前那样,任意地穿插执行进程,所以系统的执行效率可能会显著降低。
        另一个问题是,这个方案不能够直接用在多处理器的系统中。
    }

    \subsection{特殊机器指令}
    {
        当系统中的某处理器访问一个内存位置的时候,通过硬件的手段避免其他处理器对该位置的访问。
        基于这一思想,处理器的设计者设计了几条机器指令来实现两个或两个以上动作的原子性。
        当这种类型的指令执行时,被它们访问的内存位置将不允许被其他指令同时访问。

        \subsubsection{比较替换(compare \& swap)指令}
        {
            该指令会将某内存地址中的内容于一个测试值相比较,如果它们相等,该内存地址中的内容将被替换为新值,否则,它的内容不变。
            该指令包括两个部分:比较部分和替换部分。
            所有的这些动作不可被打断或被干扰,也就是说这个指令具有原子性。
        }

        \subsubsection{交换(exchange)指令}
        {
            该指令的功能是将两个内存地址的内容进行交换。
        }

        \subsection{机器指令方案的特点}
        {
            采用特殊机器指令的方案来实现互斥具有以下优点:

            \begin{itemize}
                \item 既适用于多处理器结构的计算机系统,也适用于单处理器的系统。
                \item 设计简单,易于验证。
                \item 能够被用来支持多临界区,每个临界区可以定义自己的变量来控制进入。
            \end{itemize}

            同时,该方案也有以下严重的缺点:

            \begin{itemize}
                \item 忙等问题,当进程在等待进入临界区时,需要不停检测,从而无谓消耗大量处理器时间。
                \item 可能导致饥饿,因为进程进入临界区的顺序是随机的。
                \item 可能导致死锁。
            \end{itemize}
        }
    }
}
