%%
%% Author: Clay
%% 2020/11/16
%%

\section{进程控制}
{
    \subsection{(处理器的)执行的模式}
    {
        几乎所有处理器都支持两种执行模式,一种为拥有更高特权级的模式。
        有些指令必须在这一更高特权级模式下执行,部分内存区域必须在这一模式下才能访问。
        另一种为低特权级模式,也称为\emreg{用户模式},用户进程往往在这一模式下执行。

        前一种更高特权级的处理器执行模式,往往被称为\emreg{系统模式、控制模式},或者\emreg{内核模式}。
        操作系统的内核就运行在内核模式下,并在该模式下提供所有操作系统应具备的功能。

        之所以区分处理器执行模式,其目的在于保护操作系统,以及操作系统中一些重要的表格。
        在内核模式下执行的代码,能够执行所有指令,操纵所有的寄存器和内存空间。
        出于安全的考虑,用户的进程是不允许拥有如此高的权限的。

        在处理器的程序状态字(PSW)中,会有一个特定的位,标识处理器所执行的模式。
        该位在一些特定事件发生时,会发生转换。
    }
}
