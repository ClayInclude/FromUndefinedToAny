%%
%% Author: Clay
%% 2020/11/16
%%

\section{进程控制}
{
    \subsection{(处理器的)执行的模式}
    {
        几乎所有处理器都支持两种执行模式,一种为拥有更高特权级的模式。
        有些指令必须在这一更高特权级模式下执行,部分内存区域必须在这一模式下才能访问。
        另一种为低特权级模式,也称为\emreg{用户模式},用户进程往往在这一模式下执行。

        前一种更高特权级的处理器执行模式,往往被称为\emreg{系统模式、控制模式},或者\emreg{内核模式}。
        操作系统的内核就运行在内核模式下,并在该模式下提供所有操作系统应具备的功能。

        之所以区分处理器执行模式,其目的在于保护操作系统,以及操作系统中一些重要的表格。
        在内核模式下执行的代码,能够执行所有指令,操纵所有的寄存器和内存空间。
        出于安全的考虑,用户的进程是不允许拥有如此高的权限的。

        在处理器的程序状态字(PSW)中,会有一个特定的位,标识处理器所执行的模式。
        该位在一些特定事件发生时,会发生转换。
        用户态的进程在执行过程中发出了一个调用内核服务的请求,或者一个中断触发了内核中的中断服务例程的执行,这些事件发生时,处理器将切换到内核模式,而当系统从这些事件中返回到用户进程后,处理器将切换回用户模式。
    }

    \subsection{进程创建}
    {
        当操作系统打算创建一个进程时,它会启动以下步骤:

        \begin{description}
            \item[为新进程设置一个标识符] 一个新的进程控制块会被创建,且加入系统所维护的主进程表格中,该表格的每一项都唯一地对应了一个进程。
            \item[为进程分配空间] 这一过程涉及进程镜像的各个部分,操作系统必须事先知道该进程的私有地址空间中各个部分,包括用户栈的大小。
            \item[初始化进程控制块]
            {
                进程的标识符包含了该进程的标识,以及一些其他的标识。
                此时状态大部分都是初始值,只有程序计数器和系统栈指针除外。
                进程控制信息部分按照系统的缺省值,或者用户提供的值进行初始化。
            }
            \item[设置指针]
            {
                处于调度的目的,操作系统为每一类进程设置了一个或多个队列。
                而新创建的进程处于就绪状态或就绪且挂起状态,它就需要通过链表操作,加入到操作系统对应的队列中。
            }
            \item[创建和扩充其他数据结构]
        \end{description}
    }

    \subsection{进程切换}
    {
        从字面意义上来看,进程切换意味着一个处于运行态的进程被中断,操作系统将另一个进程设置为运行状态,并且将处理器的控制权交给新换入的进程。

        \paraph{何时进行进程切换}
        {
            只有当操作系统从当前运行的进程处得到了处理器的控制权后,进程切换才有可能发生。

            \begin{table}[htb]
                \centering

                \caption{中断当前运行进程的机制}

                \begin{tabular}{c|c|c}
                    \hline
                    机制 & 原因 & 用途 \\
                    \hline
                    中断 & 外部事件的发生 & 响应异步的外部事件 \\
                    \hline
                    异常(陷阱) & 当前指令无法继续执行 & 处理错误或异常事件 \\
                    \hline
                    系统调用 & 由当前运行的进程显式发出 & 调用操作系统的服务例程 \\
                    \hline
                \end{tabular}
            \end{table}

            很多系统将中断分为两种类型:
            中断和异常。
            中断指的是与当前正在运行的进程无关的,异步发生的,独立的外部事件。
            异常指的是当前正在运行的进程产生的错误,或异常事件。
            在处理中断时,处理器控制首先转移到中断处理例程,然后再从中断处理例程中调用操作系统用于处理中断的函数,并最终完成中断的处理。
            典型的例子如下:

            \begin{description}
                \item[时钟中断]
                {
                    为了调度的公平性,操作系统往往为系统中进程的运行定义了\emspe{时间片}。
                    也就是说,时间片往往规定了该系统中,一个进程被调度后,能够使用处理器最长的时间。
                    在使用完时间片后,时钟中断的产生将导致当前运行的进程转入就绪态,同时操作系统会调入另一进程到处理器开始执行。
                }
                \item[I/O中断] 操作系统通过这类中断得知I/O事件的完成。
                \item[内存错误] 当前运行的进程如果在执行过程中访问了尚未被调入内存的区域,则会产生一个中断。
            \end{description}

            对于异常,操作系统将决定导致该异常发生的错误或异常的状态是否是致命的。
            如果是这样,当前运行的进程将被转移到结束态,且换入另一个进程开始执行。
            如果不是,操作系统的行为将取决于错误的类型和操作系统的设计原则。

            最后,进程切换也可能由当前进程通过显式地发出\emreg{系统调用}所产生。
        }

        \paraph{模式切换}
        {
            在中断处理阶段,如果还有中断未处理,处理器会进入以下流程:

            \begin{enumerate}
                \item 设置程序计数器到该中断对应的中断处理例程的入口。
                \item 从用户模式切换到内核模式,以开始中断处理例程的执行。
            \end{enumerate}

            同时,当前进程的上下文会被保存到它自己的进程控制块中。

            如果中断处理后确实有进程的切换发生,则操作系统可能会做一些额外的工作,并修改当前进程的进程控制块中除了上下文以外的部分。
            然而大多数情况下,中断的发生和处理并不会直接导致进程切换。
            通常情况下,对这些信息的保存和恢复都由硬件完成。
        }

        \paraph{进程状态的改变}
        {
            处理器的模式切换和进程切换并不是同一个概念。
            处理器模式的切换,并不会直接导致当前正在运行的进程被切换。
            这种情况下,当前进程的上下文的保存和随后的恢复并不会带来太大的系统开销。
            然而,如果当前进程转移到其他状态,操作系统就必须开始一个完整的进程切换流程:

            \begin{enumerate}
                \item 保存当前进程的上下文,包括程序计数器和其他的寄存器。
                \item 改变当前进程的进程控制块的内容,其中包括其他状态的改变。
                \item 将当前进程的控制块移到其它队列中。选择一个合适的进程开始执行。
                \item 更新被选择投入执行的进程的进程控制块,包括将其状态置为运行态。
                \item 更新内存管理模块的数据结构。
                \item 回复被选中投入执行的进程的运行现场。
            \end{enumerate}
        }
    }
}
