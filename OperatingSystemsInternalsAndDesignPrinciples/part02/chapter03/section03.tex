%%
%% Author: Clay
%% 2020/11/16
%%

\section{进程描述符}
{
    操作系统控制计算机系统中所有的事件,包括调度进程到处理器上执行、为进程分配资源、响应用户请求等。
    基本上,可以把操作系统看作是以进程为单位,管理系统资源的实体。

    \subsection{操作系统中控制资源的结构}
    {
        加入操作系统需要管理进程和它所使用的资源,那么操作系统必须知道每个进程的当前状态。
        为了达到这一目标,操作系统采用了一个直接的方式:
        操作系统为每一种类型的资源,建立一个表格进行管理。
        操作系统管理着4类资源:
        内存,I/O设备,文件和进程。

        \paraph{内存表格}
        {
            用于跟踪记录物理内存和虚拟内存的使用状况。
            一部分系统内存被保留给操作系统自己使用,而余下的内存则在用户进程之间使用和分配。
            内存表格必须包含以下信息:

            \begin{itemize}
                \item 物理内存的分配信息
                \item 二级内存的分配信息
                \item 每一块内存的访问权限信息
                \item 用于管理虚拟内存的信息
            \end{itemize}
        }

        \paraph{I/O设备表格}
        {
            该表格被用于管理I/O设备,以及计算机系统的通道。
            在任意时刻,一个I/O设备只能被分配给一个特定的进程。
        }

        \paraph{文件表格}
        {
            这些表格用于表述文件的基本信息,它们当前的状态,以及一些其他的属性信息。
            在有些操作系统中,这些信息都由文件系统所管理。
        }

        \paraph{进程表格}
        {
            操作系统必须维护进程表格以对系统中存在的进程进行管理。
            4种看似独立的表格,但是,这些表格并不是各自为政的,它们之间通过指针进行相互的链接。
            内存、I/O设备,以及文件实际上都是以进程的单位进行管理。
            这些表格都不可能凭空存在,必须放到内存的特定区域。

            操作系统必须对其运行的环境的有基本的了解。
            当操作系统在初始化阶段,它必须能够得知将要运行的基础信息,这些信息可能需要人工帮助,也可由一些自动配置软件完成。
        }
    }

    \subsection{进程控制块}
    {
        假如操作系统需要控制和管理一个进程,首先操作系统需要知道这个进程位置,其次操作系统需要知道该进程的一些基本属性。

        \paraph{进程位置}
        {
            在最基本的情况下,一个进程必须包含一段或多端即将执行的程序。
            与这些程序相关联的是一组数据。
            进程组成中,必然包含足够的内存用于存储这些程序和数据。
            另外,程序的执行必然会利用到堆栈,用于存放该程序执行过程中调用函数时的参数。
            最后,每个进程都有一组变量,用于描述自身的状态和属性信息以便于操作系统对其进行管理和控制。
            通常称操作系统中用于管理和控制进程的数据结构为\emreg{进程控制块(Process Control Block)}。
            同时,称进程所对应的程序、数据、堆栈,以及它的属性信息所构成的全集为\emreg{进程镜像}。

            进程镜像所防止的位置,取决于内存管理系统。
            在最简单的情形下,进程镜像被连续地存放在辅存种。
            当操作系统需要管理该进程的时候,该进程至少有一部分将被加载到内存。
            当需要执行该进程时,它的进程镜像将被完整地载入物理内存或虚拟内存中。

            现代操作系统假设底层硬件具有页式地址管理的功能,这样以便于在不连续内存空间中,部分地将进程装入。
            在给定时刻,一个进程镜像的一部分可能被装入内存,而其余部分则可能在辅存中。
            所以,进程控制表必须包含足够信息,使得操作系统可以知道进程镜像中页面的为孩子。

            进程位置星系的结构:
            有一个主进程表格,其中每个进程占据一个表项,每个表项至少包含一个指向进程镜像的指针。
            如果进程镜像包含多个块,该信息会在主进程表格中记录,或者通过交叉索引在内存表格中记录。
        }

        \paraph{进程属性}
        {
            复杂得多任务系统需要为每个进程记录大量信息。

            进程控制块中存放的进程信息归为以下3哥类型:

            \begin{description}
                \item[进程标识符(Process Identification)]
                \item[处理器状态信息(Processor State Information)]
                \item[进程控制信息(Process Control Information)]
            \end{description}
        }

        \paraph{进程标识符}
        {
            几乎每个操作系统都为其中的每一个进程设置了唯一的数字标识。
            如果系统允许一个进程在执行过程中创建其它进程,则进程标识符可以用来索引父进程和子进程。
            除了远程标识符外,每个进程还可以给一个用户标识符,以表征创建该进程的用户。
        }

        \paraph{处理器状态信息}
        {
            包含处理了其中寄存器的值和状态。
            当进程执行过程中被中断时,所有的寄存器内的值将被保存到进程控制块中,以便该进程被再次调度执行时的恢复。

            一般来说,所有的处理器的设计都包含一个或一组被称为\emreg{程序状态字(Program Status Word, PSW)}的寄存器。
            这组寄存器包含了指令执行后的条件码外加一些状态信息。
        }

        \paraph{进程控制信息}
        {
            这一部分信息被操作系统用于控制和协调系统内的各种进程活动。
        }

        \paraph{进程控制块的角色}
        {
            每一个进程控制块都包含操作系统为了控制它所对应的进程所需要的所有重要信息。
            进程控制块为操作系统内几乎所有重要的模块所读取、修改,其中包括调度模块、资源分配模块中断处理模块,以及性能监控和分析模块等。
            进程控制块所构成的集合直接定义了操作系统在某个时间点上的状态。

            由于操作系统中的各个模块都有可能访问进程控制块,对进程控制块的访问保护问题就变得重要起来。
            这里有两个问题:

            \begin{enumerate}
                \item 如果在某个操作系统模块出现了bug,它破坏了某进程的进程控制块,这将可能导致系统在其后无法对被破坏了的进程控制块中的进程进行有效管理。
                \item 对于进程控制块本身的结构调整,将势必影响操作系统中的各个模块对进程控制块的访问。
            \end{enumerate}

            这些问题可以通过为所有进程访问控制块的模块定义统一入口的函数得以实现。
            在这些函数中,对所有访问进程控制块的操作加以保护,以防止数据被破坏。
        }
    }
}
