%%
%% Author: Clay
%% 2020/11/16
%%

\section{什么是进程}
{
    \subsection{背景}
    {
        现代操作系统的设计,应能满足系统中多个应用程序执行的需要,且达到以下设计目标:

        \begin{itemize}
            \item 资源应该能够被多个应用程序使用
            \item 处理器需要在多个应用程序的执行过程中进行切换,以使这些应用程序的执行看起来好像在同时进行
            \item 尽可能地提高处理器和I/O设备的利用率
        \end{itemize}

        为了达到这些设计目标,所有现代操作系统都将应用程序的执行过程抽象成一个或多个进程。
    }

    \subsection{进程和进程控制块}
    {
        也可以把进程认为是一个包含多个元素的独立个体,其中有两个基本元素:
        程序代码,以及与这个代码相关的一组数据。
        现在开始这一段代码的执行,并称这一执行体为进程。
        在它执行过程的任意时间点上,所对应的进程拥有唯一表示其存在的元素,包括以下几项:

        \begin{description}
            \item[标识符] 一个系统唯一的标识符与其对应,从而将它与系统中其他可能存在的进程区分开来
            \item[进程状态] 如果该进程正在执行,那么它的状态就是执行态
            \item[优先级] 该优先级是一个相对其他进程的优先级,往往用来确定进程被调度到处理器上执行的先后次序
            \item[程序计数器] 记录下一条指令的地址
            \item[内存指针] 通过这些指针,能够找到该进程的代码、数据,以及与其他进程共享的内存部分
            \item[上下文]  即进程执行到该时间点时,处理器中寄存器的值
            \item[I/O状态] 包括I/O请求、被分配给该进程的I/O设备、正在使用的文件等
            \item[记账信息] 包括该进程累计占用了多少处理器时间,它的使用时间限制等
        \end{description}

        这些信息一般会被放在一个被称为\emreg{进程控制块}的数据结构中,往往由操作系统创建。
        进程控制块的一个重要特点在于,它包含了足够的信息,从而能够在系统产生中断时,打断它的执行过程,并在中断处理完成后通过进程控制块的信息恢复执行。
    }
}
