%%
%% Author: Clay
%% 2020/11/18
%%

\section{操作系统的执行}
{
    \subsection{独立内核}
    {
        比较传统的方案是在所有进程之外执行操作系统。
        操作系统拥有自己的内存空间,以及自己的系统栈保留自身函数调用的参数和返回值,所以能够执行自身的任何函数调用。

        在任何情形下,进程的概念只适合用户程序,而操作系统作为一个运行于特权态的代码而独立于任何进程之外。
    }

    \subsection{嵌套于用户进程}
    {
        操作系统设计的另一个可能,是将操作系统的功能嵌套在用户进程中,并在用户进程的上下文中执行这些代码。

        当有中断、异常,或者系统调用发生时,处理器将进入内核态。
        并在保存模式上下文后,转而执行操作系统中的代码。
    }

    \subsection{基于进程的操作系统}
    {
        操作系统设计的另一种可能,是将操作系统设计为一组系统进程。
    }
}
