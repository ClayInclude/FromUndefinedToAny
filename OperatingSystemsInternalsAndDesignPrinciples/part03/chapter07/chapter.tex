%%
%% Author: Clay
%% 2020/8/17
%%

\chapter{内存管理}
{
    在单道程序设计系统中,内存被分为两部分:
    一部分供操作系统使用,另一部分供当前正在执行的程序使用。
    在多道程序设计系统中,内存的用户部分被进一步细分,以满足多个进程的存放需求。
    内存细分的任务由操作系统动态完成,称为\emreg{内存管理}。

    \begin{table}[htb]
        \centering

        \caption{内存管理术语}

        \begin{tabular}{c|c}
            \hline
            页框 & 固定长度的内存块 \\
            \hline
            页 & 存储在辅存中的、固定长度的数据块。数据页可以临时复制到内存的页框中。 \\
            \hline
            段 & 存储在辅存中的、可变长度的数据块。整个段可以临时复制到内存的一个可用区域,也可以将其分为许多也,分别将每页复制到内存中。 \\
            \hline
        \end{tabular}
    \end{table}

    %%
%% Author: Clay
%% 2020/12/5
%%

\section{死锁的原理}
{
    死锁是指一组进程因为竞争系统资源或互相等待消息,而永远无法向前推进的状态。

    \subsection{可重用资源}
    {
        资源可以分为两个大类:
        可重用资源与消耗性资源。

        可重用资源指的是一次只供一个进程使用,但用后又能被别的进程使用的资源。

        从系统的角度出发,能够解决死锁问题的方法之一是对应用申请系统资源的顺序做出限定。
    }
}

}

\cleardoublepage

\endinput
