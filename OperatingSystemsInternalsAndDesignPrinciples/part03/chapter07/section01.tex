%%
%% Author: Clay
%% 2020/12/7
%%

\section{内存管理的需求}
{
    \subsection{内存重定位}
    {
        在多道程序程序设计系统中,可用的内存空间通常被多个进程共享使用。
        进程被换出到磁盘后,如果要求该程序在下次被换入内存时必须放入到与换出前相同的内存区域,那么这将是一个很苛刻的限制。
        因此,需要将进程\emreg{重定位}到不同的内存区域。

        假设进程映像占用一片连续的内存区域。
        操作系统需要知道进程控制信息的地址和执行栈的地址,以及进程开始执行的程序入口点。
        处理器必须能处理程序内的内存访问。
        处理器硬件和操作系统软件必须嗯那个将程序代码中的内存访问转换成当前程序在内存中的实际物理地址的访问。
    }

    \subsection{内存保护}
    {
        每个进程都应受到系统的保护,以避免其他进程对其进行有意或无意地非法访问。
        在某种意义上,满足内存重定位需求的同时,也增加了内存保护的难度。

        通常,用户布恩那个访问操作系统的任何区域,包括系统程序和系统数据。
        并且,一个进程中的程序不能转而去执行其他进程中的某条指令。
        如果没有特别许可,一个进程中的程序也不能访问其他进程中的数据。

        内存保护必须通过处理器以硬件方式实现,而不能通过操作系统以软件方式实现。
        因为操作系统无法预测程序可能产生的所有内存访问;
        几时可以预测,提前检测每个进程中是否可能存在非法内存访问也是非常耗时、甚至被禁止使用的。
        因此,只有在访问指令执行时,才有可能去判断该内存访问是否合法。
        处理器硬件必须具备这个能力以实现这一点。
    }

    \subsection{内存共享}
    {
        任何保护机制都需要有一定的灵活性,以允许多个进程访问内存中的同一区域。
        内存管理系统必须控制对内存共享区域的访问,但不损害基本的保护。
    }

    \subsection{逻辑结构}
    {
        绝大多数情况下,计算机系统中的内存被构造成包含字节或字序列的线性地址空间。

        如果操作系统和计算机硬件能够有效的处理以模块方式组织的用户程序和数据,就会有以下优点:

        \begin{itemize}
            \item 可以独立编写模块,模块间的相互调用交给系统,在运行过程中进行处理。
            \item 通过适当的额外开销,可以为不同的模块设置不同的保护级别。
            \item 可以引入某种机制,使得模块可以被多个进程共享使用。
        \end{itemize}

        分段是最容易满足这些需求的工具。
    }

    \subsection{物理结构}
    {
        计算机存储至少分为两层结构,即内存和外存。
        内存提供快速访问,但成本相对较高。
        内存是易失性的,即不提供永久存储。
        外存比内存更为缓慢和便宜,通常是非易失性的。

        内存和外村之间信息流的组织是系统关注的主要内容。
        虽然可以让程序员负责这个信息流的组织和分派,但这个方案是不切实际的,也是不合乎要求的,主要源于以下两个原因:

        \begin{itemize}
            \item
            {
                可供程序和数据使用的内存可能不够用。
                这种情况下,程序员需要利用\emspe{覆盖(Overlaying)}技术来组织程序和数据:
                多个模块被分配到同一个内存区域,主程序负责在需要时实现各个模块的换入或换出。
                即使有编译工具的协助,编写这样的程序也会浪费程序员的时间。
            }
            \item
            {
                在多道程序设计环境中,程序员在编写代码时并不知道有多少空间可以使用,也不清楚这些可用的空间究竟在哪里。
            }
        \end{itemize}

        在两层存储器间移动信息的任务很显然应该交给系统来完成,而该任务就是内存管理的本质。
    }
}
