%%
%% Author: Clay
%% 2020/12/7
%%

\section{内存管理的需求}
{
    \subsection{内存重定位}
    {
        在多道程序程序设计系统中,可用的内存空间通常被多个进程共享使用。
        进程被换出到磁盘后,如果要求该程序在下次被换入内存时必须放入到与换出前相同的内存区域,那么这将是一个很苛刻的限制。
        因此,需要将进程\emreg{重定位}到不同的内存区域。

        假设进程映像占用一片连续的内存区域。
        操作系统需要知道进程控制信息的地址和执行栈的地址,以及进程开始执行的程序入口点。
        处理器必须能处理程序内的内存访问。
        处理器硬件和操作系统软件必须嗯那个将程序代码中的内存访问转换成当前程序在内存中的实际物理地址的访问。
    }

    \subsection{内存保护}
    {
        每个进程都应受到系统的保护,以避免其他进程对其进行有意或无意地非法访问。
        在某种意义上,满足内存重定位需求的同时,也增加了内存保护的难度。

        通常,用户布恩那个访问操作系统的任何区域,包括系统程序和系统数据。
        并且,一个进程中的程序不能转而去执行其他进程中的某条指令。
        如果没有特别许可,一个进程中的程序也不能访问其他进程中的数据。

        内存保护必须通过处理器以硬件方式实现,而不能通过操作系统以软件方式实现。
        因为操作系统无法预测程序可能产生的所有内存访问;
        几时可以预测,提前检测每个进程中是否可能存在非法内存访问也是非常耗时、甚至被禁止使用的。
        因此,只有在访问指令执行时,才有可能去判断该内存访问是否合法。
        处理器硬件必须具备这个能力以实现这一点。
    }
}
