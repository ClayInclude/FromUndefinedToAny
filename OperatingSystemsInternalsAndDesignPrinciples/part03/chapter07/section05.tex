%%
%% Author: Clay
%% 2020/12/9
%%

\section{复习题}
{
    %1.
    \begin{reviewc}
        \begin{itemize}
            \item 内存重定位
            \item 内存保护
            \item 内存共享
            \item 逻辑结构
            \item 物理结构
        \end{itemize}
    \end{reviewc}

    %2.
    \begin{reviewc}
        将程序的逻辑地址转换成内存中物理地址的过程。
    \end{reviewc}

    %3.
    \begin{reviewc}
        \begin{itemize}
            \item 可以独立编写和编译
            \item 为不同模块设置不同的保护级别
            \item 模块可以被多个进程共享使用
        \end{itemize}
    \end{reviewc}

    %4.
    \begin{reviewc}
        内存共享
    \end{reviewc}

    %5.
    \begin{reviewc}
        大小固定存在两个问题:

        \begin{itemize}
            \item 程序太大以至于不能放到一个分区中。
            \item 内存利用率非常低。
        \end{itemize}

        大小不固定能够缓解这两个问题,但不能完全解决。
    \end{reviewc}

    %6.
    \begin{reviewc}
        内部碎片指分区内没有被利用的空间,外部碎片指分区外内存中无法被利用的空间。
    \end{reviewc}

    %7.
    \begin{reviewc}
        动态重定位和静态重定位。
    \end{reviewc}

    %8.
    \begin{reviewc}
        页是进程的块,页框是内存的块。
    \end{reviewc}

    %9.
    \begin{reviewc}
        分页要求页大小相等,而段大小可以不等。
    \end{reviewc}
}
