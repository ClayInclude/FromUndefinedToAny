%%
%% Author: Clay
%% 2020/12/9
%%

\section{分页}
{
    如果内存被划分成许多容量固定的块,同时块的容量相对较小,而每个进程也被划分成多个同样大小的块,那么进程中称为\emreg{页(Page)}的块正好可以放到内存中称为\emreg{页框(Frame)}的空闲块中。
    内存中每个进程浪费的内存空间,仅仅是进程最后一页占用一个页框时形成的内部碎片,内存是没有外部碎片的。

    在某个给定的时间点,内存中的一部分页框被进程使用,而另一部分页框则空闲,操作系统维护空闲页框组成的一个列表。

    操作系统需要为每个进程维护一个\emreg{Page Table}。
    页表给出了进程每一页对应的页框的位置。
    在程序中,每个逻辑地址包含一个页号和一个偏移量。
    在分页中,逻辑地址向物理地址的转换仍由处理器硬件完成,但处理器必须知道如何访问当前进程的页表,处理器根据该页表将逻辑地址(页号、偏移量)转换成物理地址(页框号、偏移量)。

    简单分页类似于固定分区,它们的不同之处在于,分页技术中的分区更小,程序会占用不止一个分区,而占用的这些分区可以是不连续的。

    为了使分页方案更方便,规定页的大小和页框的大小必须是2的幂次方。
    页的大小设为2的幂次方,将很容易用程序起点和逻辑地址定义的相对地址,即可以用页号和偏移量表示。

    页的大小设为2的幂次方具有双重功效。
    首先,逻辑地址方案对程序员、汇编器和链接器透明。
    程序的每个逻辑地址与它的相对地址是一致的。
    其次,在运行时用硬件实现动态地地址转换的功能相对容易。

    有了简单分页技术,内存被分成许多大小相等的小的页框,每个进程被划分成与页框同样大小的页。
    当进程被加载到内存时,进程所有的页都被加载到内存中可用的页框中,并建立相应的页表,这种方法解决了分区技术中存在的许多问题。
}
