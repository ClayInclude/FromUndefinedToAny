%%
%% Author: Clay
%% 2020/12/9
%%

\section{分段}
{
    用户程序可以用分段的方法进行细分,将程序和关联的数据划分成几个\emreg{段(Segment)}。
    尽管段有一个最大长度的限制,但并不要求所有的程序的所有段都具有相同的长度。
    和分页一样,分段技术中的逻辑地址由两部分组成:段号和偏移量。

    由于使用大小不等的段,分段类似于动态分区。
    在没有采用覆盖机制或没有使用虚拟内存的情况下,需要把程序的所有段全部加载到内存后才能执行该程序。
    与动态分区相比,其区别就在于分段方案中,一个程序可以占用多个内存分区,而这些分区可以是不连续的。

    分页对程序员来说是透明的,而分段通常是可见的。

    采用大小不等的段的另一个后果是,逻辑地址和物理地址不再具有简单的对应关系。
    与分页机制相类似,在简单分段方案中,每个进程都有一个段表,系统也会维护一个空闲内存块的列表。
    段表中的每个表项给出了相应的段在内存中的起始地址,同时还需要指明段的长度,以避免对无效地址的访问。
    物理地址需要使用逻辑地址加上段的起始地址。

    采用简单分段技术,进程被划分成许多大小不等的段。
    当进程被调入内存时,需要将该进程所有的段都加载到内存的可用分区中,并建立相应的段表。
}
