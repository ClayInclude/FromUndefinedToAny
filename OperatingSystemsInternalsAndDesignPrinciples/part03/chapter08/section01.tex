%%
%% Author: Clay
%% 2020/12/9
%%

\section{硬件和控制结构}
{
    分页和分段是内存管理取得根本性突破的关键所在。

    \begin{itemize}
        \item
        {
            进程中所有的内存访问都是基于逻辑地址的,这些逻辑地址在进程执行过程中动态地转换成物理地址。
            这意味着进程可以被换入或换出内存,使得进程在执行过程中不同时刻占用不同的内存区域。
        }
        \item 进程可以被划分成许多块,动态运行时地址转换和页表或段表的使用,使得进程在执行过程中这些块不需要占用连续的内存区域。
    \end{itemize}

    如果具备前面所说的分页和分段的特点,那么进程的执行过程中,不需要将进程所有的页或段全部装入内存。
    如果存放下一条指令的块被载入内存,同时待访问的下一个数据块也在内存,则进程可以继续执行一段时间。

    用属于\emreg{块(Piece)}来表示分页机制中的页或分段机制中的段。
    假设需要将一个新进程加载到内存,操作系统仅加载一个或几个块,这一个或几个块只包含位于程序起点的部分代码及这些代码需访问的数据。
    任何时刻都存放在内存中的这部分进程称为进程的\emreg{驻留集(Resident Set)}。
    在进程的执行过程中,凡是涉及驻留集中的内存访问,都可以顺利进行。
    如果处理器发现需访问的逻辑地址不在内存,就会产生一个中断提示内存访问出错。
    此时,操作系统会将这个被中断的进程设置为阻塞状态。
    为了能让进程继续执行,操作系统必须将包含有导致访问错误的逻辑地址的进程块加载到内存。
    为此,操作系统产生一个磁盘I/O读请求。
    在磁盘的I/O执行过程中,操作系统可以调度其他进程运行。
    一旦所需的块被加载到内存,则产生一个I/O中断,将控制权交回给操作系统,然后操作系统将被阻塞的这个进程设置为就绪状态。

    提高系统利用率的方法:

    \begin{description}
        \item[内存中可以存放更多的进程]
        {
            由于任何进程都只需要加载部分块到内存,因此内存空间可以容纳更多的进程。
            这将使处理器的使用更加高效,因为在任何时刻至少都有一个进程处于就绪状态。
        }
        \item[进程可以超过整个内存的容量]
        {
            程序受内存空间的限制,这是影响程序设计的最大限制之一,但这个限制被取消了。
            对程序员来说,要处理的只是一个大容量的内存,而内存的大小与磁盘存储器有关。
        }
    \end{description}

    由于进程只能在内存中执行,因此通常内存也称为\emreg{实存(Real Memory)}。
    但程序员或用户感觉到的是一个更大的存储器,这个存储器一般配置在磁盘上,称之为\emreg{虚存(Virtual Memory)}。
    虚存支持更有效地多道程序设计,让用户从不必要的内存限制中解脱出来。

    \subsection{局部性和虚拟内存}
    {
        任何时刻每个进程只存访一部分块在内存中,那么在内存中可以容纳更多的进程。
        甚至,因为未使用的块无需换入或换出内存,从而节省了时间。
        但是,操作系统必须要能够智能地管理这个方案。
        在稳定状态下,内存空间几乎全部被进程块占用,处理器和操作系统可以直接访问尽可能多的进程。
        当操作系统需要读入一个进程块时,就需要把内存中的某个进程快换出到磁盘。
        如果换出了一块即将被使用的进程块,则操作系统不得不很快又把它换回来。
        这类操作如果太多,将导致\emreg{抖动(Thrashing)}现象:
        系统将大量的时间花费在进程块的交换上,而不是指令的执行上。

        如何避免抖动是一个重要的研究领域,出现了一些列复杂但非常有效的算法。
        本质上,这些算法都是操作系统根据最近的历史信息来猜测那些块最有可能马上被访问。

        这些决策基于\emreg{局部性原理(Principle of Locality)}进行推断。
        简单来说,局部性原理是指进程中的程序和数据访问具有聚集性趋势。

        局部性原理说明虚拟内存机制是有效的。
        为了使虚拟内存更加实用且高效,需要以下两方面的支持:
        首先,需要有对所采用的分页或分段方案的硬件支持;
        其次,需要有能在内存和辅存之间移动页和段的操作系统管理软件。
    }

    \subsection{分页}
    {
        尽管基于分段的虚拟内存也较为常见,但属于虚拟内存通常与使用分页的系统联系在一起。

        由于每个进程只有部分页在内存中,因此也标得每个表项需要有一位P来标记它所对应的页当前是否在内存中。

        页表的表项还包含一个修改位M,标记相应页的内容自上次加载到内存后有没有被修改。

        如果内容没有变化,则当需要把该页换出内存时,不需要写出该页的内容。
    }

    \subsection{页表结构}
    {

    }
}
