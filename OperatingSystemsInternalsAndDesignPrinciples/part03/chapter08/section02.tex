%%
%% Author: Clay
%% 2020/12/11
%%

\section{操作系统软件}
{
    操作系统中内存管理的设计取决于以下3个方面的选择:

    \begin{itemize}
        \item 是否使用虚拟内存技术
        \item 使用分页还是分段,或是二者的结合
        \item 为实现各种内存管理功能采用的算法
    \end{itemize}

    前两方面的选择取决于使用的硬件平台。
    如果没有地址转换和其他基本功能的硬件支持,分页和分段技术都是不可能真正实用的。

    针对前两方面给出两个附加的建议:
    首先,除了一些老式个人计算机上的操作系统和专用系统之外,几乎所有操作系统都提供了虚拟内存的功能;
    其次,几乎没有存粹的分段系统。
    当分段与分页相结合后,操作系统设计者面临的大多数内存管理问题都来源于分页。

    第3方面的选择属于操作系统软件领域的问题。
    在各种情况下,最关键的问题就是性能:
    由于缺页带来的巨大的软件开销,所以希望尽可能减少缺页率。

    分页环境中的内存管理任务是极其复杂的。
    此外,任何一个策略集的性能取决于内存的大小、内存和辅存的相对速度、竞争使用资源的进程的大小和数量,以及单个程序的执行行为。

    对小型系统,操作系统设计者可以基于当前状态信息,尝试选择一组看上去在大多数条件都比较好的策略。
    而对大型系统,尤其是大型机,操作系统应该配被监视和监控工具,允许系统管理员根据系统状态调整操作系统。

    \subsection{读取策略}
    {
        读取策略用于确定将页读入内存的时机,常用的两种可选方案是\emreg{请求分页(Demand Paging)}和\emreg{预分页(Prepaging)}。
        对于请求分页,只有当访问某页中的单元时,才将该页加载到内存。
        如果采用合适的内存管理策略,将会发生以下情况:
        当进程第一次开始执行时,会出现大量的缺页中断。
        根据局部性原理,一段时间后,缺页异常会逐渐减少。

        对于预分页,加载到内存的页并不是缺页请求时请求的页。
        预分页利用大多数辅存设备的特性,一次读取许多连续的页比隔一段时间读取一页更加有效。
    }

    \subsection{放置策略}
    {
        在一个纯粹的分段系统中,放置策略是一个非常重要的设计问题。
        对于纯粹的分页系统或段页式系统来说,放置到哪里通常是无关紧要的。

        在所谓的\emreg{非一致存储访问(Nonuniform Memory Access), NUMA}多处理器中,机器中的分布式共享内存可以被任何一个处理器访问,但是访问某一特定物理单元的时间随着处理器与内存模块之间距离的不同而变化。
        其性能很大程度依赖于数据驻留的位置与使用该数据的处理器之间的距离。
    }
}
