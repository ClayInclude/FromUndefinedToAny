%%
%% Author: Clay
%% 2020/8/17
%%

\chapter{读者和教师指南}
{
    \section{本书概述}
    {
        本书分为8个部分。

        \begin{description}
            \item[背景] 提供计算机系统结构与组织的综述,重点讲述与操作系统设计相关的主题,并对本书其余部分的操作系统主题进行了综述。
            \item[进程] 详细分析了进程、多线程、对称多处理(SMP)和微内核,还分析了单系统中的并发机制,重点讲述了互斥和死锁等问题。
            \item[内存] 提供对内存管理技术,包括虚拟内存的综合性论述。
            \item[调度] 综合探讨进程调度的各种方法,同时还讨论了线程调度、SMP调度和实时调度。
            \item[输入/输出与文件] 分析操作系统中对I/O功能的控制问题,尤其是磁盘I/O,它是决定系统性能的关键所在。
            \item[嵌入式系统]嵌入式系统的数量远远超过通用目的的计算机系统。
            \item[安全] 对涉及计算机和网络安全的威胁和机制进行了概述。
            \item[分布式系统] 分析了计算机系统网络化的主要趋势,包括TCP/IP、客户端/服务器计算和集群。
        \end{description}
    }

    \section{实例系统}
    {

    }

    \section{读者和教师学习路线图}
    {

    }

    \section{互联网和网站资源}
    {

    }
}

\cleardoublepage

\endinput
