%%
%% Author: Clay
%% 2020/2/19
%%

\chapter{前言}
{
    其他的系统类书籍都是从\emreg{构建者的角度}来写的,讲述如何实现硬件或软件,包括操作系统、编译器和网络接口。
    而本书是从\emreg{程序员的角度}来写的,讲述应用程序员如何能够利用系统知识来编写出更好的程序。

    \section{读者应具备的背景知识}
    {
        假设你能访问一台这样的机器,并且能够登录,做一些诸如切换目录之类的简单操作。

        还假设你对\emcode{C和C++}有一定的了解。
    }

    \section{如何阅读此书}
    {
        我们相信学习系统的唯一方法就是\emreg{做(do)}系统。
    }

    \section{本书概述}
    {
        \begin{enumerate}[第1章:]
            \item \emspe{计算机系统漫游。}
            \item \emspe{信息的表示和处理。}
            \item \emspe{程序的机器级表示。}
            \item \emspe{处理器体系结构。}
            \item \emspe{优化程序性能。}
            \item \emspe{存储器层次结构。}
            \item \emspe{链接。}
            \item \emspe{异常控制流。}
            \item \emspe{虚拟内存。}
            \item \emspe{系统级I/O。}
            \item \emspe{网络编程。}
            \item \emspe{并发编程。}
        \end{enumerate}
    }
}

\cleardoublepage

\endinput
