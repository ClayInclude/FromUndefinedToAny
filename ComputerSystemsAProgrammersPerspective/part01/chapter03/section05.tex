%%
%% Author: Clay
%% 2020/4/2
%%

\section{算术和逻辑操作}
{
    大多数整数和逻辑操作都分成了指令类,这些指令类有各种带不同大小操作数的变种。
    这些操作被分成四组:
    加载有效地址、一元操作、二元操作和移位。

    \subsection{加载有效地址}
    {
        \emreg{加载有效地址(load effective address)}指令\emcode{leaq}的指令形式是从内存读数据到寄存器,但实际上它根本就没有引用内存。
        它的第一个操作数看上去是一个内存引用,但该指令并不是从指定的位置读入数据,而是将有效地址写入到目的操作数。

        %6.
        \begin{practicec}
            \begin{enumerate}[A.]
                \item $x + 6$
                \item $x + y$
                \item $x + 4y$
                \item $x + 8y + 7$
                \item $4y + 10$
                \item $x + 2y + 9$
            \end{enumerate}
        \end{practicec}

        %7.
        \begin{practicec}
            5 * x + 2 * y + 8 * z
        \end{practicec}
    }

    \subsection{一元和二元操作}
    {
        一元操作只有一个操作数,既是源又是目的。
        这个操作数可以是一个寄存器,也可以是一个内存位置。

        二元操作的第二个操作数既是源又是目的。
        第一个操作数可以是立即数、寄存器或内存位置。
        第二个操作数可以是寄存器或内存位置。
        当第二个操作数为内存地址时,处理器必须从内存读出值,执行操作,再把结果写回内存。


    }
}
