%%
%% Author: Clay
%% 2020/2/20
%%

\chapter{程序的机器级表示}
{
    计算机执行\emreg{机器代码},用字节序列编码低级的操作,包括处理数据、管理内存、读写存储设备上的数据,以及利用网络通信。
    GCC C语言编译器以\emreg{汇编代码}的形式产生输出,汇编代码是机器代码的文本表示。
    然后GCC调用\emreg{汇编器}和\emreg{链接器},根据汇编代码生成可执行的机器代码。

    使用高级语言编程的时候,机器屏蔽了程序的细节,即机器级的实现。
    使用汇编代码编程的时候,必须指定程序用来执行计算的低级指令。

    阅读编译器产生的汇编代码,需要具备的技能不同于手工编写汇编代码。
    必须了解典型的编译器在将C程序结构变换成机器代码时所做的转换。
    相对于C代码表示的计算操作,优化编译器能够重新排列执行顺序,消除不必要的计算,用快速操作替换慢速操作,甚至将递归计算变换成迭代计算。
    源代码与对应的汇编代码的关系通常不太容易理解。
    这是一种\emreg{逆向工程(reverse engineering)},通过研究系统和你想工作,来试图了解系统的创建过程。

    \emreg{精通细节是理解更深和更基本概念的先决条件}。

    %%
%% Author: Clay
%% 2020/12/5
%%

\section{死锁的原理}
{
    死锁是指一组进程因为竞争系统资源或互相等待消息,而永远无法向前推进的状态。

    \subsection{可重用资源}
    {
        资源可以分为两个大类:
        可重用资源与消耗性资源。

        可重用资源指的是一次只供一个进程使用,但用后又能被别的进程使用的资源。

        从系统的角度出发,能够解决死锁问题的方法之一是对应用申请系统资源的顺序做出限定。
    }
}

}

\cleardoublepage

\endinput
