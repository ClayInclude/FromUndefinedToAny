%%
%% Author: Clay
%% 2020/2/20
%%

\chapter{程序的机器级表示}
{
    计算机执行\emreg{机器代码},用字节序列编码低级的操作,包括处理数据、管理内存、读写存储设备上的数据,以及利用网络通信。
    GCC C语言编译器以\emreg{汇编代码}的形式产生输出,汇编代码是机器代码的文本表示。
    然后GCC调用\emreg{汇编器}和\emreg{链接器},根据汇编代码生成可执行的机器代码。

    使用高级语言编程的时候,机器屏蔽了程序的细节,即机器级的实现。
    使用汇编代码编程的时候,必须指定程序用来执行计算的低级指令。

    阅读编译器产生的汇编代码,需要具备的技能不同于手工编写汇编代码。
    必须了解典型的编译器在将C程序结构变换成机器代码时所做的转换。
    相对于C代码表示的计算操作,优化编译器能够重新排列执行顺序,消除不必要的计算,用快速操作替换慢速操作,甚至将递归计算变换成迭代计算。
    源代码与对应的汇编代码的关系通常不太容易理解。
    这是一种\emreg{逆向工程(reverse engineering)},通过研究系统和逆向工作,来试图了解系统的创建过程。

    \emreg{精通细节是理解更深和更基本概念的先决条件}。

    %%
%% Author: Clay
%% 2020/2/21
%%

\section{信息存储}
{
    大多数计算机使用8位的块,或者\emreg{字节(byte)},作为最小的可寻址的内存单位,而不是访问内存中单独的位。
    机器级程序将内存视为一个非常大的字节数组,称为\emreg{虚拟内存(virtual memory)}。
    内存的每个字节都由一个唯一的数字来标识,称为它的\emreg{地址(address)},所有可能地址的集合就称为\emreg{虚拟地址空间(virtual address space)}。

    编译器和运行时系统将存储器空间划分为更可管理的单元,来存放不同的\emreg{程序对象(program object)}。

    \subsection{十六进制表示法}
    {
        \emreg{十六进制(hexadecimal, hex)}使用数字 $0 \sim 9$ 和字符 $A \sim F$ 来表示16个可能的值。

        通过展开每个十六进制数字,将它转换为二进制格式。

        如果给定一个二进制数字,可以通过把它分为每4位一组来转换为十六进制。
        如果位总数不是4的倍数,最左边的一组可以少于4位,前面用0补足。

        %1.
        \begin{practicec}
            \begin{enumerate}[A.]
                \item $11 1001 1010 0111 1111 1000$
                \item $0xC97A$
                \item $1101 0101 1110 0100 1100$
                \item $0x26E7B5$
            \end{enumerate}
        \end{practicec}

        当值 $x = 2^n$ 时, $x$ 的二进制表示就是 $1$ 后面跟 $n$ 个 $0$ 。
        十六进制数字 $0$ 代表 4个二进制 $0$ 。
        所以当 $n$ 表示成 $i + 4j$ 的形式,其中 $0 \leq i \leq 3$ ,可以把 $x$ 写为开头的十六进制数字为 $1(i = 0), 2(i = 1), 4(i = 2), 8(i = 3)$ ,后面紧跟 $j$ 个十六进制的 $0$ 。

        %2.
        \begin{practicec}
            \begin{table}[H]
                \[
                    \begin{array}{|c|c|c|}
                        \hline
                        n & 2^n \text{(十进制)} & 2^n \text{(十六进制)} \\
                        \hline
                        9 & 512 & 0x200 \\
                        \hline
                        19 & 524288 & 0x80000 \\
                        \hline
                        14 & 16384 & 0x4000 \\
                        \hline
                        16 & 65536 & 0x10000 \\
                        \hline
                        17 & 131072 & 0x20000 \\
                        \hline
                        5 & 32 & 0x20 \\
                        \hline
                        7 & 128 & 0x80 \\
                        \hline
                    \end{array}
                \]
            \end{table}
        \end{practicec}

        将一个十进制数字 $x$ 转换为十六进制,可以反复地用 $16$ 除 $x$ ,得到一个商 $q$ 和一个余数 $r$ ,也就是 $x = q \cdot 16 + r$ 。
        然后用十六进制数字表示的 $r$ 作为最低为数字,并通过对 $q$ 反复进行这个过程得到剩下的数字。

        将一个十六进制数字转换为十进制数字,可以用相应的 $16$ 的幂乘以每个十六进制数字。

        %3.
        \begin{practicec}
            \begin{table}[H]
                \[
                    \begin{array}{|c|c|c|}
                        \hline
                        \text{十进制} & \text{二进制} & \text{十六进制} \\
                        \hline
                        0 & 0000 0000 & 0x00 \\
                        \hline
                        167 & 1010 0111 & A7 \\
                        \hline
                        62 & 0011 1110 & 3E \\
                        \hline
                        188 & 1011 1100 & BC \\
                        \hline
                        55 & 0011 0111 & 37 \\
                        \hline
                        136 & 1000 1000 & 88 \\
                        \hline
                        243 & 1111 0011 & F3 \\
                        \hline
                        82 & 0101 0010 & 0x52 \\
                        \hline
                        172 & 1010 1100 & 0xAC \\
                        \hline
                        231 & 1110 0111 & 0xE7 \\
                        \hline
                    \end{array}
                \]
            \end{table}
        \end{practicec}

        %4.
        \begin{practicec}
            \begin{enumerate}[A.]
                \item $0x5044$
                \item $0x4FFC$
                \item $0x50A0$
                \item $0x9E$
            \end{enumerate}
        \end{practicec}
    }

    \subsection{字数据大小}
    {
        每台计算机都有一个\emreg{字长(word size)},指明指针数据的\emreg{标称大小(nominal size)}。
        对于一个字长为 $w$ 位的机器而言,虚拟地址的范围为 $0 \sim 2^w - 1$ ,程序最多访问 $2^w$ 个字节。

        整数或者为\emreg{有符号的},即可以表示负数、零和正数;
        或者为\emreg{无符号的},即只能表示非负数。
    }

    \subsection{寻址和字节顺序}
    {
        在几乎所有的机器上,多字节对象都被存储为连续的字节序列,对象的地址为所使用字节中最小的地址。

        考虑一个 $w$ 位的整数,其表示为 $[x_{w - 1}, x_{w - 2}, \cdots, x_1, x_0]$ ,其中 $x_{w - 1}$ 是最高有效位,而 $x_0$ 是最低有效位。
        假设 $w$ 是 $8$ 的倍数,这样这些位就能被分组成为字节。
        某些机器选择在内存中按照从最低有效字节到最高有效字节的顺序存储对象,而另一些机器则按照从最高有效字节到最低有效字节的顺序存储。
        前一种规则称为\emreg{小端法(little endian)},后一种规则称为\emreg{大端法(big endian)}。

        有时候,字节序会成为问题。

        \begin{enumerate}
            \item
            {
                首先是在不同类型的机器之间通过网络传送二进制数据时,一个常见的问题是当小端法机器产生的数据被发送到大端机器或者反过来时,接收程序会发现,字里的字节成了反序的。
            }
            \item
            {
                第二种情况时,当阅读表示整数数据的字节序列时字节顺序也很重要。
            }
            \item
            {
                第三种情况是当编写规避正常的类型系统的程序时。
            }
        \end{enumerate}
    }
}

    %%
%% Author: Clay
%% 2020/12/15
%%

\section{静态链接}
{
    \emreg{静态链接器(static linker)}以一组可重定位目标文件和命令行参数作为输入,生成一个完全链接的、可以加载和运行的可执行目标文件作为输出。
    输入的可重定位目标文件由各种不同的代码和数据\emreg{节(section)}组成,每一节都是一个连续的字节序列。

    为了构造可执行文件,链接器必须完成两个任务:

    \begin{description}
        \item[符号解析(symbol resolution)]
        {
            目标文件定义和引用\emspe{符号},每个符号对应一个函数、一个全局变量或一个\emspe{静态变量}(即C语言中任何以\emcode{static}属性声明的变量)。
            符号解析的目的是将每个符号\emspe{引用}正好和一个符号\emspe{定义}关联起来。
        }
        \item[重定位(relocation)]
        {
            编译器和汇编器生成从地址0开始的代码和数据节。
            链接器通过把每个符号定义与一个内存位置关联起来,从而\emspe{重定位}这些节,然后修改所有对这些符号的引用,使得它们指向这个内存位置。
            链接器使用汇编器产生的\emspe{重定位条目(relocation entry)}的详细指令。
        }
    \end{description}

    目标文件纯粹是字节块的集合。
    这些块中,有些包含程序代码,有些包含程序数据,而其他的则包含引导链接器和加载器的数据结构。
    链接器将这些块连接起来,确定被连接块的运行时位置,并且修改代码和数据块中的各种位置。
    链接器对目标机器了解甚少,产生目标文件的编译器和汇编器已经完成了大部分工作。
}

    %%
%% Author: Clay
%% 2020/2/29
%%

\section{整数运算}
{
    \subsection{无符号加法}
    {
        要想完整的表示算术运算的结果,不能对字长做任何限制。
        \emcode{List}实际上就支持\emreg{无限精度}的运算,允许任意的(要在机器的内存限制之内)整数运算。
        更常见的是,编程语言支持固定精度的运算。

        \begin{defines}[无符号数加法]
            对满足 $0 \leq x, y < 2^w$ 的 $x, y$ 有:

            \begin{align}
                x +_w^u y =
                \begin{cases}
                    x + y, x + y < 2^w
                    \\
                    x + y - 2^w, 2^w \leq x + y < 2^{w + 1}
                \end{cases}
            \end{align}
        \end{defines}

        说一个算数\emreg{溢出},是指完整的整数结果不能放到数据类型的字长限制中去。

        \begin{defines}[检测无符号数加法中的溢出]
            对在范围 $0 \leq x, y \leq UMax_w$ 中的 $x, y$ ,令 $s = x +_w^u y$ 。
            则对计算 $s$ ,当且仅当 $s < x$ (或者等价地 $s < y$ )时,发生了溢出。
        \end{defines}

        %27.
        \begin{practicec}

        \end{practicec}

        模数加法形成了一种数学结构,称为\emreg{阿贝尔群(Abelian group)}。
        也就是说,它是可交换的和可结合的。
        它有一个单位元 $0$ ,并且每个元素有一个加法逆元。

        \begin{defines}[无符号数求反]
            对满足 $0 \leq x < 2^w$ 的任意 $x$ ,其 $w$ 位的无符号逆元 $-_w^ux$ 由下式给出:

            \begin{align}
                -_w^u x =
                \begin{cases}
                    x, x = 0
                    \\
                    2^w - x, x > 0
                \end{cases}
            \end{align}
        \end{defines}

        %28.
        \begin{practicec}
            \begin{table}[H]
                \[
                    \begin{array}{|c|c|c|c|}
                        \hline
                        \multicolumn{2}{|c|}{x} & \multicolumn{2}{c|}{-_4^ux} \\
                        \hline
                        \text{十六进制} & \text{十进制} & \text{十进制} & \text{十六进制} \\
                        \hline
                        0 & 0 & 0 & 0 \\
                        \hline
                        5 & 5 & 11 & B \\
                        \hline
                        8 & 8 & 8 & 8 \\
                        \hline
                        D & 13 & 3 & 3 \\
                        \hline
                        F & 15 & 1 & 1 \\
                        \hline
                    \end{array}
                \]
            \end{table}
        \end{practicec}
    }

    \subsection{补码加法}
    {
        \begin{defines}[补码加法]
            对满足 $-2^{w - 1} \leq x, y \leq 2^{w - 1} - 1$ 的整数 $x, y$ ,有:

            \begin{align}
                x+_w^ty =
                \begin{cases}
                    x + y - 2^w, 2^{w - 1} \leq x + y
                    \\
                    x + y, -2^{w - 1} \leq x + y < 2{w - 1}
                    \\
                    x + y + 2^w, x + y < -2^{w - 1}
                \end{cases}
            \end{align}
        \end{defines}

        \begin{defines}[检测补码加法中的溢出]
            对满足 $TMin_w \leq x, y \leq TMax_w$ 的 $x, y$ ,令 $s = x +_w^t y$ 。
            当且仅当 $x > 0, y > 0$ ,但 $s \leq 0$ 时,计算发生了\emspe{正溢出(positive overflow)}。
            当且仅当 $x < 0, y < 0$ ,但 $s \geq 0$ 时,计算发生了\emspe{负溢出(negative overflow)}。
        \end{defines}

        %29.
        \begin{practicec}
            \begin{table}[H]
                \[
                    \begin{array}{|c|c|c|c|c|}
                        \hline
                        x & y & x + y & x +_5^t y & \text{情况} \\
                        \hline
                        10100 & 10001 & -27 & 5 & \text{负溢出} \\
                        \hline
                        11000 & 11000 & -16 & -16 & \text{正常} \\
                        \hline
                        10111 & 01000 & -1 & -1 & \text{正常} \\
                        \hline
                        00010 & 00101 & 7 & 7 & \text{正常} \\
                        \hline
                        01100 & 00100 & 16 & -16 & \text{正溢出} \\
                        \hline
                    \end{array}
                \]
            \end{table}
        \end{practicec}

        %30.
        \begin{practicec}

        \end{practicec}

        %31.
        \begin{practicec}
            如果有 $sum = x + y$ ,由于补码加法形成了一个阿贝尔群,满足交换律,则有 $sum - x = x + y - x$ ,得出 $sum - x = y$ 。
            无论溢出都会得到相同的结果。
        \end{practicec}

        %32.
        \begin{practicec}
            $y$ 为 $TMin$ 时。
        \end{practicec}
    }

    \subsection{补码的非}
    {
        \begin{defines}[补码的非]
            对满足 $TMin_w \leq x \leq TMax_w$ 的 $x$ ,其补码的非 $-_w^tx$ 由下式给出

            \begin{align}
                -_w^tx =

                \begin{cases}
                    TMin_w, x = TMin_w
                    \\
                    -x, x ? TMin_w
                \end{cases}
            \end{align}
        \end{defines}

        %33.
        \begin{practicec}
            \begin{table}[H]
                \[
                    \begin{array}{|c|c|c|c|}
                        \hline
                        \multicolumn{2}{|c|}{x} & \multicolumn{2}{c|}{-_4^tx} \\
                        \hline
                        \text{十六进制} & \text{十进制} & \text{十进制} & \text{十六进制} \\
                        \hline
                        0 & 0 & 0 & 0 \\
                        \hline
                        5 & 5 & -5 & A \\
                        \hline
                        8 & -8 & -8 & 8 \\
                        \hline
                        D & -3 & 3 & 3 \\
                        \hline
                        F & -1 & 1 & 1 \\
                        \hline
                    \end{array}
                \]
            \end{table}
        \end{practicec}
    }

    \subsection{无符号乘法}
    {
        \begin{defines}[无符号数乘法]
            对满足 $0 \leq x, y \leq < UMax_w$ 的 $x, y$ 有:

            \begin{align}

            \end{align}
        \end{defines}
    }
}

    %%
%% Author: Clay
%% 2021/2/22
%%

\section{套接字接口}
{
    \emreg{套接字接口(socket interface)}是一组函数,它们和Unix I/O函数结合起来,用以创建网络应用。

    \subsection{套接字地址结构}
    {
        从Linux程序的角度来看,套接字就是一个有相应描述符的打开文件。

        IP地址和端口号总是以网络字节顺序存放的。
    }

    \subsection{socket函数}
    {
        客户端和服务器使用socket函数来创建一个\emreg{套接字描述符(socket descriptor)}。

        socket返回的clientfd描述符仅是部分打开的,还不能用于读写。
    }

    \subsection{connect函数}
    {
        客户端通过调用connect函数来建立和服务器的连接。
    }

    \subsection{bind函数}
    {
        bind函数告诉内核将addr中的服务器套接字地址和套接字描述符sockfd联系起来。
    }

    \subsection{listen函数}
    {
        客户端时发起连接请求的主动实体。
        服务器是等待来自客户端的连接请求的被动实体。
        默认情况下,内核会认为socket函数创建的描述对应于\emreg{主动套接字(active socket)},它存在于一个连接的客户端。
        服务器调用listen函数告诉内核,描述符是被服务器而不是客户端使用的。

        listen函数将sockfd从一个主动套接字转化为一个\emreg{监听套接字(listening socket)},该套接字可以接受来自客户端的连接请求。
    }

    \subsection{accept函数}
    {
        服务器通过调用accept函数来等待来自客户端的连接请求。

        acept函数等待来自客户端的连接请求到达侦听描述符listenfd,然后在addr中填写客户端的套接字地址,并返回一个\emreg{已连接描述符(connected descriptor)}。
    }

    \subsection{主机和服务的转换}
    {
        Linux提供了一些强大的函数实现二进制套接字地址结构和主机名、主机地址、服务名和端口号的字符串表示之间的相互转化。
        当和套接字接口一起使用时,这些函数能编写独立于任何特定版本的IP协议的网络程序。

        \subsubsection{getaddrinfo函数}
        {
            getaddrinfo函数将主机名、主机地址、服务器名和端口号的字符串表示转化成套接字地质结构。

            应用程序必须在最后调用freeaddrinfo,避免内存泄漏。
        }

        \subsubsection{getnameinfo}
        {
            getnameinfo函数和getaddrinfo是相反的,将一个套接字地址结构转换成相应的主机和服务名字符串。
        }
    }

    \subsection{套接字接口的辅助函数}
    {
        \subsubsection{open\_clientfd函数}
        {
            客户端调用open\_clientfd函数建立与服务器的连接。
        }

        \subsubsection{open\_listenfd函数}
        {
            调用open\_listenfd函数,服务器创建一个监听描述符,准备好接收连接请求。
        }
    }

    \subsection{echo客户端和服务器的示例}
    {
        简单的echo服务器一次只能处理一个客户端。
        这种类型的服务器一次一个地在客户端间迭代,称为\emreg{迭代服务器(iterative server)}。
    }
}

    %%
%% Author: Clay
%% 2020/8/20
%%

\section{分级存储体系}
{
    需要权衡考虑存储器的容量、访问时间和成本这三个关键因素。
    往往遵循如下规律:

    \begin{itemize}
        \item 存储时间越快,单位成本越高。
        \item 容量越大,单位成本越低。
        \item 容量越大,存取速度越慢。
    \end{itemize}

    充分利用\emreg{分级存储体系(Memory Hierarchy)}。
    随着层次下移,具有如下特征:

    \begin{itemize}
        \item 单位成本逐层递减
        \item 容量逐层递增
        \item 访问时间逐层递增
        \item 处理器访问存储器的频率逐层递减
    \end{itemize}

    价格较高、容量较小的告诉存储器通常会配备价格低廉、容量较大的慢速存储器作为补充。
    这种架构成功的关键是减少位于较低层的慢速设备的访问频率。原则上可以实现二级存储结构的存储策略,并具备列举的特征1到4。
    现有的存储系统通过多种技术,也都能满足特征1到3。
    幸运的是,特征4通常也是有效的。

    特征4有效的基础是访问的\emreg{局部性原理}。
    在程序的执行过程中,无论是指令还是数据,处理器对存储器的访问都呈现出聚焦访问的特点。
    较短的时间内,处理器主要访问存储器中固定的数据集。
    可以利用数据的层次来组织数据,数据组织层次越低,数据访问的次数也就越少。

    寄存器、高速缓存和内存通常具有易失性,并采用了半导体技术。
    数据一般保存在能永久保存的大容量外部存储设备中,常见的有硬盘和移动存储介质。
    非易失性的外部存储器又称为\emreg{二级存储器(Secondary Memory)}或\emreg{辅助存储器(Auxiliary Memory)},用于存储程序和数据文件。
    除此之外,硬盘还用于作为内存的扩展,称为虚拟内存。

    内存的一部分可以作为\emreg{缓冲区(Buffer)},临时保存从磁盘读出的数据。
    该技术有时称为磁盘高速缓存,可以通过下面两种方法改善性能:

    \begin{itemize}
        \item 磁盘成簇写。采用大容量数据少次传递,将数据聚集到一定容量后在进行一次性传输,以改进磁盘性能,并最大限度减少处理器的参与。
        \item 对必定会写出到磁盘的数据,在数据保存到磁盘之间,能从磁盘高速缓存中快速取出。
    \end{itemize}
}

    %%
%% Author: Clay
%% 2020/8/10
%%

\section{综合:高速缓存对程序性能的影响}
{
    \subsection{存储器山}
    {
        一个程序从存储系统中读数据的速率称为\emreg{读吞吐量(read throughput)},或者有时称为\emreg{读带宽(read bandwidth)}。

        size的值越小,得到的工作集越小,因此时间局部性越好。
        stride的值越小,得到的空间局部性越好。
        反复以不同的size和stride值调用run函数,那么就能得到一个读带宽的时间和空间局部性的二维函数,称为\emreg{存储器山(memory mountain)}。

        当程序的时间局部性很差时,空间局部性任然能补救。

        %21.
        \begin{practicec}
            $2100Mhz / (12000 MB / s) * 8B \simeq 1.5 hz \cdot s$
        \end{practicec}
    }

    \subsection{重新排列循环以提高空间局部性}
    {

    }

    \subsection{在程序中利用局部性}
    {
        理解存储器层次结构本质能够利用这些知识编写出更有效的程序,无论具体的存储系统结构是怎样的。

        \begin{itemize}
            \item 将注意力集中在内循环上,大部分计算和内存访问都发生在这里。
            \item 通过按照数据对象存储在内存中的顺序、以步长为1的来读数据,从而使得程序中的空间局部性最大。
            \item 一旦从存储器中读入了一个数据对象,就尽可能多地使用它,从而使得程序中的时间局部性最大。
        \end{itemize}
    }
}

    %%
%% Author: Clay
%% 2020/8/10
%%

\section{小结}
{
    %22.
    \begin{practicec}
        $x = 0.5$
    \end{practicec}

    %23.
    \begin{practicec}
        \begin{align*}
            T_{avg seek} &= 4ms \\
            \\
            T_{avg rotation} &= \frac{1}{RPM} * \frac{60s}{1min} * \frac{1}{2} \\
            &= 2ms \\
            \\
            T_{avg transfer} &= T_{max rotation} * \frac{1}{\text{平均扇区数/磁道}} \\
            &= 0.005ms \\
            \\
            T_{access}
            &= T_{avg seek} + T_{avg rotation} + T_{avg transfer} \\
            &= 6.005ms
        \end{align*}
    \end{practicec}

    %24.
    \begin{practicec}
        \begin{align*}
            T_{avg seek} &= 4ms \\
            \\
            T_{max rotation} &= \frac{1}{RPM} * \frac{60s}{1min} \\
            &= 4ms\\
            \\
            T_{avg rotation} &= T_{max rotation} * \frac{1}{2} \\
            &= 2ms \\
            \\
            T_{avg transfer} &= T_{max rotation} * \frac{1}{\text{平均扇区数/磁道}} \\
            &= 0.002ms
        \end{align*}

        \begin{enumerate}[A.]
            \item
            {
                \begin{align*}
                    T_{avg access} &= T_{avg seek} + T_{avg rotation} + 4 * T_{max rotation} \\
                    &= 22ms
                \end{align*}
            }
            \item
            {
                \begin{align*}
                    T_{avg access} &=  (T_{avg seek} + T_{avg rotation} + T_{avg transfer}) * 4000 \\
                    &= 24008ms
                \end{align*}
            }
        \end{enumerate}
    \end{practicec}

    %25.
    \begin{practicec}
        \begin{table}[htb]
            \[
                \begin{array}{|c|c|c|c|c|c|c|c|c|}
                    \hline
                    \text{高速缓存} & m & C & B & E & S & t & s & b \\
                    \hline
                    1 & 32 & 1024 & 4 & 4 & 64 & 24 & 6 & 2 \\
                    \hline
                    1 & 32 & 1024 & 4 & 256 & 1 & 30 & 0 & 2 \\
                    \hline
                    1 & 32 & 1024 & 8 & 1 & 128 & 22 & 7 & 3 \\
                    \hline
                    1 & 32 & 1024 & 8 & 128 & 1 & 29 & 0 & 3 \\
                    \hline
                    1 & 32 & 1024 & 32 & 1 & 32 & 22 & 5 & 5 \\
                    \hline
                    1 & 32 & 1024 & 32 & 4 & 8 & 24 & 3 & 5 \\
                    \hline
                \end{array}
            \]
        \end{table}
    \end{practicec}

    %26.
    \begin{practicec}
        \begin{table}[htb]
            \[
                \begin{array}{|c|c|c|c|c|c|c|c|c|}
                    \hline
                    \text{高速缓存} & m & C & B & E & S & t & s & b \\
                    \hline
                    1 & 32 & 2048 & 8 & 1 & 256 & 21 & 8 & 3 \\
                    \hline
                    2 & 32 & 2048 & 4 & 4 & 128 & 23 & 7 & 2 \\
                    \hline
                    3 & 32 & 1024 & 2 & 8 & 64 & 25 & 6 & 1 \\
                    \hline
                    4 & 32 & 1024 & 32 & 2 & 16 & 23 & 4 & 5 \\
                    \hline
                \end{array}
            \]
        \end{table}
    \end{practicec}

    %% practice 12
    % 0: 0 0001 0010 0000
    % 0: 0 0001 0010 0001
    % 0: 0 0001 0010 0010
    % 0: 0 0001 0010 0011

    % 1: 0 1000 1010 0100
    % 1: 0 1000 1010 0101
    % 1: 0 1000 1010 0110
    % 1: 0 1000 1010 0111
    % 1: 0 0111 0000 0100
    % 1: 0 0111 0000 0101
    % 1: 0 0111 0000 0110
    % 1: 0 0111 0000 0111

    % 3: 0 0110 0100 1100
    % 3: 0 0110 0100 1101
    % 3: 0 0110 0100 1110
    % 3: 0 0110 0100 1111

    % 4: 1 1000 1111 0000
    % 4: 1 1000 1111 0001
    % 4: 1 1000 1111 0010
    % 4: 1 1000 1111 0011
    % 4: 0 0000 1011 0000
    % 4: 0 0000 1011 0001
    % 4: 0 0000 1011 0010
    % 4: 0 0000 1011 0011

    % 5: 0 1110 0011 0100
    % 5: 0 1110 0011 0101
    % 5: 0 1110 0011 0110
    % 5: 0 1110 0011 0111

    % 6: 1 0010 0011 1000
    % 6: 1 0010 0011 1001
    % 6: 1 0010 0011 1010
    % 6: 1 0010 0011 1011

    % 7: 1 1011 1101 1100
    % 7: 1 1011 1101 1101
    % 7: 1 1011 1101 1110
    % 7: 1 1011 1101 1111
    %%

    %27.
    \begin{practicec}
        \begin{enumerate}[A.]
            \item
            {
                0 1000 1010 0100
                0 1000 1010 0101
                0 1000 1010 0110
                0 1000 1010 0111
                0 0111 0000 0100
                0 0111 0000 0101
                0 0111 0000 0110
                0 0111 0000 0111
            }
            \item
            {
                1 0010 0011 1000
                1 0010 0011 1001
                1 0010 0011 1010
                1 0010 0011 1011
            }
        \end{enumerate}
    \end{practicec}

    %28.
    \begin{practicec}
        \begin{enumerate}[A.]
            \item
            {
                无
            }
            \item
            {
                1 1000 1111 0000
                1 1000 1111 0001
                1 1000 1111 0010
                1 1000 1111 0011
                0 0000 1011 0000
                0 0000 1011 0001
                0 0000 1011 0010
                0 0000 1011 0011
            }
            \item
            {
                0 1110 0011 0100
                0 1110 0011 0101
                0 1110 0011 0110
                0 1110 0011 0111
            }
            \item
            {
                1 1011 1101 1100
                1 1011 1101 1101
                1 1011 1101 1110
                1 1011 1101 1111
            }
        \end{enumerate}
    \end{practicec}

    %29.
    \begin{practicec}
        \begin{enumerate}[A.]
            \item CT: 8; CI: 2; CO: 2
            \item
            {
                \begin{table}[htb]
                    \begin{tabular}{|c|c|}
                        \hline
                        否 & - \\
                        \hline
                        是 & - \\
                        \hline
                        是 & C0 \\
                        \hline
                    \end{tabular}
                \end{table}
            }
        \end{enumerate}
    \end{practicec}

    %30.
    \begin{practicec}
        \begin{enumerate}[A.]
            \item 128
            \item CT: 8; CI: 3; CO: 2
        \end{enumerate}
    \end{practicec}

    %31.
    \begin{practicec}
        \begin{enumerate}[A.]
            \item 00111000; 110; 10
            \item EB
        \end{enumerate}
    \end{practicec}

    %32.
    \begin{practicec}
        \begin{enumerate}[A.]
            \item 10110111; 010; 00
            \item -
        \end{enumerate}
    \end{practicec}

    %33.
    \begin{practicec}
        1 0111 1000 1000
        1 0111 1000 1001
        1 0111 1000 1010
        1 0111 1000 1011
        1 0110 1100 1000
        1 0110 1100 1001
        1 0110 1100 1010
        1 0110 1100 1011
    \end{practicec}

    %34.
    \begin{practicec}
        \begin{table}[htb]
            \begin{tabular}{|c|c|c|c|}
                \hline
                m & m & h & m \\
                \hline
                m & h & m & h \\
                \hline
                m & m & h & m \\
                \hline
                m & h & m & h \\
                \hline
            \end{tabular}
        \end{table}

        \begin{table}[htb]
            \begin{tabular}{|c|c|c|c|}
                \hline
                m & m & m & m \\
                \hline
                m & m & m & m \\
                \hline
                m & m & m & m \\
                \hline
                m & m & m & m \\
                \hline
            \end{tabular}
        \end{table}
    \end{practicec}

    %35.
    \begin{practicec}
        缓存可以放下两个数组,故所有的不命中都是初始的冷不命中。
    \end{practicec}

    %36.
    \begin{practicec}
        \begin{enumerate}[A.]
            \item $100\%$
            \item $25\%$
            \item $25\%$
            \item 不会
            \item 会
        \end{enumerate}
    \end{practicec}

    %37.
    \begin{practicec}
        都是 $25\%$
    \end{practicec}

    %38.
    \begin{practicec}
        \begin{enumerate}[A.]
            \item $1024$
            \item $128$
            \item $12.5\%$
        \end{enumerate}
    \end{practicec}

    %39.
    \begin{practicec}
        \begin{enumerate}[A.]
            \item $1024$
            \item $256$
            \item $25\%$
        \end{enumerate}
    \end{practicec}

    %40.
    \begin{practicec}
        \begin{enumerate}[A.]
            \item $1024$
            \item $256$
            \item $25\%$
        \end{enumerate}
    \end{practicec}

    %41.
    \begin{practicec}
        $25\%$
    \end{practicec}

    %42.
    \begin{practicec}
        $25\%$
    \end{practicec}

    %43.
    \begin{practicec}
        $100\%$
    \end{practicec}

    %44.
    \begin{practicec}

    \end{practicec}

    %45.
    \begin{practicec}
        使用 $2 \times 2$ 、 $4 \times 4$ 和 $8 \times 8$ 的循环尝试。
    \end{practicec}

    %46.
    \begin{practicec}
        无向图与上题相比,只需遍历一半的元素。
    \end{practicec}
}

    %%
%% Author: Clay
%% 2019/12/5
%%

\section{证明的方法和策略}
{
    \subsection{引言}
    {
        数学家工作时,他们拟定猜测并试图证明或推翻之。
    }

    \subsection{穷举证明法和分情形证明法}
    {
        有时候采用单一的论证不能在定理的所有可能情况下都成立,故不能证明该定理。
        现在介绍一种通过分别考虑不同的情况的来证明定理的方法。
        该方法是基于现在要介绍的一个推理规则。
        为了证明如下的条件语句
        $$(p_1 \vee p_2 \vee \cdots \vee p_n) \rightarrow q$$
        可以用永真式
        $$((p_1 \vee p_2 \vee \cdots \vee p_n) \rightarrow q) \leftrightarrow ((p_1 \rightarrow q) \wedge (p_2 \rightarrow q) \wedge \cdots \wedge (p_n \rightarrow q))$$
        作为推理规则。
        这个推理规则说明可以通过分别证明每个条件语句 $p_i \rightarrow q(i = 1, 2, \cdots , n)$ 来证明由命题 $p_1, p_2, \cdots , p_n$ 的析取式组成前提的原条件语句。
        这种论证称为\emreg{分情形证明法(proof by cases)}。

        \paraph{穷举证明法}
        {
            有些定理可以通过检验相对少量的例子来证明。
            这样的证明叫做\emreg{穷举证明法(exhaustive proof, proof by exhaustion)},因为这些证明是要穷尽所有可能性的。
            一个穷举证明法是分情形证明的特例,这里每一种情形检验一个例子。

            注意当不可能列出所有要检查的情形时,即使是计算机也不能检查所有情形。
        }

        \paraph{分情形证明法}
        {
            分情形证明一定要覆盖定理中出现的所有可能的情况。
        }

        \paraph{充分利用分情形证明法}
        {
            当一个证明不可能同时顾及所有情形时,应该考虑采用分情形证明法。
            一般地,当没有明显的思路开始一个证明,而每一种情形的额外信息又能推进证明时,可以寻求分情形证明法。

            在分情形证明中,有时我们能消除几乎全部而只留下少量情形。
        }

        \paraph{不失一般性}
        {
            一般地,当证明中用到\emreg{不失一般性(without loss of generality, WLOG)}一词时,我们断言通过证明定理的一种情形,不需要额外的论证来证明其他特定的情形。
            也就是说,其他的一系列情形论证可以通过对论证做一些简单的改变,或者通过补充一些简单的初始步骤来完成。
            当然,不正确地应用这个原理会导致不幸的错误发生,有时候所做的会导致失去一般性。
            这类假设通常是由于忽略了一个情形可能与其他情形有着巨大的差异。
            这样会导致一个不完整的或许不可补救的证明。
        }

        \paraph{穷举证明法和分情形证明法中的常见错误}
        {
            推理中的一种常见错误是从个例中得出不正确的结论。
            不管考虑了多少不同的个例,都不能从个例来证明定理,除非每一种可能情况都覆盖了。
        }
    }

    \subsection{存在性证明}
    {
        许多定理是断言特定类型对象的存在性。
        这种类型的定理是形如 $\exists x P(x)$ 的命题,其中 $P$ 是谓词。
        $\exists x P(x)$ 这类命题的证明称为\emreg{存在性证明(existence proof)}。
        通过找出一个使得 $P(a)$ 为真的元素 $a$ (称为一个物证)来给出 $\exists x P(x)$ 的存在性证明。
        这样的存在性证明称为是\emreg{构造性的(constructive)}。
        也可以给出一种\emreg{非构造性的}存在性证明,即不是找出使 $P(a)$ 为真的元素 $a$ ,而是以某种其他方式来证明 $\exists x P(x)$ 为真。
        给出非构造性证明的常用方法是使用归谬证明,证明该存在量化式的否定蕴含一个矛盾。
    }

    \subsection{唯一性证明}
    {
        某些定理断言具有特定性质的元素唯一存在。
        这些定理断言恰好只有一个元素具有这个性质。
        要证明这类语句,需要证明存在一个具有此性质的元素,以及没有其他元素具有此性质。
        \emreg{唯一性证明(uniqueness proof)}的两个部分如下:

        \begin{description}
            \item[存在性:] 证明存在某个元素 $x$ 具有期望的性质。
            \item[唯一性:] 证明如果 $y \neq x$ ,则 $y$ 不具有期望的性质。
        \end{description}

        我们也可以等价地证明如果 $x$ 和 $y$ 都具有期望的性质,则 $x = y$ 。

        \begin{defines}
            证明存在唯一元素 $x$ 使得 $P(x)$ 为真等同于证明语句 $\exists x (P(x) \wedge \forall y (y \neq x \rightarrow \neg P(y)))$ 。
        \end{defines}
    }

    \subsection{证明策略}
    {
        寻找证明是一项富于挑战性的工作。
        当你面对待证命题时,应该先把术语替换成其定义,再仔细分析前提结论的含义。
        这样做之后,用一种可用的证明方法去尝试证明结论。
        一般情况下,如果语句是条件语句,就应该首先尝试直接证明法;
        如果这样不行,就尝试间接证明法。
        如果这些方法都不行,就尝试归谬证明法。

        \paraph{正向和反向推理}
        {
            无论选择什么证明方法,都需要为证明找一个起点。
            条件语句的直接证明就从前提开始。
            利用这些前提以及公理和已知定理,用导向结论的一系列步骤来构造证明。
            这类推理称为\emreg{正向推理(forward reasoning)},是用来证明相对简单结论的一类最常见推理方式。
            同样,要开始间接证明,就从结论的否定开始,用一系列步骤来得出前提的否定。

            但是,正向推理常常难以用来证明更复杂的结论,因为得出想要的结论所需要的推理可能并不明显。
            在这种情况下使用\emreg{反向推理(backward reasoning)}可能会有所帮助。
            要反向推理命题 $q$ ,我们就寻找一个命题 $p$ 并可证明其具有性质 $p \rightarrow q$ 。
            寻找一个命题 $r$ 并能证明 $q \rightarrow r$ 不会有所帮助,因为从 $q \rightarrow r$ 和 $r$ 得出 $q$ 为真是一种肯定结论的谬误。
        }

        \paraph{改变现有证明}
        {
            在寻找可用于证明语句方法时,一个很好的思路是利用类似结论现有的证明。
            一个现有的证明通常可以改编用于证明其他结论。
            即使不是这样,现有证明中的一些想法也会有所帮助。
        }
    }

    \subsection{寻找反例}
    {
        当面对一个猜想时,首先可以试图去证明这个猜想,如果你的尝试没有成功,你可以试图寻找一个反例。
        如果你不能找到反例,可以再试图证明这个语句。
    }

    \subsection{证明策略实践}
    {
        以探索概念和例子开始,提出问题,形成猜想,并企图通过证明或者通过反例来解决这些猜想。
        这些就是数学家的日常活动。

        人们基于各种可能证据来拟定猜想。
        对特殊情形的考察可能够导致一个猜想,就像识别一些可能的模式。
        对已知定理的假设和结论稍作改变也能导致可信的猜想。
        有些时候,猜想的建立是基于直觉或者甚至认为结果成立的信念。
        无论猜想是怎样产生的,一旦它被形式化描述,目标就是证明或者驳斥它。
    }

    \subsection{拼接}
    {
        通过对棋盘拼接游戏的简要研究研究能够解释证明策略的各个方面。
        研究棋盘的拼接游戏是一种能快速发现多种结论并用各种证明方法来构造其证明的很有效方法。
        在这个领域几乎创造了无穷多的猜想及其研究。

        一个\emreg{棋盘}是一个由水平和垂直分割成同样大小方格组成的矩形。
        8行和8列的棋盘称为\emreg{标准棋盘(standard checkerboard)}。
        在这一节我们用术语\emreg{拼板(board)}指任意大小的矩形棋盘,以及删除一个或多个方格剩下的棋盘的组成。
        一个\emreg{骨牌(domino)}是一块一乘二的方格组成的矩形。
        当一个拼板的所有方格由不重叠的骨牌覆盖并且没有骨牌悬空时,我们就说一个拼板由骨牌所\emreg{拼接(tiled)}。

        用同样的方格沿边粘连起来构成的相同形状的板块而非骨牌来做拼接游戏。
        这样的板块称为是\emreg{多联骨牌(polyomino)}。
        一种三联骨牌是\emreg{直三联骨牌(straight triomino)},另一种是\emreg{直角三联骨牌(right triomino)}。
    }

    \subsection{开放问题的作用}
    {
        数学中的许多进展是人们在试图解决著名的悬而未决的问题时而做出的。
    }

    \subsection{其他证明方法}
    {
        本章介绍了证明中使用的基本方法。
        同时描述了如何利用这些方法来证明各种结论。
        后续章节中将会用到这些证明方法。
    }

    \subsection{练习}
    {
        %1.
        \begin{practices}
            \begin{enumerate}[i.]
                \item 当 $n = 1$ , $1^2 + 1 \geq 2^1$
                \item 当 $n = 2$ , $2^2 + 1 \geq 2^2$
                \item 当 $n = 3$ , $3^2 + 1 \geq 2^3$
                \item 当 $n = 4$ , $4^2 + 1 \geq 2^4$
            \end{enumerate}

            故当 $n$ 是 $1 \leq n \leq 4$ 的正整数时,有 $n^2 + 1 \geq 2^n$ 。
            证毕。
        \end{practices}

        %2.
        \begin{practices}
            \begin{enumerate}[i.]
                \item 当 $n = 1$ ,不存在。
                \item 当 $n = 8$ ,不存在。
                \item 当 $n = 27$ ,不存在。
                \item 当 $n = 64$ ,不存在。
                \item 当 $n = 125$ ,不存在。
                \item 当 $n = 216$ ,不存在。
                \item 当 $n = 343$ ,不存在。
                \item 当 $n = 512$ ,不存在。
                \item 当 $n = 729$ ,不存在。
                \item 当 $n = 1000$ ,不存在。
            \end{enumerate}
        \end{practices}

        %3.
        \begin{practices}
            \begin{enumerate}[i.]
                \item 当 $x \geq y$ 时, $max(x, y) + min(x, y) \equiv x + y$ 。
                \item 当 $x < y$ 时, $max(x, y) + min(x, y) \equiv x + y$ 。
            \end{enumerate}
        \end{practices}

        %4.
        \begin{practices}
            \begin{enumerate}[i.]
                \item 当 $a$ 是三个数中最小的数, $min(a, min(b, c)) = a = min(min(a, b), c)$ 。
                \item 当 $b$ 是三个数中最小的数, $min(a, min(b, c)) = b = min(min(a, b), c)$ 。
                \item 当 $c$ 是三个数中最小的数, $min(a, min(b, c)) = c = min(min(a, b), c)$ 。
            \end{enumerate}
        \end{practices}

        %5.
        \begin{practices}
            \begin{enumerate}[i.]
                \item 当 $x = y$ 时, $min(x, y) = x = (2x / 2) = (x + y - 0) / 2$ , $max(x, y) = x = (2x / 2) = (x + y - 0) / 2$ 。
                \item 不失一般性,当 $x > y$ 时, $min(x, y) = y = (x + y - x + y) / 2$ , $max(x, y) = x = (x + y - y + 2) / 2$ 。
            \end{enumerate}
        \end{practices}

        %6.
        \begin{practices}
            不失一般性,当 $x$ 为偶数 $y$ 为奇数时,$x = 2k, y = 2l + 1$ 。
            $5x + 5y = 10k + 10l +5 = 2(5k + 5l + 2) + 1$ ,故为奇数。
        \end{practices}

        %7.
        \begin{practices}
            \begin{enumerate}[i.]
                \item
                {
                    当 $x$ 和 $y$ 都大于 $0$ 时,

                    \begin{align*}
                        |x| + |y| &\geq |x + y| \\
                        x + y &\geq x + y
                    \end{align*}
                }
                \item
                {
                    当 $x$ 和 $y$ 都小于 $0$ 时,

                    \begin{align*}
                        |x| + |y| &\geq |x + y| \\
                        -x - y &\geq -x - y
                    \end{align*}
                }
                \item
                {
                    不失一般性,当 $x$ 大于 $0$ , $y$ 小于 $0$ 时。

                    \begin{align*}
                        |x| + |y| &\geq |x + y| \\
                        x - y &\geq -x - y \\
                        \text{或} x - y &\geq x + y \\
                        2x &\geq 2y \\
                        \text{或} 0 &\geq 2y
                    \end{align*}
                }

                证毕。
            \end{enumerate}
        \end{practices}

        %8.
        \begin{practices}
            1。
            构造性的。
        \end{practices}

        %9.
        \begin{practices}
            $100^2 = 10000, 101^2 = 10201$ ,这中间超过100个连续的正整数不是完全平方数。
            构造性的。
        \end{practices}

        %10.
        \begin{practices}
            令 $n = 2 \times 10^{500} + 15, n + 1 = 2 \times 10^{500} + 16$ ,若 $n$ 和 $n + 1$ 都为完全平方数,则 $n = 0, n + 1 = 1$ 。
            $n$ 显然大于 $0$ 。
            故得出矛盾,则两数中有一个必然不是完全平方数。
            非构造性的。
        \end{practices}

        %11.
        \begin{practices}
            $0$ 和 $1$ 。
        \end{practices}
    }
}
    %%
%% Author: Clay
%% 2020/2/20
%%

\section{重要主题}
{

}

    %%
%% Author: Clay
%% 2021/2/20
%%

\section{标准I/O}
{
    C语言定义了一组高级输入输出函数,称为\emreg{标准I/O}库,为程序员提供了Unix I/O的较高级别的替代。

    标准I/O库将一个打开的文件模型化为一个\emreg{流}。
}

    %%
%% Author: Clay
%% 2020/12/22
%%

\section{从应用程序中加载和链接共享库}
{
    应用程序还可能在它运行时要求动态链接器加载和链接某个共享库,而无需在编译时将那些库连接到应用中。

    \begin{itemize}
        \item 分发软件
        \item 构建高性能Web服务器
    \end{itemize}

    其思路是将每个生成动态内容的函数打包在共享库中。
    当一个来自Web浏览器的请求到达时,服务器动态地加载和链接适当的函数,然后直接调用它。
    函数会一直缓存在服务器的地址空间中,所以只要一个简单的函数调用的开销就可以处理随后的请求了。
    更进一步地说,在运行时无需停止服务器,就可以更新已存在的函数,以及添加新的函数。
}

    %%
%% Author: Clay
%% 2020/5/19
%%

\section{小结}
{
    %58.
    \begin{practicec}

    \end{practicec}

    %59.
    \begin{practicec}
        $x * y = ux * uy - (x_{63}y + y_{63}x)2^{64}$
    \end{practicec}

    %60.
    \begin{practicec}

    \end{practicec}

    %61.
    \begin{practicec}

    \end{practicec}

    %62.
    \begin{practicec}

    \end{practicec}

    %63.
    \begin{practicec}

    \end{practicec}

    %64.
    \begin{practicec}
        \begin{enumerate}[A.]
            \item $A + L(S(T \cdot i + j) + k)$
            \item 7, 5, 13
        \end{enumerate}
    \end{practicec}

    %65.
    \begin{practicec}
        \begin{enumerate}[A.]
            \item rdx
            \item rax
            \item 15
        \end{enumerate}
    \end{practicec}

    %66.
    \begin{practicec}
        \begin{enumerate}[A.]
            \item 3n
            \item 4n + 1
        \end{enumerate}
    \end{practicec}

    %67.
    \begin{practicec}
        \begin{enumerate}[A.]
            \item
            {
                (rsp) = x;
                8(rsp) = y;
                16(rsp) = \&z;
                24(rsp) = z;
            }
            \item rdi = 64(rsp)
            \item rsp
            \item rdi
            \item 64(rsp)
            \item 调用前分配好返回结构的空间。
        \end{enumerate}
    \end{practicec}

    %68.
    \begin{practicec}
        $A = 9, B = 5$
    \end{practicec}

    %69.
    \begin{practicec}
        \begin{enumerate}[A.]
            \item 7
            \item
            {
                \begin{lstlisting}[language=C]
typedef struct
{
    long int idx;
    long int x[4];
} a_struct;
                \end{lstlisting}
            }
        \end{enumerate}
    \end{practicec}

    %70.
    \begin{practicec}
        \begin{enumerate}[A.]
            \item 0, 8, 0, 8
            \item 16
            \item \emcode{up->e2.x = *(up->e2.next->e1.p) - up->e1.y}
        \end{enumerate}
    \end{practicec}

    %71.
    \begin{practicec}

    \end{practicec}

    %72.
    \begin{practicec}
        \begin{enumerate}[A.]
            \item
            {
                $s_2 = s_1 - ((n * 8 + 30) \& 0xfffffff0)$ 。
                如果 $n$ 是奇数, $s_2 = s_1 - (n * 8 + 24)$ 。
                如果 $n$ 是偶数, $s_2 = s_1 - (n * 8 + 16)$ 。
            }
            \item $p = (s_2 + 15) \& 0xfffffff0$ 。
            \item 当 $s_1 \% 16 == 1$ 时, $e_1 = 1$ 最小;当 $s_1 \% 16 == 0$ 时, $e_1 = 24$ 最大。
            \item $s_2$ 与 $s_1$ 以16对齐; $p$ 以16位对齐。
        \end{enumerate}
    \end{practicec}

    %73.
    \begin{practicec}
        \begin{lstlisting}
find_range:
    vxorps %xmm1, %xmm1, %xmm1
    vucomiss %xmm0, %xmm1
    ja .L1
    jb .L2
    je .L2
    movl $3, %eax
    ret
.L1
    movl $2, %eax
    ret
.L2
    movl $0, %eax
    ret
.L3
    movl $1, %eax
    ret
        \end{lstlisting}
    \end{practicec}

    %74.
    \begin{practicec}
        \begin{lstlisting}
find_range:
    vxorps %xmm1, %xmm1, %xmm1
    vucomiss %xmm0, %xmm1
    movl $0, %eax
    cmove $1, %eax
    cmova $2, %eax
    cmovp $3, %eax
    ret
        \end{lstlisting}
    \end{practicec}

    %75.
    \begin{practicec}
        \begin{enumerate}[A.]
            \item 两个连续的xmm寄存器,前一个存放实数部分,后一个存放虚数部分。
            \item xmm0实数部分,xmm1虚数部分。
        \end{enumerate}
    \end{practicec}
}

}

\cleardoublepage

\endinput
