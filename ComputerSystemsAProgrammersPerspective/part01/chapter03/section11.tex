%%
%% Author: Clay
%% 2020/5/7
%%

\section{浮点代码}
{
    处理器的\emreg{浮点体系结构}包括多个方面,会影响对浮点数据操作的程序如何被映射到机器上,包括:

    \begin{itemize}
        \item 如何存储和访问浮点数值。
        \item 对浮点数据操作的指令。
        \item 向函数传递浮点数参数和从函数返回浮点数结果的规则。
        \item 函数调用过程中保存寄存器的规则。
    \end{itemize}

    1997年引入了持续数代的\emreg{媒体(media)}指令,支持图形和图像处理。
    这些指令本意是允许多个操作以并行模式执行,称为\emreg{单指令多数据(SIMD)}。
    名字经过了一些列大的修改,从MMX到\emreg{SSE(Streaming SIMD Extesion,流式SIMD扩展)},以及最新的\emreg{AVX(Advanced Vector Extesion,高级向量扩展)}。
    MM寄存器是64位的,XMM寄存器是128位的,YMM是256位的。
    寄存器的值可以是整数,也可以是浮点数。

    2000年媒体指令开始包括对\emreg{标量}浮点数据进行操作的指令。

    \subsection{浮点传送和转换操作}
    {
        VMOV指令在内存和XMM寄存器之间不做任何转换的传送浮点数指令。

        VCVTT指令把一个XMM寄存器或内存中读出的浮点值进行转换,并将结果写入一个通用寄存器。
        把浮点值转换成整数时,指令会执行\emreg{截断(truncation)},把值向0舍入。

        VCVT指令把整数转换成浮点数。

        vunpcklps指令通常用来交叉放置来自两个XMM寄存器的值,把它们的值存储到第三个寄存器中。

        %50.
        \begin{practicec}
            \begin{enumerate}[A.]
                \item val1: d
                \item val2: i
                \item val3: l
                \item val4: f
            \end{enumerate}
        \end{practicec}

        %51.
        \begin{practicec}
            \begin{table}[htb]
                \begin{tabular}{|c|c|c|}
                    \hline
                    $T_x$ & $T_y$ & 指令 \\
                    \hline
                    long & double & \emcode{vcvtsi2sdq \%rdi, \%xmm0, \%xmm0} \\
                    \hline
                    double & int & \emcode{vcvttsd2si \%xmm0, \%eax} \\
                    \hline
                    double & float & \makecell{\emcode{vmovddup \%xmm0, \%xmm0, \%xmm0} \\ \emcode{vcvtpd2psx \%xmm0, \%xmm0}} \\
                    \hline
                    long & float & \emcode{vcvtsi2ssq \%rdi, \%xmm0, \%xmm0} \\
                    \hline
                    float & long & \emcode{vcvttss2siq \%xmm0, \%rax} \\
                    \hline
                \end{tabular}
            \end{table}
        \end{practicec}
    }

    \subsection{过程中的浮点代码}
    {
        XMM寄存器用来向函数传递浮点参数,以及从函数返回浮点值。

        \begin{itemize}
            \item XMM寄存器 $\%xmm0 \sim \%xmm7$ 最多可以传递8个浮点参数。
            \item 函数使用寄存器 $\%xmm0$ 来返回浮点值。
            \item 所有的XMM寄存器都是调用者保存的。
        \end{itemize}

        %52.
        \begin{practicec}
            \begin{enumerate}[A.]
                \item \emcode{\%xmm0, \%rdi, \%xmm1, \%esi}
                \item \emcode{\%edi, \%rsi, \%rdx, \%rcx}
                \item \emcode{\%rdi, \%xmm0, \%esi, \%xmm1}
                \item \emcode{\%xmm0, \%rdi, \%xmm1, \%xmm2}
            \end{enumerate}
        \end{practicec}
    }

    \subsection{浮点运算操作}
    {
        第一个源操作数可以是一个XMM寄存器或一个内存位置。
        第二个源操作数合目的操作数都必须是XMM寄存器。
        每个操作都有一条针对单精度的指令和一条针对双精度的指令。

        %53.
        \begin{practicec}
            int, long, float, double
        \end{practicec}

        %54.
        \begin{practicec}
            \begin{lstlisting}[language=C]
double funct2(double w, int x, float y, long z)
{
    return x * y - w / z;
}
            \end{lstlisting}
        \end{practicec}
    }

    \subsection{定义和使用浮点常数}
    {
        AVX浮点操作不能以立即数值作为操作数。
        编译器必须为所有的常量值分配和初始化存储空间。

        %55.
        \begin{practicec}
            IEEE 754
        \end{practicec}
    }

    \subsection{在浮点代码中使用位级操作}
    {
        XMM寄存器的位级操作作用于封装好的数据,即它们更新整个目的XMM寄存器,对两个源寄存器的所有位都实施指定的位级操作。

        %56.
        \begin{practicec}
            \begin{enumerate}[A.]
                \item \emcode{fabs(x)}
                \item \emcode{x \^ x}
                \item \emcode{-x}
            \end{enumerate}
        \end{practicec}
    }

    \subsection{浮点比较操作}
    {
        对于浮点比较,当两个操作数中任一个是NaN时,就会设置奇偶标志位PF。
        当任一操作数为NaN时,就会出现\emreg{无序}的情况。

        %57.
        \begin{practicec}
            \begin{lstlisting}[language=C]
double funct3(int *ap, double b, long int c, float *dp)
{
    float x = *dp;
    double y = (double) *ap;

    if (b > y)
    {
        b = (double) c;
        b = float (b * x);

        return (double) b;
    }
    else
    {
        x = x + x;
        float z = (float) c;
        z = z + x;

        return (double) z;
    }
}
            \end{lstlisting}
        \end{practicec}
    }

    \subsection{对浮点代码的观察结论}
    {
        用AVX2为浮点数上的操作产生的机器代码风格类似于为整数上的操作产生代码风格。
        它们都使用一组寄存器来保存和操作数据值,也都使用这些寄存器来传递函数参数。
    }
}
