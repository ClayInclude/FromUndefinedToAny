%%
%% Author: Clay
%% 2020/5/7
%%

\section{浮点代码}
{
    处理器的\emreg{浮点体系结构}包括多个方面,会影响对浮点数据操作的程序如何被映射到机器上,包括:

    \begin{enumerate}
        \item 如何存储和访问浮点数值。
        \item 对浮点数据操作的指令。
        \item 向函数传递浮点数参数和从函数返回浮点数结果的规则。
        \item 函数调用过程中保存寄存器的规则。
    \end{enumerate}

    1997年引入了持续数代的\emreg{媒体(media)}指令,支持图形和图像处理。
    这些指令本意是允许多个操作以并行模式执行,称为\emreg{单指令多数据(SIMD)}。
    名字经过了一些列大的修改,从MMX到\emreg{SSE(Streaming SIMD Extesion,流式SIMD扩展)},以及最新的\emreg{AVX(Advanced Vector Extesion,高级向量扩展)}。
    MM寄存器是64位的,XMM寄存器是128位的,YMM是256位的。
    寄存器的值可以是整数,也可以是浮点数。

    2000年媒体指令开始包括对\emreg{标量}浮点数据进行操作的指令。

    \subsection{浮点传送和转换操作}
    {
        VMOV指令在内存和XMM寄存器之间不做任何转换的传送浮点数指令。

        VCVTT指令把一个XMM寄存器或内存中读出的浮点值进行转换,并将结果写入一个通用寄存器。
        把浮点值转换成整数时,指令会执行\emreg{截断(truncation)},把值向0舍入。

        VCVT指令把整数转换成浮点数。

        vunpcklps指令通常用来交叉放置来自两个XMM寄存器的值,把它们的值存储到第三个寄存器中。
    }
}
