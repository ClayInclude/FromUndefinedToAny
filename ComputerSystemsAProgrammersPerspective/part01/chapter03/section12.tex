%%
%% Author: Clay
%% 2020/5/19
%%

\section{小结}
{
    %58.
    \begin{practicec}

    \end{practicec}

    %59.
    \begin{practicec}
        $x * y = ux * uy - (x_{63}y + y_{63}x)2^{64}$
    \end{practicec}

    %60.
    \begin{practicec}

    \end{practicec}

    %61.
    \begin{practicec}

    \end{practicec}

    %62.
    \begin{practicec}

    \end{practicec}

    %63.
    \begin{practicec}

    \end{practicec}

    %64.
    \begin{practicec}
        \begin{enumerate}[A.]
            \item $A + L(S(T \cdot i + j) + k)$
            \item 7, 5, 13
        \end{enumerate}
    \end{practicec}

    %65.
    \begin{practicec}
        \begin{enumerate}[A.]
            \item rdx
            \item rax
            \item 15
        \end{enumerate}
    \end{practicec}

    %66.
    \begin{practicec}
        \begin{enumerate}[A.]
            \item 3n
            \item 4n + 1
        \end{enumerate}
    \end{practicec}

    %67.
    \begin{practicec}
        \begin{enumerate}[A.]
            \item
            {
                (rsp) = x;
                8(rsp) = y;
                16(rsp) = \&z;
                24(rsp) = z;
            }
            \item rdi = 64(rsp)
            \item rsp
            \item rdi
            \item 64(rsp)
            \item 调用前分配好返回结构的空间。
        \end{enumerate}
    \end{practicec}

    %68.
    \begin{practicec}
        $A = 9, B = 5$
    \end{practicec}

    %69.
    \begin{practicec}
        \begin{enumerate}[A.]
            \item 7
            \item
            {
                \begin{lstlisting}[language=C]
typedef struct
{
    long int idx;
    long int x[4];
} a_struct;
                \end{lstlisting}
            }
        \end{enumerate}
    \end{practicec}

    %70.
    \begin{practicec}
        \begin{enumerate}[A.]
            \item 0, 8, 0, 8
            \item 16
            \item \emcode{up->e2.x = *(up->e2.next->e1.p) - up->e1.y}
        \end{enumerate}
    \end{practicec}

    %71.
    \begin{practicec}

    \end{practicec}

    %72.
    \begin{practicec}
        \begin{enumerate}[A.]
            \item
            {
                $s_2 = s_1 - ((n * 8 + 30) \& 0xfffffff0)$ 。
                如果 $n$ 是奇数, $s_2 = s_1 - (n * 8 + 24)$ 。
                如果 $n$ 是偶数, $s_2 = s_1 - (n * 8 + 16)$ 。
            }
            \item $p = (s_2 + 15) \& 0xfffffff0$ 。
            \item 当 $s_1 \% 16 == 1$ 时, $e_1 = 1$ 最小;当 $s_1 \% 16 == 0$ 时, $e_1 = 24$ 最大。
            \item $s_2$ 与 $s_1$ 以16对齐; $p$ 以16位对齐。
        \end{enumerate}
    \end{practicec}

    %73.
    \begin{practicec}
        \begin{lstlisting}
find_range:
    vxorps %xmm1, %xmm1, %xmm1
    vucomiss %xmm0, %xmm1
    ja .L1
    jb .L2
    je .L2
    movl $3, %eax
    ret
.L1
    movl $2, %eax
    ret
.L2
    movl $0, %eax
    ret
.L3
    movl $1, %eax
    ret
        \end{lstlisting}
    \end{practicec}

    %74.
    \begin{practicec}
        \begin{lstlisting}
find_range:
    vxorps %xmm1, %xmm1, %xmm1
    vucomiss %xmm0, %xmm1
    movl $0, %eax
    cmove $1, %eax
    cmova $2, %eax
    cmovp $3, %eax
    ret
        \end{lstlisting}
    \end{practicec}

    %75.
    \begin{practicec}
        \begin{enumerate}[A.]
            \item 两个连续的xmm寄存器,前一个存放实数部分,后一个存放虚数部分。
            \item xmm0实数部分,xmm1虚数部分。
        \end{enumerate}
    \end{practicec}
}
