%%
%% Author: Clay
%% 2020/3/21
%%

\section{访问信息}
{
    x86--64的中央处理单元(CPU)包含一组16个存储64位值的\emreg{通用目的寄存器}。
    这些寄存器用来存储整数数据和指针。
    它们的名字都以\emcode{\%r}开头。
    最初的8086中有8个16位的寄存器,即\emcode{\%ax, \%bx, \%cx, \%dx, \%si, \%di, \%bp, \%sp}。
    扩展到IA32架构时,这些寄存器也扩展成32位寄存器,标号从\emcode{\%eax}到\emcode{\%esp}。
    扩展到x86--64后,原来的8个寄存器扩展成64位,标号从从\emcode{\%rax}到\emcode{\%rsp}。

    \subsection{操作数指示符}
    {
        大多数指令有一个或多个\emreg{操作数(operand)},指示出执行一个操作中要使用的源数据值,以及放置结果的目的位置。

        各种不同的操作数的可能性被分为三种类型:

        \begin{itemize}
            \item \emreg{立即数(immediate)},用来表示常数值。
            \item \emreg{寄存器(register)},它表示某个寄存器的内容。
            \item \emreg{内存}引用,它会根据计算出来的地址(\emreg{有效地址})访问某个内存位置。
        \end{itemize}

        有多种不同的\emreg{寻址模式},允许不同形式的内存引用。
        用语法 $Imm(r_b, r_i, s)$ 表示的是最常用的形式。
        这样的引用有四个组成部分:
        一个立即数偏移 $Imm$ ,一个基址寄存器 $r_b$ ,一个变址寄存器 $r_i$ 和一个比例因子 $s$ 。
        这里 $s$ 必须是 $1, 2, 4$ 或者 $8$ 。

        %1.
        \begin{practicec}
            \begin{table}[H]
                \[
                    \begin{array}{|c|c|}
                        \hline
                        \%rax & 0x100 \\
                        \hline
                        0x104 & 0xab \\
                        \hline
                        \$0x108 & 0x108 \\
                        \hline
                        (\%rax) & 0xff \\
                        \hline
                        4(\%rax) & 0xab \\
                        \hline
                        9(\%rax, \%rdx) & 0x11 \\
                        \hline
                        260(\%rcx, \%rdx) & 0x13 \\
                        \hline
                        0xfc(, \%rcx, 4) & 0xff \\
                        \hline
                        (\%rax, \%rdx, 4) & 0x11 \\
                        \hline
                    \end{array}
                \]
            \end{table}
        \end{practicec}
    }

    \subsection{数据传送指令}
    {
        最频繁使用的指令是将数据从一个位置复制到另一个位置的指令。

        MOV类由四条指令组成:\emcode{movb, movw, movl, movq}。
        源操作数指定的值是一个立即数,存储在寄存器中或者内存中。
        目的操作数指定一个位置,要么是一个寄存器,要么是一个内存地址。
        传送指令的两个操作数不能都指向内存位置。

        大多数情况中,MOV指令只会更新目的操作数指定的那些寄存器字节或内存位置。
        唯一的例外是\emcode{movl}指令以寄存器作为目的时,它会把该寄存器的高位4字节设置为0。
        任何为寄存器生成32位值的指令都会把该寄存器的高位部分置成0。

        常规的\emcode{movq}只能以表示为32位补码数字的立即数作为源操作数,然后把这个值符号扩展得到64位的值。
        \emcode{movabsq}能够以任意64位立即数作为源操作数,并且只能以寄存器作为目的。

        在将较小的原值复制到较大的目的时,MOVZ类中的指令把目的中剩余的字节填充为0,而MOVS类中的指令通过符号扩展来填充。
        \emcode{movzlq}可以用\emcode{movl}指令来实现。

        \emcode{cltq}的效果与\emcode{movslq \%eax, \%rax}完全一致。

        %2.
        \begin{practicec}
            \begin{enumerate}[A.]
                \item \emcode{movl}
                \item \emcode{movw}
                \item \emcode{movb}
                \item \emcode{movq}
                \item \emcode{movw}
            \end{enumerate}
        \end{practicec}

        %3.
        \begin{practicec}

        \end{practicec}
    }

    \subsection{数据传送示例}
    {
        参数通过寄存器传递给函数。
        函数值通过把值存储在寄存器rax或该寄存器的某个低位部分中返回。

        所谓指针其实就是地址。
        间接引用就是将该指针放在一个寄存器中,然后在内存引用中使用这个寄存器。
        局部变量通常是保存在寄存器中,而不是内存中。
        访问寄存器比访问内存快的多。

        %.4
        \begin{practicec}
            \begin{enumerate}[A.]
                \item \emcode{movsbl (\%rdi), \%eax; movl \%eax, (\%rsi)}
                \item \emcode{movsbl (\%rdi), \%eax; movl \%eax, (\%rsi)}
                \item \emcode{movzbq (\%rdi), \%rax; movq \%rax, (\%rsi)}
                \item \emcode{movl (\%rdi), \%eax; movb \%al, (\%rsi)}
                \item \emcode{movl (\%rdi), \%eax; movb \%al, (\%rsi)}
                \item \emcode{movsbw (\%rdi), \%ax; movw \%ax, (\%rsi)}
            \end{enumerate}
        \end{practicec}

        %5.
        \begin{practicec}

        \end{practicec}
    }

    \subsection{压入和弹出栈数据}
    {
        通过push操作把数据压入栈中,通过pop操作删除数据。
        弹出的值永远是最近被压入而且仍然在栈中的值。
        栈可以实现为一个数组,总是从数组的一端插入和删除元素,这一端被称为\emreg{栈顶}。
        栈指针\%rsp保存着栈顶元素的地址。

        将一个四字值压入栈中,首先将栈指针减8,然后将值写到新的栈顶地址。
        弹出一个四字的操作包括从栈顶位置读出数据,然后将栈指针加8。
        程序可以用标准的内存寻址方法访问栈内的任意位置。
    }
}
