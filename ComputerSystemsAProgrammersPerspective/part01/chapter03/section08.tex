%%
%% Author: Clay
%% 2020/4/27
%%

\section{数组分配和访问}
{
    C语言中的数组是一种将标量数据聚集成更大数据类型的方式。

    \subsection{基本原则}
    {
        对于数据类型T和整型常数N,声明如下:

        \emcode{T A[N];}

        起始位置表示为 $x_A$ 。
        首先在内存中分配一个 $L \cdot N$ 字节的连续区域,这里 $L$ 是数据类型T的大小。
        其次,引入标识符A,可以用A来作为指向数组开头的指针,这个指针的值就是 $x_A$ 。
        可以用 $0 \sim N - 1$ 的整数索引来访问该数组元素。
        数组元素i会被存放在地址为 $x_A + L \codt i$ 的地方。

        假设 E是一个\emcode{int}型的数组,想计算\emcode{E[i]},E的存放地址在寄存器\%rdx中,而i放在寄存器\%rcx中,然后指令

        \emcode{movl (\%rdx, \%rcx, 4), \%eax}

        会执行地址计算 $x_E + 4i$ ,读这个内存位置的值,并将结果存放到寄存器\%eax中。

        %36.
        \begin{practicec}
            \begin{table}[htb]
                \centering

                \[
                    \begin{array}[|c|c|c|c|c|]
                        \hline
                        S & 2 & 14 & x_S & x_S + i * 2 \\
                        \hline
                        T & 8 & 24 & x_T & x_T + i * 8 \\
                        \hline
                        U & 8 & 48 & x_U & x_U + i * 8 \\
                        \hline
                        V & 4 & 32 & x_V & x_V + i * 4 \\
                        \hline
                        W & 8 & 4 & x_W & x_W + i * 8 \\
                        \hline
                    \end{array}
                \]
            \end{table}
        \end{practicec}
    }

    \subsection{指针运算}
    {
        对于一个表示某个对象的表达式\emcode{Expr},\emcode{\&Expr}是给出该对象地址和间接引用指针。
        对于一个表示地址的表达式\emcode{AExpr},\emcode{*AExpr}给出该地址处的值。

        %37.
        \begin{practicec}
            \begin{table}[htb]
                \centering

                \[
                    \begin{array}[|c|c|c|]
                        \hline
                        address & x_S + 2 & leaq (\%rdx, 1, 2) \%rax \\
                        \hline
                        value & M[x_S + 3 * 2] & movw 6(\%rdx) \%ax \\
                        \hline
                        address & x_S + i * 2 & leaq (\%rdx, \%rcx, 2) \%rax \\
                        \hline
                        value & M[x_S + 8 * i + 4] & movw 4(\%rdx, \%rcx, 8) \%ax \\
                        \hline
                        value & M[x_S * 2 * i - 20] & movw -20(\%rdx, \%rcx, 2) \%ax \\
                        \hline
                    \end{array}
                \]
            \end{table}
        \end{practicec}
    }

    \subsection{嵌套的数组}
    {
        要访问多维数组的元素,编译器会以数组起始为基地址,(可能需要经过伸缩的)偏移量为索引,产生计算期望的元素的偏移量。

        %38.
        \begin{practicec}
            $M = 5, N = 7$
        \end{practicec}
    }

    \subsection{定长数组}
    {
        C语言编译器能够优化定长多维数组上的操作代码。

        %39.
        \begin{practicec}

        \end{practicec}

        %40.
        \begin{practicec}
            \begin{lstlisting}[language=C]
void fix_set_diag(int *A[], int val)
{
    int *ptr = A;
    int *end = A + (N - 1) * N + N - 1;

    do
    {
        *ptr = val;
        ptr += N + 1;
    } while (ptr != end);
}
            \end{lstlisting}
        \end{practicec}
    }

    \subsection{变长数组}
    {
        变成数组必须用乘法指令对i伸缩n倍,而不能用一系列的移位和加法。
    }
}
