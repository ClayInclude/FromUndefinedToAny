%%
%% Author: Clay
%% 2020/4/8
%%

\section{控制}
{
    机器代码提供两种基本的低级机制来实现有条件的行为:
    测试数据值,然后根据测试的结果来改变控制流或者数据流。

    与数据相关的控制流是实现有条件行为的更一般和更常见的方法。
    用\emcode{jump}指令可以改变一组机器代码指令的执行顺序,\emcode{jump}指令指定控制应该被传递到程序的某个其他部分,可能是依赖于某个测试的结果。

    \subsection{条件码}
    {
        除了整数寄存器,CPU还维护着一组单个位的\emreg{条件码(condition code)}寄存器,它们描述了最近的算术或逻辑操作的属性。
        可以检测这些寄存器来执行条件分支指令。
        最常用的条件码有:

        \begin{description}
            \item[CF:进位标志] 最近的操作使最高位产生了进位。
            \item[ZF:零标志] 最近的操作得出的结果为0。
            \item[SF:符号标志] 最近的操作得到的结果为负数。
            \item[OF:溢出标志] 最近的操作导致一个补码溢出。
        \end{description}

        \emcode{leaq}不改变任何条件码。

        对于逻辑操作,进位标志和溢出标志会设置为0。
        对于移位操作,进位标志将设置为最后一个被移出的位,而溢出标志设置为0。
        \emcode{inc}和\emcode{dec}指令会设置溢出和零标志,但是不会改变进位标志。

        还有两类指令只设置条件码而不改变任何其他寄存器。

        \emcode{cmp}指令与\emcode{sub}指令的行为是一样的,除了只设置条件码而不更新目的寄存器。
        \emcode{test}指令与\emcode{and}指令的行为是一样的,除了只设置条件码而不更新目的寄存器。
    }

    \subsection{访问条件码}
    {
        条件码通常不会直接读取,常用的使用方法有三种:

        \begin{enumerate}
            \item 可以根据条件码的某种组合,将一个字节设置为0或者1
            \item 可以条件跳转到程序的某个其他的部分
            \item 可以有条件地传送数据
        \end{enumerate}

        SET类指令之间的区别就在于它们考虑的条件码的组合是什么,这些指令名字的不同后缀指明了它们所考虑的条件码的组合
    }
}
