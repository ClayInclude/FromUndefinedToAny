%%
%% Author: Clay
%% 2020/4/8
%%

\section{控制}
{
    机器代码提供两种基本的低级机制来实现有条件的行为:
    测试数据值,然后根据测试的结果来改变控制流或者数据流。

    与数据相关的控制流是实现有条件行为的更一般和更常见的方法。
    用\emcode{jump}指令可以改变一组机器代码指令的执行顺序,\emcode{jump}指令指定控制应该被传递到程序的某个其他部分,可能是依赖于某个测试的结果。

    \subsection{条件码}
    {
        除了整数寄存器,CPU还维护着一组单个位的\emreg{条件码(condition code)}寄存器,它们描述了最近的算术或逻辑操作的属性。
        可以检测这些寄存器来执行条件分支指令。
        最常用的条件码有:

        \begin{description}
            \item[CF:进位标志] 最近的操作使最高位产生了进位。
            \item[ZF:零标志] 最近的操作得出的结果为0。
            \item[SF:符号标志] 最近的操作得到的结果为负数。
            \item[OF:溢出标志] 最近的操作导致一个补码溢出。
        \end{description}

        \emcode{leaq}不改变任何条件码。

        对于逻辑操作,进位标志和溢出标志会设置为0。
        对于移位操作,进位标志将设置为最后一个被移出的位,而溢出标志设置为0。
        \emcode{inc}和\emcode{dec}指令会设置溢出和零标志,但是不会改变进位标志。

        还有两类指令只设置条件码而不改变任何其他寄存器。

        \emcode{cmp}指令与\emcode{sub}指令的行为是一样的,除了只设置条件码而不更新目的寄存器。
        \emcode{test}指令与\emcode{and}指令的行为是一样的,除了只设置条件码而不更新目的寄存器。
    }

    \subsection{访问条件码}
    {
        条件码通常不会直接读取,常用的使用方法有三种:

        \begin{enumerate}
            \item 可以根据条件码的某种组合,将一个字节设置为0或者1
            \item 可以条件跳转到程序的某个其他的部分
            \item 可以有条件地传送数据
        \end{enumerate}

        SET类指令之间的区别就在于它们考虑的条件码的组合是什么,这些指令名字的不同后缀指明了它们所考虑的条件码的组合而不是操作数大小。
        一条SET指令的目的操作数是低位单字节寄存器元素之一,或是一个字节的内存位置,指令会将这个字节设置成0或者1。
        为了得到一个32位或64位结果,必须将高位清零。

        各个SET命令的描述都适用的情况是:执行比较指令,根据计算 $t = a -^t_w b$ 设置条件码。

        %13.
        \begin{practicec}
            \begin{enumerate}[A.]
                \item \emcode{int32\_t, <}
                \item \emcode{int16\_t, >=}
                \item \emcode{uint8\_t, <=}
                \item \emcode{int64\_t, !=}
            \end{enumerate}
        \end{practicec}

        %14.
        \begin{practicec}
            \begin{enumerate}[A.]
                \item \emcode{int64\_t, >=}
                \item \emcode{int16\_t, ==}
                \item \emcode{uint8\_t, >}
                \item \emcode{int32\_t, !=}
            \end{enumerate}
        \end{practicec}
    }

    \subsection{跳转指令}
    {
        正常执行的情况下,指令按照它们出现的顺序一条一条地执行。
        \emreg{跳转(jump)}指令会导致执行切换到程序中一个全新的位置。
        在汇编代码中,这些跳转的目的地通常用一个\emreg{标号(label)}指明。

        jmp指令是无条件跳转。
        它可以是\emreg{直接}跳转,即跳转目标是作为指令的一部分编码的;
        也可以是\emreg{间接}跳转,即跳转目标是从寄存器或内存位置中读出的。

        其他跳转指令都是\emreg{有条件}的。
        这些指令的名字和跳转条件与SET指令的名字和设置条件是相匹配的。
    }

    \subsection{跳转指令的编码}
    {
        在汇编代码中,跳转目标用符号标号书写。
        汇编器,以及后来的连接器,会产生跳转目标的适当编码。
        跳转指令有几种不同的编码,但是最常用都是\emreg{PC相对的(PC-relative)}。
        第二种编码方法是给出绝对地址。

        当执行PC相对寻址时,程序计数器的值是跳转指令后面的那条指令的地址。

        %15.
        \begin{practicec}
            \begin{enumerate}[A.]
                \item $4003fe$
                \item $400425$
                \item $400543, 400545$
                \item $8d$
            \end{enumerate}
        \end{practicec}

        跳转指令提供了一种实现条件执行和几种不同循环结构的方式。

        C语言中的\emcode{if--else}语句的通用形式模板如下:

        \begin{lstlisting}[language=C]
if (test-expr)
    then-statement
else
    else-statement
        \end{lstlisting}

        对于这种通用形式,汇编实现通常会使用下面这种形式,这里,用C语法来描述控制流:

        \begin{lstlisting}
    t = test-expr;

    if (!t)
        goto false;
    then-statement;
    goto done;
false:
    else-statement;
done:
        \end{lstlisting}

        也就是,汇编器为\emcode{then--statement}和\emcode{else--statement}产生各自的代码块,它会插入条件和无条件分支,以保证能执行正确的代码块。

        %16.
        \begin{practicec}
            \begin{enumerate}[A.]
                \item
                {
                    \begin{lstlisting}[language=C]
void cond(long int a, long int *p)
{
    if (!p || a <= *p)
        goto done;

    *p = a;
done:
}
                    \end{lstlisting}
                }
                \item
                {
                    \emcode{\&\&}被拆为两个\emcode{if}。
                }
            \end{enumerate}
        \end{practicec}

        %17.
        \begin{practicec}
            \begin{enumerate}[A.]
                \item
                {
                    \begin{lstlisting}[language=C]
void cond(long int a, long int *p)
{
    if (p && a > *p)
        goto true;

    goto done;
true:
    *p = a;
done:
}
                    \end{lstlisting}
                }
                \item
                {
                    只有一个分支时,可以减少一个标签和一次跳转。
                }
            \end{enumerate}
        \end{practicec}

        %18.
        \begin{practicec}
            \begin{lstlisting}[language=C]
void test(long int x, long int y, long int z)
{
    long int val = x + y + z;

    if (x < -3)
    {
        if (y < z)
        {
            val = x * y;
        }
        else
        {
            val = y * z;
        }
    }
    else if (x > 2)
    {
        val = x * z;
    }
}
            \end{lstlisting}
        \end{practicec}
    }

    \subsection{用条件传送来实现条件分支}
    {
        实现条件操作的传统方法是通过使用\emreg{控制}的条件转移。

        一种替代的策略是使用\emreg{数据}的条件转移。
        这种方法计算一个条件操作的两种结果,然后再根据条件是否满足从中选取一个。
        只有在一些受限制的情况中,这种策略才可行,但是如果可行,就可以用一条简单的\emreg{条件传送}指令来实现,条件传送指令更符合现代处理器的性能特性。

        处理器通过使用\emreg{流水线(pipelining)}来获得高性能,在流水线中,一条指令的处理要经过一系列的阶段,每个阶段执行所需操作的一小部分。
        这种方法通过重叠连续指令的步骤来获得高性能。
        要做到这一点,要求能够事先确定要执行的指令序列,这样才能保持流水线中充满了待执行的指令。
        当机器遇到条件跳转时,只有当分支条件求值完成之后才能决定分支往哪边走。
        处理器采用非常精密的\emreg{分支预测逻辑}来猜测每条跳转指令是否会执行。
        错误预测会招致很严重的惩罚,导致程序性能严重下降。

        控制流不依赖于数据,这使得处理器更容易保持流水线是满的。

        %19.
        \begin{practicec}
            \begin{enumerate}[A.]
                \item 30
                \item 46
            \end{enumerate}
        \end{practicec}

        条件传送指令有两个操作数,源寄存器或者内存地址S,和目的寄存器R。
        只有在指定条件满足时,才会被复制到目的寄存器中。
        汇编器可以从目标寄存器的名字推断出条件传送指令的操作数长度。

        同条件跳转不同,处理器无需预测测试结果就可以执行条件传送。
        处理器只是读原值,检查条件码,然后要么更新目的寄存器,要么保持不变。

        不是所有的条件表达式都可以用条件传送来编译。
        最重要的是,无论测试结果如何,会对\emcode{then-expr}和\emcode{else-expr}都求值。
        如果这两个表达式中的任意一个可能产生错误条件或者副作用,就会导致非法的行为。

        使用条件传送也不总是会提高代码的效率。

        %20.
        \begin{practicec}
            除法
        \end{practicec}

        %21.
        \begin{practicec}
            \begin{lstlisting}[language=C]
long int test(long int x, long int y)
{
    long int val = 8 * x;
}
            \end{lstlisting}
        \end{practicec}
    }
}
