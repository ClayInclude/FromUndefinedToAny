%%
%% Author: Clay
%% 2020/5/22
%%

\section{理解现代处理器}
{
    随着试图进一步提高性能,必须考虑利用处理器\emreg{微体系结构的优化},也就是处理器用来执行指令的底层系统设计。
    要想充分提高性能,需要仔细分析程序,同时代码的生成也要针对目标处理器进行调整。

    处理器的实际操作与通过观察机器级程序所察觉到的大相径庭。
    在实际的处理器中,是同时对多条指令求值的,这个现象称为\emreg{指令级并行}。
    采用一些精细的机制来确保这种并行执行的行为,正好能获得机器级程序要求的顺序语义模型的效果。

    当一系列操作必须必须按照严格顺序执行时,就会遇到\emreg{延迟界限(latency bound)},因为在下一条指令开始之前,这条指令必须结束。
    当代码中的数据相关限制了处理器利用指令级并行的能力时,延迟界限能够限制程序性能。
    \emreg{吞吐量界限(throughput bound)}刻画了处理器功能单元的原始计算能力。
    这个界限是程序性能的终极限制。

    \subsection{整体操作}
    {
        这些处理器在工业界称为\emreg{超标量(superscalar)},意思是它可以在每个时钟周期执行多个操作,并且是\emreg{乱序的(out--of--order)},意思就是指令执行的顺序不一定要与它们在机器级程序中的顺序一致。
        整个设计有两个主要部分:\emreg{指令控制单元(Instruction Control Unit, ICU)}和\emreg{执行单元(Execution Unit, EU)}。
        前者负责从内存中读出指令序列,并根据这些指令序列生成一组针对程序数据的基本操作。
        而后者执行这些操作。
        和\emreg{按序(in-order)}流水线相比,乱序处理器需要更大更复杂的硬件,但是它们能更好地达到更高的指令级并行度。

        ICU从\emreg{指令高速缓存(instruction cache)}中读取指令,指令高速缓存是一个特殊的高速缓存器,它包含最近访问的指令。
        ICU会在当前正在执行的指令很早之前取指,这样才有足够的时间对指令译码,并把操作发送到EU。
        当程序遇到分支时,程序有两个可能的前进方向。
        现代处理器采用了一种称为\emreg{分支预测(branch prediction)}的技术,处理器会猜测是否选择分支,同时还预测分支的目标地址。
        使用\emreg{投机执行(speculative Execution)}的技术,处理器会开始取出位于它预测的分支会跳到的地方的指令,并对指令译码,甚至在它确定分支预测是否正确之前就开始执行这些操作。
        如果过后确定分支预测错误,会将状态重新设置到分支点的状态,并开始取出和执行另一个方向上的指令。
        标记为\emreg{取指控制}的块包括分支预测,以完成确定取哪些指令的任务。

        \emreg{指令译码}逻辑接受实际的程序指令,并将他们转换成一组基本操作(\emreg{微操作})。
        对于具有复杂指令的机器,一条指令可以被译码成多个操作。
        关于指令如何被译码成操作序列的细节,不同的机器都不同,这个信息可谓是高度机密。
        幸运的是,不需要知道某台机器实现的底层细节,也能优化自己的程序。

        译码逻辑对指令进行分解,允许任务在一组专门的硬件单元之间进行分割。
        这些单元可以并行地执行多条指令的不同部分。

        EU接受来自取指单元的操作。
        通常,每个时钟周期会接收多个操作。
        这些操作会被分派到一组\emreg{功能单元}中。

        加载和存储单元通过\emreg{数据高速缓存(data cache)}来访问内存。

        使用投机执行技术对操作求值,但是最终结果不会存放在程序寄存器或内存中,知道处理器能确定应该实际执行这些指令。
        分支操作被送到EU,不是确定该往哪里去,而是确定分支预测是否正确。
        如果预测错误,EU会丢弃分支点之后计算出来的结果。
        它还会发信号给分支单元,说预测是错误的,并指出正确的分支目的。
        这种情况中,分支单元开始新的位置取指。
        这样的\emreg{预测错误}会导致很大的性能开销。

        算数运算单元被特意设计成能够执行各种不同的操作。

        在ICU中,\emreg{退役单元(retirement Unit)}记录正在进行的处理,并确保它遵守机器级程序的顺序语义。
        图中展示了一个\emreg{寄存器文件},它包含整数、浮点数和最近的SSE和AVX寄存器,是退役单元的一部分,因为退役单元控制这些寄存器的更新。
        指令译码时,关于指令的信息被放置在一个先进先出的队列中。
        一旦一条指令的操作完成了,而且所有引起这条指令的分支点也都被确认为预测正确,那么这条指令就可以\emreg{退役(retired)}了,所有对程序寄存器的更新都可以被实际执行。
        如果引起该指令的某个分支点预测错误,这条指令会被\emreg{清空(flushed)},丢弃所有计算出来的结果。
        通过这种方法,预测错误就不会改变程序的状态了。

        为了加速一条指令到另一条指令的传送,许多信息都是在执行单元之间交换的。

        操作控制数在执行单元间传送的最常见的机制称为\emreg{寄存器重命名(register renaming)}。
    }

    \subsection{功能单元的行能}
    {
        每个运算都是由以下这些数值来刻画的:
        一个是\emreg{延迟(latency)},它表示完成运算所需要的总时间;
        另一个是\emreg{发射时间(issue time)},它表示两个连续的同类型的运算之间需要的最小时钟周期数;
        还有一个是\emreg{容量(capacity)},它表示能够执行该运算的功能单元的数量。

        \emreg{流水线}化的功能单元实现为一系列的\emreg{阶段(stage)},每个阶段完成一部分的运算。
        只有当要执行的运算是连续的、逻辑上独立的时候,才能利用这种功能。
        发射时间为1的功能单元被称为\emreg{完全流水线化的(fully pipelined)}:
        每个时钟周期可以开始一个新的运算。
        出现容量大于1的运算是由于有多个功能单元。

        除法器不是完全流水线化的---它的发射时间等于它的延迟。
        这就意味着在开始一条新运算前,除法器必须完成整个除法。

        表达发射时间的一种更常见的方法是指明这个功能单元的最大\emreg{吞吐量},定义为发射时间的倒数。

        CPU设计者必须小心地平衡功能单元的数量和它们各自的性能。

        \emreg{延迟界限}给出了任何必须按照严格顺序完成合并运算的函数所需要的最小CPE值。
        根据功能单元产生结果的最大速率,\emreg{吞吐量界限}给出了CPE的最小界限。
    }

    \subsection{处理器操作的抽象模型}
    {
        程序的\emreg{数据流(data--flow)}是一种图形化的表示方法,展示了不同操作之间的数据相关是如何限制它们的执行顺序的。
        这些限制形成了图中的\emreg{关键路径(critical path)},这是执行一组机器指令所需时钟周期数的一个下界。

        CPE就等于延迟界限L。

        \subsubsection{机器级代码数据流图}
        {
            程序的数据流表示是非正式的。

            将访问到的寄存器分为四类:

            \begin{description}
                \item[只读] 这些寄存器只用做源值,可以作为数据,也可以用来计算内存地址,但是在循环中它们是不会被修改的。
                \item[只写] 这些寄存器作为数据传送操作的目的。
                \item[局部] 这些寄存器在循环内部被修改和使用,迭代与迭代之间不相关。
                \item[循环] 对于循环来说,这些寄存器既作为源值,又作为目的,一次迭代中产生的值会在另一次迭代中用到。
            \end{description}

            循环寄存器之间的操作连决定了限制性能的数据相关。
        }

        \subsubsection{其它性能因素}
        {
            看上去,延迟界限是基本的限制。

            %5.
            \begin{practicec}
                \begin{enumerate}[A.]
                    \item $2n$ 次乘法和 $n$ 次加法。
                    \item 两次乘法可以并行。
                \end{enumerate}
            \end{practicec}

            %6.
            \begin{practicec}
                \begin{enumerate}[A.]
                    \item $n$ 次乘法和 $n$ 次加法。
                    \item 加法要在乘法之后进行。
                    \item 两次乘法和加法可以并行。
                \end{enumerate}
            \end{practicec}
        }
    }
}
