%%
%% Author: Clay
%% 2020/6/2
%%

\section{提高并行性}
{
    \subsection{多个累积变量}
    {
        对于一个可结合和可交换的合并运算来说,可以通过将一组合并运算分割成两个或更多的部分,并在最后合并结果来提高性能。

        既做循环展开同时也是用两路并行这种方法,打破了由延迟界限设下的界限。

        可以将多个累积变量变换归纳为将循环展开 $k$ 次,以及并行累积 $k$ 个值。

        只有保持能够执行该操作的所有功能单元的流水线都是满的,程序才能达到这个操作的吞吐量界限。
        对延迟为 $L$ ,容量为 $C$ 的操作而言,这就要求循环展开因子 $k \geq C \cdot L$ 。

        在执行 $k \times k$ 循环展开变换时,必须考虑是否要保留原始函数的功能。
    }

    \subsection{重新结合变换}
    {
        对代码做很小的改动,可以从根本上改变合并执行的方式,也极大地提高程序的性能。

        差别仅在于两个括号是如何放置的。
        称之为\emreg{重新结合变换(reassociation transformation)},因为括号改变了向量元素与累计值的合并顺序,产生了成为 $2 \times 1a$ 的循环展开形式。

        这种变换带来的性能结果与 $k \times k$ 循环展开中保持 $k$ 个累计变量的结果相似。

        重新结合变换能够减少计算中关键路径上操作的数量,更好地利用功能单元的流水线能力得到更好的性能。

        %8.
        \begin{practicec}
            \begin{enumerate}[A.]
                \item 5
                \item 3.3
                \item 1.6
                \item 1.6
                \item 3.3
            \end{enumerate}
        \end{practicec}
    }
}
