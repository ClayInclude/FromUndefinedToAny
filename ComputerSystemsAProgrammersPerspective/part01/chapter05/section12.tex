%%
%% Author: Clay
%% 2020/6/2
%%

\section{理解内存性能}
{
    所有的现代处理器都包含一个或多个\emreg{高速缓存(cache)}存储器,以及对这样少量的存储器提供快速的访问。

    \subsection{加载的性能}
    {
        一个包含加载操作的程序的性能既依赖于流水线的能力,也依赖于加载单元的延迟。
    }

    \subsection{存储的性能}
    {
        与之对应的是\emreg{存储(store)}操作,它将一个寄存器值写到内存。

        大多数情况中,存储操作能够在完全流水线化的模式中工作。

        就其本性来说,一系列存储操作不会产生数据相关。
        只有加载操作会受存储操作结果的影响,因为只有加载操作能从由存储操作写的那个位置读回值。

        \emreg{写/读相关(write/read dependency)}---一个内存读的结果依赖于一个最近的内存写。

        存储单元包含一个\emreg{存储缓冲区},它包含已经被发射到存储单元而又还没有完成的存储操作的地址和数据,这里的完成包括更新数据高速缓存。
        提供这样一个缓冲区,使得一系列存储操作不必等待每个操作都更新高速缓存就能够执行。
        当一个加载操作发生时,它必须检查存储缓冲区中的条目,看有没有地址相匹配。
        如果有地址相配,意味着在写的字节与在读的字节有相同的地址。

        %10.
        \begin{practicec}
            \begin{enumerate}[A.]
                \item 将后一个值写入前一个位置。
                \item 将前一个值写入后一个位置。
                \item B调用产生了写/读相关。
                \item CPE为1。
            \end{enumerate}
        \end{practicec}

        %11.
        \begin{practicec}
            加载、存储、加法。
        \end{practicec}

        %12.
        \begin{practicec}
            增加一个临时变量。
        \end{practicec}
    }
}
