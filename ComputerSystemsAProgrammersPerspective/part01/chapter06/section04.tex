%%
%% Author: Clay
%% 2020/8/3
%%

\section{高速缓存存储器}
{
    由于CPU和主存之间逐渐增大的差距,系统设计者被迫在CPU寄存器文件和主存之间插入高速缓存存储器。

    \subsection{通用的高速缓存存储器组织结构}
    {
        考虑一个计算机系统,其中每个存储器地址有 $m$ 位,形成 $M = 2^m$ 个不同的地址。
        这样一个机器的高速缓存被组织成一个有 $S = 2^s$ 个\emreg{高速缓存组(cache set)}的数组。
        每个组包含 $E$ \emreg{高速缓存行(cache line)}。
        每个行是由一个 $B = 2^b$ 字节的数据\emreg{(block)}组成的,一个\emreg{有效位(valid bit)}指明这个行是否包含有意义的信息,还有 $t = m - (b + s)$ 个\emreg{标记位(tag bit)},它们唯一地表示存储在这个高速缓存行中的块。

        一般而言,高速缓存的结构可以用元组 $(S, E, B, m)$ 来描述。
        高速缓存的大小 $C$ 指的是所有块的大小的和。
        标记位和有效位不包括在内,因此 $C = S \times E \times B$ 。

        %9.
        \begin{practicec}
            \begin{table}[htb]
                \[
                    \begin{array}{|c|c|c|c|c|c|c|c|c|}
                        \hline
                        \text{高速缓存} & m & C & B & E & S & t & s & b \\
                        \hline
                        1 & 32 & 1024 & 4 & 1 & 256 & 22 & 8 & 2 \\
                        \hline
                        2 & 32 & 1024 & 8 & 4 & 32 & 24 & 5 & 3 \\
                        \hline
                        3 & 32 & 1024 & 32 & 32 & 1 & 27 & 0 & 5 \\
                        \hline
                    \end{array}
                \]
            \end{table}
        \end{practicec}
    }

    \subsection{直接映射高速缓存}
    {
        每个组只有一行( $E = 1$ )的高速缓存称为\emreg{直接映射高速缓存(direct-mapped cache)}。

        高速缓存确定一个请求是否命中,然后抽取出被请求的字的过程分为三步:

        \begin{enumerate}
            \item 组选择
            \item 行匹配
            \item 字抽取
        \end{enumerate}

        \subsubsection{直接映射高速缓存中的组选择}
        {
            高速缓存从 $w$ 的地址中间抽取出 $s$ 个组索引位。
            这些位被解释成一个对应于一个组号的无符号整数。
        }

        \subsubsection{直接映射高速缓存中的行匹配}
        {
            每个组只有一行,当且仅当设置了有效位,而且高速缓存中的标记与 $w$ 地址中的标记相匹配时,这一行中包含 $w$ 的一个副本。
        }

        \subsubsection{直接映射高速缓存中的字选择}
        {
            块偏移位提供了所需要的字的第一个字节的偏移。
        }

        \subsubsection{直接映射高速缓存中不命中时的行替换}
        {
            对于直接映射高速缓存来说,替换策略非常简单:用新取出的行替换当前的行。
        }

        \subsubsection{综合:运行中的直接映射高速缓存}
        {

        }

        \subsubsection{直接映射高速缓存中的冲突不命中}
        {
            术语\emreg{抖动(thrash)}描述的是这样一种情况,即高速缓存反复地加载和驱逐相同的高速缓存块的组。
        }

        %10.
        \begin{practicec}
            $75\%$
        \end{practicec}

        %11.
        \begin{practicec}
            \begin{enumerate}[A.]
                \item $2^t$
                \item 1
            \end{enumerate}
        \end{practicec}
    }

    \subsection{组相联高速缓存}
    {
        直接映射高速缓存中冲突不命中造成的问题源于每个组只有一行这个限制。
        \emreg{组相联高速缓存(set associative cache)}放松了这条限制,所以每个组都保存有多于一个的高速缓存行。

        \subsubsection{组相联高速缓存中的组选择}
        {
            组索引位标识组。
        }

        \subsubsection{组相联高速缓存中的行匹配和字选择}
        {
            \emreg{相联存储器}是一个(key, value)对的数组,以key为输入,返回与输入相匹配的value值。
            可以把组相联高速缓存中的每个组都看成一个小的相联存储器,key是标记和有效位,而value就是块的内容。

            组中的任何一行都可以包含映射到这个组的内存块。
            所以高速缓存必须搜索组中的每一行,寻找一个有效的行,其标记与地址中的标记相匹配。
        }

        \subsubsection{组相联高速缓存中不命中时的行替换}
        {
            如果有一个空行,那它就是个很好的候选。
            但是如果该组中没有空行,那么必须从中选择一个非空的行,希望CPU不会很快引用这个被替换的行。

            越往存储器层次结构下面走,远离CPU,一次不命中的开销就会更加昂贵,用更好的替换策略使得不命中最少也变得更加值得了。
        }
    }

    \subsection{全相联高速缓存}
    {
        \emreg{全相联高速缓存(fully associative cache)}是由一个包含所有高速缓存行的组组成的。

        \subsubsection{全相联高速缓存中的组选择}
        {
            地址中没有组索引位,地址只被划分成了一个标记和一个块偏移。
        }

        \subsubsection{全相联高速缓存中的行匹配和字选择}
        {
            全相联高速缓存中的行匹配和字选择与组相联高速缓存中的是一样的,区别主要是规模大小的问题。

            因为高速缓存电路必须并行的搜索许多相匹配的标记,构造一个又大又快的相联高速缓存很困难,而且很昂贵。
        }
    }

    %12.
    \begin{practicec}
        CT: 8; CI: 3; CO: 2
    \end{practicec}

    %13.
    \begin{practicec}
        \begin{enumerate}[A.]
            \item 0 1110 0011 0100
            \item CO: 0
            \item CI: 5
            \item CT: 0x71
            \item 命中
            \item 0x0b
        \end{enumerate}
    \end{practicec}

    %14.
    \begin{practicec}
        \begin{enumerate}[A.]
            \item 0 1101 1101 0101
            \item CO: 1
            \item CI: 5
            \item CT: 0x6e
            \item 未命中
            \item -
        \end{enumerate}
    \end{practicec}

    %15.
    \begin{practicec}
        \begin{enumerate}[A.]
            \item 1 1111 1110 0100
            \item CO: 0
            \item CI: 1
            \item CT: 0xff
            \item 未命中
            \item -
        \end{enumerate}
    \end{practicec}

    %16.
    \begin{practicec}
        \begin{enumerate}[A.]
            \item 0 0000 1100 1100
            \item 0 0000 1100 1101
            \item 0 0000 1100 1110
            \item 0 0000 1100 1111
            \item 0 0110 0100 1100
            \item 0 0110 0100 1101
            \item 0 0110 0100 1110
            \item 0 0110 0100 1111
        \end{enumerate}
    \end{practicec}

    \subsection{有关写的问题}
    {
        高速缓存关于读的操作非常简单。
        首先,在高速缓存中查找所需字 $w$ 的副本。
        如果命中,立即返回字 $w$ 给CPU。
        如果不明惠宗,从存储器层次机构中较低层中取出包含 $w$ 的块,将这个块存储到某个高速缓存行中(可能会驱逐一个有效的行),然后返回字 $w$ 。

        假设要写一个已经缓存了的字 $w$ \emreg{写命中(write hit)} 。
        在高速缓存更新了副本之后,最简单的方法称为\emreg{直写(write-through)},立即将 $w$ 的高速缓存写回到紧跟着的第一层中。
        另一种方法,称为\emreg{写回(write-back)},尽可能地推迟更新,只有当替换算法要驱逐这个更新过的块时,才把它写到紧接着的低一层中。
        高速缓存必须为每个高速缓存行维护一个额外的\emreg{修改位(dirty bit)},表明这个高速缓存块是否被修改过。

        另一个问题是如何处理写不命中。
        一种方法称为\emreg{写分配(write-allocate)},加载相应的低一层中的块到高速缓存中,然后更新这个高速缓存块。
        另一种方法称为\emreg{非写分配(not-write-allocate)},避开高速缓存,直接把这个字写到低一层中。
    }

    \subsection{一个真实的高速缓存层次结构的解剖}
    {
        只保存指令的高速缓存称为\emreg{i-cache}。
        只保存数据的高速缓存称为\emreg{d-cache}。
        既保存指令又保存数据的称为\emreg{统一的高速缓存(unified cache)}。
    }

    \subsection{高速缓存参数的性能影响}
    {
        有许多指标衡量高速缓存的性能:

        \begin{description}
            \item[不命中率(miss rate)] 内存引用不命中的比率:\emreg{不命中数量/引用数量}。
            \item[命中率(hit rate)] 命中的内存引用比率:\emreg{1 - 不命中率}。
            \item[命中时间(hit time)] 从高速缓存传送一个字到CPU所需的时间。
            \item[不命中触发(miss penalty)] 由于不命中所需要的额外时间。
        \end{description}

        \subsubsection{高速缓存大小的影响}
        {
            较大的高速缓存可能会提高命中率,也可能会增加命中时间。
        }

        \subsubsection{块大小的影响}
        {
            大的块能利用程序中可能存在的空间局部性,帮助提高命中率。
            对于给定的高速缓存大小,块越大就意味着高速缓存行数越少,这会损害时间局部性比空间局部性更好的程序中的命中率。
            较大的快对不命中处罚也有负面影响。
        }

        \subsubsection{相联度的影响}
        {
            相联度的选择最终变成了命中时间和不命中处罚之间的折中。
        }

        \subsubsection{写策略的影响}
        {
            直写高速缓存比较容易实现,而且能使用独立高速缓存的\emreg{write buffer},用来更新内存。
            高速缓存越往下层,越可能使用写回。
        }
    }
}
