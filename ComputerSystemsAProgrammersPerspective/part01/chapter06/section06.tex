%%
%% Author: Clay
%% 2020/8/10
%%

\section{综合:高速缓存对程序性能的影响}
{
    \subsection{存储器山}
    {
        一个程序从存储系统中读数据的速率称为\emreg{读吞吐量(read throughput)},或者有时称为\emreg{读带宽(read bandwidth)}。

        size的值越小,得到的工作集越小,因此时间局部性越好。
        stride的值越小,得到的空间局部性越好。
        反复以不同的size和stride值调用run函数,那么就能得到一个读带宽的时间和空间局部性的二维函数,称为\emreg{存储器山(memory mountain)}。

        当程序的时间局部性很差时,空间局部性任然能补救。

        %21.
        \begin{practicec}
            $2100Mhz / (12000 MB / s) * 8B \simeq 1.5 hz \cdot s$
        \end{practicec}
    }

    \subsection{重新排列循环以提高空间局部性}
    {

    }

    \subsection{在程序中利用局部性}
    {
        理解存储器层次结构本质能够利用这些知识编写出更有效的程序,无论具体的存储系统结构是怎样的。

        \begin{itemize}
            \item 将注意力集中在内循环上,大部分计算和内存访问都发生在这里。
            \item 通过按照数据对象存储在内存中的顺序、以步长为1的来读数据,从而使得程序中的空间局部性最大。
            \item 一旦从存储器中读入了一个数据对象,就尽可能多地使用它,从而使得程序中的时间局部性最大。
        \end{itemize}
    }
}
