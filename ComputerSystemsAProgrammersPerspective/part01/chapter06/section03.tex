%%
%% Author: Clay
%% 2020/6/19
%%

\section{存储器层次结构}
{
    存储技术和计算机软件的一些基本的和持久的属性:

    \begin{description}
        \item[存储技术] 不同存储技术的访问时间差异很大。CPU和主存之间的速度差距在增大。
        \item[计算机软件] 一个编写良好的程序倾向于展示出良好的局部性。
    \end{description}

    硬件和软件的这些基本属性互相补充的很完美。
    他们这种相互补充的性质使人想到一种组织存储器系统的方法,称为\emreg{存储器层次结构(memory hierarchy)}。

    \subsection{存储器层次结构中的缓存}
    {
        \emreg{高速缓存(cache)}是一个小而快速的存储设备,它作为存储在更大、也更慢的设备中的数据对象的缓冲区域。
        使用高速缓存的过程称为\emreg{缓存(caching)}。

        存储器层次结构的中心思想是,对于每个 $k$ ,位于 $k$ 层的更快更小的存储设备作为 $k + 1$ 层的更大更慢的存储设备的缓存。
        换句话说,层次结构中的每一层都缓存来自较低一层的数据对象。

        第 $k + 1$ 层的存储器被划分成连续的数据对象组块(chunk),称为\emreg{块(block)}。

        类似地,第 $k$ 层的存储器被划分成较少的块的集合,每个块的大小与 $k + 1$ 层的块的大小一样。
        在任何时刻,第 $k$ 层的缓存包含第 $k + 1$ 层块的一个子集的副本。

        数据总是以块大小为\emreg{传送单元(transfer unit)}在第 $k$ 层和第 $k + 1$ 层来回复制的。
        虽然在层次结构中任何一对相邻的层次之间块大小是固定的,但是其他的层次对之间可以有不同的大小。
        一般而言,层次结构中较低层的设备的访问时间较长,因此为了补偿这些较长的访问时间,倾向于使用较大的块。

        \subsubsection{缓存命中}
        {

        }
    }
}
