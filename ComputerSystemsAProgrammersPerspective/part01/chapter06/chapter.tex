%%
%% Author: Clay
%% 2020/2/20
%%

\chapter{存储器层次结构}
{
    \emreg{存储器系统(memory system)}是一个具有不同容量、成本和访问时间的存储设备的层次结构。
    CPU寄存器保存着最常用的数据。
    靠近CPU的小的、快速的\emreg{高速缓存存储器(cache memory)}作为一部分存储在相对慢速的\emreg{主存储器(main memory)}中数据和指令的缓冲区域。
    主存缓存存储在容量较大的、慢速硬盘上的数据,而这些磁盘常常又作为存储在通过网络连接的其他机器的磁盘或磁带上的数据的缓冲区域。

    存储器层次结构是可行的,这是因为与下一个更低层次的存储设备相比来说,一个编写良好的程序倾向于更频繁地访问某一个层次上的存储设备。

    如果理解了系统是如何将数据在存储器层次结构中上上下下移动的,那么就可以编写自己的应用程序使得它们的数据项存储在层次结构中较高的地方,在那里CPU能更快地访问到它们。

    具有良好\emreg{局部性(locality)}的程序倾向于一次又一次的访问相同的数据项集合。

    %%
%% Author: Clay
%% 2020/12/5
%%

\section{死锁的原理}
{
    死锁是指一组进程因为竞争系统资源或互相等待消息,而永远无法向前推进的状态。

    \subsection{可重用资源}
    {
        资源可以分为两个大类:
        可重用资源与消耗性资源。

        可重用资源指的是一次只供一个进程使用,但用后又能被别的进程使用的资源。

        从系统的角度出发,能够解决死锁问题的方法之一是对应用申请系统资源的顺序做出限定。
    }
}

}

\cleardoublepage

\endinput
