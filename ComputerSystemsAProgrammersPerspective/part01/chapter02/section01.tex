%%
%% Author: Clay
%% 2020/2/21
%%

\section{信息存储}
{
    大多数计算机使用8位的块,或者\emreg{字节(byte)},作为最小的可寻址的内存单位,而不是访问内存中单独的位。
    机器级程序将内存视为一个非常大的字节数组,称为\emreg{虚拟内存(virtual memory)}。
    内存的每个字节都由一个唯一的数字来标识,称为它的\emreg{地址(address)},所有可能地址的集合就称为\emreg{虚拟地址空间(virtual address space)}。

    编译器和运行时系统将存储器空间划分为更可管理的单元,来存放不同的\emreg{程序对象(program object)}。

    \subsection{十六进制表示法}
    {
        \emreg{十六进制(hexadecimal, hex)}使用数字 $0 \sim 9$ 和字符 $A \sim F$ 来表示16个可能的值。

        通过展开每个十六进制数字,将它转换为二进制格式。

        如果给定一个二进制数字,可以通过把它分为每4位一组来转换为十六进制。
        如果位总数不是4的倍数,最左边的一组可以少于4位,前面用0补足。

        %1.
        \begin{practicec}
            \begin{enumerate}[A.]
                \item $11 1001 1010 0111 1111 1000$
                \item $0xC97A$
                \item $1101 0101 1110 0100 1100$
                \item $0x26E7B5$
            \end{enumerate}
        \end{practicec}
    }
}
