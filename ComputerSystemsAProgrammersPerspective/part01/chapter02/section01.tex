%%
%% Author: Clay
%% 2020/2/21
%%

\section{信息存储}
{
    大多数计算机使用8位的块,或者\emreg{字节(byte)},作为最小的可寻址的内存单位,而不是访问内存中单独的位。
    机器级程序将内存视为一个非常大的字节数组,称为\emreg{虚拟内存(virtual memory)}。
    内存的每个字节都由一个唯一的数字来标识,称为它的\emreg{地址(address)},所有可能地址的集合就称为\emreg{虚拟地址空间(virtual address space)}。

    编译器和运行时系统将存储器空间划分为更可管理的单元,来存放不同的\emreg{程序对象(program object)}。

    \subsection{十六进制表示法}
    {
        \emreg{十六进制(hexadecimal, hex)}使用数字 $0 \sim 9$ 和字符 $A \sim F$ 来表示16个可能的值。

        通过展开每个十六进制数字,将它转换为二进制格式。

        如果给定一个二进制数字,可以通过把它分为每4位一组来转换为十六进制。
        如果位总数不是4的倍数,最左边的一组可以少于4位,前面用0补足。

        %1.
        \begin{practicec}
            \begin{enumerate}[A.]
                \item $0011 1001 1010 0111 1111 1000$
                \item $0xc97a$
                \item $1101 0101 1110 0100 1100$
                \item $0x26e7b5$
            \end{enumerate}
        \end{practicec}

        当值 $x = 2^n$ 时, $x$ 的二进制表示就是 $1$ 后面跟 $n$ 个 $0$ 。
        十六进制数字 $0$ 代表 4个二进制 $0$ 。
        所以当 $n$ 表示成 $i + 4j$ 的形式,其中 $0 \leq i \leq 3$ ,可以把 $x$ 写为开头的十六进制数字为 $1(i = 0), 2(i = 1), 4(i = 2), 8(i = 3)$ ,后面紧跟 $j$ 个十六进制的 $0$ 。

        %2.
        \begin{practicec}
            \begin{table}[H]
                \[
                    \begin{array}{|c|c|c|}
                        \hline
                        n & 2^n \text{(十进制)} & 2^n \text{(十六进制)} \\
                        \hline
                        9 & 512 & 0x200 \\
                        \hline
                        19 & 524288 & 0x80000 \\
                        \hline
                        14 & 16384 & 0x4000 \\
                        \hline
                        16 & 65536 & 0x10000 \\
                        \hline
                        17 & 131072 & 0x20000 \\
                        \hline
                        5 & 32 & 0x20 \\
                        \hline
                        7 & 128 & 0x80 \\
                        \hline
                    \end{array}
                \]
            \end{table}
        \end{practicec}

        将一个十进制数字 $x$ 转换为十六进制,可以反复地用 $16$ 除 $x$ ,得到一个商 $q$ 和一个余数 $r$ ,也就是 $x = q \cdot 16 + r$ 。
        然后用十六进制数字表示的 $r$ 作为最低为数字,并通过对 $q$ 反复进行这个过程得到剩下的数字。

        将一个十六进制数字转换为十进制数字,可以用相应的 $16$ 的幂乘以每个十六进制数字。

        %3.
        \begin{practicec}
            \begin{table}[H]
                \[
                    \begin{array}{|c|c|c|}
                        \hline
                        \text{十进制} & \text{二进制} & \text{十六进制} \\
                        \hline
                        0 & 0000 0000 & 0x00 \\
                        \hline
                        167 & 1010 0111 & a7 \\
                        \hline
                        62 & 0011 1110 & 3e \\
                        \hline
                        188 & 1011 1100 & bc \\
                        \hline
                        55 & 0011 0111 & 37 \\
                        \hline
                        136 & 1000 1000 & 88 \\
                        \hline
                        243 & 1111 0011 & f3 \\
                        \hline
                        82 & 0101 0010 & 0x52 \\
                        \hline
                        172 & 1010 1100 & 0xac \\
                        \hline
                        231 & 1110 0111 & 0xe7 \\
                        \hline
                    \end{array}
                \]
            \end{table}
        \end{practicec}

        %4.
        \begin{practicec}
            \begin{enumerate}[A.]
                \item $0x5044$
                \item $0x4ffc$
                \item $0x507c$
                \item $0xae$
            \end{enumerate}
        \end{practicec}
    }

    \subsection{字数据大小}
    {
        每台计算机都有一个\emreg{字长(word size)},指明指针数据的\emreg{标称大小(nominal size)}。
        对于一个字长为 $w$ 位的机器而言,虚拟地址的范围为 $0 \sim 2^w - 1$ ,程序最多访问 $2^w$ 个字节。

        整数或者为\emreg{有符号的},即可以表示负数、零和正数;
        或者为\emreg{无符号的},即只能表示非负数。
    }

    \subsection{寻址和字节顺序}
    {
        在几乎所有的机器上,多字节对象都被存储为连续的字节序列,对象的地址为所使用字节中最小的地址。

        考虑一个 $w$ 位的整数,其表示为 $[x_{w - 1}, x_{w - 2}, \cdots, x_1, x_0]$ ,其中 $x_{w - 1}$ 是最高有效位,而 $x_0$ 是最低有效位。
        假设 $w$ 是 $8$ 的倍数,这样这些位就能被分组成为字节。
        某些机器选择在内存中按照从最低有效字节到最高有效字节的顺序存储对象,而另一些机器则按照从最高有效字节到最低有效字节的顺序存储。
        前一种规则称为\emreg{小端法(little endian)},后一种规则称为\emreg{大端法(big endian)}。

        有时候,字节序会成为问题。

        \begin{enumerate}
            \item
            {
                首先是在不同类型的机器之间通过网络传送二进制数据时,一个常见的问题是当小端法机器产生的数据被发送到大端机器或者反过来时,接收程序会发现,字里的字节成了反序的。
            }
            \item
            {
                第二种情况时,当阅读表示整数数据的字节序列时字节顺序也很重要。
            }
            \item
            {
                第三种情况是当编写规避正常的类型系统的程序时。
            }
        \end{enumerate}

        %5.
        \begin{practicec}
            \begin{enumerate}[A.]
                \item $21, 87$
                \item $2143, 8765$
                \item $214365, 876543$
            \end{enumerate}
        \end{practicec}

        %6.
        \begin{practicec}
            \begin{enumerate}[A.]
                \item $0000 0000 0011 0101 1001 0001 0100 0001, 0100 1010 0101 0110 0100 0101 0000 0100$
                \item 21位
                \item 前9位和后2位
            \end{enumerate}
        \end{practicec}
    }

    \subsection{表示字符串}
    {
        C语言中字符串编码为一个以\emcode{null}(其值为0)字符结尾的字符数组。
        在使用ASCII码作为字符码的任何系统上都将得到相同的结果,与字节顺序和字大小规则无关。
        因而,文本数据比二进制数据具有更强的平台独立性。

        %7.
        \begin{practicec}
            $61, 62, 63, 64, 65, 66$
        \end{practicec}
    }

    \subsection{表示代码}
    {
        不同的机器类型使用不同的且不兼容的指令和编码方式。

        从机器的角度看,程序仅仅只是字节序列。
        机器没有关于原始源程序的任何信息。
    }

    \subsection{布尔代数简介}
    {
        二进制值是计算机编码、存储和操作信息的核心。

        \emreg{布尔代数(Boolean algebra)}将逻辑值 $True$ 和 $False$ 编码为二进制值 $1$ 和 $0$ ,以研究逻辑推理的基本原则。

        最简单的布尔代数是在二元集合 $\{0, 1\}$ 基础上的定义。
        运算符\emcode{\~{}}、\emcode{\&}、\emcode{|}和\emcode{\^{}}分别对应逻辑运算NOT、AND、OR和EXCLUSIVE--OR。

        可以将上述4个布尔运算扩展到\emreg{位向量}的运算,位向量就是固定长度为 $w$ 、由 $0$ 和 $1$ 组成的串。
        位向量的运算可以定义成参数的每个对应元素之间的运算。

        %8.
        \begin{practicec}
            \begin{table}[H]
                \[
                    \begin{array}{|c|c|}
                        \hline
                        \text{运算} & \text{结果} \\
                        \hline
                        a & 0110 1001 \\
                        b & 0101 0101 \\
                        \text{\~{}}a & 1001 0110 \\
                        \text{\~{}}b & 1010 1010 \\
                        a \& b & 0100 0000 \\
                        a | b & 0111 1101 \\
                        a \oplus b & 0011 1100 \\
                        \hline
                    \end{array}
                \]
            \end{table}
        \end{practicec}

        位向量一个很有用的应用就是表示有限集合。
        可以用位向量 $[a_{w - 1}, \cdots, a_1, a_0]$ 来编码任何子集 $A \subseteq \{0, 1, \cdots, w - 1\}$ ,其中 $a_i = 1$ 当且仅当 $i \in A$ 。

        %9.
        \begin{practicec}
            \begin{enumerate}[A.]
                \item 黑色,白色。蓝色,黄色。绿色,红紫色。蓝绿色,红色。
                \item 蓝绿色。绿色。蓝色。
            \end{enumerate}
        \end{practicec}
    }

    \subsection{C语言中的位级运算}
    {
        %10.
        \begin{practicec}
            \begin{table}[H]
                \[
                    \begin{array}{|c|c|c|}
                        \hline
                        \text{步骤} & *x & *y \\
                        \hline
                        \text{初始} & a & b \\
                        \hline
                        \text{第1步} & a & a \oplus b \\
                        \hline
                        \text{第2步} & a \oplus (a \oplus b) = b & a \oplus b \\
                        \hline
                        \text{第3步} & b & (a \oplus b) \oplus b = a \\
                        \hline
                    \end{array}
                \]
            \end{table}
        \end{practicec}

        %11.
        \begin{practicec}
            \begin{enumerate}[A.]
                \item $k, k$
                \item $a \oplus a = 0$
                \item \emcode{first < last}
            \end{enumerate}
        \end{practicec}

        位级运算的一个常见用法就是实现\emreg{掩码}运算。
        这里掩码是一个位模式,表示从一个字中选出的位的集合。

        %12.
        \begin{practicec}
            \begin{enumerate}[A.]
                \item \emcode{x \& 0xFF}
                \item \emcode{x \^{} \~{} 0xFF}
                \item \emcode{x | 0xFF}
            \end{enumerate}
        \end{practicec}

        %13.
        \begin{practicec}
            \begin{enumerate}[A.]
                \item \emcode{x | m}
                \item \emcode{x \& \~{} m}
            \end{enumerate}
        \end{practicec}
    }

    \subsection{C语言中的逻辑运算}
    {
        逻辑运算认为所有非零的参数都表示\emcode{True},参数\emcode{0}表示\emcode{False}。
        它们返回\emcode{0}或者\emcode{1}。
        按位运算只有参数被限制为0或1时,才和其对应的逻辑运算有相同的行为。

        第二个区别是:如果对第一个参数求值就能确定表达式的结果,那么逻辑运算符就不会对第二个参数求值。

        %14.
        \begin{practicec}
            \begin{table}[H]
                \[
                    \begin{array}{|c|c|c|c|}
                        \hline
                        x & 0110 0110 & y & 0011 1001 \\
                        \hline
                        x \&{} y & 0010 0000 & x \&{}\&{} y & 1 \\
                        \hline
                        x | y & 0111 1111 & x || y & 1 \\
                        \hline
                        \text{\~{}}x | \text{\~{}}y & 1000 0000 & !x || !y & 0 \\
                        \hline
                        x \&{} !y & 0100 0000 & x \&{}\&{} \text{\~{}}y & 0 \\
                        \hline
                    \end{array}
                \]
            \end{table}
        \end{practicec}

        %15.
        \begin{practicec}
            \emcode{!(x \^{} y)}
        \end{practicec}
    }

    \subsection{C语言中的移位运算}
    {
        C语言还提供了一组\emreg{移位运算},向左或向右移动位模式。
        对于一个位表示为 $[x_{w - 1}, \cdots, x_1, x_0]$ 的操作数 $x$ , \emcode{x << k}会生成一个值,其位表示为 $[x_{w - k - 1}, \cdots, x_1, x_0, 0, \cdots, 0]$ 。
        也就是说, $x$ 向左移 $k$ 位,丢弃最高的 $k$ 位,并在右端补 $k$ 个 $0$ 。
        移位量应该是一个 $0 \sim w - 1$ 之间的值。
        移位运算是从左至右可结合的。

        有一个相应的右移运算\emcode{x >> k}。
        机器支持两种形式的右移:\emreg{逻辑右移}和\emreg{算术右移}。
        逻辑右移在左端补 $k$ 个 $0$ ,得到的结果是 $[0, \cdots, 0, x_{w - 1}, x_{w - 2}, \cdots, x_k]$ 。
        算术右移是在左端补 $k$ 个最高有效位的值,得到的结果是 $[x_{w - 1}, \cdots, x_{w - 1}, x_{w - 1}, x_{w - 2}, \cdots, x_k]$ 。

        %16.
        \begin{practicec}
            \begin{table}[H]
                \[
                    \begin{array}{|c|c|c|c|c|c|c|c|}
                        \hline
                        \multicolumn{2}{|c|}{x} & \multicolumn{2}{c|}{x << 3} & \multicolumn{2}{c|}{x >> 2\text{(逻辑)}} & \multicolumn{2}{c|}{x >> 2\text{(算数)}} \\
                        \hline
                        \text{十六进制} & \text{二进制} & \text{二进制} & \text{十六进制} & \text{二进制} & \text{十六进制} & \text{二进制} & \text{十六进制} \\
                        \hline
                        0xc3 & 1100 0011 & 0001 1000 & 0x18 & 0011 0000 & 0x30 & 1111 0000 & 0xF0 \\
                        0x75 & 0111 0101 & 1010 1000 & 0xA8 & 0001 1101 & 0x1E & 0001 1101 & 0x1E \\
                        0x87 & 1000 0111 & 0011 1000 & 0x38 & 0010 0001 & 0x21 & 1110 0001 & 0xE1 \\
                        0x66 & 0110 0110 & 0011 0000 & 0x30 & 0001 1001 & 0x19 & 0001 1001 & 0x19 \\
                        \hline
                    \end{array}
                \]
            \end{table}
        \end{practicec}
    }
}
