%%
%% Author: Clay
%% 2020/2/29
%%

\section{浮点数}
{
    浮点表示对形如 $V = x \times 2^y$ 的有理数进行编码。

    \subsection{二进制小数}
    {
        十进制表示法使用如下形式的表示:

        $$d_md_{m - 1}\cdots d_1d_0.d_{-1}d_{-2}\cdots d_{-n}$$

        其中每个十进制数 $d_i$ 的取值范围是 $0 \sim 9$ 。
        这个表达描述的数值 $d$ 定义如下:

        $$d = \sum_{i = -n}^m 10^i \times d_i$$

        考虑一个形如

        $$b_mb_{m - 1}\cdots b_1b_0.b_{-1}b_{-2}\cdots b_{-n}$$

        的表示法,其中每个二进制数字 $b_i$ 的取值范围是 $0$ 和 $1$ 。
        这种表示方法表示的数 $b$ 定义如下:

        $$b = \sum_{i = -n}^m \times 2^i \times b_i$$

        形如 $0.11\cdots 1_2$ 的数表示的是刚好小于1的数。

        假定仅考虑有限长度的编码,那么十进制表示法不能准确地表达像 $\frac{1}{3}$ 和 $\frac{5}{7}$ 这样的数。
        类似,小数的二进制表示法只能表示那些能够被写成 $x \times 2^y$ 的数。
        其它的值只能被近似地表示。
        增加二进制表示的长度可以提高表示的精度。

        %45.
        \begin{practicec}
            \begin{table}[H]
                \[
                    \begin{array}{|c|c|c|}
                        \hline
                        \text{小数值} & \text{二进制表示} & \text{十进制表示} \\
                        \hline
                        \frac{1}{8} & 0.001 & 0.125 \\
                        \hline
                        \frac{3}{4} & 0.11 & 0.75 \\
                        \hline
                        \frac{25}{16} & 1.1001 & 1.5625 \\
                        \hline
                        \frac{43}{16} & 10.1011 & 10.6875 \\
                        \hline
                        \frac{9}{8} & 1.001 & 1.125 \\
                        \hline
                        \frac{47}{8} & 101.111 & 5.875 \\
                        \hline
                        \frac{51}{16} & 11.0011 & 3.1875 \\
                        \hline
                    \end{array}
                \]
            \end{table}
        \end{practicec}

        %46.
        \begin{practicec}
            \begin{enumerate}[A.]
                \item $0.0000 0000 0000 0000 0000 0001 1001 1001\cdots$
                \item $0.0000000953674316461$
            \end{enumerate}
        \end{practicec}
    }
}
