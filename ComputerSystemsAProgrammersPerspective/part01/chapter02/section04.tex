%%
%% Author: Clay
%% 2020/2/29
%%

\section{浮点数}
{
    浮点表示对形如 $V = x \times 2^y$ 的有理数进行编码。

    \subsection{二进制小数}
    {
        十进制表示法使用如下形式的表示:

        $$d_md_{m - 1}\cdots d_1d_0.d_{-1}d_{-2}\cdots d_{-n}$$

        其中每个十进制数 $d_i$ 的取值范围是 $0 \sim 9$ 。
        这个表达描述的数值 $d$ 定义如下:

        $$d = \sum_{i = -n}^m 10^i \times d_i$$

        考虑一个形如

        $$b_mb_{m - 1}\cdots b_1b_0.b_{-1}b_{-2}\cdots b_{-n}$$

        的表示法,其中每个二进制数字 $b_i$ 的取值范围是 $0$ 和 $1$ 。
        这种表示方法表示的数 $b$ 定义如下:

        $$b = \sum_{i = -n}^m \times 2^i \times b_i$$

        形如 $0.11\cdots 1_2$ 的数表示的是刚好小于1的数。

        假定仅考虑有限长度的编码,那么十进制表示法不能准确地表达像 $\frac{1}{3}$ 和 $\frac{5}{7}$ 这样的数。
        类似,小数的二进制表示法只能表示那些能够被写成 $x \times 2^y$ 的数。
        其它的值只能被近似地表示。
        增加二进制表示的长度可以提高表示的精度。

        %45.
        \begin{practicec}
            \begin{table}[H]
                \[
                    \begin{array}{|c|c|c|}
                        \hline
                        \text{小数值} & \text{二进制表示} & \text{十进制表示} \\
                        \hline
                        \frac{1}{8} & 0.001 & 0.125 \\
                        \hline
                        \frac{3}{4} & 0.11 & 0.75 \\
                        \hline
                        \frac{25}{16} & 1.1001 & 1.5625 \\
                        \hline
                        \frac{43}{16} & 10.1011 & 10.6875 \\
                        \hline
                        \frac{9}{8} & 1.001 & 1.125 \\
                        \hline
                        \frac{47}{8} & 101.111 & 5.875 \\
                        \hline
                        \frac{51}{16} & 11.0011 & 3.1875 \\
                        \hline
                    \end{array}
                \]
            \end{table}
        \end{practicec}

        %46.
        \begin{practicec}
            \begin{enumerate}[A.]
                \item $0.0000 0000 0000 0000 0000 0001 1001 1001\cdots$
                \item $0.0000000953674316461$
                \item $0.3433227539$
                \item $686.6455$
            \end{enumerate}
        \end{practicec}
    }

    \subsection{IEEE浮点表示}
    {
        定点表示法不能很有效地表示非常大的数字。

        IEEE浮点标准用 $V = (-1)^s \times M \times 2^E$ 的形式来表示一个数:

        \begin{description}
            \item[符号(sign)] $s$ 决定这是负数还是正数,对于数值 $0$ 的符号位解释作为特殊情况处理。
            \item[尾数(significand)] $M$ 是一个二进制小数,它的范围是 $1 \sim 2 - \epsilon$ ,或者是 $0 \sim 1 - \epsilon$ 。
            \item[阶码(exponent)] $E$ 的作用是对浮点加权,这个权重是 $2$ 的 $E$ 次幂(可能是负数)。
        \end{description}

        将浮点数的位表示划分为三个字段,分别对这些值进行编码:

        \begin{itemize}
            \item 一个单独的符号位 $s$ 直接编码符号 $s$ 。
            \item $k$ 位的阶码字段 $exp = e_{k - 1}\cdots e_1e_0$ 编码阶码 $E$ 。
            \item $n$ 位小数字段 $frac = f_{n - 1}\cdots f_1f_0$ 编码尾数 $M$ ,但是编码出来的值也依赖于阶码字段的值是否等于 $0$ 。
        \end{itemize}

        在单精度浮点格式中,字段分别为 $s = 1, k = 8, n = 23$ 位。
        在双精度浮点格式中,字段分别为 $s = 1, k = 11, n = 53$ 位。

        根据 $exp$ 的值,被编码的值可以分成三种不同的情况:

        \begin{description}
            \item[规格化的值]
            {
                当 $exp$ 的位模式既不全为0,也不全为1时,都属于这类情况。
                这种情况中,阶码字段被解释为以\emspe{偏置(biased)}形式表示的有符号整数。
                阶码的值是 $E = e - Bias$ ,其中 $e$ 是无符号数,其位表示为 $e_{k - 1}\cdots e_1e_0$ ,而 $Bias$ 是一个等于 $2^k - 1$ 的偏置值。

                小数字段 $frac$ 被解释描述小数值 $f$ ,其中 $0 \leq f < 1$ ,其二进制表示为 $0.f_{n - 1}\cdots f_1f_0$ ,也就是二进制小数点在最高有效位的左边。
                尾数定义为 $M = 1 + f$ 。
                有时这种方式也叫做\emspe{隐含的以1开头(implied leading 1)}表示,因为可以把 $M$ 看成一个二进制表达式为 $1.f_{n - 1}\cdots f_1f_0$ 的数字。
                既然总是能调整阶码 $E$ ,使得尾数 $M$ 在范围 $1 \leq M < 2$ 之中,那么这种表示方法是一种轻松获得一个额外精度位的技巧。
            }
            \item[非规格化的值]
            {
                当阶码域为全0时,所表示的数是\emreg{非规格化}形式。
                在这种情况下,阶码值是 $E = 1 - Bias$ ,而尾数的值是 $M = f$ ,也就是小数字段的值,不包含隐含的开头的 $1$ 。

                非规格化数有两个用途。
                首先提供了一种表示数值 $0$ 的方法( $+0, -0$ )。

                非规格化数的另外一个功能是表示那些非常接近于 $0.0$ 的数。
                它们提供了一种属性,称为\emspe{逐渐溢出(gradual underflow)}。
                其中,可能的数值分布均匀地接近于 $0.0$ 。
            }
            \item[特殊值]
            {
                最后一类数值是当阶码全位 $1$ 的时候出现的。
                当小数域全为0时,得到的值表示无穷($+\infty, -\infty$)。

                当两个非常大的数相乘,或者除以零时,无穷能够表示\emspe{溢出}的结果。
                当小数域为非零时,结果值被称为\emspe{NaN(Not a Number)}。
            }
        \end{description}
    }

    \subsection{数字示例}
    {
        最大非规格化数和最小规格化数之间的平滑转变归功于对非规格化数 $E$ 的定义。
        通过将 $E$ 定义为 $1 - Bias$ 而不是 $-Bias$ ,可以补偿非规格化数的尾数没有隐含开头的1。

        将浮点数的位表示解释为无符号整数,他们就是按升序排列的,就像他们表示的浮点数一样。

        %47.
        \begin{practicec}
            \begin{table}[H]
                \[
                    \begin{array}{|c|c|c|c|c|c|c|c|}
                        \hline
                        \text{位} & e & E & 2^E & f & M & 2^e \times M & \text{十进制} \\
                        \hline
                        0 00 00 & 0 & 0 & 1 & 0 & 0 & 0 & 0 \\
                        \hline
                        0 00 01 & 0 & 0 & 1 & \frac{1}{4} & \frac{1}{4} & \frac{1}{4} & 0.25 \\
                        \hline
                        0 00 10 & 0 & 0 & 1 & \frac{1}{2} & \frac{1}{2} & \frac{1}{2} & 0.5 \\
                        \hline
                        0 00 11 & 0 & 0 & 1 & \frac{3}{4} & \frac{3}{4} & \frac{3}{4} & 0.75 \\
                        \hline
                        0 01 00 & 1 & 0 & 1 & 0 & 1 & 1 & 1 \\
                        \hline
                        0 01 01 & 1 & 0 & 1 & \frac{1}{4} & \frac{5}{4} & \frac{5}{4} & 1.25 \\
                        \hline
                        0 01 10 & 1 & 0 & 1 & \frac{1}{2} & \frac{3}{2} & \frac{3}{2} & 1.5 \\
                        \hline
                        0 01 11 & 1 & 0 & 1 & \frac{3}{4} & \frac{7}{4} & \frac{7}{4} & 1.75 \\
                        \hline
                        0 10 00 & 2 & 1 & 2 & 0 & 1 & 2 & 2 \\
                        \hline
                        0 10 01 & 2 & 1 & 2 & \frac{1}{4} & \frac{5}{4} & \frac{5}{2} & 2.5 \\
                        \hline
                        0 10 10 & 2 & 1 & 2 & \frac{1}{2} & \frac{3}{2} & 3 & 3 \\
                        \hline
                        0 10 11 & 2 & 1 & 2 & \frac{3}{4} & \frac{7}{4} & \frac{7}{2} & 3.5 \\
                        \hline
                        0 11 00 & - & - & - & - & - & \infty & - \\
                        \hline
                        0 11 01 & - & - & - & - & - & NaN & - \\
                        \hline
                        0 11 10 & - & - & - & - & - & Nan & - \\
                        \hline
                        0 11 11 & - & - & - & - & - & NaN & - \\
                        \hline
                    \end{array}
                \]
            \end{table}
        \end{practicec}

        有 $k$ 位阶码和 $n$ 位小数的浮点表示的一般属性:

        \begin{itemize}
            \item 值 $+0.0$ 总有一个全为0的位表示。
            \item
            {
                最小的正非规格化的值位表示,是由最低有效位为1而其他所有位为0构成的。
                它具有小数值 $M = f = 2^{-n}$ 和阶码值 $E = -2^{k - 1} + 2$ 。
                因此数值 $V = 2^{-n - 2^{k - 1} + 2}$ 。
            }
            \item
            {
                最大的非规格化值的位模式是由全为0的阶码字段和全为1的小数字段组成的。
                它有小数值 $M = f = 1 - 2^{-n}$ 和阶码值 $E = -2^{k - 1} + 2$ 。
                因此数值 $V = (1 - 2^{-n}) \times 2^{-2^{k - 1} + 2}$ 。
            }
            \item
            {
                最小的正规格化值的位模式的阶码字段的最低有效位为1,其他位都等于0。
                它的尾数值 $M = 1$ ,阶码值 $E = -2^{k - 1} + 2$ 。
                因此数值 $V = 2^{-2^{k - 1} + 2}$ 。
            }
            \item
            {
                值 $1.0$ 的位表示的阶码字段除了最高有效位等于0以外,其他位都等于1。
                它的尾数值为 $M = 1$ ,而它的阶码为 $E = 0$ 。
            }
            \item
            {
                最大的规格化值的位表示的符号位为0,阶码的最低有效位为0,其他位为1。
                它的小数值 $M = 1 + f = 2 - 2^{-n}$ ,它的阶码为 $E = 2^{k - 1} - 1$ 。
                得到数值 $V = (2 - 2^{-n}) \times 2^{2^{k - 1} - 1} = (1 - 2^{-n - 1} \times 2^{2^{k - 1}})$ 。
            }
        \end{itemize}

        %48.
        \begin{practicec}
            整数位模式: $00000000001 101011001000101000001$ 。
            浮点位模式: $ 0 10010100 10101100100010100000100$ 。
        \end{practicec}

        %49.
        \begin{practicec}
            \begin{enumerate}[A.]
                \item $\lceil log_{10} 2^n \rceil$
                \item 7
            \end{enumerate}
        \end{practicec}
    }

    \subsection{舍入}
    {
        表示方法限制了浮点数的范围和精度,所以浮点运算只能近似地表示实数运算。
        对于值 $x$ ,一般想用一种系统的方法,能够找到最接近的匹配值 $x'$ ,它可以用期望的浮点形式表示出来。
        这就是\emreg{舍入(rounding)}运算的任务。
        一个关键问题是在两个可能值的中间确定舍入方向。

        \emreg{向偶数舍入(round-to-even)}也被称为\emreg{向最接近的值舍入(round-to-nearest)},是默认的方式。

        其他三种方式产生实际值的\emreg{确界(guaranteed bound)}。

        %50.
        \begin{practicec}
            \begin{enumerate}[A.]
                \item $10.0$
                \item $10.1$
                \item $11.0$
                \item $11.0$
            \end{enumerate}
        \end{practicec}

        %51.
        \begin{practicec}
            \begin{enumerate}[A.]
                \item $0.00011001100110011001101$
                \item $0.00000000000000000000001$
                \item $0.08583068847656$
                \item $171.661376953$
            \end{enumerate}
        \end{practicec}

        %52.
        \begin{practicec}
            \begin{table}[H]
                \[
                    \begin{array}{|c|c|c|c|}
                        \hline
                        \multicolumn{2}{|c|}{\text{格式A}} & \multicolumn{2}{c|}{\text{格式B}} \\
                        \hline
                        bit & value & bit & value \\
                        \hline
                        011 0000 & 1 & 0111 000 & 1 \\
                        \hline
                        101 1110 & \frac{7}{2} & 1001 111 & \frac{7}{2} \\
                        \hline
                        010 1001 & \frac{9}{32} & 0110 100 & \frac{1}{4} \\
                        \hline
                        110 1111 & \frac{15}{2} & 1001 111 & \frac{15}{2} \\
                        \hline
                        000 0001 & \frac{1}{128} & 0000 000 & 0 \\
                        \hline
                    \end{array}
                \]
            \end{table}
        \end{practicec}
    }
}
