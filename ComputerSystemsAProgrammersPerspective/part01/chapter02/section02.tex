%%
%% Author: Clay
%% 2020/2/27
%%

\section{整数表示}
{
    \subsection{整型数据类型}
    {
        C语言标准只要求正数和负数的取值范围是对称的。
        固定大小的数据类型保证数值与典型数值一致,包括负数与正数的不对称性。
    }

    \subsection{无符号数的编码}
    {
        \begin{defines}[无符号数编码的定义]
            对向量 $\vec x = [x_{w - 1}, x_{w -2}, \cdots, x_0]$ :

            \begin{align}
                B2U_w(\vec x) = \sum_{i = 0}^{w - 1} x_i2^i
            \end{align}
        \end{defines}

        最小值是用位向量 $[00\cdots0]$ 表示,也就是整数值 $0$ 。
        最大值是用位向量 $[11\cdots1]$ 表示,也就是

        $$UMax_w = \sum_{i = 0}^{w - 1} 2^i = 2^w - 1$$

        因此函数 $B2U_w$ 能被定义为一个映射 $B2U_w: \{0, 1\}^w \rightarrow \{0, \cdots, 2^w - 1\}$ 。

        \begin{defines}[无符号数编码的唯一性]
            函数 $B2U_w$ 是一个双射。
        \end{defines}
    }

    \subsection{补码编码}
    {
        最常见的有符号数的计算机表示方式就是\emreg{补码(two's-complement)}形式。
        在这个定义中,将字的最高有效位解释为\emreg{负权(negative weight)}。

        \begin{defines}[补码编码的定义]
            对向量 $\vec x = [x_{w - 1}, x_{w - 2}, \cdots, x_0]$

            \begin{align}
                B2T_w(\vec x) = -x_{w - 1}2^{w - 1} + \sum_{i = 0}^{w - 2}x_i2^i
            \end{align}
        \end{defines}

        最高有效位 $x_{w - 1}$ 也称为\emreg{符号位},它的权重为 $-2^{w - 1}$ ,是无符号表示中权重的负数。

        它能表示的最小值是位向量 $[10\cdots0]$ ,其整数值为

        $$TMin_w = -2^{w - 1}$$

        而最大值是 $[01\cdots1]$ ,其整数值为

        $$TMax_w = 2^{w - 1} - 1$$

        可以看出 $B2T_w$ 是一个从长度为 $w$ 的位模式到 $TMin_w \sim TMax_w$ 之间数字的映射 $B2T_w: \{0, 1\}^w \rightarrow \{TMin_w, \cdots, TMax_w\}$ 。

        \begin{defines}补码编码的唯一性
            函数 $B2T_w$ 是一个双射。
        \end{defines}

        %17.
        \begin{practicec}
            \begin{table}[H]
                \[
                    \begin{array}{|c|c|c|c|}
                        \hline
                        \multicolumn{2}{|c|}{\vec x} & \multirow{2}*{$B2U_4(\vec x)$} & \multirow{2}*{$B2T_4(\vec x)$} \\
                        \cline{1-2}
                        \text{十六进制} & \text{二进制} & & \\
                        \hline
                        0xE & 1110 & 14 & -2 \\
                        0x0 & 0000 & 0 & 0 \\
                        0x5 & 0101 & 5 & 5 \\
                        0x8 & 1000 & 8 & -8 \\
                        0xD & 1101 & 13 & -3 \\
                        0xF & 1111 & 15 & -1 \\
                    \end{array}
                \]
            \end{table}
        \end{practicec}

        补码的范围是不对称的: $|TMin| = |TMax| + 1$ ,也就是说, $TMin$ 没有与之对应的正数。
        最大的无符号数值刚好比补码的最大值的两倍大一点: $UMax_w = 2TMax_w + 1$ 。

        %18.
        \begin{practicec}
            \begin{enumerate}[A.]
                \item $0x2E0, 736$
                \item $-0x58, -88$
                \item $0x28, 40$
                \item $-0x30, -48$
                \item $0x78, 120$
                \item $0x88, 136$
                \item $0x1F8, 504$
                \item $0x8, 8$
                \item $0xC0, 192$
                \item $-0x48, -92$
            \end{enumerate}
        \end{practicec}
    }

    \subsection{有符号数和无符号数之间的转换}
    {
        对于大多数的C语言实现,处理同样字长的有符号数和无符号数之间相互转换的一般规则是:
        数值可能会改变,但是位模式不变。
        定义函数 $U2B_w$ 和 $T2B_w$ ,它们将数值映射为无符号数和补码形式的位表示。

        定义函数 $T2U_w$ 为 $T2U_w(x) = B2U_w(T2B_w(x))$ 。
        类似的,定义函数 $U2T_w$ 为 $B2T_w(U2B_w(x))$ 。

        %19.
        \begin{practicec}
            \begin{table}[H]
                \[
                    \begin{array}{|c|c|}
                        \hline
                        x & T2U_4(x) \\
                        \hline
                        -8 & 8 \\
                        \hline
                        -3 & 13 \\
                        \hline
                        -2 & 14 \\
                        \hline
                        -1 & 15 \\
                        \hline
                        0 & 0 \\
                        \hline
                        5 & 5 \\
                        \hline
                    \end{array}
                \]
            \end{table}
        \end{practicec}

        \begin{defines}[补码转换为无符号数]
            对满足 $TMin_w \leq x \leq TMax_w$ 的 $x$ 有:

            \begin{align}
                T2U_w(x) =
                \begin{cases}
                    x + 2^w, x < 0
                    \\
                    x, x \geq 0
                \end{cases}
            \end{align}
        \end{defines}

        \begin{defines}[无符号数转为补码]
            对满足 $0 \leq u \leq UMax_w$ 的 $u$ 有

            \begin{align}
                U2T_w(u) =
                \begin{cases}
                    u, u \leq TMax_w
                    \\
                    u - 2^w, u > TMax_w
                \end{cases}
            \end{align}
        \end{defines}
    }

    \subsection{C语言中的有符号数与无符号数}
    {
        %21.
        \begin{practicec}
            \begin{table}[H]
                \[
                    \begin{array}{|c|c|c|}
                        \hline
                        \text{表达式} & \text{类型} & \text{求值} \\
                        \hline
                        -2147483647 - 1 == 2147483647U & unsigned & 1 \\
                        \hline
                        -2147483647 - 1 < 2147483647 & signed & 1 \\
                        \hline
                        -2147483647 - 1U < 2147483647 & unsigned & 0 \\
                        \hline
                        -2147483647 - 1 < -2147483647 & signed & 1 \\
                        \hline
                        -2147483647 - 1U < -2147483647 & unsigned & 0 \\
                        \hline
                    \end{array}
                \]
            \end{table}
        \end{practicec}
    }

    \subsection{扩展一个数字的位表示}
    {
        要将一个无符号数转换为一个更大的数据类型,我们只要简单地在表示的开头添加 $0$ 。
        这种运算被称为\emreg{零扩展(zero extension)}。

        \begin{defines}[无符号数的零扩展]
            定义宽度为 $w$ 的位向量 $\vec u = [u_{w - 1}, u_{w - 2}, \cdots, u_0]$  和宽度为 $w'$ 的位向量 $\vec u' = [0, \cdots, 0, u_{w - 1}, u_{w - 2}, \cdots, u_0]$ ,其中 $w' > w$ 。
            则 $B2U_w(\vec u) = B2U_{w'}(\vec u')$ 。
        \end{defines}

        要将一个补码数字转换为一个更大的数据类型,可以执行一个\emreg{符号扩展(sign extension)},在表示中添加最高有效位的值。

        \begin{defines}
            定义宽度为 $w$ 的位向量 $\vec x = [x_{w - 1}, x_{w - 2}, \cdots, x_0]$  和宽度为 $w'$ 的位向量 $\vec x' = [x_{w - 1}, \cdots, x_{w - 1}, x_{w - 1}, x_{w - 2}, \cdots, x_0]$ ,其中 $w' > w$ 。
            则 $B2U_w(\vec x) = B2U_{w'}(\vec x')$ 。
        \end{defines}

        %22.
        \begin{practicec}
            都是对 $1011$ 执行了符号扩展。
        \end{practicec}

        从一个数据大小到另一个数据大小的转换,以及无符号和有符号数字之间的转换的相对顺序能够影响一个程序的行为。

        %23.
        \begin{practicec}
            \begin{table}[H]
                \[
                    \begin{array}{|c|c|c|}
                        \hline
                        w & fun1(w) & fun2(w) \\
                        \hline
                        0x00000076 & 0x00000076 & 0x00000076 \\
                        \hline
                        0x87654321 & 0x00000021 & 0x00000021 \\
                        \hline
                        0x000000c9 & 0x000000c9 & 0x111111c9 \\
                        \hline
                        0xEDCBA987 & 0x00000087 & 0x11111187 \\
                        \hline
                    \end{array}
                \]
            \end{table}
        \end{practicec}
    }

    \subsection{截断数字}
    {
        截断一个数字可能会改变它的值---溢出的一种形式。

        \begin{defines}[截断无符号数]
            令 $\vec x$ 等于位向量 $[x_{w - 1}, x_{w - 2}, \cdots, x_0]$ ,而 $\vec x'$ 是将其截断为 $k$ 位的结果: $\vec x' = [x_{k - 1}, x_{k - 2}, \cdots, x_0]$ 。
            则 $B2U_k(\vec x') = B2U_w(\vec x) \ mod \ 2^k$ 。
        \end{defines}

        \begin{defines}[截断补码数值]
            令 $\vec x$ 等于位向量 $[x_{w - 1}, x_{w - 2}, \cdots, x_0]$ ,而 $\vec x'$ 是将其截断为 $k$ 位的结果: $\vec x' = [x_{k - 1}, x_{k - 2}, \cdots, x_0]$ 。
            则 $B2T_k(\vec x') = U2T_k(B2U_w(\vec x) \ mod \ 2^k)$ 。
        \end{defines}

        %24.
        \begin{practicec}
            \begin{table}[H]
                \[
                    \begin{array}{|c|c|c|c|c|c|}
                        \hline
                        \multicolumn{2}{|c|}{\text{十六进制}} & \multicolumn{2}{c|}{\text{无符号}} & \multicolumn{2}{c|}{\text{补码}} \\
                        \hline
                        \text{原始值} & \text{截断值} & \text{原始值} & \text{截断值} & \text{原始值} & \text{截断值} \\
                        \hline
                        0000 & 000 & 0 & 0 & 0 & 0 \\
                        0010 & 010 & 2 & 2 & 2 & 2 \\
                        1001 & 001 & 9 & 1 & -7 & 1 \\
                        1011 & 011 & 11 & 3 & -5 & 3 \\
                        1111 & 111 & 15 & 7 & -1 & 7 \\
                        \hline
                    \end{array}
                \]
            \end{table}
        \end{practicec}
    }

    \subsection{关于有符号数与无符号数的建议}
    {
        有符号数到无符号数的隐士强制类型转换导致了某些非直观的行为。

        %25.
        \begin{practices}
            改为\emcode{i < length}。
        \end{practices}

        %26.
        \begin{practicec}
            \begin{enumerate}[A.]
                \item 当\emcode{strlen(s) < strlen(t)}时。
                \item 两个无符号数相减永远大于等于 $0$ 。
                \item 返回\emcode{strlen(s) > strlen(t)}。
            \end{enumerate}
        \end{practicec}

        避免这类错误的一种方式就是绝不使用无符号数。

        当仅仅把字看作是位的集合而没有任何数字意义时,无符号数值是非常有用的。
        往一个字中放入描述各种布尔条件的\emreg{标记(flag)}时就是这样。
        地址自然地就是无符号的。
    }
}
