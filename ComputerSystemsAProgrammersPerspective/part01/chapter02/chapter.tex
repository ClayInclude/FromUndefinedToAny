%%
%% Author: Clay
%% 2020/2/21
%%

\chapter{信息的表示和处理}
{
    把\emreg{位(bit)}组合在一起,再加上某种\emreg{解释(interpretation)},即赋予不同的可能位模式以含意,就能够表示任何有限集合的元素。

    \emreg{无符号(unsigned)}编码基于传统的二进制表示方法,表示大于或者等于零的数字。
    \emreg{补码(two's--complement)}编码是表示有符号整数的最常见的方式,有符号整数就是可以为正或者为负的数字。
    \emreg{浮点数(floating-point)}编码是表示实数的科学计数法的以2为基数的版本。

    计算机的表示法是用有限数量的位来对一个数字编码,当结果太大以至不能表示时,某些运算就会\emreg{溢出(overflow)}。

    %%
%% Author: Clay
%% 2020/12/5
%%

\section{死锁的原理}
{
    死锁是指一组进程因为竞争系统资源或互相等待消息,而永远无法向前推进的状态。

    \subsection{可重用资源}
    {
        资源可以分为两个大类:
        可重用资源与消耗性资源。

        可重用资源指的是一次只供一个进程使用,但用后又能被别的进程使用的资源。

        从系统的角度出发,能够解决死锁问题的方法之一是对应用申请系统资源的顺序做出限定。
    }
}

}

\cleardoublepage

\endinput
