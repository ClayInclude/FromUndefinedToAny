%%
%% Author: Clay
%% 2020/2/29
%%

\section{整数运算}
{
    \subsection{无符号加法}
    {
        要想完整的表示算术运算的结果,不能对字长做任何限制。
        \emcode{List}实际上就支持\emreg{无限精度}的运算,允许任意的(要在机器的内存限制之内)整数运算。
        更常见的是,编程语言支持固定精度的运算。

        \begin{defines}[无符号数加法]
            对满足 $0 \leq x, y < 2^w$ 的 $x, y$ 有:

            \begin{align}
                x +_w^u y =
                \begin{cases}
                    x + y, x + y < 2^w
                    \\
                    x + y - 2^w, 2^w \leq x + y < 2^{w + 1}
                \end{cases}
            \end{align}
        \end{defines}

        说一个算数\emreg{溢出},是指完整的整数结果不能放到数据类型的字长限制中去。

        \begin{defines}[检测无符号数加法中的溢出]
            对在范围 $0 \leq x, y \leq UMax_w$ 中的 $x, y$ ,令 $s = x +_w^u y$ 。
            则对计算 $s$ ,当且仅当 $s < x$ (或者等价地 $s < y$ )时,发生了溢出。
        \end{defines}

        %27.
        \begin{practicec}

        \end{practicec}

        模数加法形成了一种数学结构,称为\emreg{阿贝尔群(Abelian group)}。
        也就是说,它是可交换的和可结合的。
        它有一个单位元 $0$ ,并且每个元素有一个加法逆元。

        \begin{defines}[无符号数求反]
            对满足 $0 \leq x < 2^w$ 的任意 $x$ ,其 $w$ 位的无符号逆元 $-_w^ux$ 由下式给出:

            \begin{align}
                -_w^u x =
                \begin{cases}
                    x, x = 0
                    \\
                    2^w - x, x > 0
                \end{cases}
            \end{align}
        \end{defines}

        %28.
        \begin{practicec}
            \begin{table}[H]
                \[
                    \begin{array}{|c|c|c|c|}
                        \hline
                        \multicolumn{2}{|c|}{x} & \multicolumn{2}{c|}{-_4^ux} \\
                        \hline
                        \text{十六进制} & \text{十进制} & \text{十进制} & \text{十六进制} \\
                        \hline
                        0 & 0 & 0 & 0 \\
                        \hline
                        5 & 5 & 11 & B \\
                        \hline
                        8 & 8 & 8 & 8 \\
                        \hline
                        D & 13 & 3 & 3 \\
                        \hline
                        F & 15 & 1 & 1 \\
                        \hline
                    \end{array}
                \]
            \end{table}
        \end{practicec}
    }

    \subsection{补码加法}
    {
        \begin{defines}[补码加法]
            对满足 $-2^{w - 1} \leq x, y \leq 2^{w - 1} - 1$ 的整数 $x, y$ ,有:

            \begin{align}
                x+_w^ty =
                \begin{cases}
                    x + y - 2^w, 2^{w - 1} \leq x + y
                    \\
                    x + y, -2^{w - 1} \leq x + y < 2{w - 1}
                    \\
                    x + y + 2^w, x + y < -2^{w - 1}
                \end{cases}
            \end{align}
        \end{defines}

        \begin{defines}[检测补码加法中的溢出]
            对满足 $TMin_w \leq x, y \leq TMax_w$ 的 $x, y$ ,令 $s = x +_w^t y$ 。
            当且仅当 $x > 0, y > 0$ ,但 $s \leq 0$ 时,计算发生了\emspe{正溢出(positive overflow)}。
            当且仅当 $x < 0, y < 0$ ,但 $s \geq 0$ 时,计算发生了\emspe{负溢出(negative overflow)}。
        \end{defines}

        %29.
        \begin{practicec}
            \begin{table}[H]
                \[
                    \begin{array}{|c|c|c|c|c|}
                        \hline
                        x & y & x + y & x +_5^t y & \text{情况} \\
                        \hline
                        10100 & 10001 & -27 & 5 & \text{负溢出} \\
                        \hline
                        11000 & 11000 & -16 & -16 & \text{正常} \\
                        \hline
                        10111 & 01000 & -1 & -1 & \text{正常} \\
                        \hline
                        00010 & 00101 & 7 & 7 & \text{正常} \\
                        \hline
                        01100 & 00100 & 16 & -16 & \text{正溢出} \\
                        \hline
                    \end{array}
                \]
            \end{table}
        \end{practicec}

        %30.
        \begin{practicec}

        \end{practicec}

        %31.
        \begin{practicec}
            如果有 $sum = x + y$ ,由于补码加法形成了一个阿贝尔群,满足交换律,则有 $sum - x = x + y - x$ ,得出 $sum - x = y$ 。
            无论溢出都会得到相同的结果。
        \end{practicec}

        %32.
        \begin{practicec}
            $y$ 为 $TMin$ 时。
        \end{practicec}
    }

    \subsection{补码的非}
    {
        \begin{defines}[补码的非]
            对满足 $TMin_w \leq x \leq TMax_w$ 的 $x$ ,其补码的非 $-_w^tx$ 由下式给出

            \begin{align}
                -_w^tx =

                \begin{cases}
                    TMin_w, x = TMin_w
                    \\
                    -x, x ? TMin_w
                \end{cases}
            \end{align}
        \end{defines}

        %33.
        \begin{practicec}
            \begin{table}[H]
                \[
                    \begin{array}{|c|c|c|c|}
                        \hline
                        \multicolumn{2}{|c|}{x} & \multicolumn{2}{c|}{-_4^tx} \\
                        \hline
                        \text{十六进制} & \text{十进制} & \text{十进制} & \text{十六进制} \\
                        \hline
                        0 & 0 & 0 & 0 \\
                        \hline
                        5 & 5 & -5 & A \\
                        \hline
                        8 & -8 & -8 & 8 \\
                        \hline
                        D & -3 & 3 & 3 \\
                        \hline
                        F & -1 & 1 & 1 \\
                        \hline
                    \end{array}
                \]
            \end{table}
        \end{practicec}
    }

    \subsection{无符号乘法}
    {
        \begin{defines}[无符号数乘法]
            对满足 $0 \leq x, y \leq < UMax_w$ 的 $x, y$ 有:

            \begin{align}

            \end{align}
        \end{defines}
    }
}
