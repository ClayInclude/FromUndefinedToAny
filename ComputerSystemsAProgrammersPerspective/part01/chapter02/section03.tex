%%
%% Author: Clay
%% 2020/2/29
%%

\section{整数运算}
{
    \subsection{无符号加法}
    {
        要想完整的表示算术运算的结果,不能对字长做任何限制。
        \emcode{List}实际上就支持\emreg{无限精度}的运算,允许任意的(要在机器的内存限制之内)整数运算。
        更常见的是,编程语言支持固定精度的运算。

        \begin{defines}[无符号数加法]
            对满足 $0 \leq x, y < 2^w$ 的 $x, y$ 有:

            \begin{align}
                x +_w^u y =
                \begin{cases}
                    x + y, x + y < 2^w
                    \\
                    x + y - 2^w, 2^w \leq x + y < 2^{w + 1}
                \end{cases}
            \end{align}
        \end{defines}

        说一个算数\emreg{溢出},是指完整的整数结果不能放到数据类型的字长限制中去。

        \begin{defines}[检测无符号数加法中的溢出]
            对在范围 $0 \leq x, y \leq UMax_w$ 中的 $x, y$ ,令 $s = x +_w^u y$ 。
            则对计算 $s$ ,当且仅当 $s < x$ (或者等价地 $s < y$ )时,发生了溢出。
        \end{defines}

        %27.
        \begin{practicec}

        \end{practicec}

        模数加法形成了一种数学结构,称为\emreg{阿贝尔群(Abelian group)}。
        也就是说,它是可交换的和可结合的。
        它有一个单位元 $0$ ,并且每个元素有一个加法逆元。

        \begin{defines}[无符号数求反]
            对满足 $0 \leq x < 2^w$ 的任意 $x$ ,其 $w$ 位的无符号逆元 $-_w^ux$ 由下式给出:

            \begin{align}
                -_w^u x =
                \begin{cases}
                    x, x = 0
                    \\
                    2^w - x, x > 0
                \end{cases}
            \end{align}
        \end{defines}

        %28.
        \begin{practicec}

        \end{practicec}
    }
}
