%%
%% Author: Clay
%% 2021/2/18
%%

\section{虚拟内存作为内存保护的工具}
{
    任何现代计算机系统必须为操作系统提供手段来控制对内存系统的访问。
    不应该允许一个用户进程修改它的只读代码段。
    也不应该允许它读写任何内核中的代码和数据结构。
    不应该允许它独写其他进程的私有内存,并且不允许它修改任何与其他进程共享的虚拟页面,除非所有的共享者都显示地允许它这么做。

    因为每次CPU生成一个地址时,地址翻译硬件都会读一个PTE,所以通过在PTE上添加一些额外的许可位来控制对一个虚拟页面的内容的访问十分简单。

    如果一条指令违反了这些许可条件,那么CPU就触发一个一般保护故障,将控制传递给一个内核中的异常处理程序。
    Linux shell 一般将这种异常报告为\emreg{段错误(segmentation fault)}。
}
