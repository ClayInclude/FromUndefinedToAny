%%
%% Author: Clay
%% 2020/2/20
%%

\chapter{虚拟内存}
{
    为了更加有效地管理内存并且少出错,现代系统提供了一种对主存的抽象概念,叫做\emreg{虚拟内存(VM)}。
    虚拟内存是硬件异常、硬件地址翻译、主存、磁盘文件和内核软件的完美交互,它为每个进程提供了一个大的、一致的和私有的地址空间。
    通过一个很清晰的机制,虚拟内存提供了三个重要的能力:
    它将主存看成是一个存储在磁盘上的地址空间的高速缓存,在主存中只保存活动区域;
    它为每个进程提供了一致的地址空间,从而简化了内存管理;
    它保护了每个进程的地址空间不被其他进程破坏。

    虚拟内存是计算机系统最重要的概念之一。

    \begin{description}
        \item[虚拟内存是核心的] 虚拟内存遍及计算机系统的所有层面,在硬件异常、汇编器、链接器、加载器、共享对象、文件和进程的设计中扮演着重要角色。
        \item[虚拟内存是强大的] 虚拟内存给予应用程序强大的能力,可以创建和销毁\emspe{内存片(chunk)}、将内存片映射到磁盘文件的某个部分,以及其它进程共享内存。
        \item[虚拟内存是危险的] 如果虚拟内存使用不当,应用将遇到复杂危险的与内存有关的错误。
    \end{description}

    %%
%% Author: Clay
%% 2020/12/5
%%

\section{死锁的原理}
{
    死锁是指一组进程因为竞争系统资源或互相等待消息,而永远无法向前推进的状态。

    \subsection{可重用资源}
    {
        资源可以分为两个大类:
        可重用资源与消耗性资源。

        可重用资源指的是一次只供一个进程使用,但用后又能被别的进程使用的资源。

        从系统的角度出发,能够解决死锁问题的方法之一是对应用申请系统资源的顺序做出限定。
    }
}

    %%
%% Author: Clay
%% 2020/11/16
%%

\section{进程状态}
{
    对每一个应用程序的执行,系统都会为其创建一个进程与之对应。
    从处理器的角度来看,它只是从程序计数器指向的地址中取得指令并执行。
    而程序计数器在系统运行过程中,可能时不时地从指向一个进程的代码转到指向另一个进程的代码,从而实现进程切换。
    而从单个的应用程序来看,它的执行对应的是它的代码所规定的一连串指令。

    现在将单个进程的行为抽象为一串指令,并称它为该进程的\emreg{执行轨迹(Trace)}。
    有了这个定义,就能够通过进程的执行轨迹来描述处理器的行为。

    \emreg{调度器(Dispatcher)}在内存的低端,用于调度进程的执行。

    \subsection{双状态模型}
    {
        操作系统的首要任务是控制进程的执行过程,这意味着它必须确定多个进程如何交替执行,以及如何为进程分配资源。
        为了达到这一目标,必须对进程的行为加以描述。
        先建立一个简单的模型,在任意时刻,进程只可能处于两种状态:
        要么执行,要么不执行。

        当操作系统创建一个新的进程时,创建该进程对应的进程控制块,并将该进程投入不执行态。
        之后,如果当前正在运行的进程用完了分配给它的时间片,它将被中断并进入不执行态,而调度程序将调入一个处于不执行态的进程投入运行。

        首先,每个进程必须能够被操作系统跟踪。
        一个进程必须包含一些必要的信息使得操作系统能够感知它的存在,并控制它的执行。
        其次,系统必须设计某种队列结构,来存放被暂停执行或暂时无法执行的进程的信息。

        通过双状态模型,可以描述调度器的行为:
        当前运行的进程要么因为用完时间片而被中断执行,并加入队列中等待,要么因为执行完毕而离开系统。
        无论哪一种情况发生,调度器都将从队列中取出一个进程,并将其投入运行。
    }

    \subsection{进程的创建和结束}
    {
        先讨论进程的\emreg{创建(Create)}和\emreg{结束(Termination)},因为它们标志着一个进程的开始和结束。

        \subsubsection{进程的创建}
        {
            当创建一个进程的时候,操作系统将建立管理它的数据结构,并将其代码加载到内存空间。

            \begin{table}[htb]
                \centering

                \caption{创建进程的典型原因}

                \begin{tabular}{l|l}
                    \hline
                    新建一个批处理任务 & 在一个批处理作业流中,当操作系统完成了上一个作业,从作业流中读出下一个作业是 \\
                    \hline
                    交互式登录 & 用户通过一个交互式终端登录到某个系统中时 \\
                    \hline
                    操作系统创建,以提供某种服务 & 现代操作系统往往需要创建某种服务以辅助用户进程完成其工作 \\
                    \hline
                    被已存在的进程所创建 & 用户进程创建的一个子进程 \\
                    \hline
                \end{tabular}
            \end{table}

            一般来说,操作系统往往以对用户透明的方式来创建进程。
            然而,允许用户进程在执行过程中\emreg{显式(Explicit)}地通过调用系统调用创建进程,将对应用的构建非常有帮助。
            称这种通过显式系统调用创建一个新进程的方法称为\emreg{进程繁殖(Process Spawning)}。
            假如在运行过程中,进程A通过进程繁殖创建了进程B,就乘进程A为\emreg{父进程},而进程B为\emreg{子进程}。
        }

        \subsubsection{进程结束}
        {
            \begin{table}[htb]
                \centering

                \caption{进程\emspe{结束(Termination)}}的典型原因

                \begin{tabular}{l|l}
                    \hline
                    正常结束(Completion) & 进程完成其工作任务,并调用系统调用完成撤销动作 \\
                    \hline
                    超时 & 进程的用时超过了用户定义的上限 \\
                    \hline
                    内存不足 & 进程需要的内存容量超出了系统所能够提供的上线 \\
                    \hline
                    访存超界 & 进程访问了超过其访问界限的内存空间 \\
                    \hline
                    保护错误 & 进程访问了某受保护的资源,或者使用了非法的访问方式访问该资源 \\
                    \hline
                    算术错误 & 进程在执行过程中执行了不被允许的算术指令 \\
                    \hline
                    等待时间过长 & 进程等待某事件发生的时间太长,超过了系统所规定的上限 \\
                    \hline
                    I/O失败 & 系统发生了I/O错误 \\
                    \hline
                    无效指令 & 执行了处理器所不能识别的指令 \\
                    \hline
                    特权态指令 & 用户态执行了只能在特权态下执行的指令 \\
                    \hline
                    数据错误 & 进程使用的数据类型不对,或者使用前未被合理地初始化 \\
                    \hline
                    操作员或操作系统介入 & 因为某种原因操作员或操作系统介入,强制杀死某进程 \\
                    \hline
                    父进程终止 & 当父进程终止的时候,某些极端情况下,操作系统会自动终止其所有子进程的执行 \\
                    \hline
                    父进程请求 & 父进程终止它的某个子进程 \\
                    \hline
                \end{tabular}
            \end{table}

            任何计算机系统都必须为进程提供某种方式,让它在结束执行时通知系统。
        }

        \subsubsection{五状态模型}
        {
            将不执行态加以区分,将其划分为两种状态:
            就绪态和阻塞态。
            这样,系统就存在5种状态:

            \begin{description}
                \item[运行态(Running)] 进程正在运行。
                \item[就绪态(Ready)] 进程已准备好执行,如果给它处理器,他就可以立即投入运行。
                \item[阻塞态/等待态(Blocked/Waiting)] 进程在等待某事件的发生。
                \item[新创建态(New)] 进程刚刚被创建,但还未被操作系统加入可执行队列。
                \item[结束态(Exit)] 进程结束,它所占用的系统资源被释放,但操作系统扔保留其控制结构。
            \end{description}

            进程状态可能发生的转换如下:

            \begin{description}
                \item[无 ---> 新创建态] 进程创建。
                \item[新创建态 ---> 就绪态] 当操作系统准备好一个新进程的执行时,它将挑选一个新创建态进程,将该进程的状态转到就绪态。
                \item[就绪态 ---> 运行态] 当处理器空闲时,操作系统从处于就绪态的进程中挑选一个进程,并将其投入运行。
                \item[运行态 ---> 结束态] 进程结束,或被操作系统撤销。
                \item[运行态 ---> 就绪态]
                {
                    导致这一状态改变的最常见的原因,是进程在执行过程中用完了分配给它的时间片。
                    对于这种情况,通常说操作系统\emreg{抢占(preempt)}了执行权。
                    最后,一个进程可能在运行过程中,\emreg{自愿(volumtarily)}放弃了自己的执行权。
                }
                \item[运行态 ---> 阻塞态] 进程在执行过程中,因为请求一些暂时得不到而必须等待的服务。
                \item[阻塞态 ---> 就绪态] 进程等待的事件到达后。
                \item[就绪态 ---> 结束态]
                \item[阻塞态 ---> 结束态]
            \end{description}

            为每一种可能到达的事件建立一个阻塞队列,将提高操作系统查找进程的速度。
            当某时间到达后,该事件对应的阻塞队列中所有的进程都将被转移到就绪态。

            如果操作系统系统采用了优先级调度策略,可以为每一个优先级组织一个就绪队列,也就是采用多就绪队列的方法。
            这样,操作系统能够较快速地找到高优先级的进程或等待时间最长的进程。
        }

        \subsubsection{进程挂起}
        {
            \paraph{交换(Swapping)的需要}
            {
                可能所有被载入内存的进程都进入了阻塞态,而处理器仍无事可做。

                一个解决方案是虚拟存储技术。
                这一技术可以将进程的一部分或者整个进程从内存交换到硬盘。
                当系统中没有进程处于就绪态时,操作系统就可以用虚拟存储技术,将其中一个处于阻塞态的进程交换到硬盘,将该进程的状态设置为\emreg{挂起状态(Suspend)},并放到挂起状态对应的队列中。

                随着对虚拟存储技术的采用,引入了一个新状态:挂起状态。
                当系统中所有的进程进入阻塞态时,操作系统可以将其中一个阻塞态的进程转换为挂起状态,并将其内存镜像交换到硬盘中,从而腾出空间载入另一个进程。
                这时,系统面临两个选择:
                接受执行一个新创建的进程,或者调度执行一个之前处于挂起状态的进程。
                如果系统选择了后者,就会面临一个新的困难,那就是处于挂起状态的进程,在被换出之前都是处于阻塞态的。

                这里有两个互相独立的概念:
                进程是否在等待某个事件,以及进程是否从内存中被换出。
                显然,这是一个 $2 \times 2$ 的组合问题:

                \begin{description}
                    \item[就绪态(Ready)]
                    \item[阻塞态(Blocked)]
                    \item[阻塞且挂起态(Blocked-Suspend)]
                    \item[就绪且挂起态(Ready=Suspend)]
                \end{description}

                可能进行的状态转换:

                \begin{description}
                    \item[阻塞态 ---> 阻塞且挂起态] 如果系统中没有就绪态进程可调度,则其中至少一个阻塞态的进程将被交换到外存中。
                    \item[阻塞且挂起态 ---> 就绪且挂起态] 进程所等待的事件到达。
                    \item[就绪且挂起态 ---> 就绪态] 当系统中无就绪态进程时,操作系统将调入一个处于就绪且挂起态的进程到内存中。
                    \item[就绪态 ---> 就绪且挂起态]
                    {
                        一般情况下,操作系统更倾向于选择一个处于阻塞态的进程,将其换出到外存中。
                        然后,有时选择一个就绪态的进程将其换出到外村,可能时腾出大量连续内存空间的唯一方法。
                    }
                \end{description}

                还有一些值得讨论的状态转换:

                \begin{description}
                    \item[新创建态 ---> 就绪且挂起态及新创建态 ---> 就绪态]
                    {
                        当一个新创建的进程进入系统时,它要么被加入到就绪队列,要么被加入到就绪且挂起队列中。
                        将新创建的进程放到就绪且挂起状态是非常必要的,因为此时无需将它们完全载入内存。
                        同时,由于这种新创建进程的\emreg{用时调入机制(Just-In-Time)},操作系统能够在自身非常繁忙的时候有效地处理大量新创建进程的任务。
                    }
                    \item{阻塞且挂起态 ---> 阻塞态}
                    {
                        看起来,这一状态转换对于一个实际运行的操作系统没有任何意义。
                        考虑以下情形:
                        当前执行的进程结束执行,并腾出了大量的内存空间。
                    }
                    \item{运行态 ---> 就绪且挂起态}
                    {
                        如果操作系统采用了抢占式调度方案,且有一个高优先级的进程从阻塞且挂起队列中被唤醒,从而抢占了当前运行进程的执行权,且此时因为该高优先级进程的换入导致了内存空间的紧张。
                        这种情况下,操作系统不得不中断正在运行的进程的执行,并将其换出到外存中,以腾出足够的内存空间。
                    }
                    \item{任意态 ---> 结束态}
                    {
                        一般情况下,只有处于运行态的进程会进入结束态。
                        其原因可能是它执行完了所有的预定任务,或者运行过程中发生了致命的错误。
                        然而,在有些操作系统因为某进程的终止,强制中止它所有的子进程。
                        这样,就意味着该系统中的进程可能从任意状态直接进入结束态。
                    }
                \end{description}
            }

            \paraph{挂起的其他用途}
            {
                挂起状态进程的概念:
                一个处于该状态的进程没有被调入到内存中,所以它无法被立即调度执行。

                处于该状态的进程的特点:

                \begin{enumerate}
                    \item 进程不能够被立即调入执行。
                    \item 进程可能在等待某事件的发生,如果这一条成立,阻塞态跟挂起态将同时存在且相互独立。
                    \item 进程必然是被另一个实体置为挂起状态的:可能时它的父进程、操作员或操作系统。
                    \item 一个处于挂起状态的进程必须等到将它挂起的实体发出明确的命令后,才能从挂起状态装换为其他状态。
                \end{enumerate}

                \begin{table}[htb]
                    \centering

                    \caption{进程挂起的原因}

                    \begin{tabular}{l|l}
                        \hline
                        交换 & 操作系统需要腾出内存空间,来调度执行另一个处于就绪态的进程 \\
                        \hline
                        其他的系统原因 & 操作系统可能会挂起一个背景进程或守护进程,也可能是一个普通用户进程以避免它可能导致的一些问题 \\
                        \hline
                        交互式系统中的用户请求 & 交互式系统中,用户可能需要挂起某进程以方便对其进行调试,或强制其暂时释放它所占据的资源 \\
                        \hline
                        时间因素 & 操作系统中有些进程,他们会被周期性唤醒,当未被唤醒时,就可能被置于挂起态 \\
                        \hline
                        父进程的请求 & 父进程在运行过程中可能会挂起它的一个子进程,以协调其它子进程的活动 \\
                        \hline
                    \end{tabular}
                \end{table}

                还有一些原因是由交互式终端的使用产生的。

                被挂起的进程的激活都需要最开始将其挂起的哪个实体明确的发出激活的命令。
            }
        }
    }
}

    %%
%% Author: Clay
%% 2020/11/16
%%

\section{进程描述符}
{
    操作系统控制计算机系统中所有的事件,包括调度进程到处理器上执行、为进程分配资源、响应用户请求等。
    基本上,可以把操作系统看作是以进程为单位,管理系统资源的实体。

    \subsection{操作系统中控制资源的结构}
    {
        加入操作系统需要管理进程和它所使用的资源,那么操作系统必须知道每个进程的当前状态。
        为了达到这一目标,操作系统采用了一个直接的方式:
        操作系统为每一种类型的资源,建立一个表格进行管理。
        操作系统管理着4类资源:
        内存,I/O设备,文件和进程。

        \paraph{内存表格}
        {
            用于跟踪记录物理内存和虚拟内存的使用状况。
            一部分系统内存被保留给操作系统自己使用,而余下的内存则在用户进程之间使用和分配。
            内存表格必须包含以下信息:

            \begin{enumerate}
                \item 物理内存的分配信息
                \item 二级内存的分配信息
                \item 每一块内存的访问权限信息
                \item 用于管理虚拟内存的信息
            \end{enumerate}
        }

        \paraph{I/O设备表格}
        {
            该表格被用于管理I/O设备,以及计算机系统的通道。
            在任意时刻,一个I/O设备只能被分配给一个特定的进程。
        }

        \paraph{文件表格}
        {
            这些表格用于表述文件的基本信息,它们当前的状态,以及一些其他的属性信息。
            在有些操作系统中,这些信息都由文件系统所管理。
        }

        \paraph{进程表格}
        {
            操作系统必须维护进程表格以对系统中存在的进程进行管理。
            4种看似独立的表格,但是,这些表格并不是各自为政的,它们之间通过指针进行相互的链接。
            内存、I/O设备,以及文件实际上都是以进程的单位进行管理。
            这些表格都不可能凭空存在,必须放到内存的特定区域。

            操作系统必须对其运行的环境的有基本的了解。
            当操作系统在初始化阶段,它必须能够得知将要运行的基础信息,这些信息可能需要人工帮助,也可由一些自动配置软件完成。
        }
    }

    \subsection{进程控制块}
    {
        假如操作系统需要控制和管理一个进程,首先操作系统需要知道这个进程位置,其次操作系统需要知道该进程的一些基本属性。

        \paraph{进程位置}
        {
            在最基本的情况下,一个进程必须包含一段或多端即将执行的程序。
            与这些程序相关联的是一组数据。
            进程组成中,必然包含足够的内存用于存储这些程序和数据。
            另外,程序的执行必然会利用到堆栈,用于存放该程序执行过程中调用函数时的参数。
            最后,每个进程都有一组变量,用于描述自身的状态和属性信息以便于操作系统对其进行管理和控制。
            通常称操作系统中用于管理和控制进程的数据结构为\emreg{进程控制块(Process Control Block)}。
            同时,称进程所对应的程序、数据、堆栈,以及它的属性信息所构成的全集为\emreg{进程镜像}。

            进程镜像所防止的位置,取决于内存管理系统。
            在最简单的情形下,进程镜像被连续地存放在辅存种。
            当操作系统需要管理该进程的时候,该进程至少有一部分将被加载到内存。
            当需要执行该进程时,它的进程镜像将被完整地载入物理内存或虚拟内存中。

            现代操作系统假设底层硬件具有页式地址管理的功能,这样以便于在不连续内存空间中,部分地将进程装入。
            在给定时刻,一个进程镜像的一部分可能被装入内存,而其余部分则可能在辅存中。
            所以,进程控制表必须包含足够信息,使得操作系统可以知道进程镜像中页面的为孩子。

            进程位置星系的结构:
            有一个主进程表格,其中每个进程占据一个表项,每个表项至少包含一个指向进程镜像的指针。
            如果进程镜像包含多个块,该信息会在主进程表格中记录,或者通过交叉索引在内存表格中记录。
        }

        \paraph{进程属性}
        {
            复杂得多任务系统需要为每个进程记录大量信息。

            进程控制块中存放的进程信息归为以下3哥类型:

            \begin{description}
                \item[进程标识符(Process Identification)]
                \item[处理器状态信息(Processor State Information)]
                \item[进程控制信息(Process Control Information)]
            \end{description}
        }

        \paraph{进程标识符}
        {
            几乎每个操作系统都为其中的每一个进程设置了唯一的数字标识。
            如果系统允许一个进程在执行过程中创建其它进程,则进程标识符可以用来索引父进程和子进程。
            除了远程标识符外,每个进程还可以给一个用户标识符,以表征创建该进程的用户。
        }

        \paraph{处理器状态信息}
        {
            包含处理了其中寄存器的值和状态。
            当进程执行过程中被中断时,所有的寄存器内的值将被保存到进程控制块中,以便该进程被再次调度执行时的恢复。

            一般来说,所有的处理器的设计都包含一个或一组被称为\emreg{程序状态字(Program Status Word, PSW)}的寄存器。
            这组寄存器包含了指令执行后的条件码外加一些状态信息。
        }

        \paraph{进程控制信息}
        {
            这一部分信息被操作系统用于控制和协调系统内的各种进程活动。
        }

        \paraph{进程控制块的角色}
        {
            每一个进程控制块都包含操作系统为了控制它所对应的进程所需要的所有重要信息。
            进程控制块为操作系统内几乎所有重要的模块所读取、修改,其中包括调度模块、资源分配模块中断处理模块,以及性能监控和分析模块等。
            进程控制块所构成的集合直接定义了操作系统在某个时间点上的状态。

            由于操作系统中的各个模块都有可能访问进程控制块,对进程控制块的访问保护问题就变得重要起来。
            这里有两个问题:

            \begin{enumerate}
                \item 如果在某个操作系统模块出现了bug,它破坏了某进程的进程控制块,这将可能导致系统在其后无法对被破坏了的进程控制块中的进程进行有效管理。
                \item 对于进程控制块本身的结构调整,将势必影响操作系统中的各个模块对进程控制块的访问。
            \end{enumerate}

            这些问题可以通过为所有进程访问控制块的模块定义统一入口的函数得以实现。
            在这些函数中,对所有访问进程控制块的操作加以保护,以防止数据被破坏。
        }
    }
}

    %%
%% Author: Clay
%% 2019/9/23
%%

\section{命题等价式}
{
    \subsection{引言}
    {
        \emreg{谓词逻辑}用来表达数学和计算机科学中各种语句的意义,并允许推理和探索对象之间的关系。
        \emreg{量词}可以对这样的语句进行推理:某一性质对于某一类型的所有对象均成立,存在一个对象使得某一特性成立。
    }

    \subsection{谓词}
    {
        语句``$x$ 大于3'' 有两个部分。
        第一部分即变量 $x$ 是语句的主语。
        第二部分(\emreg{谓词``大于3''})表明语句的主语具有的一个性质。

        语句 $P(x)$ 也可以说成是命题函数 $P$ 在 $x$ 的值。
        一旦给变量 $x$ 赋一个值,语句 $P(x)$ 就成为命题并具有真值。

        一般地,设计 $n$ 个变量 $x_1, x_2, \cdots, x_n$ 的语句可以表示成
        $$P(x_1, x_2, \cdots, x_n)$$
        形式为 $P(x_1, x_2, \cdots, x_n)$ 的语句是 \emreg{命题函数} $P$ 在 $n$ 元组 $(x_1, x_2, \cdots, x_n)$ 的值, $P$ 也称为\emreg{ $n$ 位谓词} 或 \emreg{ $n$ 元谓词}。 

        \paraph{前置和后置条件}
        {
            谓词还可以用来验证计算机程序,也就是证明当给定合法输入时计算机程序总是能产生所期望的输出。(除非建立了程序的正确性,否则无论测试了多少次都不能证明程序对所有输入都产生期望的输出,除非能测试到每个输入值。)
            描述合法输入的语句叫做\emreg{前置条件},而程序运行的输出应该满足的条件称为\emreg{后置条件}。
        }
    }

    \subsection{量词}
    {
        有一种称为\emreg{量化}的重要方式也可以从命题函数生成一个命题。
        量化表示在何种程度上谓词对于一个范围的个体成立。
        这里集中讨论两类量化:
        全称量化,它告诉我们一个谓词在所考虑的范围内对每一个个体都为真;
        存在量化,它告诉我们一个谓词在所考虑范围内的一个或多个个体为真。
        处理谓词和量词的逻辑领域称为\emreg{谓词演算}。

        \paraph{全称量词}
        {
            许多数学命题断言某一性质对于变量在某一特定域内的所有值均为真,这一特定域称为变量的\emreg{论域 (domain of discourse)}(或\emreg{全体域 (universe of discourse)}),时常简称为\emreg{域 (domain)}。
            使用全称量词时必须指定论域,否则语句的\emreg{全称量化}就是无定义的。

            \begin{defines}
                $P(x)$ 的\emspe{全称量化}是语句
                $$P(x)\text{对}x\text{在其论域的所有值为真。}$$
                符号 $\forall xP(x)$ 表示 $P(x)$ 的全称量化,其中 $\forall$ 称为\emspe{全称量词}。
                命题 $\forall xP(x)$ 读做``对所有 $x, P(x)$''或``对每个 $x, P(x)$''。
                一个使 $P(x)$ 为假的个体称为 $\forall xP(x)$ 的\emspe{反例}。
            \end{defines}

            \begin{table}[htb]
                \centering

                \begin{tabular}{c|c|c}
                    \hline
                    命题 & 什么时候为真 & 什么时候为假 \\
                    \hline
                    $\forall xP(x)$ & 对每一个 $x, P(x)$ 都为真 & 有一个 $x$ 使 $P(x)$ 为假 \\
                    $\exists xP(x)$ & 有一个 $x$ 使 $P(x)$ 为真 & 对每一个 $x, P(x)$ 都为假 \\
                    \hline
                \end{tabular}

                \caption{量词}
            \end{table}

            如果论域为空,那么 $\forall xP(x)$ 对任何命题函数 $P(x)$ 都为真。

            要证明当 $x$ 在论域中时 $P(x)$ 不总为真,方法之一就是寻找一个 $\forall xP(x)$ 的反例。
        }

        \paraph{存在量词}
        {
            许多数学定理断言:有一个个体使得某种性质成立。

            \begin{defines}
                $P(x)$ 的存在量化是命题
                $$\text{论域中存在一个个体}x\text{满足}P(x)\text{。}$$
                用符号 $\exists xP(x)$ 表示 $P(x)$ 的存在量化,其中 $\exists$ 称为存在量词。
            \end{defines}

            使用存在量词时必须指定论域,否则语句的\emreg{存在量化}就是无定义的。

            如果论域为空,那么 $\exists xP(x)$ 对任何命题函数 $P(x)$ 都为假。
        }

        \paraph{唯一性量词}
        {
            其他量词中最常见的是\emreg{唯一性量词},用符号 $\exists !$ 或 $\exists_1$ 表示。
        }
    }

    \subsection{约束论域的量词}
    {
        要限定一个量词的论域时,变量必须满足的条件直接放在量词的后面。

        全称量化的约束和一个条件语句的全称量化等价。
        存在量化的约束和一个合取式的存在量化等价。
    }

    \subsection{量词的优先级}
    {
        量词 $\forall$ 和 $\exsits$ 比命题演算中的所有逻辑运算符都具有更高的优先级。
    }

    \subsection{变量绑定}
    {
        当量词作用与变量 $x$ 的时候,此变量的这次出现为\emreg{约束的}。
        一个变量的出现被成为是\emreg{自由的},如果没有被量词约束或设置为某一特定值。
        命题函数中的所有变量出现必须是约束的或者被设置未等于某个特定值的,才能把它转变为一个命题。
    }

    \subsection{练习}
    {
        %1.
        \begin{practices}
            \begin{enumerate}[A.]
                \item $true$
                \item $true$
                \item $false$
            \end{enumerate}
        \end{practices}

        %2.
        \begin{practices}
            \begin{enumerate}[A.]
                \item $true$
                \item $false$
                \item $false$
                \item $true$
            \end{enumerate}
        \end{practices}

        %3.
        \begin{practices}
            \begin{enumerate}[A.]
                \item $true$
                \item $false$
                \item $false$
                \item $false$
            \end{enumerate}
        \end{practices}

        %4.
        \begin{practices}
            $x = 2$
        \end{practices}
    }
}

    %%
%% Author: Clay
%% 2021/1/14
%%

\section{信号}
{
    \emreg{Linux信号}允许进程和内核中断其它进程。

    一个\emreg{信号}就是一条小消息,它通知进程系统中发生了一个某种类型的事件。

    底层的硬件异常是由内核异常处理程序处理的,正常情况下,对用户进程而言是不可见的。
    信号提供了一种机制,通知用户进程发生了异常。

    \subsection{信号术语}
    {
        传送一个信号到目的进程是由两个不同步骤组成的:

        \begin{description}
            \item[发送信号]
            {
                内核通过更新目的进程上下文中的某个状态,\emspe{发送(递送)}一个信号给目的进程。
                发送信号可以有如下两种原因:
                内核检测到一个系统事件。
                一个进程调用了kill函数,显示地要求内核发送一个信号给目的进程。
                一个进程可以发送信号给它自己。
            }
            \item[接收信号]
            {
                当目的进程被内核强迫以某种方式对信号的发送做出反应时,它就\emspe{接收}了信号。
                进程可以忽略这个信号,终止或者通过执行一个称为\emspe{信号处理程序(signal handler)}的用户层函数\emspe{捕获}这个信号。
            }
        \end{description}

        一个发出而没有被接受的信号叫做\emreg{待处理信号(pending signal)}。
        在任何时刻,一种类型至多只会有一个待处理信号。
        如果一个进程有一个类型为k的待处理信号,那么接下来发送到这个进程的类型为k的信号都\emreg{不会}排队等待,它们只是被简单地丢弃。
        一个进程可以有选择地\emreg{阻塞某种信号}。
        当一种信号被阻塞时,它仍可以被发送,但是产生的待处理信号不会被接收,直到进程取消对这种信号的阻塞。

        一个待处理信号最多只能被接受一次。
    }

    \subsection{发送信号}
    {
        Unix系统提供了大量向进程发送信号的机制。
        所有这些机制都是基于\emreg{进程组(process group)}这个概念的。

        \subsubsection{进程组}
        {
            每个进程都只属于一个\emreg{进程组},进程组是由一个正整数\emreg{进程组ID}来标识的。

            默认地,一个子进程和它的父进程属于一个进程组。
        }

        \subsubsection{用/bin/kill程序发送信号}
        {
            /bin/kill程序可以向另外的进程发送任意的信号。

            一个为负的pid会导致信号被发送到进程组PID中的每个进程。
        }

        \subsubsection{从键盘发送信号}
        {
            Unix shell使用\emreg{作业(job)}这个抽象概念来表示为对一条命令行求值而创建的进程。
            在任何时刻,至多只有一个前台作业和0个或多个后台作业。
        }

        \subsubsection{用kill函数发送信号}
        {
            进程通过调用kill函数发送给信号给其他进程。
        }

        \subsubsection{用alarm函数发送信号}
        {
            进程可以通过调用alarm函数向它自己发送SIGALRM信号。
        }
    }

    \subsection{接收信号}
    {
        当内核把进程p从内核模式切换到用户模式时,它会检查进程p的未被阻塞的待处理信号的集合。
        如果这个集合是非空的,那么内核选择集合中的某个信号k,并强制p\emreg{接收}信号k。
        每个信号类型都有一个预定义的\emreg{默认行为},是下面的一种:

        \begin{itemize}
            \item 进程终止。
            \item 进程终止并转储内存。
            \item 进程停止直到被SIGCONT信号重启。
            \item 进程忽略该信号。
        \end{itemize}

        进程可以通过signal函数修改和信号相关联的默认行为。
        唯一的例外是SIGSTOP和SIGKILL,它们的默认行为是不能修改的。

        signal函数可以通过下列三种方法之一来改变和信号signum相关联的行为。

        \begin{itemize}
            \item 如果handler是SIG_IGN,那么忽略信号。
            \item 如果handler是SIG_DEL,那么恢复默认行为。
            \item
            {
                否则,handler就是用户定义的函数的地址,这个函数被称为\emreg{信号处理程序},只要进程接收到一个类型为signum的信号,就会调用这个程序。
                通过把处理程序的地址传递到signal函数从而改变默认行为,这叫做\emreg{设置信号处理程序(installing the handler)}。
                调用信号处理程序称为\emreg{捕获信号}。
                执行信号处理程序称为\emreg{处理信号}。
            }
        \end{itemize}

        当一个进程捕获了一个类型为k的信号时,会调用为信号k设置的处理程序,一个整数参数被设置为k,这个参数允许同一个处理函数捕获不同类型的信号。

        当处理程序执行它的return语句时,控制通常传递回控制流中进程信号接收中断位置处的指令。

        信号处理程序可以被其他信号处理程序中断。
    }

    \subsection{阻塞和解除阻塞信号}
    {
        Linux提供阻塞信号的隐式和显式的机制:

        \begin{description}
            \item[隐式阻塞机制] 内核默认阻塞任何当前处理程序正在处理信号类型的待处理的信号。
            \item[显示阻塞机制] 应用程序可以使用sigprocmask函数和它的辅助函数,明确地阻塞和解除阻塞选定的信号。
        \end{description}
    }

    \subsection{编写信号处理程序}
    {
        处理程序有几个属性使得它们很难推理分析:
        处理程序与主程序并发运行,共享同样的全局变量,因此可能与主程序和其他的处理程序互相干扰。
        如何以及何时接收信号的规则常常有违人的直觉。
        不同的系统有不同的信号处理语义。

        \subsubsection{安全的信号处理}
        {
            一些保守的编写处理程序的原则,使得这些处理程序能安全地并发运行。

            \begin{description}
                \item[处理程序要尽可能简单] 避免麻烦的最好方法是保持处理程序尽可能小和简单。
                \item[在处理程序中只调用异步信号安全的函数]
                {
                    所谓\emspe{异步信号安全}的函数能够被信号处理程序安全地调用,原因有二:
                    要么它是\emreg{可重入的},要么它不能被信号处理程序中断。

                    信号处理程序中产生输出唯一安全的方法是使用write函数。
                }
                \item[保存和恢复errno] 只有在处理程序要返回时才有必要。
                \item[阻塞所有的信号,保护对共享全局数据结构的访问]
                \item[用volatile声明全局变量] 用volatile类型限定符来定义一个变量,告诉编译器不要缓存这个变量。
                \item[用sig_atomic_t声明标志] C提供一种整型数据类型,对它的读和写保证会是原子的。
            \end{description}
        }

        \subsubsection{正确的信号处理}
        {
            信号的一个与直觉不服的方面是未处理的信号是不排队的,因为pending位向量中每种类型的信号只对应有一位,所以每种类型最多只能有一个未处理的信号。
            如果存在一个未处理的信号就表明\emreg{至少}有一个信号到达了。

            父进程必须回收子进程以避免在系统中留下僵死进程。

            \emreg{不可以用信号来对其他进程中发生的事件计数}。

            %8.
            \begin{practicec}
                213
            \end{practicec}
        }

        \subsubsection{可移植的信号处理}
        {

        }
    }
}

    %%
%% Author: Clay
%% 2020/4/8
%%

\section{控制}
{
    机器代码提供两种基本的低级机制来实现有条件的行为:
    测试数据值,然后根据测试的结果来改变控制流或者数据流。

    与数据相关的控制流是实现有条件行为的更一般和更常见的方法。
    用\emcode{jump}指令可以改变一组机器代码指令的执行顺序,\emcode{jump}指令指定控制应该被传递到程序的某个其他部分,可能是依赖于某个测试的结果。

    \subsection{条件码}
    {
        除了整数寄存器,CPU还维护着一组单个位的\emreg{条件码(condition code)}寄存器,它们描述了最近的算术或逻辑操作的属性。
        可以检测这些寄存器来执行条件分支指令。
        最常用的条件码有:

        \begin{description}
            \item[CF:进位标志] 最近的操作使最高位产生了进位。
            \item[ZF:零标志] 最近的操作得出的结果为0。
            \item[SF:符号标志] 最近的操作得到的结果为负数。
            \item[OF:溢出标志] 最近的操作导致一个补码溢出。
        \end{description}

        \emcode{leaq}不改变任何条件码。

        对于逻辑操作,进位标志和溢出标志会设置为0。
        对于移位操作,进位标志将设置为最后一个被移出的位,而溢出标志设置为0。
        \emcode{inc}和\emcode{dec}指令会设置溢出和零标志,但是不会改变进位标志。

        还有两类指令只设置条件码而不改变任何其他寄存器。

        \emcode{cmp}指令与\emcode{sub}指令的行为是一样的,除了只设置条件码而不更新目的寄存器。
        \emcode{test}指令与\emcode{and}指令的行为是一样的,除了只设置条件码而不更新目的寄存器。
    }

    \subsection{访问条件码}
    {
        条件码通常不会直接读取,常用的使用方法有三种:

        \begin{enumerate}
            \item 可以根据条件码的某种组合,将一个字节设置为0或者1
            \item 可以条件跳转到程序的某个其他的部分
            \item 可以有条件地传送数据
        \end{enumerate}

        SET类指令之间的区别就在于它们考虑的条件码的组合是什么,这些指令名字的不同后缀指明了它们所考虑的条件码的组合而不是操作数大小。
        一条SET指令的目的操作数是低位单字节寄存器元素之一,或是一个字节的内存位置,指令会将这个字节设置成0或者1。
        为了得到一个32位或64位结果,必须将高位清零。

        各个SET命令的描述都适用的情况是:执行比较指令,根据计算 $t = a -^t_w b$ 设置条件码。

        %13.
        \begin{practicec}
            \begin{enumerate}[A.]
                \item \emcode{int32\_t, <}
                \item \emcode{int16\_t, >=}
                \item \emcode{uint8\_t, <=}
                \item \emcode{int64\_t, !=}
            \end{enumerate}
        \end{practicec}

        %14.
        \begin{practicec}
            \begin{enumerate}[A.]
                \item \emcode{int64\_t, >=}
                \item \emcode{int16\_t, ==}
                \item \emcode{uint8\_t, >}
                \item \emcode{int32\_t, !=}
            \end{enumerate}
        \end{practicec}
    }

    \subsection{跳转指令}
    {
        正常执行的情况下,指令按照它们出现的顺序一条一条地执行。
        \emreg{跳转(jump)}指令会导致执行切换到程序中一个全新的位置。
        在汇编代码中,这些跳转的目的地通常用一个\emreg{标号(label)}指明。

        jmp指令是无条件跳转。
        它可以是\emreg{直接}跳转,即跳转目标是作为指令的一部分编码的;
        也可以是\emreg{间接}跳转,即跳转目标是从寄存器或内存位置中读出的。

        其他跳转指令都是\emreg{有条件}的。
        这些指令的名字和跳转条件与SET指令的名字和设置条件是相匹配的。
    }

    \subsection{跳转指令的编码}
    {
        在汇编代码中,跳转目标用符号标号书写。
        汇编器,以及后来的连接器,会产生跳转目标的适当编码。
        跳转指令有几种不同的编码,但是最常用都是\emreg{PC相对的(PC-relative)}。
        第二种编码方法是给出绝对地址。

        当执行PC相对寻址时,程序计数器的值是跳转指令后面的那条指令的地址。

        %15.
        \begin{practicec}
            \begin{enumerate}[A.]
                \item $4003fe$
                \item $400425$
                \item $400543, 400545$
                \item $8d$
            \end{enumerate}
        \end{practicec}

        跳转指令提供了一种实现条件执行和几种不同循环结构的方式。

        C语言中的\emcode{if--else}语句的通用形式模板如下:

        \begin{lstlisting}[language=C]
if (test-expr)
    then-statement
else
    else-statement
        \end{lstlisting}

        对于这种通用形式,汇编实现通常会使用下面这种形式,这里,用C语法来描述控制流:

        \begin{lstlisting}
    t = test-expr;

    if (!t)
        goto false;
    then-statement;
    goto done;
false:
    else-statement;
done:
        \end{lstlisting}

        也就是,汇编器为\emcode{then--statement}和\emcode{else--statement}产生各自的代码块,它会插入条件和无条件分支,以保证能执行正确的代码块。

        %16.
        \begin{practicec}
            \begin{enumerate}[A.]
                \item
                {
                    \begin{lstlisting}[language=C]
void cond(long int a, long int *p)
{
    if (!p || a <= *p)
        goto done;

    *p = a;
done:
}
                    \end{lstlisting}
                }
                \item
                {
                    \emcode{\&\&}被拆为两个\emcode{if}。
                }
            \end{enumerate}
        \end{practicec}
    }
}

}

\cleardoublepage

\endinput
