%%
%% Author: Clay
%% 2021/2/18
%%

\section{虚拟内存作为内存管理的工具}
{
    操作系统为每个进程提供了一个独立的页表,因而也就是一个独立的虚拟地址空间。
    多个虚拟页面可以映射到一个共享物理页面上。

    按需页面调度和独立的虚拟地址空间的结合,对系统中内存的使用和管理造成了深远的影响。
    特别地,VM简化了链接和加载、代码和数据共享,以及应用程序的内存分配。

    \begin{description}
        \item[简化链接] 独立的地址空间允许每个进程的内存映像使用相同的基本格式,而不管代码和数据实际存放在物理内存的何处。
        \item[简化加载] 加载器从不从磁盘到内存实际复制任何数据。
        \item[简化共享] 操作系统通过将不同进程中适当的虚拟页面映射到相同的物理页面,从而安排多个进程共享这部分代码的一个副本。
        \item[简化内存分配]
        {
            当一个运行在用户进程中的程序要求额外的堆空间时,操作系统分配一个适当数字个连续的虚拟内存页面,并且将它们映射到物理内存中任意位置的的物理页面。
            由于页表的工作方式,页面可以随机地分散在物理内存中。
        }
    \end{description}

    将一组连续的虚拟页映射到一个文件中的任意位置的表示法称作\emreg{内存映射(memory mapping)}。
}
