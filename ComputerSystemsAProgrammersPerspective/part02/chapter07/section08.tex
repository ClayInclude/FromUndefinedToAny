%%
%% Author: Clay
%% 2020/12/22
%%

\section{可执行目标文件}
{
    可执行目标文件的格式类似于可重定位目标文件的格式。
    ELF头描述文件的总体格式。
    它还包括程序的\emreg{入口点(entry point)},也就是当程序运行时要执行的第一条指令的地址。
    .text、.rodata和.data节与可重定位目标文件中的节是相似的,除了这些节已经被重定位到他们最终的运行时内存地址以外。
    .init节定义了一个小函数,叫做_init,程序的初始化代码会调用它。
    因为可执行文件是\emreg{完全链接的}(已被重定位),所以它不再需要.rel节。

    ELF可执行文件被设计得很容易加载到内存,可执行文件的连续的\emreg{片(chunk)}被映射到连续的内存段。
    \emreg{程序头部表(program header table)}描述了这种映射关系。

    对于任何段s,连接器必须选择一个起始地址vaddr,使得

    \begin{align*}
        vaddr mod alilgn = off mod align
    \end{align*}

    off是目标文件中段的第一个节的偏移量,align是程序头部中指定的对齐。

    这个对其要求是一种优化,使得当程序执行时,目标文件中的段能够很有效率地传送到内存中。
}
