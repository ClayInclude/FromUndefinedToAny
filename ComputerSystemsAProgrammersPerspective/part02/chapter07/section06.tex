%%
%% Author: Clay
%% 2020/12/21
%%

\section{符号解析}
{
    链接器解析符号引用的方法是将每个引用与它输入的可重定位目标文件的符号表中的一个确定的符号定义关联起来。

    \subsection{链接器如何解析多重定义的全局符号}
    {
        在编译时,编译器向汇编器输出每个全局符号,或者是\emreg{强(strong)}或者是\emreg{弱(weak)},二汇编器把这个信息隐含底编码在可重定位目标文件的符号表里。
        函数和已初始化的全局变量是强符号,未初始化的全局变量是弱符号。

        根据强弱符号的定义,Linux链接器使用下面的规则来处理多重定义的符号名:

        \begin{itemize}
            \item 不允许有多个同名的强符号。
            \item 如果有一个强符号和多个弱符号同名,那么选择强符号。
            \item 如果有多个弱符号同名,那么从这些弱符号中任意选择一个。
        \end{itemize}

        当编译器在翻译某个模块时,遇到一个弱全局符号,它并不知道其他模块是否也定义了,无法预测链接器应该使用多重定义中的哪一个。
        所以编译器分配到了COMMON,把决定权留给链接器。

        %2.
        \begin{practicec}
            \begin{enumerate}[A.]
                \item main.1; main.1
                \item 错误
                \item 未知
            \end{enumerate}
        \end{practicec}
    }

    \subsection{与静态库链接}
    {
        所有的编译系统都提供一种机制,将所有相关的目标模块打包成为一个单独的文件,称为\emreg{静态库(static library)},它可以用作链接器的输入。
        当链接器构造一个输出的可执行文件时,它只复制静态库里被应用程序引用的目标模块。
    }
}
