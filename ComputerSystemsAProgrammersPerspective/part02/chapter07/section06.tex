%%
%% Author: Clay
%% 2020/12/21
%%

\section{符号解析}
{
    链接器解析符号引用的方法是将每个引用与它输入的可重定位目标文件的符号表中的一个确定的符号定义关联起来。

    \subsection{链接器如何解析多重定义的全局符号}
    {
        在编译时,编译器向汇编器输出每个全局符号,或者是\emreg{强(strong)}或者是\emreg{弱(weak)},二汇编器把这个信息隐含底编码在可重定位目标文件的符号表里。
        函数和已初始化的全局变量是强符号,未初始化的全局变量是弱符号。

        根据强弱符号的定义,Linux链接器使用下面的规则来处理多重定义的符号名:

        \begin{itemize}
            \item 不允许有多个同名的强符号。
            \item 如果有一个强符号和多个弱符号同名,那么选择强符号。
            \item 如果有多个弱符号同名,那么从这些弱符号中任意选择一个。
        \end{itemize}

        当编译器在翻译某个模块时,遇到一个弱全局符号,它并不知道其他模块是否也定义了,无法预测链接器应该使用多重定义中的哪一个。
        所以编译器分配到了COMMON,把决定权留给链接器。

        %2.
        \begin{practicec}
            \begin{enumerate}[A.]
                \item main.1; main.1
                \item 错误
                \item x.2;x.2
            \end{enumerate}
        \end{practicec}
    }

    \subsection{与静态库链接}
    {
        所有的编译系统都提供一种机制,将所有相关的目标模块打包成为一个单独的文件,称为\emreg{静态库(static library)},它可以用作链接器的输入。
        当链接器构造一个输出的可执行文件时,它只复制静态库里被应用程序引用的目标模块。

        在Linux系统中,静态库以一种称为\emreg{存档(archive)}的特殊文件格式存放在磁盘中。
        存档文件是一组连接起来的可重定位目标文件的集合,有一个头部用来描述每个成员目标文件的大小和位置。
    }

    \subsection{链接器如何使用静态库来解析引用}
    {
        在符号解析阶段,链接器从左到右按照它们在编译器驱动程序命令行上出现的顺序来扫描可重定位目标文件和存档文件。
        在这次扫描中,连接器维护一个可重定位目标文件的集合E,一个未解析的符号集合U,以及一个在前面输入文件中已定义的符号集合D。
        初始时,E、U和D均为空。

        \begin{itemize}
            \item 如果f是一个目标文件,那么链接器把f添加到E,修改U和D来反映f中的符号定义和引用,并继续下一个输入文件。
            \item
            {
                如果f是一个存档文件,那么链接器就尝试匹配U中未解析的符号和由存档文件成员定义的符号。
                如果某个存档文件成员定义了一个符号来解析U中的一个引用,那么就加到E中,并且链接器修改U和D来反映m中的符号定义和引用。
                对存档中所有的成员目标文件都依次进行这个过程,直到U和D都不再发生变化。
                此时,任何不包含在E中的成员目标都被简单地丢弃,而链接器继续处理下一个输入文件。
            }
            \item
            {
                如果当链接器完成对命令行上输入文件的扫描后,U是非空的,那么链接器就会输出一个错误并终止。
                否则,它会合并和重定位E中的目标文件,构建输出的可执行文件。
            }
        \end{itemize}

        如果需要满足依赖需求,可以在命令行上重复库。

        %3.
        \begin{practicec}
            \begin{enumerate}[A.]
                \item \emcode{p.o libx.a}
                \item \emcode{p.o libx.a liby.a}
                \item \emcode{p.o libx.a liby.a libx.a p.o}
            \end{enumerate}
        \end{practicec}
    }
}
