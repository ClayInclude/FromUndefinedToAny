%%
%% Author: Clay
%% 2020/12/15
%%

\section{静态链接}
{
    \emreg{静态链接器(static linker)}以一组可重定位目标文件和命令行参数作为输入,生成一个完全链接的、可以加载和运行的可执行目标文件作为输出。
    输入的可重定位目标文件由各种不同的代码和数据\emreg{节(section)}组成,每一节都是一个连续的字节序列。

    为了构造可执行文件,链接器必须完成两个任务:

    \begin{description}
        \item[符号解析(symbol resolution)]
        {
            目标文件定义和引用\emspe{符号},每个符号对应一个函数、一个全局变量或一个\emspe{静态变量}(即C语言中任何以\emcode{static}属性声明的变量)。
            符号解析的目的是将每个符号\emspe{引用}正好和一个符号\emspe{定义}关联起来。
        }
        \item[重定位(relocation)]
        {
            编译器和汇编器生成从地址0开始的代码和数据节。
            链接器通过把每个符号定义与一个内存位置关联起来,从而\emspe{重定位}这些节,然后修改所有对这些符号的引用,使得它们指向这个内存位置。
            链接器使用汇编器产生的\emspe{重定位条目(relocation entry)}的详细指令。
        }
    \end{description}

    目标文件纯粹是字节块的集合。
    这些块中,有些包含程序代码,有些包含程序数据,而其他的则包含引导链接器和加载器的数据结构。
    链接器将这些块连接起来,确定被连接块的运行时位置,并且修改代码和数据块中的各种位置。
    链接器对目标机器了解甚少,产生目标文件的编译器和汇编器已经完成了大部分工作。
}
