%%
%% Author: Clay
%% 2020/12/21
%%

\section{符号和符号表}
{
    在链接器的上下文中,有三种不同的符号:

    \begin{itemize}
        \item 由模块m定义并能够被其他模块引用的\emspe{全局符号}。
        \item 由其它模块定义并被模块m引用的\emspe{全局符号},这些符号称为\emspe{外部符号}。
        \item 只被模块m定义和引用的局部符号。
    \end{itemize}

    符号表是由汇编器够造的,使用编译器输出到汇编语言.s文件中的符号。

    name是字符串表中的字节偏移,指向符号的以null结尾的字符串名字。
    value是符号的地址。
    size是目标的大小。
    type通常要么是数据,要么是函数。
    符号表还可以包含各个节的条目,以及对应原始源文件的路径名的条目。
    binding字段表示符号是本地的还是全局的。

    每个符号都被分配到目标文件的某个节,由section字段表示。
    有三个特殊的\emreg{伪节(pseudosection)},它们在节头部表中是没有条目的:
    ABS代表不该被重定位的符号;
    UNDEF代表未定义的符号;
    COMMON表示还未被分配的未初始化的数据目标。

    COMMON和.bss的区别很细微,
    COMMON:维持化的全局变量,.bss:未初始化的静态变量,以及初始化为0的全局或静态变量。

    %1.
    \begin{practicec}
        \begin{table}{htb}
            \centering

            \begin{tabular}{|c|c|c|c|c|}
                \hline
                符号 & .symtab条目? & 符号类型 & 在哪个模块中定义 & 节 \\
                \hline
                buf & Y & 外部 & m.o & .data \\
                \hline
                bufp0 & Y & 全局 & swap.o & .data \\
                \hline
                bufp1 & Y & 全局 & swap.o & COMMON \\
                \hline
                swap & Y & 全局 & swap.o & .text \\
                \hline
                temp & N & - & - & - \\
                \hline
            \end{tabular}
        \end{table}
    \end{practicec}
}
