%%
%% Author: Clay
%% 2020/12/17
%%

\section{可重定位目标文件}
{
    \emreg{ELF头(ELF header)}以一个16字节的序列开始,这个序列描述了生成该文件的系统的字的大小和字节顺序。
    ELF头剩下的部分包含帮助连接器语法分析和解释目标文件的信息。
    其中包括ELF头的大小、目标文件的类型、机器类型、\emreg{节头部表(section header table)}的文件偏移,以及节头部表中条目的大小和数量。
    不同节的位置和大小是由节头部表描述的,其中目标文件每个节都有一个固定大小的\emreg{条目(entry)}。

    一个典型的ELF可重定位目标文件包含下面几个节:

    \begin{description}
        \item[.text] 已编译程序的机器代码。
        \item[.rodata] 只读数据,比如字符串和开关语句的跳转表。
        \item[.data] 已初始化的全局和静态C变量。
        \item[.bss] 未初始化的全局和静态C变量。
    \end{description}
}
