%%
%% Author: Clay
%% 2020/12/22
%%

\section{加载可执行目标文件}
{
    shell通过调用某个驻留在存储器中称为\emreg{加载器(loader)}的操作系统代码来运行它。
    加载器将可执行目标文件中的代码和数据从磁盘复制到内存中,然后通过跳转到程序的第一条指令或\emreg{入口点}来运行该程序。
    这个将程序复制到内存并运行的过程叫做\emreg{加载}。

    每个Linux程序都有一个运行时内存映像。
    代码段总是从地址0x400000开始,后面是数据段。
    运行时\emreg{堆}在数据段之后,
    通过调用malloc往上增长。
    堆后面的区域是为共享模块保留的。
    用户栈总是从最大的合法用户地址( $2^48 - 1$ )开始,向叫嚣内存地址增长。
    栈上的区域,从地址 $2^48$ 开始,是为\emreg{内核(kernel)}中的代码和数据保留的,所谓内核就是操作系统驻留在内存的部分。
}
