%%
%% Author: Clay
%% 2020/12/22
%%

\section{库打桩机制}
{
    Linux链接器支持一个很强大的技术,称为\emreg{库打桩(library interpositioning)},它允许截取对共享库函数的调用,取而代之执行自己的代码。

    给定一个需要打桩的\emreg{目标函数},创建一个\emreg{包装函数},它的原型与目标函数完全一样。
    使用某种特殊的打桩机制,你就可以欺骗系统调用包装函数胡而不是目标函数了。

    \subsection{编译时打桩}
    {

    }

    \subsection{链接时打桩}
    {

    }

    \subsection{运行时打桩}
    {
        编译时打桩需要能够访问程序的源代码,链接时打桩需要能够访问程序的可重定位对象文件。
        有一种机制能在运行时打桩,只需要能够访问可执行目标文件。
    }
}
