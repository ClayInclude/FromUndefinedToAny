%%
%% Author: Clay
%% 2020/2/20
%%

\chapter{异常控制流}
{
    从给处理器加电开始,直到断电为止,程序计数器假设一个值的序列

    \begin{align*}
        a_0, a_1, \cdots, a_{n -1}
    \end{align*}

    其中,每个 $a_k$ 是某个相应的指令 $I_k$ 的地址。
    每次从 $a_k$ 到 $a_{k + 1}$ 的过渡称为\emreg{控制转移(control transfer)}。
    这样的控制转移序列叫做处理器的\emreg{控制流(flow of control 或 control flow)}。

    最简单的控制流是一个平滑的序列,其中每个 $\I_k$ 和 $\I_{k + 1}$ 在内存中都是相邻的。
    这种平滑流的突变通常是由诸如跳转、调用和返回这样一些熟悉的程序指令造成的。
    这样一些指令都是必要的机制,使得程序能够对由程序变量表示的内部程序状态中的变化作出反应。

    但是系统也必须能够对系统状态的变化做出反应。
    这些系统状态不是被内部程序变量捕获的,而且也不一定要和程序的执行相关。

    现代系统通过使控制流发生突变来对这些情况做出反应。
    这些突变称为\emreg{异常控制流(Exceptional Control Flow, ECF)}。
    异常控制流发生在计算机系统的各个层次。

    理解ECF很重要,有很多原因:

    \begin{description}
        \item[理解ECF将帮助你理解重要的系统概念] ECF是操作系统用来实现I/O、进程和虚拟内存的基本机制。
        \item[理解ECF将帮助你理解应用程序是如何与操作系统交互的] 应用程序通过使用一个叫做\emspe{陷阱(trap)}或者\emspe{系统调用(system call)}的ECF形式,向操作系统请求服务。
        \item[理解ECF将帮助你编写有趣的新应用程序] 操作系统为应用程序提供饿了强大的ECF机制,用来创建新进程、等待进程终止、通知其他进程系统中的异常事件,以及检测和响应这些事件。
        \item[理解ECF将帮助你理解并发] ECF是计算机系统中实现并发的基本机制。
        \item[理解ECF将帮助你理解软件异常如何工作] 软件异常允许程序进行\emspe{非本地} 跳转来响应错误情况。
    \end{description}

    %%
%% Author: Clay
%% 2020/12/5
%%

\section{死锁的原理}
{
    死锁是指一组进程因为竞争系统资源或互相等待消息,而永远无法向前推进的状态。

    \subsection{可重用资源}
    {
        资源可以分为两个大类:
        可重用资源与消耗性资源。

        可重用资源指的是一次只供一个进程使用,但用后又能被别的进程使用的资源。

        从系统的角度出发,能够解决死锁问题的方法之一是对应用申请系统资源的顺序做出限定。
    }
}

}

\cleardoublepage

\endinput
