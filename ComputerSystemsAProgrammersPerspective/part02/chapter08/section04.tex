%%
%% Author: Clay
%% 2020/12/25
%%

\section{进程控制}
{
    \subsection{获取进程ID}
    {
        每个进程都有一个唯一的正数\emreg{进程ID(PID)}。
        \emcode{getpid}函数返回调用进程的PID。
        \emcode{getppid}函数返回它的父进程的PID。
    }

    \subsection{创建和终止进程}
    {
        进程总是处于下面三种状态之一:

        \begin{description}
            \item[运行] 进程要么在CPU上执行,要么在等待执行且最终会被内核调度。
            \item[停止] 进程的执行被\emspe{挂起(suspended)},且不会被调度。
            \item[终止] 进程永远的停止了。
        \end{description}

        \emcode{exit}函数以\emcode{status}\emreg{退出状态}来终止进程。

        \emreg{父进程}通过调用\emcode{fork}函数创建一个新的运行的\emreg{子进程}。

        子进程得到与父进程用户级虚拟地址空间相同的一份副本,包括代码的数据段、堆、共享库以及用户栈。
        子进程还获得与父进程任何打开文件描述符相同的副本。

        fork函数只被调用\emreg{一次},却会返回\emreg{两次}。
        在子进程中,fork返回0。

        \begin{description}
            \item[调用一次,返回两次]
            \item[并发执行] 不能对不同进程中指令的交替做任何假设。
            \item[相同但是独立的地址空间]
            \item[共享文件]
        \end{description}

        对于运行在单处理器上的程序,对应进程图中所有顶点的\emreg{拓扑排序(topological sort)}表示程序中语句的一个可行的全序排列。

        %2.
        \begin{practicec}
            \begin{enumerate}[A.]
                \item 2;1
                \item 0
            \end{enumerate}
        \end{practicec}
    }

    \subsection{回收子进程}
    {
        当一个进程由于某种原因终止时,内核并不是立即把它从系统中清除。
        相反,进程被保持在一种已终止的状态中,直到被它的父进程\emreg{回收(reaped)}。
        当父进程回收已终止的子进程时,内核将子进程的退出状态传递给父进程,然后抛弃已终止的进程,从此时开始该进程就不存在了。
        一个终止了但还未被回收的进程称为\emreg{僵死进程(zombie)}。

        如果一个父进程终止了,内核会安排init进程成为它的孤儿进程的养父。
        init进程的PID为1,是在系统启动时由内核创建的,它不会终止,是所有进程的祖先。
        如果父进程没有回收它的僵死进程就终止了,那么内核会安排init进程去回收他们。

        一个进程可以通过调用\emcode{waitpid}函数来等待它的子进程终止或者停止。

        默认情况下(当options=0)时,waitpid挂起调用进程的执行,直到它的\emreg{等待集合(wait set)}中的一个子进程终止。
        如果等待集合中的一个进程在刚调用的时刻就已经终止了,那么waitpid就立即返回。
        在这两种情况中,waitpid返回导致waitpid返回的已终止子进程的pid。

        \subsubsection{判定等待集合的成员}
        {
            \begin{itemize}
                \item 如果pid>0,那么等待集合就是一个单独的子进程
                \item 如果pid=-1,那么等待集合就是由父进程所有的子进程组成的
            \end{itemize}
        }

        \subsubsection{修改默认行为}
        {
            \begin{description}
                \item[WNOHANG] 如果等待集合中的任何子进程都还没有终止,那么就立即返回。
                \item[WUNTRACED] 挂起调用进程的执行,知道等待集合中的一个进程变成已终止或者被停止。
                \item[WCONTINUED] 挂起调用进程的执行,知道等待集合中一个正在运行的进程终止或等待集合中一个被停止的进程收到SIGCONT信号重新开始执行。
            \end{description}
        }

        \subsubsection{检查已回收子进程的退出状态}
        {
            \begin{description}
                \item[WIFEXITED] 如果子进程通过调用exit或者一个返回正常终止,就返回真。
                \item[WEXITSTATUS] 返回一个正常终止的子进程的退出状态。
                \item[WIFSIGNALED] 如果子进程是因为一个未被捕获的信号终止的,那么就返回真。
                \item[WTERMSIG] 返回导致子进程终止的信号的编号。
                \item[WIFSTOPPED] 如果引起返回的子进程当前是停止的,那么就返回真。
                \item[WSTOPSIG]  返回引起子进程停止的信号的编号。
                \item[WIFCONTINUED] 如果子进程收到SIGCONT信号重新启动,则返回真。
            \end{description}
        }

        \subsubsection{错误条件}
        {
            如果调用进程没有子进程,那么waitpid返回-1,并设置errno为ECHILD。
            如果waitpid函数被一个信号中断,那么它返回-1,并设置errno为EINTR。

            %3.
            \begin{practicec}
                \begin{enumerate}[A.]
                    \item bacc
                    \item abcc
                    \item acbc
                \end{enumerate}
            \end{practicec}
        }

        \subsubsection{wait函数}
        {
            调用wait等价于wait(\-1, \&status, 0)。
        }

        \subsubsection{使用waitpid的示例}
        {
            当回收了所有的子进程后,再调用waitpid就返回-1,并且设置errno为ECHILD。

            子进程回收的顺序是这台特定的计算机系统的属性。
            这是\emreg{非确定性}行为的一个示例,这种非确定性行为使得对并发进行推理非常困难。
            唯一正确的假设是每一个可能的结果都同样可能出现。

            %4.
            \begin{practicec}
                \begin{enumerate}[A.]
                    \item 6
                    \item
                    {
                        Hello;
                        0;
                        1;
                        Bye;
                        0;
                        Bye;
                    }
                \end{enumerate}
            \end{practicec}
        }
    }

    \subsection{让进程休眠}
    {
        如果请求的时间时间量已经到了,sleep返回零,否则返回还剩下的要休眠的秒数。
        后一种情况是可能的,如果因为sleep函数被一个\emreg{信号}中断而过早地返回。

        pause函数让调用函数休眠,直到该进程收到一个信号。

        %5.
        \begin{practicec}

        \end{practicec}
    }

    \subsection{加载并运行程序}
    {
        execve函数在当前进程的上下文中加载并运行一个新程序。

        只有当出现错误时,execve才会返回到调用程序。
        execve调用一次并从不返回。

        argv指向一个以null结尾的指针数组,其中每个指针都指向一个参数字符串。
        envp变量指向一个以null结尾的指针数组,其中每个指针指向一个环境变量字符串,每个串都是形如name=value的名字-值对。

        %6.
        \begin{practicec}

        \end{practicec}
    }

    \subsection{利用fork和execve运行程序}
    {
        shell执行一系列的\emreg{读/求值(read/evaluate)}步骤,然后终止。
        读步骤读取来自用户的一个命令行,求值步骤解析命令行,并代表用户运行程序。

        对命令行求值的首要任务是调用parseline函数。
        这个函数解析了以空格分隔的命令行参数,并够造最终会传递给execve的argv向量。
        第一个参数被假设为要么是一个内置的shell命令名,马上就会解释这个命令,要么是一个可执行目标文件,会在一个新的子进程的上下文中加载并运行这个文件。

        如果最后一个参数是一个\&字符,那么标识应该在\emreg{后台}执行该程序(shell不会等待它完成)。

        这个简单的shell是有缺陷的,因为它并不回收它的后台子进程。
        修改这个缺陷要求使用信号。
    }
}
