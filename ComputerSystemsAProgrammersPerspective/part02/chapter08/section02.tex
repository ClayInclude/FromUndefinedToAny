%%
%% Author: Clay
%% 2020/12/23
%%

\section{进程}
{
    异常是允许操作系统内核提供\emreg{进程(process)}概念的基本够造块,进程是计算机科学中最深刻、最成功的概念之一。

    进程的经典定义就是一个\emreg{执行中程序的实例}。
    系统中的每个程序都运行在某个进程的\emreg{上下文(context)}中。
    上下文是由程序正确运行所需的状态组成的。
    这个状态包括存放在内存中的程序的代码和数据,它的栈、通用目的寄存器的内容、程序计数器、环境变量以及打开文件描述符的集合。

    进程提供给应用程序的关键抽象:

    \begin{itemize}
        \item 一个独立的逻辑控制流
        \item 一个私有的地址空间
    \end{itemize}

    \subsection{逻辑控制流}
    {
        如果想用调试器单步执行程序,会看到一系列的程序计数器的值,这些唯一地对应于包含在程序的可执行目标文件中的指令,是包含在运行时动态链接到程序的共享对象中的指令。
        这个PC值的序列叫做\emreg{逻辑控制流},简称\emreg{逻辑流}。

        进程是轮流使用处理器的。
        每个进程执行它的流的一部分,然后被\emreg{抢占(preempted)}(暂时挂起),然后轮到其他进程。
    }

    \subsection{并发流}
    {
        一个逻辑流的执行在时间上与另一个流重叠,称为\emreg{并发流(concurrent flow)},这两个流被称为\emreg{并发地运行}。
        更准确地说,流X和Y互相并发,当且仅当X在Y开始后和Y结束之前开始。

        多个流并发地执行的一般现象被称为\emreg{并发(concurrency)}。
        一个进程和其他进程轮流运行的概念称为\emreg{多任务(multitasking)}。
        一个进程执行它的控制流的一部分的每一时间段叫做\emreg{时间片(time slice)}。
        因此,多任务也叫做\emreg{时间分片(time slicing)}。

        如果两个流并发地运行在不同的处理器核或者计算机上,那么称它们为\emreg{并行流(parallel flow)},它们\emreg{并行地运行(running in parallel)},且\emreg{并行地执行(parallel execution)}。

        %1.
        \begin{practicec}
            \begin{enumerate}[A.]
                \item Y
                \item N
                \item Y
            \end{enumerate}
        \end{practicec}
    }

    \subsection{私有地址空间}
    {
        在一台 $n$ 位地址的机器上,\emreg{地址空间}是 $2^n$ 个可能地址的集合。
        进程为每个程序提供它自己的\emreg{私有地址空间}。
        一般而言,和这个空间中某个地址相关联的那个内存字节是不能被其他进程读或写的,从这个意义上说,这个地址空间是私有的。
    }

    \subsection{用户模式和内核模式}
    {
        为了使操作系统内核提供一个无懈可击的进程抽象,处理器必须提供一种机制,限制一个应用可以执行的指令以及它可以访问的地址空间范围。

        处理器通常是用控制寄存器中的一个\emreg{模式位(mode bit)}来提供这种功能的,该寄存器描述了进程当前享有的特权。
        当设置了模式位时,进程就运行在\emreg{内核模式}中。
        一个运行在内核模式的进程可以执行指令集中的任何指令,并且访问系统中的任何内存位置。

        没有设置模式时,进程就运行在\emreg{用户模式}中。
        用户模式中的进程不允许执行\emreg{特权指令(privileged instruction)},比如停止处理器、改变模式位,或者发起一个I/O操作。
        也不允许用户模式中的进程直接引用地址空间中内核区内的代码和数据。
        任何这样的尝试都会导致致命的保护故障。
        用户程序必须通过系统调用接口简介的访问内核代码和数据。

        进程从用户模式变为内核模式的唯一方法是通过诸如中断、故障或者陷入系统调用这样的异常。
    }

    \subsection{上下文切换}
    {
        操作系统内核使用一种称为\emreg{上下文切换(context switch)}的较高层形式的异常控制流来实现多任务。

        内核为每个进程维持一个\emreg{上下文(context)}。
        上下文就是内核重新启动一个被抢占的进程所需的状态。

        在进程执行的某些时刻,内核可以决定抢占当前进程,并重新开始一个先前被抢占了的进程。
        这种决策就叫做\emreg{调度(scheduling)},是由内核中称为\emreg{调度器(scheduler)}的代码处理的。
        当内核选择一个新的进程运行时,就说内核\emreg{调度}了这个进程。
        在内核调度了一个新的进程运行后,它就抢占当前进程,并使用一种称为\emreg{上下文切换}的机制来将控制转移到新的进程。

        内核执行系统调用时,可能会发生上下文切换。

        中断也可能引发上下文切换。
    }
}
