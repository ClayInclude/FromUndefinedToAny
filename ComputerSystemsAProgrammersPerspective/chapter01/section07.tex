%%
%% Author: Clay
%% 2020/2/21
%%

\section{操作系统管理硬件}
{
    可以把\emreg{操作系统}看成是应用程序和硬件之间插入的一层软件。
    所有应用程序对硬件的操作尝试都必须通过操作系统。

    操作系统有两个基本功能:

    \begin{itemize}
        \item 防止硬件被失控的应用程序滥用。
        \item 向应用程序提供简单一致的机制来控制复杂而又通常大不相同的低级硬件设备。
    \end{itemize}

    操作系统通过几个基本的抽象概念(\emreg{进程}、\emreg{虚拟内存}和\emreg{文件})来实现这两个功能。
    文件是对I/O设备的抽象表示,虚拟内存是对主存和磁盘I/O设备的抽象表示,进程则是对处理器、主存和I/O设备的抽象表示。

    \subsection{进程}
    {
        \emreg{进程}是操作系统对一个正在运行的程序的一种抽象。
        在一个系统上可以同时运行多个进程,而每个进程都好像在独占地使用硬件。
        而\emreg{并发运行},则是说一个进程的指令和另一个进程的指令是交错执行的。
        一个CPU看上去像是在并发地执行多个进程,这是通过处理器在进程间切换来实现的。
        操作系统实现这种交错执行的机制称为\emreg{上下文切换}。

        操作系统保持跟踪进程运行所需的所有状态信息。
        这种状态,也就是\emreg{上下文},包括许多信息,比如PC和寄存器文件的当前值,以及主存的内容。
        当操作系统决定要把控制权从当前进程转移到某个新进程时,就会进行\emreg{上下文切换},即保存当前进程的上下文、恢复新进程的上下文。

        从一个进程到另一个进程的转换是由操作系统\emreg{内核(kernel)}管理的。
        内核是操作系统代码常驻主存的部分。
        当应用程序需要操作系统的某些操作时,它就执行一条特殊的\emreg{系统调用(system call)}指令,将控制权传递给内核。
        然后内核执行被请求的操作并返回应用程序。
        内核并不是一个独立的进程。
        相反,它是系统管理全部进程所用代码和数据结构的集合。
    }

    \subsection{线程}
    {
        一个进程实际上可以由多个称为\emreg{线城}的执行单元组成,每个线程都运行在进程的上下文中,并共享同样的代码和全局数据。
        多线程之间比多进程之间更容易共享数据,线程一般来说都比进程更高效。
    }

    \subsection{虚拟内存}
    {
        \emreg{虚拟内存}是一个抽象概念,它为每个进程提供了一个假象,即每个进程都在独占地使用内存。
        每个进程看到的内存都是一致的,称为\emreg{虚拟地址空间}。

        \begin{description}
            \item[程序代码和数据]
            {
                对所有的进程来说,代码是从同一固定地址开始,紧接着的是和C全局变量相对应的数据位置。
                代码和数据区是直接按照可执行目标文件的内容初始化的。
            }
            \item[堆]
            {
                代码和数据区后紧随着的是运行时\emreg{堆}。
                堆可以在运行时动态地扩展和收缩。
            }
            \item[共享库]
            {
                大约在地址空间的中间部分是一块用来存放像C标准库和数学库这样的共享库的代码和数据的区域。
            }
            \item[栈]
            {
                位于用户虚拟地址空间顶部的是\emreg{用户栈},编译器用它来实现函数调用。
                特别地,每调用一个函数,栈就会增长;
                从一个函数返回时,栈就会收缩。
            }
            \item[内核虚拟内存]
        \end{description}
    }

    \subsection{文件}
    {
        \emreg{文件}就是字节序列。
        每个I/O设备都可以看成是文件。
    }
}
