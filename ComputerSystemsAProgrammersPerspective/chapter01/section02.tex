%%
%% Author: Clay
%% 2020/2/20
%%

\section{程序被其他程序翻译成不同的格式}
{
    为了在系统上运行C程序,每条C语句都必须被其他程序转化为一系列的低级\emreg{机器语言}指令。
    然后这些指令按照一种称为\emreg{可执行目标程序}的格式打好包,并以二进制文件的形式存放起来。
    目标程序也称为\emreg{可执行目标文件}。

    从源文件到目标文件的转化是由\emreg{编译器驱动程序}完成的。

    翻译过程分为四个阶段完成。
    执行这四个阶段的程序(\emreg{预处理器}、\emreg{编译器}、\emreg{汇编器}和\emreg{链接器})一起构成了\emreg{编译系统(compilation system)}。

    \begin{description}
        \item[预处理阶段] 预处理器(cpp)根据以字符\emcode{\#}开头的命令,修改原始的C程序。
        \item[编译阶段] 编译器(ccl)将文本文件翻译成一个\emspe{汇编语言程序}。
        \item[汇编阶段] 汇编器(as)将汇编语言翻译成机器语言指令,把这些指令打包成一种叫做\emspe{可重定位目标程序(relocatable object program)}的格式。
        \item[链接阶段] 链接器(ld)处理合并,结果得到一个\emspe{可执行目标文件}(简称为\emspe{可执行文件}),可以被加载到内存中,由系统执行。
    \end{description}
}
