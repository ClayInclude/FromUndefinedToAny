%%
%% Author: Clay
%% 2020/2/20
%%

\section{处理器读并解释存储在内存中的指令}
{
    \emcode{shell}是一个命令行解释器,它输出一个提示符,等待输入一个命令行,然后执行这个命令。
    如果该命令行的第一个单词不是一个内置的shell命令,那么shell就会假设这是一个可执行文件的名字,它将加载并运行这个文件。

    \subsection{系统的硬件组成}
    {
        \subsubsection{总线}
        {
            贯穿整个系统的是一组电子管道,称作\emreg{总线},它携带信息字节并负责在各个部件间传递。
            通常总线被设计成传送定长的字节块,也就是\emreg{字(word)}。
        }

        \subsubsection{I/O设备}
        {
            I/O(输入/输出)设备是系统与外部世界的联系通道。
            示例系统包括四个I/O设备:作为用户输入的键盘和鼠标,作为用户输出的显示器,以及用于长期存储数据和程序的磁盘驱动器。

            每个I/O设备都通过一个\emreg{控制器}或\emreg{适配器}与I/O总线相连。
            控制器与适配器之间的区别主要在于它们的封装方式。
            控制器是I/O设备本身或者系统的主印制电路板(通常称作\emreg{主板})上的芯片组。
            而适配器则是一块插在主板插槽上的卡。
        }

        \subsubsection{主存}
        {
            \emreg{主存}是一个临时存储设备,在处理器执行程序时,用来存放程序和程序处理的数据。
            从物理上来说,主存是由一组\emreg{动态随机存取存储器(DRAM)}芯片组成的。
            从逻辑上来说,存储器是一个线性的字节数组,每个字节都有其唯一的地址(数组索引),这些地址是从零开始的。
            一般来说,组成程序的每条机器指令都由不同数量的字节构成。
        }

        \subsubsection{处理器}
        {
            \emreg{中央处理单元(CPU)},简称\emreg{处理器},是解释(或\emreg{执行})存储在主存中指令的引擎。
            处理器的核心是一个大小为一个字的存储设备(或\emreg{寄存器}),称为\emreg{程序计数器(PC)}。
            在任何时刻,PC都指向主存中的某条机器语言指令。

            处理器\emreg{看上去}是按照一个非常简单的指令执行模型来操作的,这个模型是由\emreg{指令集架构}决定的。

            这些操作围绕着主存、\emreg{寄存器文件(register file)}和\emreg{算数/逻辑单元(ALU)}进行。
            寄存器文件是一个小的存储设备,由一些单个字长的寄存器组成,每个寄存器都有唯一的名字。
            ALU计算新的数据和地址。
            CPU在指令的要求下可能会执行这些操作:

            \begin{description}
                \item[加载] 从主存复制一个字节或一个字到寄存器,以覆盖寄存器原来的内容。
                \item[存储] 从寄存器复制一个字节或者一个字到主存的某个位置,以覆盖这个位置上原来的内容。
                \item[操作] 把两个寄存器的内容复制到ALU,ALU对这两个字做算术运算,并将结果存放到一个寄存器中,以覆盖该寄存器中原来的内容。
                \item[跳转] 从指令本身抽取一个字,并将这个字复制到PC中,以覆盖PC中原来的值。
            \end{description}

            指令集架构描述的是每条机器代码指令的效果;
            而\emreg{微体系结构}描述的是处理器实际上是如何实现的。
        }
    }

    \subsection{运行hello程序}
    {
        初始时,shell程序执行它的指令,等待输入命令。
        当在键盘上输入\emcode{./hello}后,shell程序将字符逐一读入寄存器,再存放到内存中。

        当在键盘上敲回车键时,shell程序就知道已经结束了命令的输入。
        然后shell程序执行一系列指令来加载可执行文件,将目标文件中的代码和数据从磁盘复制到主存。

        利用\emreg{直接存储器存取(DMA)}技术,数据可以不通过处理器而直接从磁盘到达主存。

        一旦目标文件中的代码和数据被加载到主存,处理器就开始执行程序中的机器语言指令。
    }
}
