%%
%% Author: Clay
%% 2020/2/21
%%

\section{重要主题}
{
    \subsection{Amdahl定律}
    {
        \begin{defines}[Amdahl定律]
            当我们对系统的某个部分加速时,其对系统整体性能的影响取决于该部分的重要性和加速程度。
            若系统执行某应用程序需要时间为 $T_{old}$ 。
            假设系统某部分所需执行时间与改时间的比例为 $\alpha$ ,而该部分性能提升比例为 $k$ 。
            总时间为:
            $$T_{new} = (1 - \alpha)T_{old} + \frac{\alpha T_{old}}{k} = T_{old}[(1 - \alpha) + \frac{\alpha}{k}]$$
            加速比 $S = T_{old} / T_{new}$ 为:
            $$S = \frac{1}{(1 - \alpha) + \alpha / k}$$
        \end{defines}

        要想显著加速整个系统,必须提升全系统中相当大的部分的速度。

        %1.
        \begin{practicec}
            \begin{enumerate}[A.]
                \item $1.25$
                \item $300$
            \end{enumerate}
        \end{practicec}

        %2.
        \begin{practicec}
            $2.67$
        \end{practicec}

        Amdahl定律一个有趣的特殊情况是考虑 $k$ 趋向于 $\infty$ 时的效果。
        这就意味着,可以取系统的某一部分将其加速到一个点上,在这个点上,这部分时间可以忽略不计,于是得到:
        $$S_\infty = \frac{1}{1 - \alpha}$$
    }

    \subsection{并发和并行}
    {
        数字计算机的整个历史中,有两个需求是驱动进步的持续动力:
        一个是想要计算机做得更多,另一个是想要计算机运行得更快。
        \emreg{并发(concurrency)}是一个通用的概念,指一个同时具有多个活动的系统;
        \emreg{并行(parallelism)}指的是用并发来使一个系统运行得更快。
        并行可以在计算机系统的多个抽象层次上运用。

        \subsubsection{线程级并发}
        {
            构建在进程这个抽象上,能够设计出同时有多个程序执行的系统,这就导致了并发。
            传统意义上,这种并发执行只是\emreg{模拟}出来的,是通过使一台计算机在它正在执行的进程间快速切换来实现的。
            在以前,即使处理器必须在多个任务间切换,大多数实际的计算也都是由一个处理器来完成的。
            这种配置称为\emreg{单处理器系统}。

            当构建一个由但操作系统内核控制的多处理器组成的系统时,就得到了一个\emreg{多处理器系统}。

            \emreg{多核处理器}是将多个CPU集成到一个电路芯片上。

            \emreg{超线程},有时称为\emreg{同时多线程(simultaneous multi-threading)},是一项允许一个CPU执行多个控制流的技术。

            多处理器的使用可以从两方面提高系统性能。
            首先,它减少了在执行多个任务时模拟并发的需要。
            其次,它可以使应用程序运行的更快,当然,这必须要求程序是以多线程方式来书写的,这些线程可以并行地高效执行。
        }

        \subsubsection{指令级并行}
        {
            现代处理器可以同时执行多条指令的属性称为\emreg{指令级并行}。
        }
    }
}
