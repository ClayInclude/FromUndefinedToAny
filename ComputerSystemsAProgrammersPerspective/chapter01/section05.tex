%%
%% Author: Clay
%% 2020/2/20
%%

\section{高速缓存至关重要}
{
    系统花费了大量的时间把信息从一个地方挪到另一个地方。
    这些复制就是开销,减慢了程序真正的工作。
    因此,系统设计这的另一个主要目标就是使这些复制操作尽可能快地完成。

    根据机械原理,较大的存储设备要比较小的存储设备运行得慢,而快速设备的造价远高于同类的低速设备。

    随着半导体技术的进步,这种\emreg{处理器与主存之间的差距}还在持续增大。
    加快处理器的运行速度比加快主存的运行速度要容易和便宜得多。

    针对这种处理器与主存之间的差异,系统设计者采用了更小更快的存储设备,称为\emreg{高速缓存存储器(cache memory,简称cache或高速缓存)},作为暂时的集结区域,存放处理器近期可能会需要的信息。
    L1和L2高速缓存使用一种叫做\emreg{静态随机访问存储器(SRAM)}的硬件技术实现的。
    系统可以获得一个很大的存储器,同时访问速度也很快,原因是利用了高速缓存的\emreg{局部性原理},即程序具有访问局部区域里的数据和代码的趋势。
    通过让高速缓存里存放可能经常访问的数据,大部分的内存操作都能在快速的高速缓存中完成。
}
