%%
%% Author: Clay
%% 2021/2/20
%%

\section{共享文件}
{
    内核用三个相关的数据结构来表示打开的文件:

    \begin{description}
        \item[描述符表(descriptor table)]
        {
            每个进程都有它独立的描述符表,它的表项是由进程打开的文件描述符来索引的。
            每个打开的描述符表项指向\emspe{文件表}中的一个表项。
        }
        \item[文件表(file table)]
        {
            打开文件的集合是由一张文件表来表示的,所有的进程共享这张表。
            每个文件表的表项组成包括当前的文件位置、\emspe{引用计数(reference count)},以及一个指向\emspe{v--node表}中对应表项的指针。
            关闭一个描述符会减少相应的文件表表项中的引用计数。
            内核知道表项的引用计数为零才会删除这个文件表表项。
        }
        \item[v-node表(v--node table)] 所有的进程共享这张表。
    \end{description}

    多个描述符可以通过不同的文件表项来引用同一个文件。
    例如同一个filename调用open函数两次。

    子进程有一个父进程描述符表的副本。
    父子进程共享相同的打开文件表集合,因此共享相同的文件位置。
    在内核删除相应文件表表项之前,父子进程都必须关闭了他们的描述符。

    %2.
    \begin{practicec}
        f
    \end{practicec}

    %3.
    \begin{practicec}
        o
    \end{practicec}
}
