%%
%% Author: Clay
%% 2021/2/19
%%

\section{文件}
{
    每个Linux文件都有一个\emreg{类型(type)}来表明它在系统中的角色:

    \begin{description}
        \item[普通文件(reguluarfile)]
        {
            包含任意数据。
            应用程序常常要区分\emspe{文本文件(text file)}和\emspe{二进制文件(binary file)}。
            对内核而言,文本文件和二进制文件没有区别。
        }
        \item[目录(directory)]
        {
            包含一组\emspe{链接(link)}的文件,其中每个链接都将一个\emspe{文件名(filename)}映射到一个文件,这个文件可能是另一个目录。
            每个目录都至少包含两个条目:
            \emcode{.}是到该目录自身的映射,以及\emcode{..}是到目录层次结构中\emspe{父目录(parent directory)}的链接。
        }
        \item[套接字(socket)] 用来与另一个进程进行跨网络通信的文件。
    \end{description}

    其他文件类型包含\emreg{命名通道(named pipe)}、\emreg{符号链接(symbolic link)}以及\emreg{字符和块设备(character and block device)}。

    Linux内核将所有文件都组织成一个\emreg{目录层次结构(directory hierarch)},由\emreg{根目录}确定。
    系统每个文件都是根目录的直接或间接的后代。

    作为其上下文的一部分,每个进程都有一个\emreg{当前工作目录(current working directory)}来确定其在目录层次结构中的当前位置。

    目录层次结构中的位置用\emreg{路径名(pathname)}来指定。
    路径是一个字符串,包括一个可选斜杠,其后紧跟一系列的文件名,文件名之间用斜杠分隔。
    路径名有两种形式:

    \begin{description}
        \item[绝对路径名(absolute pathname)] 以一个斜杠开始,表示从根节点开始的路径。
        \item[相对路径名(relative pathname)] 以文件名开始,表示从当前工作目录开始的路径。
    \end{description}
}
