%%
%% Author: Clay
%% 2021/2/20
%%

\section{读和写文件}
{
    应用程序是通过分别调用read和write函数来执行输入和输出的。

    read函数从描述符为fd的当前文件位置复制最多 $n$ 个字节到内存位置buf。
    write函数从内存位置buf至多 $n$ 个字节到描述符fd的当前文件位置。

    通过调用lseek函数,应用程序能够显示地修改当前文件的的位置。

    在某些情况下,read和write传送的字节比应用程序要求的要少。
    这些\emreg{不足值(short count)}不表示有错误。
    出现这样情况的原因有:

    \begin{description}
        \item[读时遇到EOF]
        \item[从终端读文本行]
        \item[读写网络套接字(socket)和Linux管道(pipe)]
    \end{description}
}
