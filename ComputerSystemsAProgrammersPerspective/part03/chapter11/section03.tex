%%
%% Author: Clay
%% 2021/2/20
%%

\section{全球IP因特网}
{
    每台因特网主机都运行实现\emreg{TCP/IP协议(Transmission Control Protocol/Internet Protocol, 传输控制协议/互联网协议)}的软件,几乎每个现代计算机系统都支持这个协议。
    因特网的客户端和服务器混合使用\emreg{套接字接口}函数和Unix I/O函数来进行通信。

    TCP/IP实际是一个协议族,其中每一个都提供不同的功能。

    从程序员的角度,可以把因特网看作一个世界范围的主机集合,满足以下特性:

    \begin{itemize}
        \item 主机集合被映射为一组32位的\emspe{IP地址}
        \item 这组IP地址被映射为一组称为\emspe{因特网域名(Internet domain name)}的标识符
        \item 因特网主机上的进程能够通过\emspe{连接(connection)}和任何其他因特网主机上的进程通信。
    \end{itemize}

    \subsection{IP地址}
    {
        一个IP地址就是一个32位无符号整数。

        TCP/IP为任意整数数据项定义了统一的\emreg{网络字节顺序(network byte order)}(大端字节顺序)。

        IP地址通常是以一种称为\emreg{点分十进制表示法}来表示的。

        %1.
        \begin{practicec}
            \begin{table}
                \[
                    \begin{array}{|c|c|}
                        \hline
                        \text{十六进制地址} & \text{点分十进制地址} \\
                        \hline
                        0x0 & 0.0.0.0 \\
                        \hline
                        0xffffffff & 255.255.255.255 \\
                        \hline
                        0x7f000001 & 127.0.0.1 \\
                        \hline
                        0xcdbca079 & 205.188.160.121 \\
                        \hline
                    \end{array}
                \]
            \end{table}
        \end{practicec}
    }

    \subsection{因特网域名}
    {
        大整数是很难记住的,所以因特网定义了一组更加人性化的\emreg{域名(domain name)},以及一种将域名映射到IP地址的机制。

        域名集合形成了一个层次结构,每个域名编码了它在这个层次中的位置。

        树的节点表示域名,反向到根的路径形成了域名。
        子树称为\emreg{子域(subdomain)}。
        第一层是一个未命名的根节点。
        下一层是一组\emreg{一级域名(first--level domain name)},由非营利组织\emreg{ICANN(Internet Corporation for Assigned Names and Numbers, 因特网分配名字数字协会)}定义。

        下一层是\emreg{二级(second--level)}域名。
        一旦一个组织得到了一个二级域名,那么它就可以在这个子域中创建任何新的域名了。

        因特网定义了域名集合和IP地址集合之间的映射。
        知道1988年,这个映射都是通过一个叫做HOSTS.TXT的文本文件来手工维护的。
        从那以后,这个映射是通过分布世界范围内的数据库(称为\emreg{DNS(Domain Name System, 域名系统)})来维护的。
        从概念上而言,DNS数据库由上百万的\emreg{主机条目结构(host entry structure)}组成,其中每条定义了一组域名和一组IP地址之间的映射。

        每台因特网主机都有本地定义的域名localhost,这个域名总是映射为\emreg{回送地址(127.0.0.1)}。

        域名和IP地址之间可以实一一映射,多个域名也可以映射为同一个IP地址,多个域名可以映射到同一组的多个IP地址.
        某些合法的域名没有映射到任何IP地址。
    }

    \subsection{因特网连接}
    {
        因特网客户端和服务器在\emreg{连接}上发送和接收字节流来通信。
        从连接一对进程的意义上而言,连接是\emreg{点对点}的。
        从数据可以同时双向流动的角度来说,它是\emreg{全双工}的。
        并且从由源进程发出的字节流最终被目的进程以它发出的顺序收到它的角度来说,它也是\emreg{可靠}的。

        一个\emreg{套接字}是连接的一个端点。
        每个套接字都有相应的\emreg{套接字地址},是由一个因特网地址和一个16位的整数\emreg{端口}组成的。

        当客户端发起一个连接请求时,客户端套接字地址中的端口是由内核自动分配的,称为\emreg{临时端口(ephemeral port)}。
        服务器套接字地址中的端口通常是某个\emreg{知名端口},是和这个服务相对应的。

        一个连接是由它两端的套接字地址唯一确定的。
        这对套接字地址叫做\emreg{套接字对(socket pair)}。

        $$(cliaddr:cliport, servaddr:servport)$$
    }
}
