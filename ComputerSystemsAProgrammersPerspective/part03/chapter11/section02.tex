%%
%% Author: Clay
%% 2021/2/20
%%

\section{网络}
{
    客户端和服务器通常运行在不同的主机上,并且通过\emreg{计算机网络}的硬件和软件资源来通信。

    对主机而言,网络只是又一种I/O设备,是数据源和数据接收方。

    物理上而言,网络是一个按照地理远近组成的层次系统。
    最底层是\emreg{LAN(Local Area Network, 局域网)},在一个建筑或者校园范围内。

    一个\emreg{以太网段(Ethernet segment)}包括一些电缆和一个叫做\emreg{集线器}的小盒子。
    每根电缆都有相同的最大位带宽。
    一端连接到主机的适配器,而另一端则连接到集线器的一个\emreg{端口}上。

    每个以太网适配器都有一个全球唯一的48位地址,它存储在这个适配器的非易失性存储器上。
    一台主机可以发送一段位(称为\emreg{帧(frame)})到这个网段内的任何其他主机。
    每个帧包括一些固定数量的\emreg{头部(header)}位,用来标识此帧的源和目的地址以及此帧的长度,此后紧随的就是数据位的\emreg{有效载荷(payload)}。
    每个主机适配器都能看到这个帧,但是只有目的主机实际读取他。

    使用一些电缆和叫做\emreg{网桥(bridge)}的小盒子,多个以太网段可以连接成较大的局域网,称为\emreg{桥接以太网(bridged Ethernet)}。

    网桥比集线器更充分地利用了电缆宽带。
    利用一种聪明的分配算法,它们随着时间自动学习哪个主机可以通过哪个端口可达,然后只在有必要时,有选择地将帧从一个端口复制到另一个端口。

    多个不兼容的局域网可以通过叫做\emreg{路由器(router)}的特殊计算机连接起来,组成一个\emreg{internet(互联网络)}。
    每台路由器对于它所连接到的每个网络都有一个适配器(端口)。
    路由器也能连接高速点到点电话连接,这是称为\emreg{WAN(Wide-Area Network, 广域网)}的网络示例。

    互联网络至关重要的特性是,它能由采用完全不同和不兼容技术的各种局域网和广域网组成。
    每台主机和其它主机都是物理相连的,但是如何能够让某台\emreg{源主机}跨过所有这些不兼容的网络发送数据位到另一台\emreg{目的主机}呢?

    解决办法是每一层运行在每台主机和路由器上的\emreg{协议软件}它消除了不同网络之间的差异。
    这个软件实现一种\emreg{协议},这种协议控制主机和路由器如何协同工作来实现数据传输。
    这种协议必须提供两种基本能力:

    \begin{description}
        \item[命名机制]
        {
            不同的局域网技术有不同和不兼容的方式来为主机分配地址。
            互连网络协议通过定义一种一致的主机地址格式消除了这些差异。
            每台主机会被分配至少一个这种\emspe{互联网络地址(internet address)},这个地址唯一标识了这台主机。
        }
        \item[传送机制]
        {
            互联网协议通过定义一种把数据位捆扎成不连续的片(称为\emspe{包})的统一方式消除了差异。
            一个包是由\emspe{头}和\emspe{有效载荷}组成的。
        }
    \end{description}
}
