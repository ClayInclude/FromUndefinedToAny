%%
%% Author: Clay
%% 2020/8/12
%%

\chapter{Preface}
{
    \section{Content of This Book}
    {
        The text is organized in ten major parts:

        \begin{description}
            \item[Overview] Explain what operating systems are, what they do, and how the are designed and constructed.
            \item[Process management]
            {
                A \emreg{process} is the unit of work in a system.
                Such a system consists of a collection of \emreg{concurrently} executing processes, some executing operating-system code and others executing user code.
            }
            \item[Process synchronization] Cover methods for process synchronization and deadlock handling.
            \item[Memory management] Deal with the management of main memory during the execution of a process.
            \item[Storage management] Describe how mass storage and I/O are handled in a modern computer system.
            \item[File systems] Discuss how file systems are handled in a modern computer system.
            \item[Security and protection] discuss the mechanisms necessary for the security and protection of computer systems.
            \item[Advanced topics] Discuss virtual machines and networks / distributed systems.
            \item[Case studies] Present detailed case studies of two real operating systems.
            \item[Appendices]
        \end{description}
    }

    \section{Programming Environments}
    {
        The text provides several example programs written in C and Java.
        These programs are intended to run in the following programming environments:

        \begin{description}
            \item[POSIX] \emreg{POSIX} (which stands for \emreg{Portable Operating System Interface}) represents a set of standards implemented primarily for UNIX-based operating systems.
            \item[Java] Java is a widely used programming language with a rich API and build-in language support for concurrent and parallel programming.
            \item[Windows systems] The primary programming environment for Windows systems is the Windows API, which provides a comprehensive set of functions for managing processes, threads, memory, and peripheral devices.
        \end{description}
    }
}
